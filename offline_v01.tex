\documentclass[onecolumn,10pt,titlepage]{article}
\usepackage[a4paper,top=2.5cm,bottom=2cm,left=3cm,right=2cm,marginparwidth=1.75cm,headheight=28pt]{geometry}
\usepackage[utf8]{inputenc}
\usepackage[ngerman]{babel} 
\usepackage{natbib}
\usepackage{multicol}
\usepackage{multirow}
\usepackage{mathastext}
%BORRAR O DEJAR: Cambia estilo de la fuente matemática y la deja con aspecto mas "técnico".
\usepackage{graphicx}
\usepackage{amssymb}
\usepackage{wrapfig}
\usepackage{float}
\usepackage{gensymb}
\usepackage[tc]{titlepic}
\usepackage{hyperref}
\hypersetup{
    colorlinks,
    citecolor=black,
    filecolor=black,
    linkcolor=black,
    urlcolor=black
}

%Helvetica
%\renewcommand{\familydefault}{\sfdefault}
%\usepackage[scaled=1]{helvet}
%\usepackage[format=plain,
%            labelfont={bf,it},
%            textfont=it]{caption}
            
%Latin modern sans


\renewcommand{\familydefault}{\sfdefault}
\usepackage{lmodern}
\usepackage[format=plain,
            labelfont={bf,it},
            textfont=it]{caption}
\fontfamily{lmss}\selectfont
%--------------------------------------

\renewcommand{\labelitemii}{$\circ$}
\renewcommand{\labelitemiii}{$\triangleright$}
%==============================================
\usepackage{fancyhdr}

%\renewcommand{\chaptermark}[1]{\markboth{#1}{}}
%
%\fancyhf{} % clear the headers
%\fancyhead[R]{%
%	% We want italics
%	\itshape
%%	% The chapter number only if it's greater than 0
%%	\ifnum\value{chapter}>0 \chaptername\ \thechapter. \fi
%%	% The chapter title
%%	\leftmark}
%\fancyfoot[C]{\thepage}
%
%\fancypagestyle{plain}{
%	\renewcommand{\headrulewidth}{0pt}
%	\fancyhf{}
%	\fancyfoot[C]{\thepage}
%}
%
%\setlength{\headheight}{14.5pt}
%\usepackage[automark]{scrpage2}
%\pagestyle{scrheadings}


\lhead{\nouppercase{\rightmark} } %(\nouppercase{\leftmark})}
\chead{}
\rhead{}
%\lfoot{\today}
\cfoot{}
\rfoot{\thepage}
\renewcommand{\headrulewidth}{0.4pt}
\renewcommand{\footrulewidth}{0.4pt}

%\renewcommand{\chaptermark}[1]{\markboth{#1}{}}
%\renewcommand{\chaptermark}[1]{\markboth{\MakeUppercase{\chaptername\ \thechapter.\ #1}}{}}


% % % % % % % % %gloss
%\usepackage{glossaries}
%\makeglossaries % create makeindex files
%%\newglossaryentry{}{name={},description={}}
%\newglossaryentry{pv}{name={PV},description={Photovoltaik}}
%\newglossaryentry{wea}{name={WEA},description={Windenergieanlage}}
%\newglossaryentry{ee}{name={EE},description={Erneuerbare Energien}}
%\newglossaryentry{ns}{name={NS},description={Niederspannung}}
%\newglossaryentry{ms}{name={MS},description={Mittelspannung}}
%\newglossaryentry{hs}{name={HS},description={Hochspannung}}
%
%\newglossaryentry{kw}{name={KW},description={Kraftwerk}}
%\newglossaryentry{vn}{name={VN},description={Verteilnetz}}
%\newglossaryentry{vnb}{name={VNB},description={Verteilnetzbetreiber}}
%
%
%\newglossaryentry{psw}{name={PSW},description={Pumpspeicherkraftwerk}}
%\newglossaryentry{facts}{name={FACTS},description={flexible alternating current transmission system}}
%\newglossaryentry{eeg}{name={EEG},description={ErneuerbareEnergienGesetz}}
%\newglossaryentry{brd}{name={BRD},description={Bundesrepublik Deutschland}}
%\newglossaryentry{nwa}{name={NWA},description={Netzwiederaufbau}}
%\newglossaryentry{thg}{name={THG},description={Treibhausgas}}
%\newglossaryentry{dea}{name={DEA},description={Dezentrale Erzeugungs Anlagen}}
%\newglossaryentry{ikt}{name={IKT},description={Informations- und Kommunikationstechnologie}}
%\newglossaryentry{dsm}{name={DSM},description={DemandSidemanagement}}
%\newglossaryentry{kkw}{name={KKW},description={Kernkraftwerk}}
%\newglossaryentry{ptg}{name={PtG},description={Power-to-Gas}}
%\newglossaryentry{scada}{name={SCADA},description={Supervisory Control and Data Aquisition}}
%\newglossaryentry{ael}{name={AEL},description={Alkaline Electrolyzer}}
%\newglossaryentry{pem}{name={PEM},description={Proton Exchange Membrane}}
%\newglossaryentry{soel}{name={SOEL},description={Solide Oxide Elektrolyzer}}
%\newglossaryentry{el}{name={EL},description={Elektrolyse/ Elektrolyseur}}
%%\newcommand{\glossentry}[2]{$#1$ \indent #2 \par \vspace{.4cm} }

%======================== test ============
\usepackage{siunitx}
\usepackage[acronym,toc]{glossaries}              % use glossaries-package


\setlength{\glsdescwidth}{15cm}

\newglossary[slg]{symbolslist}{syi}{syg}{Symbolslist} % create add. symbolslist


\glsaddkey{unit}{\glsentrytext{\glslabel}}{\glsentryunit}{\GLsentryunit}{\glsunit}{\Glsunit}{\GLSunit}

\makeglossaries                                   % activate glossaries-package


% ==== EXEMPLARY ENTRY FOR SYMBOLS LIST =========================================
\newglossaryentry{symb:Pi}{name=\ensuremath{\pi},
	description={Geometrical value},
	unit={},
	type=symbolslist}

\newglossaryentry{height}{name=\ensuremath{h},
	description={Height of tower},
	unit={\si{m}},
	type=symbolslist}

\newglossaryentry{energyconsump}{name=\ensuremath{P},
	description={Energy consumption},
	unit={\si{kW}},
	type=symbolslist}




% ==== EXEMPLARY ENTRY FOR ACRONYMS LIST ========================================
\newacronym{VRBD}{VRBD}{Violet-Red-Bile-Glucose-Agar}


% ==== EXEMPLARY ENTRY FOR MAIN GLOSSARY ========================================
\newglossaryentry{Biofouling}{name=Biofouling,description={Some description}}
\newglossaryentry{ael}{name={AEL},description={Alkaline Electrolyzer}}
\newglossaryentry{pem}{name={PEM},description={Proton Exchange Membrane}}
\newglossaryentry{soel}{name={SOEL},description={Solide Oxide Elektrolyzer}}
\newglossaryentry{el}{name={EL},description={Elektrolyse/ Elektrolyseur}}
\newglossaryentry{nt}{name={NT},description={Niedertemperatur-}}
\newglossaryentry{ht}{name={HT},description={Hochtemperatur-}}
\newglossaryentry{eis}{name={EIS},description={Einspeise-Management (Abkürzung für diesen Betriebstyp)}}
\newglossaryentry{res}{name={RES},description={Residual (-last, -erzeugung)(Abkürzung für diesen Betriebstyp)}}
\newglossaryentry{wea}{name={WEA},description={Windenergieanlage(Abkürzung für diesen Betriebstyp)}}
\newglossaryentry{cost}{name={COST},description={Kosten-abhängiges-Signal (Abkürzung für diesen Betriebstyp)}}
%%\newcommand{\glossentry}[2]{$#1$ \indent #2 \par \vspace{.4cm} }

\newglossarystyle{symbunitlong}{%
	\setglossarystyle{long3col}% base this style on the list style
	\renewenvironment{theglossary}{% Change the table type --> 3 columns
		\begin{longtable}{lp{0.6\glsdescwidth}>{\centering\arraybackslash}p{2cm}}}%
		{\end{longtable}}%
	%
	\renewcommand*{\glossaryheader}{%  Change the table header
		\bfseries Sign & \bfseries Description & \bfseries Unit \\
		\hline
		\endhead}
	\renewcommand*{\glossentry}[2]{%  Change the displayed items
		\glstarget{##1}{\glossentryname{##1}} %
		& \glossentrydesc{##1}% Description
		& \glsunit{##1}  \tabularnewline
	}
}

% ======================= end test =============


% \publishers{}

% \thanks{} %% use it instead of footnotes (only on titlepage)

% \dedication{} %% generates a dedication-page after titlepage


%%% uncomment following lines, if you want to:
%%% reuse the maketitle-entries for hyperref-setup
%\newcommand\org@maketitle{}
%\let\org@maketitle\maketitle
%\def\maketitle{%
%  \hypersetup{
%    pdftitle={\@title},
%    pdfauthor={\@author}
%    pdfsubject={\@subject}
%  }%
%  \org@maketitle
%}



% Definition of \maketitle
%\makeatletter         
%\def\@maketitle{
%	%\raggedright%
%	\includegraphics[height = 15mm]{logo_fb4_Bild2.png}%
%	\hfill
%	\includegraphics[height = 12mm]{Logo_uni_Bild1.png}\\[8ex]
%	\begin{center}
%		{\Huge \bfseries \sffamily \@title }\\[4ex] 
%		{\Large  \@author}\\[4ex] 
%		\@date\\[8ex]
%		%\includegraphics[width = 40mm]{image.png}
%	\end{center}}
%	\makeatother
%\titlepic{\includegraphics[width=0.3\textwidth]{logo_fb4_Bild2.png} \hspace{2cm} %\includegraphics[width=0.3\textwidth]{Logo_uni_Bild1.png}}
%\title{Modellierung des dynamischen Betriebsverhaltens von PtG-Anlagen als Systemdienstleistung für Stromnetze}

%\date{\today}

%\author{Lorenz Beck (3093543) \and 
%David Fuhrländer (3145486) \and 
%Jörn Lönneker (3145360)}


%========================> Comienza Documento
\begin{document}

%\maketitle
%\newpage
% \begin{titlepage}
% \centering

% { \large Universität Bremen  \par }
% \vspace{2cm}
% {\Large \scshape ?? \par}
% \vspace{2cm}
% {\Huge \scshape Modellierung des dynamischen Betriebsverhaltens von PtG-Anlagen als Systemdienstleistung für Stromnetze \par }
% %\vspace{1cm}
% %{\large \bf Grupo 10 \par}
% \vspace{0cm}
% \textsc{\large Autor -- Legajo \\ Coautor -- Legajo}
% \vspace{2cm}
% {\par \large Fecha de realización: \today \par}
% \vspace{1cm}
% {\large Fecha de entrega: .......................................\par}
% \vspace{2.5cm}
% {\large Firma del docente: .......................................}
% \vspace{3cm}
% \begin{figure}[htb!]
% \centering
% \includegraphics[width=6cm]{logoitba.png}
% \end{figure}
% \end{titlepage}

\begin{titlepage}
	\centering
	%\includegraphics[width=0.15\textwidth]{example-image-1x1}\par\vspace{1cm}
	
	{\includegraphics[height = 12mm]{Logo_uni_Bild1.png}\par}
	\vspace{1cm}
	{\scshape\Large Universität Bremen \par}
	\vspace{1cm}
	{\scshape\LARGE MScPT1 - Projektarbeit \par}
	\vspace{1.5cm}
	{\huge\bfseries Modellierung des dynamischen Betriebsverhaltens von PtG-Anlagen als Systemdienstleistung für Stromnetze\par}
	\vspace{2cm}
    {\large \today\par}
    \vspace{2cm}
	{\Large\itshape Lorenz Beck (3093543)\par}
    \vspace{0.3cm}
    {\Large\itshape David Fuhrländer (3145486)\par}
    \vspace{0.3cm}
    {\Large\itshape Jörn Lönneker (3145360)\par}
	\vfill
	betreuender Professor:\par
	Prof. Dr. rer. nat. ~S. \textsc{Gößling Reisemann}\\
    \vfill
   	\includegraphics[height = 20mm]{RES_EN_logo_de_rot_mittel.png}
	\hfill	
	\includegraphics[height = 20mm]{logo_fb4_Bild2.png}%
	%\vfill

% Bottom of the page
	
\end{titlepage}

\pagenumbering{Roman}

\section*{Übersicht}
Hallo  David

{\textbf{Abstract}}\par
% Un resumen del problema a tratar debería tener menos de 200 palabras y no incluir referencias. Un resumen del problema a tratar debería tener menos de 200 palabras y no incluir referencias. Un resumen del problema a tratar debería tener menos de 200 palabras y no incluir referencias. Un resumen del problema a tratar debería tener menos de 200 palabras y no incluir referencias. Un resumen del problema a tratar debería tener menos de 200 palabras y no incluir referencias.

%\begin{multicols}{2}
% \section*{Nomenklatur}
% \glossentry{\pi}{pi}
% \glossentry{g}{pff}
%\glossentry{x}{}
%\end{multicols}

\newpage

\tableofcontents

\newpage

%  ==== glossary
\glsaddall

\printglossary[type=\acronymtype,style=long]  % list of acronyms
\printglossary[type=symbolslist,style=symbunitlong]   % list of symbols
\printglossary[type=main]                     % main glossary
% ===== 
\pagenumbering{arabic}
\pagestyle{fancy}
\section{Einleitung}
Schröer: Rolle von PtG -> günstiger als BAtterien


%mathe-gedöns \verb|mathastext|:
%$e^{i\pi}+1=0$ 

%$$\int^\infty_0 \frac{1}{x^2}.dx $$

%Esto se puede cambiar simplemente borrando la linea de código \verb|7|, por defecto.
Gleichung:
\begin{equation}
5 \cdot x = 25
% * <dafu@posteo.de> 2018-06-12T07:30:05.822Z:
% 
% da hab ich mal nen kommentar geschrieben
% 
% ^.
\end{equation}
\subsection{Vorbemerkungen}
Farbige Grafiken können in der Druckversion ggf. Kontrastschwächen aufweisen. Diesbezüglich sei auf die Digitalversion dieser Arbeit verwiesen.\\


\subsection{Problemstellung}
-IPCC Bericht -> Klimaveränderung, höhere Anstrengungen notwendig
-sog. Energiewende noch am Anfang -> Anteil EE an Energiebedarf
- absehbar wesentlich höherer Anteil EE an E-Bedarf (insbes. Strom)
->-> steigender Anteil fluktuierender Erzeuger
->-> Speicher notwendig
->-> EE-Kraftstoffe!
Elektrolyse -> Sektorenkopplung
-Projekt in Heide
bereits aktuell hoher EE-Anteil
-> -> techno-ökonomische Abschätzung der Wasserstoff-Erzeugungskosten notwendig
über Abschreibungs/ Lebensdauer d. Anlagen
Einbezug von Degradationseffekten

\subsection{Netzdienliches verhalten}
----->>> siehe auch SDL
-->> Definition
-- in: Bewertungskriterien

\subsection{Zielsetzung}
\subsubsection{Modell}
Als Resultat der vorliegenden Arbeit soll ein dynamisches Modell der aktuell gängigen Elektrolyse-Technologien hervorgehen, welches auch nach Abschluss der vorliegenden Arbeit weiter optimiert und adaptiert werden kann.\\
Der Aufbau erfolgt objekt-orientiert (ja???); bzw. im Baukasten-Prinzip, wodurch bei Bedarf eine möglichst effiziente Anpassung und/oder Optimierung einzelner Funktionsgruppen erfolgen kann.
\subsubsection{Simulation}
Die Betriebs-Simulation erfolgt mit möglichst realgetreuen Eingangsgrößen. Zum aktuellen Zeitpunkt werden folgende Größen als sinnvol erachtet(???Formulierung):
\begin{itemize}
	\item Strompreis-Signal
	\item Residual-Last
	\item Eis-Man
\end{itemize}  
Allen Eingangsgrößen liegt zu Grunde, dass Sie ein indirektes Abbild der Netzsituation liefern: 

\subsubsection{transientes/dynamisches Verhalten}
\begin{itemize}
	\item Aufbau der Doppelschicht: Spannungs-/Strom-Verzögerung (in wenigen Sekunden -->>QUELLE???)
	
\end{itemize}

%\subsubsection{Gaskompress.}
%ggf. Redlich-Kwong Eq



\section{Theoretische Grundlagen}
\subsection{Grundlagen Elektrische Energienetze}
\subsubsection{Funktion und Herausforderungen}
--> historisch: Verteilung von oben nach unten
--> aktuell: veränderung durch EE im MS/NS-Netz

\subsubsection{Systemdienlichkeit}

\subsubsection{Systemdienstleistungen}
-->>Frequenzbereiche Schulz Abb 87.58 (
-->>Vgl. Erzeugungsanlagen: Preis, CO2, Volllaststd. (Schulz tab 87.10)

\subsubsection{Speicher-Einrichtungen}
Energiespeichereinrichtungen werden in Zukunft eine zunehmend wichtige Rolle im Netzbetrieb einnehmen. \cite{Schmiegel2019}
Da in zukünftigen Stromnetzen der Anteil Erneuerbare Energien deutlich höher als aktuell ausfallen wird, sind entsprechend veränderte Eigenschaften des Energiesystems absehbar. Volatilen Eigenschaften von Photovoltaik- und Windenergieanlagen stehen im Gegensatz zu möglichst frei regelbarer bzw. sogenannter einlastbarer Leistung.\cite{Matthes2018}\\
Durch die Anforderungen an einen ständigen Last- und Erzeugungsausgleich können Power-to-Gas Anlagen einerseits als regelbare Lasten betrieben werden und gleichzeitig die notwendige Grundlage alternativer Brenngase für flexible Gaskraftwerke liefer. [????????]

- X kW notwendige install Leistung im Jahr 20?? [???]
- PtG langfristig wichtige Technologie für relativ große Energiemengen günstiger \cite{Schroeer2015}
ZITAT: ''PtG
ist dabei ein zentrales Element in der kostenoptimalen Konfiguration für eine 100% EE-
Stromversorgung''
- Wasserstoff für Verkehr langfristig günstiger \cite{Robinius2018}

\subsection{Grundlagen Wasser-Elektrolyse}
\label{subs_Grundl_EL}
Innerhalb dieser Arbeit werden 3 Technologie-Typen der Wasser-Elektrolyse unterschieden bzw. betrachtet:
\begin{itemize}
	\item Alkalische Elektrolyse (AEL)
	\item[]ebenfalls abgekürzt als \glqq AEC \grqq $\rightarrow$ alkaline electrolyzer cell
	\item Polymer-Elektrolyt-Elektrolyse ??? (PEM-EL)
	\item[]ebenfalls abgekürzt als \glqq PEMEC \grqq $\rightarrow$ polymere electrolyte membrane electrolyzer cell \textit{oder} protone exchange membrane electrolyzer cell
	\item Hochtemperatur Elektrolyse (SOEL)
	\item[]ebenfalls abgekürzt als \glqq SOEC \grqq $\rightarrow$ solide oxide electrolyzer cell
	
\end{itemize}





%\subsection{PEM - EL}
%MYRTE???
%\subsection{Grundsätzliches}
- hohe Temperaturen -> niedrigere Spannung an Zelle anzulegen
aber: hohe Temp. -> stärkere Degradation
- deshalb lieber hohe Drücke aber: -> Sicherheitsaspekte! H2-permeation -> Konz in O2
[Schalenbach 2013]


\subsection{Teilreaktionen}
-->> Tabelle nach DLR (QUELLE???)
\begin{figure}
	\label{EL-Teilreakt}
	\includegraphics[width=\textwidth]{Abb/sc_Schmidt2017_EL-Rech.png}
	\caption{Schematische Darstellung der 3 betrachteten Elektrolysetypen; aus Schmidt2017 QUELLE???}
\end{figure}

\subsubsection{PEM}
Anoden-Reaktion:
\begin{equation}\label{Teilreakt_Anode_PEM}
H_2O \Rightarrow \frac{1}{2} O_2 + 2H^+ +2e^-
\end{equation}
Kathoden-Reaktion:
\begin{equation}\label{Teilreakt-Kathode_PEM}
2H^+ 2e^- \Rightarrow H_2
\end{equation}

\subsubsection{SOEL}
Anoden-Reaktion:
\begin{equation}\label{Teilreakt_Anode_SOEL}
H_2O \Rightarrow \frac{1}{2} O_2 + 2H^+ +2e^-
\end{equation}
Kathoden-Reaktion:
\begin{equation}\label{Teilreakt-Kathode_SOEL}
2H^+ 2e^- \Rightarrow H_2
\end{equation}

\subsubsection{weitere (hier nicht betrachtete) EL-Typen}
\begin{itemize}
	\item AEM (?) % % Anion exchange
	\item membraneless EL (Esposito 2017, Joule)
	\item co-electrolysis
	\item high temp. PEM
\end{itemize}



\subsubsection{Aufbau und Funktion}


-> Temperaturkontrolle:
>> Aufbau Lettenmeier (Diss):
-Temp.-Sensor im Anodenseitigen Wasseraustritts-Strom
%-$60°C$ max (Modell bedingt)
-Aufheizen nur über Verluste (Überspannung)

---Ionentauscher...?

\subsubsection{Stand der Technik}
\label{SdT}
PEMEL , PEMWE, PEMEC, ...

\subsubsection{AEL}
zu AEL: Hammoudi + HAug !!!!!

\subsubsection{Überspannung}

\subsubsection*{Diffusionsüberspannung}
Hamann: Grundlagen der Kinetik
-> 
-> 
-> S. 246
-->> siehe Bemerkung S. 248!!


\subsubsection{Performance - Kriterien}
\label{peformance-krit}
https://www.sciencedirect.com/science/article/pii/S0360319917336868

https://ieeexplore.ieee.org/abstract/document/8494523
 
siehe Abschnitt(\ref{Bewertungs-Krit})!!!
\begin{itemize}
	\item Reaktionszeit
	\item Produktgasqualität
	\item Effizienz / spez. Energiebedarf
\end{itemize}


\subsubsection{Marktsituation}


\subsection{Degradation}
Wie jedes real betriebene System unterliegen auch Elektrolyseure unterschiedlichen Verschleißmechanismen, welche Gegenstand der aktuellen Forschung sind.\\
Die Einwirkung auf sämtliche Komponenten der drei betrachteten EL(am Anfang Abkürzen!!!???)-Typen wird von unterschiedlichen Vorgängen beeinflusst.\\
Unterschiedliche Auswirkungen: Materialabbau, Widerstand-Zunahme, Überspannungs-Zunahme...

Beeinflussung von Performance-Kriterien von Interesse. % Verweis auf
Abschnitt(\ref{peformance-krit})

\subsubsection{Elektroden}


\subsubsection{Membran-Dregadation}
% % paper Lettenmeier2016

PEM:
% % bei Chandesris nur teilw. Temperatrurabhängigkeit!


\cite{wu2008review}
-> WU - PEM fuel cell degradation: Tab2: degr. rates in  lit.

\cite{Rakousky2016}
!!! -> verringerung von $i_0$ auf (!) 37 \% des Ursprungswertes
%-> Rakousky: vA durch MEA-assembly beeinflusst (schwierig modellierbar
% NAfion 117 Katalysatoren: Anode: IrO2, TiO2 (Ir-Loading of $2,25 mg/cm^2$)
%Kathode: Pt/C (Pt-Loading of 0.8 g/cm2)
%cath.- / an.-PTLs (?)
%A=17.64

%Test über 1009 h
%high purity feed water $18.2 M Ohm/cm (?), 25 ml/min, 75 C$
%cell-temp.: const. @ 80 grad C

%niedrige Stromdichte $(1 A/cm^2 // 1,70 V)$: quasi keine Degr.
%ab $1.84 V // 2 A/cm^2$ --Spannung UND Stromdichte ausschlaggebend
%Verweis -> Ayers_2015 -->> hohe Zellspannung vermutlich nicht einziger Treiber f. Degr.

%bei größerem i:
%-zunehmend Gasblasenbildung -> mechaniche beanspruchung
%- höhere Sauerstoff-Konz in Anoden-Kompartement (?) -> Korrosion/ Passivierung
% größerer Wasserbedarf -> beschleunigte Ansammlung potentieller "contaminants" in CCM
% potentielle Bildung von Hotspots bei mangelhaftem Wasser-Management

%Stromdichte UND Schalt-Zeiten beeinflussen Degr
%- Intervall-Länge beeinflusst maßgeblich Degr. (größere Intervalle -> geringere Degr)

%2 unterschiedl. Degradations-Werte:
%- durchschnittl. Degr.: ges. Spannungs-Steigerung über ges. Test-Zeit
%- molar volt. degr. rate: durchschn. degr.-rate bzgl. 1 mol H2 prod.

%degradation ist unabängig von prod. H2-menge

%findings:
%- konstanter Betrieb:
%-> 2 A/cm2 -->>hohe Degr.
%-> 1 A/cm2 -->> keine Degr.
%--->>> lange Intervalle mit großen Stromdichten vermeiden!

%- dynamischer Betrieb:
%-degr.-raten (q 2 A/cm2) verringert, wenn zwischenzeitlich geringere Stromstärken vorliegen
%-> besonders starke verringerung der Degr. bei vollst. Strom-Unterbrechung (effekt noch nicht vollständig verstanden)

- Strom-Unterbrechungen sind vorsichtig einzusetzen:
-> größere Perioden sinnvoller
-> häufige Abschaltungen vermeiden

-gemessen Halbzellen-Pot. ??? -> 3.2?!
- Anoden-Degr. höher als KAth.

- reversible degr. phenomenom stems from cathode side (für versch. zellen unterschiedlich...also unklar, wo)

EIS:
%Zellen mit großem Spannungs-Anstieg (hohe Stromdichte) weisen ebenfalls größere Zunahme von R_ohm auf
%-Widerstandsverringerung durch CCM-topology changes
%--- was ist R_PTL?

-> hohe Stromdichte: Zunahme von R-ohm
-> niedrige Stromdichte: R-PTL = const. -->> R-ohm nimmt ab
-> im Fall hoher Stromdichten sorgen Unterbrechungen für geringere Zunahme von R-ohm (kann aktuell noch nicht erklärt werden)
-------
2 Paramter für Degr.-Effekt in Polarisationskurven-modell:
-Austausch-Stromdichte (Abnahme: Gesamt-Verlust d. Elektroden-Performance) //keine An./Kath.-Unterscheidung
- R-total -> ohmscher Gesamtwiderstand d. zelle

in ersten 400h etwa 50\% Verringerung d. Austausch-Stromdichte (unabhängig von betriebsform) -> effektive betriebszeit intermittierend kürzer!

R-total korreliert mit Spannungszunahme -> wichtiger Degr.-Parameter
-> reduktion d Zellspannung auf 1.7 V (über 6 h) -> gleicher effekt, wie 10-minütige Abschaltung 

- degradationswerte -> Berechnung??

j-0 verringerung: vermutlich durch Ti-Ionen in Anode



\begin{itemize}
\item mechanisch

\begin{itemize}
\item Perforation
\item Risse
\item Poren
\item Ursachen\\
Durch Herstellungsprozess oder fehlerhafte Montage\\
Interfaces: ungleichmäßige mechan. Beanspruchung\\
low-humdification, low-humidification, relative humidity (?)\\
\end{itemize}

\end{itemize}

-------------------Chandesris (2015):\\
Regarding PEMWE, one of the most complete studies that
gives evidence of membrane degradation was conducted at
PSI in the 1990's [25] where substantial thinning of the mem-
branes has been detected. Regarding the dissolution process,
the ion exchange capacity measurements on thinned mem-
branes reveal that the composition of the remaining polymer
is not changed with respect to ionic groups: the degradation
mechanism does not involve preferential attack on the ion
exchange groups. Furthermore, complementary experiments
indicated that the membrane degrading reaction can be
localized on the cathode side of the cell. The interfaces with the anode as well as the bulk of the membrane were quite less
affected [17,25]. In the present study, as reported in section
!!!
Single cell experimental set-up, we have observed a non
negligible fluoride release only on the cathode side, which also
support the hypothesis that membrane degrading reactions
occur mainly at the cathode side for PEMWE.
!!!

( While performance can be tested quite rapidly,
lifetime estimation is much more difficult to evaluate.
Furthermore, in PEM water electrolyzer, degradation mecha-
nisms occur very slowly with a typical characteristic time of
thousands of hours, compared to hundreds of hours in PEMFC.)

the PEMFC is not the exact opposite of
an electrolysis cell. In PEMWEs, rutile oxydes like IrO 2 and
RuO 2 are used as anode catalysts and the membrane is far
thicker.
The main sources of performance losses are
related to catalysts and catalyst layers degradation, mem-
brane degradation and bipolar plates and current collectors
corrosion, however it is quite difficult to isolate the different
degradation phenomena. Nevertheless, in PEMWEs, perfor-
mance decreases and durability restrictions are mostly
attributed to membrane pollution or degradation, reason why
we will focus on this mechanism.

Nevertheless, these data should not be taken into
account as the explosivity limit has been reached. Indeed, an
important criterion for the electrolyzer is the molar percent-
age of H 2 in O 2 , as this mixture becomes explosive above 4% of
H 2 in O 2 Ref. [39]. Fig. 15 plots the evolution of this quantity at
333 and 353 K. An exponential increase is observed due to the
coupling between the thinning of the membrane leading to
the gas cross-over increase and the chemical degradation of
the membrane. This trend is completely similar to the one
observed experimentally by Inaba et al. [40] when performing
accelerated stress test in PEMFC.

Diffusion:
First, as oxygen (resp. hydrogen) is present under gas form
only at the anode (resp. cathode), a concentration gradient
appears, once the gases are dissolved in the ionomer of the
membrane. This concentration gradient induces an oxygen
flux from the anode to the cathode and a hydrogen flux from
the cathode to the anode via a diffusive process.

Furthermore,
water is crossing the membrane from the anode to the cath-
ode due to both electro-osmotic phenomenon and water
diffusion. Since part of the produced gas gets soluble in the
water, it can be assumed that it is convectively carried
through the membrane by this water flow.

---params:\\
The values for the solubility and diffusion coefficients are
chosen among a large base of experimentally given laws
[20,26,27].
--->>>> Table 4 ( medium permeation behaviour)
-----\\

tion. The cathode side is at very low potential (< 0 V vs. SHE),
as hydrogen evolution reaction occurs at the platinum elec-
troactive sites. At potential lower than 0.4 V, ORR is considered
to predominantly occurs via H 2 O 2 formation pathway [32] and
one can neglect water recombination: (?????????)

\subsection{Elektrische Energieversorgung und Energiemarkt}

\section{Rahmenbedingungen}
angewandter Betriebsbereich:
-typische Stromdichten
-Reaktionszeit-Zeit (Signal) -> Rampen
- maximale Leistungsgradienten


%\subsection{Standort -> in betriebsarten aufführen} 


%\subsection{Eingangssignale}
\section{Betriebsarten}
\label{sect_Betriebsarten}

\subsection{Einspeisemanagement-Maßnahmen (EIS)}
\subsubsection{Datengrundlage}
\subsubsection{Datenaufbereitung}
\subsubsection{Anlagenskalierung}
Abbildung: Rohdaten und JDL+Popt
\begin{figure}[H]
	
	\centering
	\includegraphics[width=0.49\textwidth]{/home/dafu/Schreibtisch/Master-Projekt/Doku/Abb/test.png}
	\includegraphics[width=0.49\textwidth]{/home/dafu/Schreibtisch/Master-Projekt/Doku/Abb/test.png}
	\caption{ }
	\label{fig:JDL_Eis} 
\end{figure}

\subsection{Negative Residuallast (RES)}
\subsubsection{Datengrundlage}
\subsubsection{Datenaufbereitung}
\subsubsection{Anlagenskalierung}
Abbildung: Rohdaten und JDL+Popt
\begin{figure}[H]
	
	\centering
	\includegraphics[width=0.49\textwidth]{/home/dafu/Schreibtisch/Master-Projekt/Doku/Abb/test.png}
	\includegraphics[width=0.49\textwidth]{/home/dafu/Schreibtisch/Master-Projekt/Doku/Abb/test.png}
	\caption{ }
	\label{fig:JDL_RES} 
\end{figure}


\subsection{Reale Wind-Einspeisung (WEA)}
\subsubsection{Datengrundlage}
\subsubsection{Datenaufbereitung}
\subsubsection{Anlagenskalierung}
Abbildung: Rohdaten und JDL+Popt
\begin{figure}[H]
	
	\centering
	\includegraphics[width=0.49\textwidth]{/home/dafu/Schreibtisch/Master-Projekt/Doku/Abb/test.png}
	\includegraphics[width=0.49\textwidth]{/home/dafu/Schreibtisch/Master-Projekt/Doku/Abb/test.png}
	\caption{ }
	\label{fig:JDL_WEA} 
\end{figure}

\subsection{Börsenstrompreis (COST)}
\subsubsection{Datengrundlage}
\subsubsection{Datenaufbereitung}
\subsubsection{Anlagenskalierung}
Abbildung: Rohdaten und JDL+Popt
\begin{figure}[H]
	
	\centering
	\includegraphics[width=0.49\textwidth]{/home/dafu/Schreibtisch/Master-Projekt/Doku/Abb/test.png}
	\includegraphics[width=0.49\textwidth]{/home/dafu/Schreibtisch/Master-Projekt/Doku/Abb/test.png}
	\caption{ }
	\label{fig:JDL_COST} 
\end{figure}
%\subsubsection{Residuallast}
%an welchem punkt? Knoten-Daten verfügbar? sonst skalierung?


%\subsubsection{Datenlage}











\section{Bewertungskriterien}
\label{Bewertungs-Krit}

Vergleichskriterien der unterschiedlichen Technologien.

Produktgas reinheit
-> \url{https://www.riessner.de/documents/gase/wasserstoff.pdf}


\subsection{Produktgasqualität}
Tjarks S. 22: Durch die hohen
Reinheitsanforderungen an den produzierten Wasserstoff wird als Regenerationsgas
ebenfalls Wasserstoff eingesetzt, um eine Fremdgas-kontamination zu vermeiden.

>>>>was ist mit O2 in H2 ??????? >>>>>>>>>Permeation für AEL und SOEL nicht ermittelt\\


\subsection{Wasserstoffproduktionsrate}
-proportional zur Stromdichte -> erhöhung der Stromdichte -->> Erhöhung der Ausbeute, ohne Vergrößerung des Stack |Lettenmeier, Diss
aber: Degradation

\subsection{Systemdienstleistung}
Systemdienstleistungen sind für Anschlussnehmer des elektrischen Verbundnetzes (??? stimmt das?) nach -->>Verordnung XY ???<<< definiert.
Im hier betrachteten Zusammenhang wird die Entlastung von Teilbereichen der verwendeten Netzebene im Fall hoher Einspeiseleistung durch Erneuerbare Energien als Systemdienstleistung verstanden.

\subsection{Utilisationrate Enutz/Epot}

\subsection{Effizienz}
%\section{Charakterisierung der Szenarien}



\section{Modellbildung}

Das Modell des Elektrolyse-Vergleichsystems wird mit Hilfe der Programmiersprache (???) \textsc{python} und auf Grundlage aktueller Veröffentlichungen zur Modellbildung der Elektrolyse erstellt (vgl. Abschnitt(\ref{subs_Grundl_EL})).\\
In vorliegender Literatur bestehen verschieden komplexe Ansätze für nahezu sämtliche technologisch relevanten System-Komponenten der betrachteten Elektrolyse Systeme. Die Zahl verfügbarer Veröffentlichungen repräsentiert den derzeitigen Forschungsstand bzw. -schwerpunkt. Entsprechend besteht für Hochtemperatursysteme eine geringere Dichte und Vielfalt an Veröffentlichungen.\\
Die zu Grunde liegenden elektrochemischen Prinzipien werden für sämtliche Systeme strukturell in gleicher Art und Weise modelliert. Simulation und Auswertung erfolgen falls möglich gegenüberstellend mit entsprechender Parametrisierung (Abschnitt(\ref{subs_Par_je_TEch})) sowie Skalierung (Abschnitt(\ref{subs_Skal_TEch})) je Technologie .\\
Ziel ist es, innerhalb der Modellierung die entsprechende Technologie ausreichend detailliert abzubilden, um wesentliche Einflüsse auf die Performance in über signifikante Zeitbereiche %Langzeit-Betrieb (??? angestrebte Betriebsdauer)
herauszustellen. Der Vergleich der 3 betrachteten Elektrolyse-Typen soll als Entscheidungsgrundlage für die tatsächlich umgesetzte Installation einer Demonstrationsanlage dienen(???).\\
Vereinfachungen [???] an geeigneter Stelle vorgenommen und entsprechend (Abschnitt(\ref{subsubs_Mod_Vereinfach})) dokumentiert.
\begin{figure}[H]
	
	\centering
	%\includegraphics[width=0.49\textwidth]{/home/dafu/Schreibtisch/Master-Projekt/Doku/Abb/test.png}
	\includegraphics[width=\textwidth]{/home/dafu/Schreibtisch/Master-Projekt/Doku/Abb/Modell/modell_ges_v34.png}
	\caption{Schematische Inhalt und Systemgrenzen der Modellierung}
	\label{fig:Mod_Zielsetz} 
\end{figure}
Abbildung(\ref{fig:Mod_Zielsetz}) stellt die wichtigsten Elemente der Modellierung dar:
Auf Grundlage verschiedener dynamischer Eingangsdaten bezüglich der elektrischen Versorgung der Anlage werden die drei ausgewählten EL-Typen ???ABK inklusive wichtiger Nebenaggregate abgebildet. Festgelegte Systemgrenzen schließen Anlagenperipherie, wie Gleichrichter/ Leistungselektronik, Förderung des Edukt- und Kühlwasserstromes sowie eine Gastrocknungsanlage ein. Des Weiteren sind für das AEL???-Modell Elektrolyt-Aufbereitung und für SOEL???-Modell ein Zuheizer??? vorgesehen. Über die Systemgrenzen tritt der Edukt-und Kühlwasserstrom sowie elektrische Energie in das System ein. Produktgase (bislang lediglich Wasserstoff???) verlassen das System über die Grenze. Vorausgesetzt bzw. nicht innerhalb des Modells berücksichtigt ist die Bereitstellung ausreichend reinen Wassers sowie die Nachverdichtung des Produktgases auf entsprechende Druckniveaus.
Im linken Drittel von Abbildung(\ref{fig:Mod_Zielsetz}) sind außerdem verwendete Eingangssignale dargestellt. Diese entsprechen den unter (\ref{sect_Betriebsarten}) erläuterten Betriebsarten. 
Im Folgenden wird häufig eine Unterscheidung zwischen Hoch- und Niedertemperatur-\gls{el} vorgenommen. Ersteres entspricht der \gls{soel}- während letztere Bezeichnung auf \gls{ael}- und \gls{pem}-\gls{el} bezogen ist.???WOHIN???
\subsection{Vorüberlegungen / Modellausrichtung}
ABZUSTIMMEN

- Zeit-Abhängigkeit durch deltaT des Eingangsignals
-->> Dynamik eher makroskopisch?
elektrochemische Dynamik nicht berücksichtigt

-> dynamisches Verhalten über Rampen angenähert
-->> Leistungsgradient als Begrenzung?

-> Druckbetrieb: Tjarks S. 93 ....bis zu 29 \% steigerung d. spez. Arbeit 

- Anpassung der Regelungs-Komponenten (?) -->> Wassermanagement // Temperatur
- optimierter Betriebsbereich (p, T)

- notwendige Ausgangsgrößen für technoökonomische Betrachtung/ Bewertung:
-> Wasserstoff-Volumen / Volumenstrom
(-> Sauerstoff)
-> Wasserbedarf
-> Leistung/ Energie-Input

Betriebskosten
-Fix:

--> Energiekosten




\subsection{Peripherie-Anlagenschemata}
\label{subs_peri_schem}

\begin{figure}[H]
	
	\centering
	\includegraphics[height ={4cm}]{/home/dafu/Schreibtisch/Master-Projekt/Doku/Abb/Modell/st_ael.png}
	\includegraphics[height ={4cm}]{/home/dafu/Schreibtisch/Master-Projekt/Doku/Abb/Modell/st_pem.png}
	\includegraphics[height ={4cm}]{/home/dafu/Schreibtisch/Master-Projekt/Doku/Abb/Modell/st_soel.png}
	\caption{Schematische Darstellung verwendeter Stack-Modelle; links: \gls{ael}; mitte: \gls{pem}; rechts: \gls{soel} }
	\label{fig:peri_schem_Modell} 
\end{figure}
Für die Modellierung der drei betrachteten \gls{el}-Technologien werden vereinfachte Stack-Topologien??? zu Grunde gelegt. Annahmen zu Komponenten und Aufbau sind aktueller Fachliteratur (vgl. Abschnitt(-> Grundlagen)) entnommen. In Abbildung(\ref{fig:peri_schem_Modell}) sind die in dieser Arbeit verwendeten Schemata abgebildet.\\
\subsubsection{NT-Elektrolyse}
\label{subsubs_peri_schem_NT}
Der Wasserkreislauf beider \gls{nt}-\gls{el}-Anlagen Kombiniert die Edukt-Wasser-Zufuhr??? sowie den Kühlwassermassenstrom. Der notwendige Edukt-Wasserstrom tritt mit Kühlwassertemperatur (siehe Tabelle(??? PARAMS)) in den Wasserkreislauf ein $T_0$ und wird zum Zirkulierenden Kühlwasserstrom beigemischt, wodurch sich eine Mischtemperatur $T_1$ einstellt. Eine Wärmeabgabe der Pumpe an den Wassermassenstrom ist theoretisch zu berücksichtigen, wird jedoch in der vorliegenden Arbeit vernachlässigt. Aus diesem Grund entspricht die Stack-Eintritts-Temperatur $T_2$ der Mischtemperatur $T_1$. Der aus dem Stack austretende Massenstrom stellt eine Mischung aus Wasser und dem Produktgasstrom dar und weist Stack-Temperatur $T_3$ auf. Für die Rückführung des Kühlwassers ist ein Wasserabscheider erforderlich, welcher jedoch im Modell nicht weiter berücksichtigt wird. Der zurückgeführte Wasserstrom wird mittels Wärmeübertrager abgekühlt ($T_4$) und dem Eduktstrom beigemischt. Der Produktgasstrom verlässt das System an Stelle (6) und wird anschließend der Gastrocknung zugeführt. Die \gls{ael}-\gls{el} weist zudem ein Elektrolyt-Aufbereitungssystem auf.

\subsubsection{HT-Elektrolyse}  
\label{subsubs_peri_schem_HT}
Die Wasser-Zirkulation ($0$),($1$),($2$),($3$),($4$) folgt der für \gls{nt}-\gls{el} beschriebenen Systematik. Dieser unterscheidet sich jedoch insofern, dass sämtliche Stoffströme in gasförmigem Zustand erfolgen/vorliegen???. Innerhalb dieser Arbeit wird die Annahme getroffen, dass am Anlagenstandort ein Dampf-Strom mit typischer Abdampftemperatur??? verfügbar ist. Da der Dampfstrom von einer externen Wärmequelle stammt, wird die Notwendigkeit einer weiteren Förder-Einheit??? ausgeschlossen. Das Aufheizen auf Soll-Temperatur des Stack erfolgt mittels eines elektrischen Heiz-??? ($1-2$). Innerhalb der Modellierung entsprechender Stoffströme ist zu beachten, dass ein molarer Mindestanteil innerhalb des Stack-Zustroms zu gewährleisten ist [???]. Entsprechend ist die Implementierung einer Rezirkulation mittels Stoffstrom-Aufteilung ($3-4$ und $3-5$), unter Berücksichtigung sich verändernder Stoffstromanteile von Dampf und Wasserstoff, vorzunehmen. Der rezirkulierte Stoffstrom muss ausreichend groß gewählt werden, um den Mindestanteil von Wasserstoff im Stack-Zustrom zu erreichen. Eine Wärmerückgewinnung aus dem Produktgasstrom ist im vorliegenden Modell nicht berücksichtigt.

\subsubsection{Gastrocknung}
\label{subsubs_peri_schem_Gastrockn}
Um einen Produkgasstrom ausreichender Reinheit bereitstellen zu können, ist der Betrieb einer Gastrocknungsanlage erforderlich. Anlagenschema und Funktionsprinzip sind \cite{Tjarks2017} entnommen und in Abbildung(\ref {fig:schema_TB}) dargestellt.

\begin{figure}[H]
	
	\centering

	\includegraphics[height ={4cm}]{/home/dafu/Schreibtisch/Master-Projekt/Doku/Abb/Modell/TB_01.png}
	\caption{Schema der verwendeten Temperatur-Wechsel-Absorptionsanlage nach \cite{Tjarks2017} ???Zahlen???ABSTIMMEN}
	\label{fig:schema_TB} 
\end{figure}
Grundsätzlich erfolgt die Gastrocknung in zwei Stufen. Nach einer reinen Kondensation ($K$) gegen Umgebungstemperatur erfolgt eine nahezu vollständige Abscheidung des Restwassers mittels Adsorption aus dem Produktgasstrom entfernt. 
Eine Temperatur-Wechsel-Adsorption wird mittels??? zwei (oder mehr) abwechselnd durchströmten Trockenbetten durchgeführt. Sofern beispielsweise das erste Trockenbett ($TB1$) zur Wasseraufnahme betrieben wird, erfolgt gleichzeitig die Regeneration des zweiten ($TB2$) Trockenbettes. Die Regeneration (Desorption des aufgenommenen Wassers) erfolgt mittels Rückleitung und Aufheizen eines Teils des trockenen Gasstromes. Sobald das regulär durchströmte Trockenbett ($TB1$) eine gesetzte Kapazitätsgrenze erreicht hat, werden entsprechende Stoffströme so umgeleitet, dass eine entgegengesetzte Durchströmung eintritt (). Dadurch wird eine Regeneration des ersten Trockenbettes und Wasseraufnahme im zweiten Trockenbettes bewirkt. 
In Anlehnung an das Vorgehen in oben genannter Quelle wird vereinfachend ein kontinuierlicher Betrieb angenommen.\\
Dieses Modell wird für sämtliche Technologien gleichermaßen angewandt.

\subsection{Grundlagen}
\subsubsection{Numerische Simulation}
Möglichst detailgetreue Beschreibung bzw. Abbildung einer realen Komponente oder eines Systems durch Zahlenwerte, welche durch Grundlagengleichungen das System-Verhalten beschreiben. [??? Quelle]


\subsubsection{Programmiersprache: Python}
Die Modellierung sowie sämtliche zusätzlichen Berechnungen werden innerhalb einer \textsc{python}-IDE vorgenommen.\\
Die ''open source high-level'' Programmiersprache \textsc{python} wird durch die \textsc{Python Software Foundation\footnote{\url{https://www.python.org/psf/}}} bereitgestellt und beinhaltet eine große Zahl verschiedener Funktionen und Erweiterungen für eine Vielzahl programmiertechnischer Anwendungen.\\
Die gewählte Entwicklungsumgebung ermöglicht eine objektorientierte (OOP) Programmierung auf Grundlage von Klassen und Methoden (Funktionen).
Da seitens der Autoren keine einschlägigen Kenntnisse bezüglich OOP vorliegen, weißt die erstellte Programmstruktur lediglich teilweise entsprechende Eigenschaften auf: Zwar sind verschiedene Funktionen auf verschiedenen Skript-Ebenen (an-)geordnet und miteinander verknüpft, jedoch wird keine Klassen-Struktur verwendet. 


%\subsection{Modell}

\subsubsection{Vereinfachungen /Modellrestriktionen}
\label{subsubs_Mod_Vereinfach}
Dynamisches Verhalten entsteht innerhalb der Modellierung makroskopisch durch das zeitabhängige Eingangssignal sowie hinsichtlich der elektrochemischen Eigenschaften durch dynamische Temperaturänderung. Das entsprechend verwendete Temperaturmodell bedient sich der vereinfachten Annahme einer zusammengefassten Wärmekapazität der Stacks sowie einer vereinfachten??? Betrachtung entsprechender Wassermassenströme (vgl. Abschnitt(\ref{subs_peri_modelle})). Zum aktuellen Zeitpunkt (Stand???) des Modells werden weitere dynamische Größen bzw. Vorgänge, wie:
\begin{itemize}
	\item Druckdynamik (makroskopisch)[vcgl???]
	\item Doppelschicht-Vorgänge (Elektrostatik)
	\item Diffusions-Vorgänge (???)
\end{itemize}
nicht berücksichtigt. Bezüglich der Druckdynamik sind bereits entsprechende Variablen vorgesehen bzw. angelegt, wodurch eine nachträgliche Implementierung sehr einfach möglich ist. Sofern bezüglich beider letztgenannten elektrochemischen Vorgänge?? hinreichend genaue Modelle erstellt werden können, ist auch deren Implementierung innerhalb des Elektrochemie-Teils (Abschnitt(\ref{subsubs_mod_elchem})) verhältnismäßig unproblematisch.\\
Trägheiten hinsichtlich der Reaktion auf Eingangsleistungsänderungen werden mittels Rampen auf Grundlage von Literaturwerten \cite{Buttler2018} angenähert. 
Des Weiteren werden Eigenschaften von Elektrode und Membranen bzw. Elektrolyt vereinfachend homogen angenommen. Eine räumlich aufgelöste Betrachtung sämtlicher Einflussgrößen und Eigenschaften ist nicht vorgesehen.\\ 
Ebenfalls Einflüsse von Konzentrationsänderungen (WO???) auf entsprechende Zellspannungen werden auf Grund unsicherer Berechnungsgrundlagen und lediglich geringer Größenordnung bzw. geringem Einfluss vernachlässigt.\\
Eine Wärmerückgewinnung aus Produktgasströmen der \gls{ht}-\gls{el} ist nicht implementiert.???\\

%- Membran / Elektroden eindimensional abgebildet
- keine Massenbilanzen im Elektrolyt
%- keine Druck-Dynamik
- Zeitvesetzte berechnug der Gasaufbereitung

%-> wasserkreisläufe: keine Mischung bei NT
%-> kein produktgas-Wärmerückgewinn

%-keine Konzentrationsüberspannungen
%- keine diffusionsdynamik
%- keine doppelschicht-dynamik


-Polarisationskurve auf Grundlage d nernst-Gleichung
->Aktivität der beteiligten Reaktionspartner erforderlich
->
Modellierung nach [??? Schalenbach]



\subsection{Skript-Struktur und funktionaler Aufbau}
\label{subs_Skript-Strukt}
Aus der in Abbildung(\ref{fig:Mod_Zielsetz}) verdeutlichten Zielsetzung ist ein Berechnungs-Tool hervorgegangen, welches insgesamt $12$ Skripte beinhaltet.
Die Berechnung wird von einem zentralen Skript aus gesteuert, welches sämtliche weiteren Skripte aufruft bzw. verwaltet. Das Rückrad des Programms bilden das erwähnte Steuerungs-Skript sowie zwei weitere Haupt-Skripte zum Koordinieren unterschiedlicher Schleifen Ebenen. 

\subsubsection{Funktionale Struktur}
\label{subs_Funkt-Strukt}
Das Modell basiert auf zwei verschachtelten Schleifen. Während die äußere Schleife (\textsc{main\_loop}) hauptsächlich dazu dient, Variablen, Parameter sowie Eingangsdaten aufzubereiten bzw. zu verwalten, findet innerhalb der inneren Schleife (\textsc{sub\_loop}) die Koordination nahezu sämtlicher notwendiger Berechnungen statt. Lediglich Berechnungen und Funktionen, welche je Zeitschritt des Datensatzes zu konstanten Werten führen, werden innerhalb der äußeren Schleife ausgeführt. Die Anzahl der äußeren Schleifen-Durchläufe entspricht der Länge des Eingangs-Datensatzes. Dagegen kann die Anzahl der inneren Schleifendurchläufe je Leistungs- bzw. Zeitwert festgelegt werden.\\

Die Funktionale Struktur bzw. Interaktion der Haupt-Skripte ist in Abbildung(\ref{fig:Strukt_Skript}) dargestellt. 

\subsubsection{Skript-Struktur}
\label{subsubs_skrpt-strukt}
Die Struktur erstellter Skripte- bzw. Ordner sowie wichtige Dateiinhalte sind wie folgt aufgebaut:

\begin{itemize}
	\item master\_ctrl\_ (\textit{startet bzw. steuert Berechnung})
	\item log-Datei (\textit{listet Berechnungszeit})
	\item log-Datei f. ökonomische Berechnung (\textit{direkte Ausgabe ökonomischer Kennwerte})
	\item Haupt-Funktions-Ordner (mainf)
	\begin{itemize} 
		\item \_main\_loop\_
		\item \_sub\_loop\_
		\item \_plot\_
		\item Funktions-Ordner (Deff)
		\begin{itemize}
			\item \_def\_glob\_(\textit{globale Funktionen})
			\item \_def\_AEL\_ (\textit{AEL-spezifische Funktionen})
			\item \_def\_PEM\_ (\textit{PEM-spezifische Funktionen})
			\item \_def\_SOEL\_ (\textit{SOEL-spezifische Funktionen})
		\end{itemize}
		\item Parameter-Ordner (Par)
		\begin{itemize}
			\item \_par\_glob\_(\textit{globale Parameter})
			\item \_par\_AEL\_ (\textit{AEL-spezifische Parameter})
			\item \_par\_PEM\_ (\textit{PEM-spezifische Parameter})
			\item \_par\_SOEL\_ (\textit{SOEL-spezifische Parameter})
		\end{itemize}
	\end{itemize}
	\item Input-Ordner (In)
	\item[] Leistungs-Datensatz von
	\begin{itemize}
		\item WEA
		\item RES
		\item EIS
		\item COST
	\end{itemize}
	\item Output-Ordner (Out)
	\begin{itemize}
		\item sämtl. Ausgangs-Datensätze
	\end{itemize}
	
\end{itemize}

In Abbildung(\ref{fig:Strukt_Skript}) ist die zu Grunde liegende Interaktion der Modell-Skripte schematisch dargestellt.\\
Die Steuerung der Berechnung erfolgt innerhalb des \textsc{ctrl}-Skripts (dreifach umrandet) sowie das gezielte Setzen bzw. Verändern von Werten innerhalb der Parameter-Skripte par\_glob und par\_tech (unterbrochene Umrandung), wobei ''tech'' für die entsprechend berechnete Technologie (\gls{ael},\gls{pem},\gls{soel}) steht. Sämtliche weiteren Skripte bleiben während der Anwendung unverändert.??? \\
Als Daten-Input wird eine ''.csv''-Datei erwartet, welche  eine Datums-Spalte sowie eine weitere Spalte mit entsprechenden Leistungswerten enthält. Auf dieser Grundlage erfolgen sämtliche Berechnungen. Bei erfolgreichem Programmdurchlauf wird eine ''.csv''-Datei ausgegeben, welche die Größen relevante Größen für sämtliche berechneten Zeitschritte enthält.

\begin{table}
	\caption{Liste der Ausgabewerte???}
	\begin{tabular}{llc}
		Größe&Zeichen&Einheit\\
		\hline
		\hline
		berechnete Zeit&$t_{abs}$ 	&$s$\\
		Stack-Temperatur&$T$ 		& $^\circ C$\\
		(Kühl-)Wassermassenstrom&$m_c$		&$kg/s$\\
		Zellspannung&$u_cell$	&$V$\\
		Stromdichte&$i_cell$	&$A/cm^2$\\
		Eingangsleistung&$P_{in}$	&$W$\\
		Anlagen-Leistungsaufnahme&$P_{act}$	&$W$\\
		Stack-Leistungsaufnahme&$P_{st}$	&$W$\\
		Peripherie-Leistungsaufnahme&$P_{aux}$	&$W$\\
		Kathodendruck&$p_K$		&$bar$\\
		Anodendruck&$p_A$		&$bar$\\
		Produktgas-Stoffstrom, Wasserstoff&$n_{H_2}$	&$mol/s$\\
		Produktgas-Stoffstrom, Sauerstoff & $n_{O_2}$		&$mol/s$\\
		Reations-Wasser-Verbrauch&$n_{H_{2}O}$&$mol/s$\\
		Stoffmengenanteil $H_2$ in $O_2$&$\theta_{H_2inO_2}$&$mol/mol$\\
		
	\end{tabular}
	\label{tab: Ausgabewerte}
\end{table}

  
 
\begin{figure}[H]
	
	\centering
	%\includegraphics[width=0.49\textwidth]{/home/dafu/Schreibtisch/Master-Projekt/Doku/Abb/test.png}
	\includegraphics[width=0.6\textwidth]{/home/dafu/Schreibtisch/Master-Projekt/Doku/Abb/Skript/skript_strukt_v32.png}
	\caption[Skript-Interaktions-Schema]{Vereinfachtes Interaktions-Schema erstellter Skripte sowie deren Interaktion; große Pfeile: Aufrufe. Zugriffe sowie Variablen- und Parameterübergabe; kleine Pfeile: Parameter- und Berechnungsrückgabe}
	\label{fig:Strukt_Skript} 
\end{figure}


\begin{figure}[H]
	
	\centering
	\includegraphics[width=0.49\textwidth]{/home/dafu/Schreibtisch/Master-Projekt/Doku/Abb/test.png}
	\includegraphics[width=0.49\textwidth]{/home/dafu/Schreibtisch/Master-Projekt/Doku/Abb/test.png}
	\caption{ }
	\label{fig:Strukt_Funkt_oa} 
\end{figure}
%\begin{itemize}
%	\item Elektrochemie der jeweiligen Technologie
%	\begin{itemize}
%		\item AEL
%		\begin{itemize}
%			\item Gibbs
%			\item Austauschstromdichte
%			\item Partialdrücke
%			\item Überspannungen
%			\item Polarisationskurve
%		\end{itemize}
%	\end{itemize}
%	\begin{itemize}
%		\item PEM
%		\begin{itemize}
%			\item 
%		\end{itemize}
%	\end{itemize}
%	\begin{itemize}
%		\item SOEL
%		\begin{itemize}
%			\item 
%		\end{itemize}
%	\end{itemize}
%\end{itemize}
\subsection{Haupt-Bestandteile der Berechnung}
Im Folgenden werden die wichtigsten Modell-Komponenten aufgeführt und erläutert.\\
 Teilmodelle und Funktionen, welche für sämtliche \gls{el}-Typen gültig sind, werden mittels $\blacktriangleright$ gekennzeichnet, während ausschließlich technologiespezifisch anzuwendende Teilmodelle mit $\triangleright$ markiert werden.\\
 
\subsubsection{Überblick???}
Ziel des vorgestellten Modell ist es, das Verfahren der Wasser-Elektrolyse mit hinreichendem Detaillierungsgrad abzubilden, um Aussagekräftigkeit über das systemrelevante??? Verhalten über einen aussagekräftigen Zeitbereich zu erlangen.\\ Wichtige Kenngrößen sind Leistungsaufnahme und Produkgasmenge sowie ableitbare Effizienz-Kennwerte.   

Die Betriebstemperatur des Stacks besitzt maßgeblichen Einfluss auf viele wichtige Größen der \gls{el}-Berechnung. 
In Abbildung(\ref{fig:Strukt_Funkt_innerloop}) wird die Struktur der dynamischen Berechnung wichtiger Kenngrößen für \gls{nt}- und \gls{ht}-\gls{el} dargestellt.\\

\begin{figure}[H]
	
	\centering
	\includegraphics[width=0.49\textwidth]{/home/dafu/Schreibtisch/Master-Projekt/Doku/Abb/Modell/inner_NT_12.png}
	\includegraphics[width=0.49\textwidth]{Abb/Modell/inner_HT_12.png}
	\caption{Struktur des dynamischen Kern-Modells; links: \gls{nt}-, rechts: \gls{ht}-\gls{el} }
	\label{fig:Strukt_Funkt_innerloop} 
\end{figure}
Die Stacktemperatur hängt unter anderem von Stackspannung und -strom ab.
Auf Grundlage der Stacktemperaturänderung $T(U,I)$ sowie korrelierenden Stoffströmen ($n_{i}$,$m_c$) ist die Leistungsaufnahme der Peripherie ($P_{aux} = P_{gt}+P_{p}(+P_{ec}(+P_{h}))$) bestimmbar. Unter Berücksichtigung der Leistungselektronik sowie zulässiger Leistungsgradienten ($P_{grad}$) und Betriebsbereichen wird anhand der verfügbaren Eingangsleistung $P_{in}$die verfügbare Leistung $P_{diff}$  des betrachteten Zeitschrittes bestimmt. Anhand technologiespezifischer, elektrochemischer Eigenschaften, welche in die Polarisationskurve (bzw. Strom-Spannungs-Verhältnis der Zelle) einfließen, ist anschließend die optimale Stromdichte bestimmbar. Auf deren Grundlage ist anschließend die Ermittlung der Produkgasströme $n_{H_2}, ~n_{O_2}$, der tatsächlichen Stackleistung $P_{st}$ sowie weitere Eingangswerte des nachfolgenden Berechnungsschrittes bildet.  
???...eine ausreichend hohe Zahl an Berechnungsdurchläufen je Eingangswert erforderlich.
In den folgenden Abschnitten werden weitere Funktionen sowie Details zu Temperaturmodell, Peripherie und grundlegender elektrochemischer Berechnungen erläutert.  

% % Temperatur-Modelle nach Tjarks, Lettenmeier, Ruuskanen, Gabrielli

\begin{figure}[H]
	
	\centering
	\includegraphics[width=0.49\textwidth]{/home/dafu/Schreibtisch/Master-Projekt/Doku/Abb/Graph/test_JDL_WEA_scale-marker_Markerpoints_01.pdf}
	
	\includegraphics[width=0.49\textwidth]{/home/dafu/Schreibtisch/Master-Projekt/Doku/Abb/2019-03-02--12-06_barplot_EISMAN_RESIDUAL_WEA_E_spec.pdf}
	\caption{TEST }
	\label{fig:Test} 
\end{figure} 

 \subsubsection{Temperaturmodell}
 \label{subsubs_mod_TEmp}
Das Temperaturmodell wird auf vereinfachender Grundlage einer zusammengefassten Wärmekapazität sämtlicher Stack-Komponenten aufgebaut. Dieses Modell ist in verschiedenen Ansätzen zu finden (\cite{Espinosa-Lopez2018,Gabrielli2016}), wobei bezüglich des zu Grunde liegenden Ansatzes \cite{Ulleberg2003} zitiert wird.\\
Elektrochemie bzw. Polarisationskurve im Detail, siehe folgender Abschnitt(\ref{subsubs_mod_elchem})

$\blacktriangleright$ Den zentralen Baustein des Temperaturmodells bildet die Bestimmung der Temperaturänderung des Stacks mittels build-in DGL-solver. 
\begin{wrapfigure}{l}{1cm}
	\includegraphics[width=1.0cm]{/home/dafu/Schreibtisch/Master-Projekt/Doku/Abb/Modell/it_dTdt.png}
	%\caption{Bildunterschrift}
\end{wrapfigure}
 Die Differentialgleichung der Wärmebilanz berücksichtigt Wärmegewinne durch Überspannungen und  Wärmeabgabe an die Umgebung sowie an den Kühlwasserstrom. Für den ersten Berechnungsschritt müssen erwartete Eingangswerte festgelegt werden, während im Laufe der weiteren Berechnung (Anzahl der Berechnungsschritte $n~>~0$) jeweils auf entsprechende Werte des vorherigen Berechnungsschrittes ($n-1$) zurückgegriffen werden kann.

Wichtige Parameter sind Wärmekapazität sowie Wärmeübergangskoeffizient (HAX???) in Abhängigkeit der Stack- bzw. Anlagengröße. 
\newline


%[Gabrielli???]
%- wichtig: PID und HAx
%-> lumped heat cap. // vereinfachung für WÜ -> T\_st  const
%
%-da\\
%-kommt noch text hin\\
%\newline

$\blacktriangleright$ Zur Regelung der Stacktemperatur wird ein PID-Regler implementiert, wobei einzelne Regelglieder mittels finiter Differenzen angenähert werden.[Quelle ???]

\begin{wrapfigure}{l}{1cm}
	%\vspace{-0.8cm}
	\includegraphics[width=1.0cm]{/home/dafu/Schreibtisch/Master-Projekt/Doku/Abb/Modell/it_PID.png}
	%\caption{Bildunterschrift}
\end{wrapfigure}
Dieser berechnet im Falle der Niedertemperatur-Elektrolyse den Kühlwassermassenstrom bzw. für die \gls{ht}-\gls{el} die elektrische Zuheizleistung des Zirkulationsstromes. Im Fall der \gls{ht}-\gls{el} wird je nach Eingangs-Datensatz eine konstante oder variable Soll-Temperatur verwendet: Sofern Eingangs-Zeitreihen lediglich kurze Zeiträume positiver Leistungswerte aufweisen, ist es erforderlich, eine konstante Soll-Temperatur auf Niveau der Arbeitstemperatur zu wählen. Bei ausreichend großen Intervallen positiver Leistungswerte kann die Soll-Temperatur dynamisch gewählt werden. Für Zeitbereiche ohne Leistungs-Input ist dadurch die notwendige Zuheiz-Leistung des Rezirkulationsstromes (vgl. Abschnitt(\ref{subsubs_peri_schem_HT})).
%Da die Temperaturänderung des Stacks von der Höhe des Kühlwassermassenstromes abhängt, dieser jedoch auf Grundlage der Soll-Ist-Differenz innerhalb der PID-Funktion ermittelt wird, 
\newline


\subsubsection{Steuerungstechnische Implementierungen}
\label{subsubs_mod_Anlagen-strg}

$\triangleright$ 
\newline

$\blacktriangleright$ 
%Für die Sicherstellung einer validen Betriebssimulation werden Leistungswerte  von $P~<~0$???????, Leistungsreduktion bei Übertemperatur und Berücksichtigung des 
Die Abbildung eines realistischen Betriebsverhaltens erfolgt mittels Berücksichtigung von maximalen Leistungsgradienten und Mindestleistungswerten.
\begin{wrapfigure}{l}{1cm}
	%\vspace{-0.8cm}
	\includegraphics[width=1.0cm]{/home/dafu/Schreibtisch/Master-Projekt/Doku/Abb/Modell/it_powgrad.png}
	%\caption{Bildunterschrift}
\end{wrapfigure} Ersteres ermöglicht die Implementierung von Einschalt- bzw. Lastwechselverhalten mittels Leistungsrampen. Die Implementierung??? von Mindestleistungswerten bewirkt das Abschalten der Anlage bei Unterschreiten des entsprechenden Wertes. Sämtliche dafür notwendigen Parameter sind Fachliteratur \cite{Buttler2018} entnommen und in Tabelle(XY ANHANG) dargestellt. 
\newline

$\blacktriangleright$ Die Berechnung des optimalen Anlagenstromes erfolgt an Hand einer build-in Optimierungsfunktion.
\begin{wrapfigure}{l}{1cm}
	%\vspace{-0.8cm}
	\includegraphics[width=1.0cm]{/home/dafu/Schreibtisch/Master-Projekt/Doku/Abb/Modell/it_Popt.png}
	%\caption{Bildunterschrift}
\end{wrapfigure} Auf Grundlage der Polarisationskurve wird die optimale Stromdichte des Stacks in Abhängigkeit verfügbarer Leistung und unter Berücksichtigung des Peripherie-Leistungsbezugs ermittelt. Die zu Grunde liegende Optimierungsfunktion wird hier mittels \textsc{Methode kleinster Quadrate} durchgeführt.




%\subsubsection{Inner Loop (HT)}
%- Temp. Modell nach???
%- zusätzliche Stoffstrom-Mischung Betrachtet



\subsubsection{Elektrochemie}
\label{subsubs_mod_elchem}
Das jeweilige Haupt-Funktionsskript der betrachteten Technologien enthält verschiedene Funktionen für die Berechnung der Spannungs- und Strom-Abhängigkeit. %Polarisationskurven-Berechn.:

\begin{figure}[H]
	
	\centering
	\includegraphics[width=0.49\textwidth]{/home/dafu/Schreibtisch/Master-Projekt/Doku/Abb/test.png}
	\includegraphics[width=0.49\textwidth]{/home/dafu/Schreibtisch/Master-Projekt/Doku/Abb/test.png}
	\caption{ }
	\label{fig:Strukt_Funkt_elchem} 
\end{figure}


%\newline
$\blacktriangleright$ Gibbs
\newline

$\blacktriangleright$ Austauschstromdichte
\newline

$\blacktriangleright$ (Partialdrücke)
\newline

$\blacktriangleright$ Überpotentiale
\newline

$\blacktriangleright$ Zell-Spannung (Polarisation)

\subsubsection{Stoffströme}

$\blacktriangleright$ Stoffströme

%\subsubsection{Gibbs}
%Enthalpie des Wassers:
%\begin{equation}\label{Enthalpie}
%\Delta H = \Delta G + T\Delta S
%\end{equation}
%Die reversible Zellspannung je Halbreaktion ist definiert nach
%\begin{equation}\label{E_rev}
%E_{rev}=\frac{\Delta G}{n \cdot F}
%\end{equation}
%Mit der Anzahl beteiliger Elektronen $n$ und der Faraday-Konstanten $F=96485,3~C/mol$
%[Schalenbach] setzt folgende Werte ein: $\Delta S=-159,6~J/kg~mol$ und $\Delta H =2,847 \cdot 10^{5}~J/mol$
%-> GibbsFreeEnergy berechnung
%
%\subsubsection{reversible Zellspannung; Nernst-Gleichung}
%Nernst-Spannung für Teilreaktion der Anode mit $E_0 = 0V$ gegen NHE:
%\begin{equation}
%E^{An} = E_0 + \frac{RT}{nF} ln \bigg( \frac{a(H^+)^2 \cdot \sqrt[]{a(O_2)}}{a(H_2O)} \bigg)
%\end{equation}
%
%Nernst-Spannung für Teilreaktion der Anode mit $E_0 = E_{rev}$ gegen NHE:
%\begin{equation}
%E^{Kat} = E^{Kat}_0 + \frac{RT}{nF} ln \bigg( \frac{a(H^+)^2}{a(H_2)} \bigg)
%\end{equation}
%
%\subsubsection{Überspannungen}
%\subsubsection*{Ohmsche}
%\subsubsection*{Diffusive}
%\subsubsection*{Konzentrations-}
%-> HAmann S. 188ff. (! S.191) \cite{Hamann2005}
%
%
%\subsubsection{Degradation}
%
%
%\subsection{Stoffströme}
%
%\subsubsection{AEL - Stoffströme}
%??? Skizze !
%\subsubsection{PEM - Stoffströme}
%Diffusions- und Permeations-Vorgänge vereinfachend (-> linear prop. zu Stromdichte) abgebildet nach [Tjarks, Diss]\\
%eigentlich: 3D-DGL 
%
%\subsubsection*{Schalenbach}
%S.14925!
%S. 14926 -> oben:In this model, the partial pressures of hydrogen in the
%anodic catalyst layer and oxygen in the cathodic catalyst layer
%are assumed to be independent on the gas crossover.
%Furthermore, the influence of pressure on the solubility and
%permeability is neglected, which is below 1% for the assumed
%conditions [27].
%
%bzgl. Partialdruck: (!)
%gas outlets. Since the production rate densities of the gases are
%proportional to the current density (eq (13)), a partial pressure
%enhancement of hydrogen in the cathodic catalyst layer with
%the same dependence in the current density is assumed. The
%
%While gas crossover dominates the Faraday efficiency, the
%voltage efficiency is affected by activation losses, mass
%transport losses and ohmic losses. Furthermore, as shown
%before, pressurized electrolysis enhances the required ther-
%modynamical voltage according to the Nernst equation (eq
%(10)).
%
%membrane influences faraday AND voltage efficiency!!
%
%
%
%\subsubsection{SOEL - Stoffströme}


%\subsection{???Temperatur-Regelung}
%Saisonale Simulation...Außentemperaturanpassung?
%
%Temperatur-Modelle nach Tjarks, Lettenmeier, Ruuskanen
%
%Temp.-Vränderung des internen Wasserkreislaufs durch Pumpenverluste (siehe Abschnitt (\ref{overall-water-managem}))
%
%
%\subsection{???Wasser-Management}
%\label{overall-water-managem}
%Wasserbedarf aus Stoffströmen
%+ Kühlwasser-Zustrom
%
%-> Pumpenwirkungsgrad (Espinoza-Lopez2018)


\subsubsection{Peripherie}
\label{subsubs_mod_Periph}
Zusätzlich zur notwendigen Stackleistung werden außerdem auftretende Leistungsbezüge von Nebenaggregaten ermitttelt. Innerhalb der vorliegenden Arbeit werden Fördereinheiten, Zuheizer sowie eine Gastrockung zur Produktgasaufbereitung berücksichtigt. 

$\triangleright$ Leistungsbezug der Nebenaggregate (Wasserpumpe und Elektrolytpumpe(\gls{ael})) werden mit vereinfacht angenommen, konstanten Druckverlusten, proportional zu Massenströmen berechnet.
\newline


$\blacktriangleright$ Der Leistungsbezug der Produktgasaufbereitung ist abhängig von Massenstrom des Produktgases und Höhe der Reinheitsanforderungen und basiert auf der Arbeit von [Tjarks???]. Erstere besitzen eine direkte Abhängigkeit vom Anlagenstrom, wobei Permeationsverluste (???) berücksichtigt werden (müssen???). Letztere werden innerhalb der vorliegenden Arbeit konstant betrachte(Verweis??? ggf. Sensitivität?).

%\subsubsection{Leistungselektronik}

%\subsubsection{Produktgasbehandlung}

%\subsubsection{Elektrolyt-Aufbereitung (AEL)}

%\subsubsection{Gastrocknung}

%\subsubsection*{Peripherie}

$\triangleright$ Innerhalb der \gls{nt}-\gls{el} ist das vereinfachte Modell einer Pumpe zur Förderung des Edukt- bzw. Kühlwasserstromes implementiert.
\begin{wrapfigure}{l}{1cm}
	%\vspace{-0.8cm}
	\includegraphics[width=1.0cm]{/home/dafu/Schreibtisch/Master-Projekt/Doku/Abb/Modell/it_pump.png}
	%\caption{Bildunterschrift}
\end{wrapfigure} Die Berechnung erfolgt auf Grundlage von Gleichung(???), mit einem vereinfachend konstant angenommenen Druckverlust sowie einem konservativ gewählten Gesamtwirkungsgrad.
\newline
$\triangleright$ Die Peripherie der \gls{ael} beinhaltet zusätzlich eine gesonderte Elektrolyt-Aufbereitung, um einen optimalen $KOH$-Anteil??? sicherzustellen. 
\begin{wrapfigure}{l}{1cm}
	%\vspace{-0.8cm}
	\includegraphics[width=1.0cm]{/home/dafu/Schreibtisch/Master-Projekt/Doku/Abb/Modell/it_KOH.png}
	%\caption{Bildunterschrift}
\end{wrapfigure}
Das Teilmodell ist nach [???] erstellt und beinhaltet zum aktuellen Zeitpunkt lediglich eine weitere Pumpe zur Bewegung des Elektrolyten. Die zusätzliche Leistungsaufnahme wird im Modell in ($P_ec$) ??? (vgl. auch Abbildung  )
\newline

$\triangleright$ Innerhalb der \gls{ht}-\gls{el} beinhaltet die Peripherie eine elektrische Heizeinrichtung, um den eintretenden Dampfstrom auf Stack-Soll-Temperatur zu erhitzen.
\begin{wrapfigure}{l}{1cm}
	%\vspace{-0.8cm}
	\includegraphics[width=1.0cm]{/home/dafu/Schreibtisch/Master-Projekt/Doku/Abb/Modell/it_Ph.png}
	%\caption{Bildunterschrift}
\end{wrapfigure}

%\subsubsection{Kompression}
%-isentrope Kompression nach |Thermodynamics of pressurized gas storage
%-->> Angaben für isentropen exponent kappa und compressibility factor Z // aktuell gemittelt
%ggf. zu interpolieren
%
%\subsection{???Effizienz}
%Ein Teilziele dieser Arbeit besteht in der Bestimmung eines optimalen Arbeitspunktes für jede auftretende Eingangsleistung.\\
%In eine entsprechende Optimierungsfunktion fließen die Folgenden Größen ein:
%\begin{itemize}
%	\item verfügbare Leistung
%	\item ggf. Zeitraum der verfügbaren Leistung
%	\item produzierte H2-Menge -> Effizienz
%	\item ggf. Arbeitspreis der bezogenen Energie 
%\end{itemize}
%Es wird angestrebt, 
------------------------------------------\\
$\blacktriangleright$ Weiterhin wird der Wirkungsgrad der Leistungselektronik mittels Curve-Fit-Funktion anhand von Literaturwerten für entsprechende Betriebsbereiche berechnet und auf den Stack-Leistungsbezug $P_{st}$ angewandt.

\subsection{???Reaktionszeit/ Dynamik}
Ein- und Ausschaltvorgänge elektrochemischer Systeme werden maßgeblich durch Doppelschicht- und Diffusionsvorgänge geprägt.

Doppelschichtvorgänge können deshalb durch Totzeiten (???) angenähert werden, während Diffusionsvorgänge konkret modelliert werden müssen.

\subsubsection{Doppelschichtvorgänge}
in Form von Rampen / Totzeiten bzgl. d. Wasserstoff-Produktion berücksichtigt

\subsubsection{Diffusionsvorgänge}
% % Elektrochemie, Hamann // S. 188 ff.! insbes.: 196...?
% dazu auch Abdin2015
% % -> Han2015 !! eq.:12 / 13
Diffusionsvorgänge unterscheiden sich maßgeblich durch den Zustand des Elektrolytes: bewegt (erzwungene Konvektion) oder ruhend. Während erstere innerhalb sehr kleiner Zeitbereiche Ablaufen ($10^{-1} \dots 1~s$), weisen letztere Zeiten (???stimmt so nicht!!!->nachlesen?????) im Bereich von $30 \dots 60~s$ auf. Die in dieser Arbeit angestrebte zeitliche Auflösung beträgt maximal $>1$ Minute, wodurch Diffusionsvorgänge nicht stationär betrachtet werden können.\\

Diffusionsgranzstromdichte (für zeitlich konstante Diffusionsschicht-Dicke)
\begin{equation}
i_0 = n \cdot F \cdot D \cdot \frac{c^0}{\delta_N}
\end{equation}

Mit Anzahl ausgetauschter Elektronen $n$, Faraday-Konstante $F$, Diffusionskoeffizient $D$, Ausgangskonzentration Elektrolyt $c_0$ sowie der Dicke der Nernstschen Diffusionsschicht $\delta_N$.





\section{Simulation}

Festlegen des inneren Berechnungszeitraumes auf 10 Sekunden.???

Im folgenden Abschnitt werden Validierung der elektrochemischen Modelle, Auswahl der Anlagengrößen für unterschiedliche Betriebsbereiche, entsprechende Parametrisierung sowie Simulationsablauf und Sensitivitätsanalyse beschrieben.

\subsection{Parametrisierung Technologie-spezifisch}
\label{subs_Par_je_TEch}
Keine keine zeitlichen und infrastrukturellen Ressourcen für Experimente vorhanden.
In Ermangelung von Versuchsaufbauten zur Bestimmung von bzw. dem fehlenden Zugriff auf Laborwerten ist lediglich eine Parametrisierung mittels Literaturangaben möglich. Durch eine entsprechende Notwendigkeit, Werte verschiedener Quellen und damit unterschiedlicher Skalierung bzw. Mess-Bedingungen zu verwenden, wird die Aussagekraft vorliegender Ergebnisse eingeschränkt. Einzelne Teilmodelle können häufig nur qualitativ überprüft bzw. verifiziert werden.\\
Dennoch ist dadurch die Grundsätzliche Funktion des Modells nachweisbar. % % verschieben!!

% % Espinoza-Lopez beschreibt PSO (S.167 unten links); dennoch sind experimente notwendig....
% % Goeßling -> Tab 1

\begin{table}[]
	\label{tab-param}
	\caption{Tabelle verwendeter Parameter}
		\begin{tabular*}{\textwidth}{lllllccc}
			
		\multirow{ 2}{*}{ \textbf{Kategorie}} & \multirow{ 2}{*}{\textbf{Größe}} &\multirow{ 2}{*}{\textbf{Symbol}}&\multirow{ 2}{*}{\textbf{Einheit}}& \multicolumn{3}{c}{\textbf{Wert}}&\multirow{ 2}{*}{\textbf{Quelle}}\\
		&&&& AEL & PEM & SOEL&\\
		\hline \hline
		&&&&&&&\\
		&Membrandicke &$\delta_m$& &&&&\cite{Espinosa-Lopez2018}\\
		&Austauschstromdichte&$i_0$&& &&&\cite{Goesling2018}\\
		&&&&&&&\\
		Temp.&Wärmekapazität, Stack&$C_th$&$J/K$&&$162116$&&\cite{Espinosa-Lopez2018}\\
		&Wärmeübergangswiderstand&$k/W$&$0.0668$&$$&$$&$$&\cite{Espinosa-Lopez2018}\\
		&Pumpen-Wirkungsgrad&$\eta_{Pu}$&$$&$$&$$&$$&\\
		&&$$&$$&$$&$$&$$&\\
		&&$$&$$&$$&$$&$$&\\
		&&$$&$$&$$&$$&$$&\\
	\end{tabular*}
\end{table}

\begin{table}[]
	\label{tab-param2}
	\caption{Tabelle verwendeter Parameter}
	\begin{tabular*}{\textwidth}{llllccccc}
		% \multirow{ 2}{*}{\textbf{Quelle}}
		%\multirow{ 2}{*}{ \textbf{Kategorie}} 
		 \multirow{ 2}{*}{\textbf{Größe}} &\multirow{ 2}{*}{\textbf{Symbol}}&\multirow{ 2}{*}{\textbf{Einheit}}& \multicolumn{3}{c}{\textbf{Wert}}&&&\\
		&&& AEL & Quelle&PEM &Quelle& SOEL&Quelle\\
		\hline \hline
		&&&&&&&&\\
		Membrandicke &$\delta_m$& &&\cite{Espinosa-Lopez2018}&&&&\\
		Austauschstromdichte&$i_0$&&&\cite{Goesling2018}&&&&\\
		&&&&&&&&\\
		Wärmekapazität, Stack&$C_th$&$J/K$&&$162116$&&\cite{Espinosa-Lopez2018}&&\\
		Wärmeübergangswiderstand&$k/W$&$0.0668$&$$&$$&$$&\cite{Espinosa-Lopez2018}&&\\
		Pumpen-Wirkungsgrad&$\eta_{Pu}$&$$&$$&$$&$$&&&\\
		
		&$$&$$&$$&$$&$$&&&\\
		&$$&$$&$$&$$&$$&&&\\
		&$$&$$&$$&$$&$$&&&\\
	\end{tabular*}
\end{table}

\subsection{Anlagenskalierung}
\label{subs_Skal_TEch}

\subsubsection{Anlagen Parameter}

\begin{table}[]
	\begin{tabular}{cllll}
		&&AEL&PEM&SOEL\\
		Betriebsart &Größe& &&\\
		\hline \hline
		\multirow{ 4}{*}{ \textbf{EIS}}&&&&\\
		&hallo&&&\\
		&hallo&&&\\
		\multirow{ 4}{*}{ \textbf{RES}}&&&&\\
		\multirow{ 4}{*}{ \textbf{WEA}}&&&&\\
		\multirow{ 12}{*}{ \textbf{COST}}&&&&\\
		&\multirow{ 4}{*}{ \textbf{s}}&&&\\
		&&&&\\
		&\multirow{ 4}{*}{ \textbf{m}}&&&\\
		&\multirow{ 4}{*}{ \textbf{m}}&&&\\
	\end{tabular}
\end{table}



\section{Simulation}
Festlegen des inneren Berechnungszeitraumes auf 10 Sekunden.???
\subsection{Parametrisierung Technologie-spezifisch}
\label{subs_Par_je_TEch}
Keine keine zeitlichen und infrastrukturellen Ressourcen für Experimente vorhanden.
In Ermangelung von Versuchsaufbauten zur Bestimmung von bzw. dem fehlenden Zugriff auf Laborwerten ist lediglich eine Parametrisierung mittels Literaturangaben möglich. Durch eine entsprechende Notwendigkeit, Werte verschiedener Quellen und damit unterschiedlicher Skalierung bzw. Mess-Bedingungen zu verwenden, wird die Aussagekraft vorliegender Ergebnisse eingeschränkt. Einzelne Teilmodelle können häufig nur qualitativ überprüft bzw. verifiziert werden.\\
Dennoch ist dadurch die Grundsätzliche Funktion des Modells nachweisbar. % % verschieben!!

% % Espinoza-Lopez beschreibt PSO (S.167 unten links); dennoch sind experimente notwendig....
% % Goeßling -> Tab 1

\begin{table}[]
	\label{tab-param}
	\caption{Tabelle verwendeter Parameter}
		\begin{tabular*}{\textwidth}{lllllccc}
			
		\multirow{ 2}{*}{ \textbf{Kategorie}} & \multirow{ 2}{*}{\textbf{Größe}} &\multirow{ 2}{*}{\textbf{Symbol}}&\multirow{ 2}{*}{\textbf{Einheit}}& \multicolumn{3}{c}{\textbf{Wert}}&\multirow{ 2}{*}{\textbf{Quelle}}\\
		&&&& AEL & PEM & SOEL&\\
		\hline \hline
		&&&&&&&\\
		&Membrandicke &$\delta_m$& &&&&\cite{Espinosa-Lopez2018}\\
		&Austauschstromdichte&$i_0$&& &&&\cite{Goesling2018}\\
		&&&&&&&\\
		Temp.&Wärmekapazität, Stack&$C_th$&$J/K$&&$162116$&&\cite{Espinosa-Lopez2018}\\
		&Wärmeübergangswiderstand&$k/W$&$0.0668$&$$&$$&$$&\cite{Espinosa-Lopez2018}\\
		&Pumpen-Wirkungsgrad&$\eta_{Pu}$&$$&$$&$$&$$&\\
		&&$$&$$&$$&$$&$$&\\
		&&$$&$$&$$&$$&$$&\\
		&&$$&$$&$$&$$&$$&\\
	\end{tabular*}
\end{table}

\begin{table}[]
	\label{tab-param2}
	\caption{Tabelle verwendeter Parameter}
	\begin{tabular*}{\textwidth}{llllccccc}
		% \multirow{ 2}{*}{\textbf{Quelle}}
		%\multirow{ 2}{*}{ \textbf{Kategorie}} 
		 \multirow{ 2}{*}{\textbf{Größe}} &\multirow{ 2}{*}{\textbf{Symbol}}&\multirow{ 2}{*}{\textbf{Einheit}}& \multicolumn{3}{c}{\textbf{Wert}}&&&\\
		&&& AEL & Quelle&PEM &Quelle& SOEL&Quelle\\
		\hline \hline
		&&&&&&&&\\
		Membrandicke &$\delta_m$& &&\cite{Espinosa-Lopez2018}&&&&\\
		Austauschstromdichte&$i_0$&&&\cite{Goesling2018}&&&&\\
		&&&&&&&&\\
		Wärmekapazität, Stack&$C_th$&$J/K$&&$162116$&&\cite{Espinosa-Lopez2018}&&\\
		Wärmeübergangswiderstand&$k/W$&$0.0668$&$$&$$&$$&\cite{Espinosa-Lopez2018}&&\\
		Pumpen-Wirkungsgrad&$\eta_{Pu}$&$$&$$&$$&$$&&&\\
		
		&$$&$$&$$&$$&$$&&&\\
		&$$&$$&$$&$$&$$&&&\\
		&$$&$$&$$&$$&$$&&&\\
	\end{tabular*}
\end{table}

\subsection{Anlagenskalierung}
\label{subs_Skal_TEch}

\subsubsection{Anlagen Parameter}

\begin{table}[h]
	\begin{tabular}{cllll}
		&&AEL&PEM&SOEL\\
		Betriebsart &Größe& &&\\
		\hline \hline
		\multirow{ 4}{*}{ \textbf{EIS}}&&&&\\
		&hallo&&&\\
		&hallo&&&\\
		\multirow{ 4}{*}{ \textbf{RES}}&&&&\\
		\multirow{ 4}{*}{ \textbf{WEA}}&&&&\\
		\multirow{ 12}{*}{ \textbf{COST}}&&&&\\
		&\multirow{ 4}{*}{ \textbf{s}}&&&\\
		&&&&\\
		&\multirow{ 4}{*}{ \textbf{m}}&&&\\
		&\multirow{ 4}{*}{ \textbf{m}}&&&\\
	\end{tabular}
\end{table}

\subsection{Funktionstest und Abgleich der Einzelkomponenten}
\subsubsection{Elektrochemie}
\subsubsection*{AEL}
Abbildung: Polarisationskurve
\begin{figure}[!tbp]
	\centering
	\begin{minipage}[b]{0.49\textwidth}
		\includegraphics[width=\textwidth]{/home/dafu/Schreibtisch/Master-Projekt/Doku/Abb/Graph/polar/ael/AEL_Polar_Tvar_5.pdf}
		
		\caption{ Polarisationskurven Tvar}
		\label{fig:polk_ael_Tvar} 
	\end{minipage}
	\hfill
	\begin{minipage}[b]{0.49\textwidth}
		\includegraphics[width=\textwidth]{/home/dafu/Schreibtisch/Master-Projekt/Doku/Abb/Graph/polar/ael/AEL_Polar__AEL_polarsens_i0_at609.pdf}
		
		\caption{ Polarisationskurveni0var}
		\label{fig:polk_ael_i0var}  
	\end{minipage}
\end{figure}


\subsubsection*{PEM}
Abbildung: Polarisationskurve
\begin{figure}[!tbp]
	\centering
	\begin{minipage}[b]{0.49\textwidth}
		\includegraphics[width=\textwidth]{/home/dafu/Schreibtisch/Master-Projekt/Doku/Abb/Graph/polar/pem/PEM_Polar__polar_plaus_base13.pdf}
		
		\caption{ }
		\label{fig:polk_PEM_Tvar} 
	\end{minipage}
	\hfill
	\begin{minipage}[b]{0.49\textwidth}
		\includegraphics[width=\textwidth]{/home/dafu/Schreibtisch/Master-Projekt/Doku/Abb/Graph/polar/pem/PEM_Polar_i0var_polar_plaus_R_ele_calc16.pdf}
		\caption{ }
		\label{fig:polk_PEM_i0var} 
	\end{minipage}
\end{figure}

- gute Übereinstimmung mit z.B. Liso
- starke Abweichung von Espinoza Lopez!!!


\subsubsection*{SOEL}
Abbildung: Polarisationskurve

\begin{figure}[!tbp]
	\centering
	\begin{minipage}[b]{0.49\textwidth}
		\includegraphics[width=\textwidth]{/home/dafu/Schreibtisch/Master-Projekt/Doku/Abb/Graph/polar/soel/SOEL_Polar__SOEL_polarsensitivity_700-1000_tmem=0h_next18.pdf}
		
		\caption{ }
		\label{fig:polk_SOEL_Tvar} 
	\end{minipage}
	\hfill
	\begin{minipage}[b]{0.49\textwidth}
		\includegraphics[width=\textwidth]{/home/dafu/Schreibtisch/Master-Projekt/Doku/Abb/Graph/polar/soel/SOEL_Polar__SOEL_polarsensitivity_T=850_tmem=0h_4500h_9000h_20000h_next20.pdf}
		\caption{ }
		\label{fig:polk_SOEL_i0var} 
	\end{minipage}
\end{figure}



\subsubsection{Thermo-Verhalten}
?
\subsubsection{chrakteristisches Verhalten (Doublet-Test)}
\subsubsection*{AEL}
Abbildung: Doublet- oder charakteristischer Test
\begin{figure}[H]
	
	\centering
	%\includegraphics[width=0.49\textwidth]{/home/dafu/Schreibtisch/Master-Projekt/Doku/Abb/ .png}
	\includegraphics[width=0.8\textwidth]{/home/dafu/Schreibtisch/Master-Projekt/Doku/Abb/test.png}
	\caption{ }
	\label{fig:doublet_AEL} 
\end{figure}
\subsubsection*{PEM}
Abbildung: Doublet- oder charakteristischer Test
\begin{figure}[H]
	
	\centering
	%\includegraphics[width=0.49\textwidth]{/home/dafu/Schreibtisch/Master-Projekt/Doku/Abb/ .png}
	\includegraphics[width=0.8\textwidth]{/home/dafu/Schreibtisch/Master-Projekt/Doku/Abb/test.png}
	\caption{ }
	\label{fig:doublet_PEM} 
\end{figure}
\subsubsection*{SOEL}
Abbildung: Doublet- oder charakteristischer Test

\begin{figure}[h]
	
	\centering
	%\includegraphics[width=0.49\textwidth]{/home/dafu/Schreibtisch/Master-Projekt/Doku/Abb/ .png}
	\includegraphics[width=0.8\textwidth]{/home/dafu/Schreibtisch/Master-Projekt/Doku/Abb/test.png}
	\caption{ }
	\label{fig:doublet_SOEL} 
\end{figure}

\subsection{Anlagenskalierung}
\subsubsection{EIS}
\begin{figure}[h]
	
	\centering
	%\includegraphics[width=0.49\textwidth]{/home/dafu/Schreibtisch/Master-Projekt/Doku/Abb/ .png}
	\includegraphics[width=0.8\textwidth]{/home/dafu/Schreibtisch/Master-Projekt/Doku/Abb/Graph/JDL_EIS_scale-marker_new-range_scal___12.pdf}
	\caption{ }
	\label{fig:Skal_EIS} 
\end{figure}
\subsubsection{RES}
\begin{figure}[h]
	
	\centering
	%\includegraphics[width=0.49\textwidth]{/home/dafu/Schreibtisch/Master-Projekt/Doku/Abb/ .png}
	\includegraphics[width=0.8\textwidth]{/home/dafu/Schreibtisch/Master-Projekt/Doku/Abb/Graph/JDL_RES_scale-marker__02.pdf}
	\caption{ }
	\label{fig:Skal_RES} 
\end{figure}
\subsubsection{WEA}
\begin{figure}[h]
	
	\centering
	%\includegraphics[width=0.49\textwidth]{/home/dafu/Schreibtisch/Master-Projekt/Doku/Abb/ .png}
	\includegraphics[width=0.8\textwidth]{/home/dafu/Schreibtisch/Master-Projekt/Doku/Abb/Graph/JDL_WEA_scale-marker__02.pdf}
	\caption{ }
	\label{fig:Skal_WEA} 
\end{figure}

\subsubsection{Testläufe / Abbildung großer Zeitreihen}
-> EIS
-> RES
-> WEA
-> COSt (1 Beisp.?)

\subsection{Sensitivität}
\begin{itemize}
	\item Anlagen verhalten
	\begin{itemize}
		\item ...
		\item elchem -> welche Größe? -> anhand polarc rausfinden!
		\item dPdt (?) -> ökon?
		\item PID (?) -> PI statt PID? 
		\item dU\_fact (?)
	\end{itemize}
	\item ökonom. Ergebn
	\begin{itemize}
		\item ...
		%\item elchem -> welche Größe? -> anhand polarc rausfinden!
		\item Inv.-Kosten
		\item Amort
		%\item PID (?) -> PI statt PID?
		%\item dU\_fact (?)
	\end{itemize}
\end{itemize}



\subsection{Grenzen des Modells}
\subsubsection{Technologie-Übergreifend}
\begin{itemize}
	\item keine Druck-Dynamik
	\item Permeation unzureichend abgebildet
	\item Degradation unzur. abgebildet
	\item eingeschränkte Vergleichbarkeit der Technologien
\end{itemize}

\subsubsection{PEM}
\begin{itemize}
	\item Elektrolytwiderstand
	\item Dynamik
	\item Thermo (unvollständige Bilanz)
	\item Peripherie
	\item Degradation:
	\begin{itemize}
		\item nur über curve-fit
		\item 
	\end{itemize}
\end{itemize}



\section{Analyse und Auswertung}
\subsection{Methodik}
\subsection{Analyse grundsätzlichen Verhaltens}
Dynamik... Verhalten bei P-Sprüngen....
aus großen Plots
\subsection{Kennwert-Analyse}
Um die betrachteten Technologien bewerten zu können, ist ein Abgleich verschiedener Kennwerte (vgl. Abschnitt(???->Grundlagen Kenndaten???)) notwendig. Im Folgenden werden entsprechende Kennwerte je Technologietyp und Eingangssignal??? vergleichend aufgetragen. In sämtlichen Diagrammen besteht eine Gruppierung nach Betriebsart über welchen die \gls{el}-Typen \gls{ael} (blau),\gls{pem} (grün) und \gls{soel} (rot) aufgetragen sind. Zum Abgleich von Systemwirkungsgrad und spezifischem Energiebedarf mit Literaturwerten sind entsprechende Referenzwerte durch graue Bereiche dargestellt.
\subsubsection{Systemwirkungsgrad}
Wie in Abschnitt() erläutert, gibt der Systemwirkungsgrad Auskunft über die Effizienz der Umsetzung elektrischer Energie während des Produktionsprozesses. Berechnete Werte für die Betriebsarten \gls{eis}, \gls{res} und \gls{wea} sind in Abbildung(\ref{fig:analy_nC_eta}) aufgetragen.
\begin{figure}[H]
	
	\centering
	%\includegraphics[width=0.49\textwidth]{/home/dafu/Schreibtisch/Master-Projekt/Doku/Abb/ .png}
	\includegraphics[width=0.8\textwidth]{/home/dafu/Schreibtisch/Master-Projekt/Doku/Abb/Graph/eco/2019-03-02--11-16_barplot_EISMAN_RESIDUAL_WEA_eta.pdf}
	\caption{ }
	\label{fig:analy_nC_eta} 
\end{figure}
Für die \gls{eis}-Betriebsart weisen die drei \gls{el}-Technologien Systemwirkungsgrade zwischen $55 $ und $69$ bzw. $70~\%$ auf. Während\gls{pem} und \gls{soel} ein ähnliches Niveau aufweisen, liegt die Effizienz der \gls{ael} deutlich darunter, befindet sich jedoch damit im Bereich aktueller Literaturangaben.\cite{Buttler2018}   
Dagegen übersteigt die \gls{pem}-\gls{el} Literaturwerte deutlich (um etwa mindestens $10\%$). Die \gls{soel} hingegen liegt mit $69~\%$ um etwa $6-8~\%$ unterhalb erwartbarer Werte.\\
Ein ähnliches Bild ergibt sich für die \gls{res}-Betriebsweise: \gls{ael} und \gls{soel} liegen etwa $1~\%$ oberhalb der für \gls{eis} beschriebenen Werte. Die \gls{pem}-\gls{el} übersteigt den entsprechenden \gls{eis}-Wert um weitere $5~\%$. Abweichungen zur Literatur liegen hier in gleichem Maße vor, wie zuvor beschrieben.\\
Auch für den Betriebsfall \gls{wea} bleiben diese Verhältnisse erhalten, wobei die Effizinez der \gls{pem}-\gls{el} nochmals um $1~ \%$ steigt.\\

%EIS/RES/WEA:
%
%- eta
%• Eis: AEL = 0,56; PEM=0,7; SOEL= 0,69
% aufsteigend von AEL zu PEM zu SOEL
%◦ PEM und SOEL ungefähr gleich
%◦ AEL im Lit Bereich; PEM deutlich über Lit Bereich; SOEL deutlich unter Lit Bereich
%• Res: AEL = 0,57; PEM=0,75; SOEL= 0,70
%◦ aufsteigend von AEL zu SOEL zu PEM
%◦ AEL im Lit; PEM weit über Lit; SOEL deutlich unter Lit
%• Wea: AEL = 0,57; PEM=0,77; SOEL= 0,69
%◦ aufsteigend von AEL zu SOEL zu PEM
%◦ AEL im Lit; PEM deutlich über Lit; SOEL deutlich unter Lit.
%• AEL schlechter, weil Pol-Kurve zu hoch
%◦ Gleichbleibende eta bei allen Input-Sig
%◦ pminfrac keine Auswirkung, weil eta\_sys bezogen auf Pact ist
%◦ keinen Bezug zum aktuellen Leistungsbezug
%• PEM zu gut, weil gute Pol-Kurve  niedrige Überspannungen
%◦ Aufsteigende eta von Eis zu Res zu Wea
%◦ Bessere Teillastausnutzung
%• SOEL 
%◦ Gleichbleibende eta
%◦ Keine Unterschiede Teil- (RES/WEA) zu Volllastbetrieb (EIS)

\begin{figure}[H]
	
	\centering
	%\includegraphics[width=0.49\textwidth]{/home/dafu/Schreibtisch/Master-Projekt/Doku/Abb/ .png}
	\includegraphics[width=0.8\textwidth]{/home/dafu/Schreibtisch/Master-Projekt/Doku/Abb/Graph/eco/2019-03-02--10-25_barplot_Cost_S_Cost_M_Cost_L_eta.pdf}
	\caption{ }
	\label{fig:analy_COST_eta} 
\end{figure}
Im Kontrast zu den zuvor beschriebenen Betriebsarten entsteht für die \gls{cost}-Betriebsweise ein abweichendes, jedoch innerhalb der Skalierungen sehr ähnliches Bild.\\
Hier liegt die  \gls{soel} mit gleichbleibend $80~-~82~ \%$ leicht oberhalb von Literaturangaben und damit deutlich über Effizienzwerten von \gls{ael} ($55~\%$) und \gls{pem} ($66~-~67~\%$). Für diese beiden Technologien ergeben sich für jede Skalierungsstufe (S, M , L) ähnliche bis gleiche Werte. Auch hier befindet sich die \gls{pem}-\gls{el} deutlich oberhalb von Literaturwerten, während Werte der \gls{ael} durchweg in der Mitte des Literaturwertebereichs liegen.

%Cost S/M/L:
%- eta
%• Cost S: aufsteigend von AEL zu PEM zu SOEL
%◦ AEL = 0,55; PEM=0,66; SOEL= 0,82
%• Cost M: aufsteigend von AEL zu PEM zu SOEL
%◦ AEL = 0,55; PEM=0,67; SOEL= 0,80
%• Cost M: aufsteigend von AEL zu PEM zu SOEL
%◦ AEL = 0,55; PEM=0,67; SOEL= 0,82
%• Cost S: AEL trifft Lit mittig; PEM über dem Lit, SOEL leicht über Lit
%• Cost M: AEL trifft mittig den Lit.; PEM über dem Lit.; SOEL am oberen Rand der Lit.
%• Cost L: AEL trifft mittig, PEM über dem Lit; SOEL knapp über dem Lit

\subsubsection{Nutzungsgrad verfügbarer Energiemenge}
\begin{figure}[H]
	\centering
	%\includegraphics[width=0.49\textwidth]{/home/dafu/Schreibtisch/Master-Projekt/Doku/Abb/ .png}
	\includegraphics[width=0.8\textwidth]{/home/dafu/Schreibtisch/Master-Projekt/Doku/Abb/Graph/eco/2019-03-02--11-44_barplot_EISMAN_RESIDUAL_WEA_eta_N.pdf}
	\caption{ }
	\label{fig:analy_nC_etaN} 
\end{figure}
Bezüglich der Ausnutzung verfügbarer Energie entsteht für \gls{eis}, \gls{res} und \gls{wea} ein deutlich inhomogenes Bild (Abbildung(\ref{fig:analy_nC_etaN})).
Für die \gls{eis}-Betriebsart weisen \gls{ael} und \gls{pem}-\gls{el} ähnlich hohe Werte im Bereich von $96$ bzw. $97~\%$ auf, während die \gls{soel} mit $72~\%$ deutlich darunter liegt.\\
Dagegen weist die \gls{ael} für beide weiteren Betriebsarten deutlich niedrigere Werte mit fallender Tendenz ($77$ bzw. $69~\%$) auf. Während für den Fall des \gls{res}-Betriebs die \gls{pem}-\gls{el} mit $94~\%$ ein hohes Niveau beibehält, steigt die \gls{soel} in beiden Fällen um etwa $15~\%$ auf $87~\%$. 
  
%Eta\_util
%• Eis: AEL = 0,96; PEM=0,97; SOEL= 0,72
%◦ aufsteigend von SOEL zu AEL zu PEM
%◦ PEM und AEL ungefähr gleich
%◦ AEL hoher Wirkungsgrad da fast immer P_nenn, kaum pminfrac
%◦ SOEL schlecht, weil häufig 0 Sig  häfig hochfahren  viel Ph
%
%• Res: AEL = 0,77; PEM=0,94; SOEL= 0,87
%◦ aufsteigend von AEL zu SOEL zu PEM
%◦ AEL schlecht, weil hoher Anteil Pin unterhalb pminfrac
%
%• Wea: AEL = 0,69; PEM=0,78; SOEL= 0,87
%◦ aufsteigend von AEL zu PEM zu SOEL
%◦ PEM regelt viel weg, wegen Temp. Schutz (Einst. vom PID)
%▪ 4mio sec. – 5,5mio sec.
%▪ Für ne bessere Aussage neuer Lauf
%• SOEL bei RES/WEA quasi gleiche eta_util, da durchgängiges Sig
%◦ Verluste durch Abschneiden des Sig bei überschreiten der Pnenn



\subsubsection{Spezifischer Energiebedarf}
Der spezifische Energiebedarf beschreibt die für die durchschnittliche Erzeugung eines Normkubikmeters Wasserstoffs notwendige Energiemenge. Auch hier ist ein Abgleich mit Literaturwerten möglich, wobei Abbildungsskalierungen so gewählt sind, dass deren Obergrenze nicht mit abgebildet wird.\\
\begin{figure}[H]
	
	\centering
	%\includegraphics[width=0.49\textwidth]{/home/dafu/Schreibtisch/Master-Projekt/Doku/Abb/ .png}
	\includegraphics[width=0.8\textwidth]{/home/dafu/Schreibtisch/Master-Projekt/Doku/Abb/Graph/eco/2019-03-02--12-06_barplot_EISMAN_RESIDUAL_WEA_E_spec.pdf}
	\caption{ }
	\label{fig:analy_nC_Espec} 
\end{figure}
Für die Betriebsarten \gls{eis}, \gls{res} und \gls{wea} entsteht insgesamt ein homogenes Bild. Die \gls{ael} liegt mit $5.2$ bis $5.4~kWh/m^3_N$ für sämtliche Fälle im Literaturbereich und deutlich oberhalb der anderen Technologien. Auch die \gls{soel} befindet sich in allen Fällen mit $4.3~kWh/m^3_N$ im mittleren Bereich von Literaturangaben. Dagegen unterschreitet die \gls{pem}-\gls{el} Werte der Literatur, wie auch der anderen Technologien, von \gls{eis} ($4.3~kWh/m^3_N$) zu \gls{wea} ($3.9~kWh/m^3_N$) zunehmend.  

%e_spez
%• (Ref. Werte AEL und PEM Obergrenze nicht angezeigt)
%• Eis: AEL = 5,40; PEM= 4,3; SOEL= 4,3
%◦ aufsteigend von PEM zu SOEL zu AEL
%◦ AEL im Lit. A und B Bereich; PEM deutlich unterhalb des Lit. A und B Bereichs; SOEL deutlich über Lit A und innerhalb von Lit. B
%◦ PEM/ SOEL vom Wert ähnlich
%◦ 
%
%• Res: AEL = 5,2; PEM= 4,0; SOEL= 4,3
%◦ aufsteigend von PEM zu SOEL zu AEL
%◦ 
%
%• Wea: AEL = 5,20; PEM= 3,90; SOEL= 4,30
%◦ aufsteigend von PEM zu SOEL zu AEL
%◦ 
%
%• AEL deutlich schlechter, weil 
%◦ weniger H2 produziert wird
%◦ von Pol-Kurv T\_betrieb < T\_betrieb PEM
%• PEM 
%◦ Eis  wenig Betriebsstunden aber große Anlagenleistung  e\_spez = hoch
%◦ Res weniger Volllaststd. als Wea  bessere Auslastung der Anlage bei Wea  im vgl. kleinere Anlage Res kaum Einfluss auf die e\_spez
%• SOEL 
%◦ gleichbleibende e-spez für EIS/RES/WEA
%▪ EIS: Zwar wenig Betriebsstd. (viel Volllast) dafür in der Zeit hohes P\_in (hohe H2 Prod)
%▪ WEA/ RES: mehr Betriebsstd. (wenig Vollast) dafür niedrigeres P\_in niveau

\begin{figure}[H]
	
	\centering
	%\includegraphics[width=0.49\textwidth]{/home/dafu/Schreibtisch/Master-Projekt/Doku/Abb/ .png}
	\includegraphics[width=0.8\textwidth]{/home/dafu/Schreibtisch/Master-Projekt/Doku/Abb/Graph/eco/2019-03-02--10-25_barplot_Cost_S_Cost_M_Cost_L_E_spec.pdf}
	\caption{ }
	\label{fig:analy_COST_Espec} 
\end{figure}
%#########################################################
%COST:
%- e\_spez
%• Cost S: aufsteigend von SOEL zu PEM zu AEL
%◦ SOEL = 3,7; PEM=4,5; AEL = 5,4
%◦ AEL über Lit A innerhalb lit B
%◦ PEM knapp innerhalb A, unterhalb B
%◦ SOEL oberhalb Lit A, knapp unterhalb Lit. B
%• Cost M: aufsteigend von SOEL zu PEM zu AEL
%◦ SOEL = 3,7; PEM=4,5; AEL = 5,4
%◦ AEL über Lit A innerhalb lit B
%◦ PEM knapp innerhalb A, unterhalb B
%◦ SOEL oberhalb Lit A, knapp unterhalb Lit. B
%• Cost M: aufsteigend von SOEL zu PEM zu AEL
%◦ SOEL = 3,7; PEM=4,5; AEL = 5,4
%◦ AEL über Lit A innerhalb lit B
%◦ PEM knapp innerhalb A, unterhalb B
%◦ SOEL oberhalb Lit A, knapp unterhalb Lit. B


\subsubsection{Sensitivität (9800er)}

auch: Wasserstoffgestehungskosten


\subsection{Ökonomische Analyse}
\subsubsection{Spezifische (H2) Produktionskosten ???}
%WEA EIS RES:
%cost_spez 
%• Eis: AEL = 1,04; PEM= 1,08; SOEL= 1,61
%◦ aufsteigend von AEL zu PEM zu SOEL
%◦ PEM/ AEL ähnlich groß
%
%• Res: AEL = 0,94; PEM= 0,86; SOEL= 1,09
%◦ aufsteigend von PEM zu AEL zu SOEL
%◦ SOEL schlecht  viel Dynamik durch das Input-Sig.
%• Wea: AEL = 0,86; PEM= 0,72; SOEL= 0,87
%◦ aufsteigend von PEM zu SOEL zu AEL
%◦ viel Dynamik, aber hohe Anzahl an Volllaststd.
-----------------------------------------------------------
\begin{figure}[!tbp]
	\centering
	\begin{minipage}[b]{0.49\textwidth}
	\includegraphics[width=\textwidth]{/home/dafu/Schreibtisch/Master-Projekt/Doku/Abb/Graph/eco/2019-03-02--14-37_barplot_EISMAN_RESIDUAL_WEA_Cost.pdf}
\caption{ }
\label{fig:analy_nC_speCo} 
	\end{minipage}
	\hfill
	\begin{minipage}[b]{0.49\textwidth}
	\includegraphics[width=\textwidth]{/home/dafu/Schreibtisch/Master-Projekt/Doku/Abb/Graph/eco/2019-03-02--10-29_barplot_Cost_S_Cost_M_Cost_L_Cost.pdf}
\caption{ }
\label{fig:analy_COST_speCo} 
	\end{minipage}
\end{figure}
%COST:
%cost\_spez
%• Cost S: AEL = 0,81; PEM=0,71; SOEL= 0,59
%• Cost M: AEL = 0,82; PEM=0,70; SOEL= 0,61
%• Cost L: AEL = 0,82; PEM=0,70; SOEL= 0,60
%• Cost S: aufsteigend SOEL zu PEM zu AEL
%• Cost M: aufsteigend SOEL zu PEM zu AEL
%• Cost L: aufsteigend SOEL zu PEM zu AEL
%• Skalierung im Volllastbereich kaum Effekt, weil viele lineare Abh. zB ct…
%• AEL am teuersten  allg. Betriebszustand schlecht
%• AEL und PEM unterschiedlich, weil PEM Auslegung sehr gut, AEL Auslegung eher schlecht

\subsubsection{Spezifische (H2) Produktionskosten und spezifischer Invest}
\begin{figure}[H]
	
	\centering
	%\includegraphics[width=0.49\textwidth]{/home/dafu/Schreibtisch/Master-Projekt/Doku/Abb/ .png}
	\includegraphics[width=\textwidth]{/home/dafu/Schreibtisch/Master-Projekt/Doku/Abb/Graph/eco/2019-03-02--15-41_plot_EIS_RES_WEA.pdf}
	\caption{ }
	\label{fig:analy_ecoVAls_all} 
\end{figure}

\begin{figure}[H]
	
	\centering
	%\includegraphics[width=0.49\textwidth]{/home/dafu/Schreibtisch/Master-Projekt/Doku/Abb/ .png}
	\includegraphics[width=\textwidth]{/home/dafu/Schreibtisch/Master-Projekt/Doku/Abb/Graph/eco/2019-03-02--10-33_plot_Costs.pdf}
	\caption{ }
	\label{fig:analy_ecoVAls_COST} 
\end{figure}




\subsection{Einzel - Ergebnisse}
\subsubsection{Strom-Spannungszusammenhänge}

\subsection{Verifizierung}
-> Abgleich mit Literatur
\subsection{Gesamt-Simulation}

\section{Ökonomische Betrachtung}
\subsection{Komponenten-Wahl}
\subsubsection{PEM}
-> S. 13 [Lettenmeier-Diss]-> Katalysator, Anode!!!

\subsection{Skalierung}
\subsection{Wasserstoff-Vertriebsmöglichkeiten}
- demibras (Hiwi/H2-Distr/sonstige) -> Tab 4, ff

\subsection{Sauerstoff-Vertriebsmöglichkeiten}

\section{Bewertung}
\subsection{Simulationsergebnisse}
\subsection{Ökonomie}
\subsection{Systemdienlichkeit}


\section{Diskussion/ Fazit}




\subsection{Ausblick}
\subsubsection{Alternative Anlagen-Konfig}
\begin{itemize}
	\item zur Steigerung der Anlagendynamik bzw. sinnvoller Eingrenzung ''verträglicher'' Stromdichten/-bereiche -> -> Kombination mit Akkumulator
	\item -->>dadurch ggf. durch Kombination mit EE Inselnetz-Aufbau möglich? 
\end{itemize}
\subsubsection{ggf. perspektivisch zu implementieren}
\begin{itemize}
	\item elektrochemische Dynamik
	\item[] möglich, da Zahl innerer Schleifendurchläufe beliebig hoch wählbar, ABER: Berechnungszeit
	\item Druck-Dynamik
	\item Ausgabe von Wirkungsgraden ( Faraday, Spannungs-)???
	\item saisonale Umgebungs-Temperatur (Einhausung d. EL) bzw. Heizenergie
	\item Modellierung über mehrere Jahre:
	\item isentrope Kompression nach |Thermodynamics of pressurized gas storage
	\begin{itemize}
		\item Korrelation mit Wetterdaten
		\item[] $\rightarrow$ jährlich verschiedene Datensätze
	\end{itemize}	
	\item optimale Betriebsweise über x Jahre:
	\begin{itemize}
		\item Druck <-> Membrandicke
		\item Temp./ Degr.
		
	\end{itemize}
	\ zu Stand-By / Off
	\begin{itemize}
		\item ggf. zusätzliche Spalte in df zu An/Aus -> über Zeitabstände von Null-Leistung
		\item für Echtzeit-Argumentation: ggf. auch über Wetterdaten % %https://openfredproject.wordpress.com/
	\end{itemize} 
	\item Thermo-Management
	\begin{itemize}
		\item Saisonale Simulation...Außentemperaturanpassung?
	\end{itemize}
\end{itemize}
\subsubsection{Ökonomie}
-> Day-Ahead-Market implementieren 

\subsubsection{Reflektion Projektmanagement}
\begin{itemize}
	\item Zeitplaneinhaltung schwierig
	\item Protokollierung ok
	\item Ergebnissicherung mangelhaft
	\item -> Striktere Überprüfung des individuellen Fortschritts -> Ehrlichkeit
	\item Projektpartner <-> Kommilitonen....schwierig
\end{itemize}

%En general no se numera las subsecciones. Esto se logra agregando un asterisco: %\verb|\subsection*{titulo}|
% \subsection*{Literaturverzeichnis}
% Citar es facil:
% \citet*[pag. 4]{silverstein1964giving}
% y \cite{serway2010fisica}, \citep{Erdos65}. Agreguen sus referencias a \verb|labibliografia.bib|.

% \subsubsection*{Minisección}
% Se puede ir hasta un nivel bajo.
% \section{Método}
% Otros deberían poder repetir el trabajo con la información disponible aquí.
% \section{Resultados}
% Hagan click sobre el indice o la referencias para navegar el documento.
% \section{Discusión o conclusión}
% La discusión debería ser corta y no tener subsecciones.

%\clearpage 
%\newpage

\bibliographystyle{alpha}
%\bibliographystyle{natdin}
\bibliography{proj-refs.bib}

\appendix
%\addcontentsline{toc}{chapter}{Anhang}
%\addtocontents{toc}{}%\protect\addtokomafont{chapterentry}{Anhang\ }
\section{Anni 1}


\end{document}