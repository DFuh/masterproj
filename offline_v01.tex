\documentclass[onecolumn,10pt,titlepage]{article}
\usepackage[a4paper,top=2.5cm,bottom=2cm,left=3cm,right=2cm,marginparwidth=1.75cm,headheight=28pt]{geometry}

\usepackage[utf8]{inputenc}
\usepackage[ngerman]{babel} 
\usepackage[onehalfspacing]{setspace}

\usepackage{natbib}
\usepackage{multicol}
\usepackage{multirow}
\usepackage{mathastext}
\usepackage{siunitx}

\usepackage[official]{eurosym}
%BORRAR O DEJAR: Cambia estilo de la fuente matemática y la deja con aspecto mas "técnico".
\usepackage{graphicx}
\usepackage{amssymb}
\usepackage{amsmath}
\usepackage{wrapfig}
\usepackage{float}
\usepackage{gensymb}
\usepackage[tc]{titlepic}
\usepackage{hyperref}
\hypersetup{
    colorlinks,
    citecolor=black,
    filecolor=black,
    linkcolor=black,
    urlcolor=black
}

%Helvetica
%\renewcommand{\familydefault}{\sfdefault}
%\usepackage[scaled=1]{helvet}
%\usepackage[format=plain,
%            labelfont={bf,it},
%            textfont=it]{caption}
            
%Latin modern sans


\renewcommand{\familydefault}{\sfdefault}
\usepackage{lmodern}
\usepackage[format=plain,
            labelfont={bf,it},
            textfont=it]{caption}
\fontfamily{lmss}\selectfont
%--------------------------------------

\renewcommand{\labelitemii}{$\circ$}
\renewcommand{\labelitemiii}{$\triangleright$}
%==============================================
\usepackage{fancyhdr}

%\renewcommand{\chaptermark}[1]{\markboth{#1}{}}
%
%\fancyhf{} % clear the headers
%\fancyhead[R]{%
%	% We want italics
%	\itshape
%%	% The chapter number only if it's greater than 0
%%	\ifnum\value{chapter}>0 \chaptername\ \thechapter. \fi
%%	% The chapter title
%%	\leftmark}
%\fancyfoot[C]{\thepage}
%
%\fancypagestyle{plain}{
%	\renewcommand{\headrulewidth}{0pt}
%	\fancyhf{}
%	\fancyfoot[C]{\thepage}
%}
%
%\setlength{\headheight}{14.5pt}
%\usepackage[automark]{scrpage2}
%\pagestyle{scrheadings}


\rhead{\nouppercase{\rightmark} } %(\nouppercase{\leftmark})}
\chead{}
\lhead{}
%\lfoot{\today}
\cfoot{}
\rfoot{\thepage}
\renewcommand{\headrulewidth}{0.4pt}
\renewcommand{\footrulewidth}{0.4pt}

%\renewcommand{\chaptermark}[1]{\markboth{#1}{}}
%\renewcommand{\chaptermark}[1]{\markboth{\MakeUppercase{\chaptername\ \thechapter.\ #1}}{}}


% % % % % % % % %gloss
%\usepackage{glossaries}
%\makeglossaries % create makeindex files
%%\newglossaryentry{}{name={},description={}}
%\newglossaryentry{pv}{name={PV},description={Photovoltaik}}
%\newglossaryentry{wea}{name={WEA},description={Windenergieanlage}}
%\newglossaryentry{ee}{name={EE},description={Erneuerbare Energien}}
%\newglossaryentry{ns}{name={NS},description={Niederspannung}}
%\newglossaryentry{ms}{name={MS},description={Mittelspannung}}
%\newglossaryentry{hs}{name={HS},description={Hochspannung}}
%
%\newglossaryentry{kw}{name={KW},description={Kraftwerk}}
%\newglossaryentry{vn}{name={VN},description={Verteilnetz}}
%\newglossaryentry{vnb}{name={VNB},description={Verteilnetzbetreiber}}
%
%
%\newglossaryentry{psw}{name={PSW},description={Pumpspeicherkraftwerk}}
%\newglossaryentry{facts}{name={FACTS},description={flexible alternating current transmission system}}
%\newglossaryentry{eeg}{name={EEG},description={ErneuerbareEnergienGesetz}}
%\newglossaryentry{brd}{name={BRD},description={Bundesrepublik Deutschland}}
%\newglossaryentry{nwa}{name={NWA},description={Netzwiederaufbau}}
%\newglossaryentry{thg}{name={THG},description={Treibhausgas}}
%\newglossaryentry{dea}{name={DEA},description={Dezentrale Erzeugungs Anlagen}}
%\newglossaryentry{ikt}{name={IKT},description={Informations- und Kommunikationstechnologie}}
%\newglossaryentry{dsm}{name={DSM},description={DemandSidemanagement}}
%\newglossaryentry{kkw}{name={KKW},description={Kernkraftwerk}}
%\newglossaryentry{ptg}{name={PtG},description={Power-to-Gas}}
%\newglossaryentry{scada}{name={SCADA},description={Supervisory Control and Data Aquisition}}
%\newglossaryentry{ael}{name={AEL},description={Alkaline Electrolyzer}}
%\newglossaryentry{pem}{name={PEM},description={Proton Exchange Membrane}}
%\newglossaryentry{soel}{name={SOEL},description={Solide Oxide Elektrolyzer}}
%\newglossaryentry{el}{name={EL},description={Elektrolyse/ Elektrolyseur}}
%%\newcommand{\glossentry}[2]{$#1$ \indent #2 \par \vspace{.4cm} }

%======================== test ============
\usepackage{siunitx}
\usepackage[acronym,toc,nomain]{glossaries}              % use glossaries-package


\setlength{\glsdescwidth}{15cm}

\newglossary[slg]{symbolslist}{syi}{syg}{Mathematische Symbole} % create add. symbolslist


\glsaddkey{unit}{\glsentrytext{\glslabel}}{\glsentryunit}{\GLsentryunit}{\glsunit}{\Glsunit}{\GLSunit}

\makeglossaries                                   % activate glossaries-package


% ==== EXEMPLARY ENTRY FOR SYMBOLS LIST =========================================
\newglossaryentry{symb:Pi}{name=\ensuremath{\pi},
	description={Geometrical value},
	unit={},
	type=symbolslist}

\newglossaryentry{height}{name=\ensuremath{h},
	description={Height of tower},
	unit={\si{m}},
	type=symbolslist}

\newglossaryentry{energyconsump}{name=\ensuremath{P},
	description={Energy consumption},
	unit={\si{kW}},
	type=symbolslist}

\newglossaryentry{Gibbs}{name=\ensuremath{\Delta G},
	description={Freie Gibbsche Enthalpie},
	unit={\si{kJ/mol}},
	type=symbolslist}

\newglossaryentry{i0}{name=\ensuremath{i_0},
	description={Austauschstromdichte},
	unit={\si{A/cm^2}},
	type=symbolslist}

\newglossaryentry{eta_sys}{name=\ensuremath{\eta_{Sys}},
	description={Systemwirkungsgrad},
	unit={\si{1}},
	type=symbolslist}

%==================
\newglossaryentry{Umgebungstemperatur}{name=\ensuremath{T_{amb}},
	description={Umgebungstemperatur},
	unit={$^\circ C$},
	type=symbolslist}

\newglossaryentry{Umgebungsdruck}{name=\ensuremath{p_{amb}},
	description={Umgebungsdruck},
	unit={\si{bar}},
	type=symbolslist}	

\newglossaryentry{Betriebsdruck}{name=\ensuremath{p},
	description={Betriebsdruck},
	unit={\si{bar}},
	type=symbolslist}	

\newglossaryentry{Betriebstemperatur}{name=\ensuremath{T_{tar}},
	description={Betriebstemperatur},
	unit={$^\circ C$},
	type=symbolslist}	

\newglossaryentry{max_Temperatur}{name=\ensuremath{T_{max}},
	description={maximale Temperatur},
	unit={$^\circ C$},
	type=symbolslist}	

\newglossaryentry{min_Temperatur}{name=\ensuremath{T_{min}},
	description={minimale Temperatur},
	unit={$^\circ C$},
	type=symbolslist}	

\newglossaryentry{Kuhlwassereingangstemperatur}{name=\ensuremath{T_{cw_{in}}},
	description={Kühlwassereingangstemperatur},
	unit={$^\circ C$},type=symbolslist}

\newglossaryentry{max_Stromdichte}{name=\ensuremath{i_{max}},
	description={maximale Stromdichte},
	unit={\si{A/{cm^2}}},
	type=symbolslist}	
%
\newglossaryentry{Zellen_pro_Stack}{name=\ensuremath{N_{ce_{st}}},
	description={Zellen pro Stack},
	unit={\si{}},
	type=symbolslist}	

%\newglossaryentry{spez_Warmekapazität}{name=\ensuremath{C_{t_{spec}}},
%	description={spez. Wärmekapazität},
%	unit={$kJ/KN_{ce}$},
%	type=symbolslist}	

\newglossaryentry{spez_Warmedurchgangskoeffizient}{name=\ensuremath{H_{AX_{spec}}},
	description={spez. Waermedurchgangskoeffizient},
	unit={$W/Kcm^2 N_{ce}$},
	type=symbolslist}
%
\newglossaryentry{spez_Waermedurchgangswiderstand}{name=\ensuremath{R_{t_{spec}}},
	description={spez. Waermedurchgangswiderstand},
	unit={$K/W$},
	type=symbolslist}

\newglossaryentry{Degradationsgradient}{name=\ensuremath{deg},
	description={Degradationsgradient},
	unit={${\mu}V/h$},
	type=symbolslist}
%
\newglossaryentry{Elektrodenoberflaeche}{name=\ensuremath{A},
	description={Elektrodenoberflaeche},
	unit={$cm^{2}$},
	type=symbolslist}
%
\newglossaryentry{Elektrodendurchmesser}{name=\ensuremath{\delta},
	description={Elektrodendurchmesser},
	unit={\si{cm}},
	type=symbolslist}
%
\newglossaryentry{Seperatordistanz}{name=\ensuremath{l},
description={Seperatordistanz},
unit={\si{cm}},
type=symbolslist}
%
\newglossaryentry{Tortuositat}{name=\ensuremath{\tau},
description={Tortuosität},
unit={\si{}},
type=symbolslist}

\newglossaryentry{Porositat}{name=\ensuremath{\epsilon},
description={Porosität},
unit={\si{}},
type=symbolslist}
%
%
\newglossaryentry{Porenradius_Knud}{name=\ensuremath{r_{P}},
description={Porenradius (Knudsen-Diff.)},
unit={\si{cm}},
type=symbolslist}
%
\newglossaryentry{Rugositaet}{name=\ensuremath{\zeta},
description={Rugositaet},
unit={\si{m^{2}/m^{2}}},
type=symbolslist}
%
\newglossaryentry{Elektrodenrauheit}{name=\ensuremath{\gamma},
description={Elektrodenrauheit},
unit={\si{}},
type=symbolslist}

\newglossaryentry{Ladungstransferkoeffizient}{name=\ensuremath{\alpha},
description={Ladungstransferkoeffizient},
unit={\si{}},
type=symbolslist}
%
\newglossaryentry{Freie_Aktivierungsenergie}{name=\ensuremath{{\Delta}G_{c}},
description={Freie Aktivierungsenergie},
unit={\si{J/mol}},
type=symbolslist}

\newglossaryentry{Referenzaustauschstromdichte}{name=\ensuremath{i_{0ref}},
description={Referenzaustauschstromdichte},
unit={\si{A/cm^{2}}},
type=symbolslist}
%
\newglossaryentry{Referenztemperatur $i_{0ref_{an}}$}{name=\ensuremath{T_{ref}},
description={Referenztemperatur},
unit={$^\circ C$},
type=symbolslist}

\newglossaryentry{Electrode_rate_parameter}{name=\ensuremath{k_{0}},
description={Electrode rate parameter},
unit={$mol/Ksm^{2}$},
type=symbolslist}
%
\newglossaryentry{Elektrodenleitfahigkeit}{name=\ensuremath{\sigma},
description={Elektrodenleitfaehigkeit},
unit={\si{S/m}},
type=symbolslist}

\newglossaryentry{Wettability}{name=\ensuremath{\omega},
description={Wettability},
unit={\si{}},
type=symbolslist}

\newglossaryentry{Spezifische_Resistivitt}{name=\ensuremath{\rho},
description={Spezifische Resistivität},
unit={$\Omega/cm$},
type=symbolslist}
%
\newglossaryentry{Membranfeuchtigkeit}{name=\ensuremath{\lambda_{mem}},
description={Membranfeuchtigkeit},
unit={\si{}},
type=symbolslist}

\newglossaryentry{Diff_Permeabilitat}{name=\ensuremath{\epsilon},
description={Diff. Permeabilitaet $O_{2}$ $O_{2}$},
unit={$mol/cmsbar$},
type=symbolslist}

\newglossaryentry{KOH_Konzentration}{name=\ensuremath{w_{KOH}},
description={KOH-Konzentration},
unit={$wt\%$},
type=symbolslist}	
%
\newglossaryentry{KOH_VolumenstromproStack}{name=\ensuremath{V_{KOH}},
description={KOH Volumenstrom pro Stack},
unit={$l/min$},
type=symbolslist}

\newglossaryentry{volllaststd}{name=\ensuremath{t_{vl}},
	description={Volllaststunden},
	unit={$h$},
	type=symbolslist}

\newglossaryentry{umgesEnergie}{name=\ensuremath{E_{ut}},
	description={umsetzbare Energiemenge},
	unit={$kWh$},
	type=symbolslist}

\newglossaryentry{T_stack}{name=\ensuremath{T_{st}},
	description={Stack-Temperatur},
	unit={$^\circ C$},
	type=symbolslist}

\newglossaryentry{m_c}{name=\ensuremath{m_{c}},
	description={Kühlwassermassenstrom},
	unit={$kg/S$},
	type=symbolslist}

\newglossaryentry{u_cell}{name=\ensuremath{U_{Zell}},
	description={Zellspannung},
	unit={$V$},
	type=symbolslist}

\newglossaryentry{i}{name=\ensuremath{i},
	description={Stromdichte},
	unit={$A/cm^2$},
	type=symbolslist}

\newglossaryentry{P_in}{name=\ensuremath{P_{in}},
	description={Eingangsleistung},
	unit={$W$},
	type=symbolslist}

\newglossaryentry{P_st}{name=\ensuremath{P_{st}},
	description={Stack-Leistungsaufnahme},
	unit={$W$},
	type=symbolslist}

\newglossaryentry{P_act}{name=\ensuremath{P_{act}},
	description={Anlagen-Leistungsaufnahme},
	unit={$W$},
	type=symbolslist}

\newglossaryentry{n_H2}{name=\ensuremath{n_{H_2}},
	description={Stoffstrom, Wasserstoff},
	unit={$mol/s$},
	type=symbolslist}

\newglossaryentry{n_O2}{name=\ensuremath{n_{O_2}},
	description={Stoffstrom, Sauerstoff},
	unit={$mol/s$},
	type=symbolslist}

\newglossaryentry{H2inO2}{name=\ensuremath{\theta_{H_2??}},
	description={Stoffstrom, Sauerstoff},
	unit={$mol/mol$},
	type=symbolslist}

\newglossaryentry{U0}{name=\ensuremath{U_0},
	description={reversible Zellspannung},
	unit={$V$},
	type=symbolslist}

\newglossaryentry{deltaH}{name=\ensuremath{\Delta H^0},
	description={Standardenthalpie},
	unit={$kJ/mol$},
	type=symbolslist}

\newglossaryentry{deltaGibbs}{name=\ensuremath{\Delta G^0},
	description={Freie Gibbsche Enthalpie},
	unit={$kJ/mol$},
	type=symbolslist}

\newglossaryentry{deltaS0}{name=\ensuremath{\Delta S^0},
	description={Standardentropie},
	unit={$kJ/Kmol$},
	type=symbolslist}

\newglossaryentry{E_cat}{name=\ensuremath{E_{Kat}},
	description={Kathodenpotential},
	unit={$V$},
	type=symbolslist}

\newglossaryentry{E_an}{name=\ensuremath{E_{An}},
	description={Anodenpotential},
	unit={$V$},
	type=symbolslist}

\newglossaryentry{Faraday}{name=\ensuremath{F},
	description={Anodenpotential},
	unit={$C/mol$},
	type=symbolslist}

\newglossaryentry{z}{name=\ensuremath{z},
	description={Anzahl ausgetauschter Elektronen},
	unit={},
	type=symbolslist}

\newglossaryentry{U_kons}{name=\ensuremath{z},
	description={Anzahl ausgetauschter Elektronen},
	unit={},
	type=symbolslist}

\newglossaryentry{rDEG}{name=\ensuremath{r_{deg}},
	description={Degradationsrate},
	unit={$\mu V/h$},
	type=symbolslist}

\newglossaryentry{top}{name=\ensuremath{t_{op}},
	description={absolute Betriebszeit},
	unit={$h$},
	type=symbolslist}

% ==== EXEMPLARY ENTRY FOR ACRONYMS LIST ========================================
\newacronym{VRBD}{VRBD}{Violet-Red-Bile-Glucose-Agar}
\newacronym{ael}{AEL}{Alkaline Electrolyzer}
\newacronym{pem}{PEM}{Proton Exchange Membrane}
\newacronym{soel}{SOEL}{Solide Oxide Elektrolyzer}
\newacronym{el}{EL}{Elektrolyse/ Elektrolyseur}
\newacronym{nt}{NT}{Niedertemperatur-}
\newacronym{ht}{HT}{Hochtemperatur-}
\newacronym{eis}{EIS}{Einspeise-Management }
\newacronym{res}{RES}{Residual (-last, -erzeugung)}
\newacronym{wea}{WEA}{Windenergieanlage}
\newacronym{cost}{COST}{Kosten-abhängiges-Signal (Abkürzung für diesen Betriebstyp)}

% ==== EXEMPLARY ENTRY FOR MAIN GLOSSARY ========================================
%\newglossaryentry{Biofouling}{name=Biofouling,description={Some description}}
%\newglossaryentry{ael}{name={AEL},description={Alkaline Electrolyzer}}
%\newglossaryentry{pem}{name={PEM},description={Proton Exchange Membrane}}
%\newglossaryentry{soel}{name={SOEL},description={Solide Oxide Elektrolyzer}}
%\newglossaryentry{el}{name={EL},description={Elektrolyse/ Elektrolyseur}}
%\newglossaryentry{nt}{name={NT},description={Niedertemperatur-}}
%\newglossaryentry{ht}{name={HT},description={Hochtemperatur-}}
%\newglossaryentry{eis}{name={EIS},description={Einspeise-Management (Abkürzung für diesen Betriebstyp)}}
%\newglossaryentry{res}{name={RES},description={Residual (-last, -erzeugung)(Abkürzung für diesen Betriebstyp)}}
%\newglossaryentry{wea}{name={WEA},description={Windenergieanlage(Abkürzung für diesen Betriebstyp)}}
%\newglossaryentry{cost}{name={COST},description={Kosten-abhängiges-Signal (Abkürzung für diesen Betriebstyp)}}
%%\newcommand{\glossentry}[2]{$#1$ \indent #2 \par \vspace{.4cm} }

\newglossarystyle{symbunitlong}{%
	\setglossarystyle{long3col}% base this style on the list style
	\renewenvironment{theglossary}{% Change the table type --> 3 columns
		\begin{longtable}{lp{0.6\glsdescwidth}>{\centering\arraybackslash}p{2cm}}}%
		{\end{longtable}}%
	%
	\renewcommand*{\glossaryheader}{%  Change the table header
		\bfseries Sign & \bfseries Description & \bfseries Unit \\
		\hline
		\endhead}
	\renewcommand*{\glossentry}[2]{%  Change the displayed items
		\glstarget{##1}{\glossentryname{##1}} %
		& \glossentrydesc{##1}% Description
		& \glsunit{##1}  \tabularnewline
	}
}

% ======================= end test =============


% \publishers{}

% \thanks{} %% use it instead of footnotes (only on titlepage)

% \dedication{} %% generates a dedication-page after titlepage


%%% uncomment following lines, if you want to:
%%% reuse the maketitle-entries for hyperref-setup
%\newcommand\org@maketitle{}
%\let\org@maketitle\maketitle
%\def\maketitle{%
%  \hypersetup{
%    pdftitle={\@title},
%    pdfauthor={\@author}
%    pdfsubject={\@subject}
%  }%
%  \org@maketitle
%}



% Definition of \maketitle
%\makeatletter         
%\def\@maketitle{
%	%\raggedright%
%	\includegraphics[height = 15mm]{logo_fb4_Bild2.png}%
%	\hfill
%	\includegraphics[height = 12mm]{Logo_uni_Bild1.png}\\[8ex]
%	\begin{center}
%		{\Huge \bfseries \sffamily \@title }\\[4ex] 
%		{\Large  \@author}\\[4ex] 
%		\@date\\[8ex]
%		%\includegraphics[width = 40mm]{image.png}
%	\end{center}}
%	\makeatother
%\titlepic{\includegraphics[width=0.3\textwidth]{logo_fb4_Bild2.png} \hspace{2cm} %\includegraphics[width=0.3\textwidth]{Logo_uni_Bild1.png}}
%\title{Modellierung des dynamischen Betriebsverhaltens von PtG-Anlagen als Systemdienstleistung für Stromnetze}

%\date{\today}

%\author{Lorenz Beck (3093543) \and 
%David Fuhrländer (3145486) \and 
%Jörn Lönneker (3145360)}


%========================> Comienza Documento
\begin{document}

%\maketitle
%\newpage
% \begin{titlepage}
% \centering

% { \large Universität Bremen  \par }
% \vspace{2cm}
% {\Large \scshape ?? \par}
% \vspace{2cm}
% {\Huge \scshape Modellierung des dynamischen Betriebsverhaltens von PtG-Anlagen als Systemdienstleistung für Stromnetze \par }
% %\vspace{1cm}
% %{\large \bf Grupo 10 \par}
% \vspace{0cm}
% \textsc{\large Autor -- Legajo \\ Coautor -- Legajo}
% \vspace{2cm}
% {\par \large Fecha de realización: \today \par}
% \vspace{1cm}
% {\large Fecha de entrega: .......................................\par}
% \vspace{2.5cm}
% {\large Firma del docente: .......................................}
% \vspace{3cm}
% \begin{figure}[htb!]
% \centering
% \includegraphics[width=6cm]{logoitba.png}
% \end{figure}
% \end{titlepage}

\begin{titlepage}
	\centering
	%\includegraphics[width=0.15\textwidth]{example-image-1x1}\par\vspace{1cm}
	
	{\includegraphics[height = 12mm]{Logo_uni_Bild1.png}\par}
	\vspace{1cm}
	{\scshape\Large Universität Bremen \par}
	\vspace{1cm}
	{\scshape\LARGE MScPT1 - Projektarbeit \par}
	\vspace{1.5cm}
	{\huge\bfseries Modellierung des dynamischen Betriebsverhaltens von PtG-Anlagen als Systemdienstleistung für Stromnetze\par}
	\vspace{2cm}
    {\large \today\par}
    \vspace{2cm}
	{\Large\itshape Lorenz Beck (3093543)\par}
    \vspace{0.3cm}
    {\Large\itshape David Fuhrländer (3145486)\par}
    \vspace{0.3cm}
    {\Large\itshape Jörn Lönneker (3145360)\par}
	\vfill
	betreuender Professor:\par
	%Prof. Dr. rer. nat. ~S. \textsc{Gößling Reisemann}\\
	Prof. Dr. Ing. ~J. \textsc{Thöming}\\

    \vfill
   	\includegraphics[height = 20mm]{RES_EN_logo_de_rot_mittel.png}
	\hfill	
	\includegraphics[height = 20mm]{logo_fb4_Bild2.png}%
	%\vfill

% Bottom of the page
	
\end{titlepage}

\pagenumbering{Roman}
\onehalfspace

\section*{Übersicht}
testetetete
naja...ob das was nützt?
ich änder hier jetzt was
{\textbf{Abstract}}\par
% Un resumen del problema a tratar debería tener menos de 200 palabras y no incluir referencias. Un resumen del problema a tratar debería tener menos de 200 palabras y no incluir referencias. Un resumen del problema a tratar debería tener menos de 200 palabras y no incluir referencias. Un resumen del problema a tratar debería tener menos de 200 palabras y no incluir referencias. Un resumen del problema a tratar debería tener menos de 200 palabras y no incluir referencias.

%\begin{multicols}{2}
% \section*{Nomenklatur}
% \glossentry{\pi}{pi}
% \glossentry{g}{pff}
%\glossentry{x}{}
%\end{multicols}

\newpage

\tableofcontents
\newpage
\listoffigures
\newpage
%  ==== glossary
\glsaddall

\printglossary[type=\acronymtype,style=long]  % list of acronyms
\newpage
\printglossary[type=symbolslist,style=symbunitlong]   % list of symbols
\newpage
\printglossary[type=main]                     % main glossary
% ===== 

\newpage
\pagenumbering{arabic}
\pagestyle{fancy}
\section{Einleitung}
%Schröer: Rolle von PtG -> günstiger als BAtterien


%===============
Zur Vermeidung unkontrollierbarer Folgen einer Klimaveränderung besteht spätestens seit dem Unterzeichnen des Pariser Klimaabkommen die verbindliche Anerkennung einer Notwendigkeit und Zusage der BRD, Emissionen von Treibhausgasen (THG) drastisch zu reduzieren \cite{Umweltbundesamt.}. Ein Großteil der in der BRD verursachten THG-Emissionen stammen mit einem Anteil von rund $80~\%$ aus der Energieversorgung \cite{Umweltbundesamt.}. Neben der Senkung des absoluten Bedarfes ist zur Reduktion der Emissionen eine Steigerung des Anteils erneuerbarer Erzeugungseinrichtungen erforderlich, was einen Wandel des etablierten Energieversorgungssystems hervorruft  \cite{Hey.26.Oktober2012}.\\
 Dieses System, welches sich aus der Historie heraus auf der Basis von fossilen oder nuklearen Energieerzeugungsanlagen aufbaut, befindet sich nun im Umschwung. Wurde zuvor die Energieerzeugung flexibel dem Verbrauch angepasst, so lässt sich durch den steigenden Anteil an erneuerbaren Energien nur in einem bestimmten Maß vorhersagen wie viel Energie produziert werden muss bzw. kann. Die Ursache hierfür liegt in der eingeschränkten Regelbarkeit von Windkraft- oder Photovoltaikanlagen, da ihre Erzeugerleistung von Umwelteinflüssen abhängig ist. Wenngleich der Wandel zu erneuerbaren Energien nicht ohne Aus- bzw. Umbau der aktuellen Netzstrukturen auskommt, wird zunehmend die Installation sinnvoller Speichersysteme notwendig \cite{Umweltbundesamt.}.\\
  Dies garantiert eine Erhöhung der Flexibilität auf der Verbraucherseite um überschüssige Energie zu speichern. Eine zentrale Rolle nimmt diesbezüglich die Wasserstofftechnologie als langfristige Speichermöglichkeit mit hohem infrastrukturellem Potential ein.
Diesbezüglich steht im Fokus dieser Arbeit die Produktion von Wasserstoff mittels Wasserelektrolyse, bei der Wasser mit Hilfe von elektrischer Energie in die Bestandteile Wasserstoff und Sauerstoff gespalten wird. Damit die Elektrolyse flexibel betrieben werden kann, werden vier verschiedene Eingangssignale (EisMan, Preis, Residuallast und Windpark) betrachtet, anhand derer entschieden wird, wann Wasserstoff produziert wird. Neben den vier Eingangssignalen werden zusätzlich drei verschiedene Typen von Elektrolyseuren mit einander verglichen und bewertet.\\
Verglichen werden der Protonen-Austausch-Membran Elektrolyseur (engl. „Proton Exchange Membrane“ Abk. PEM), die alkalische Elektrolyse (AEL) und die Hochtemperatur Elektrolyse (engl. „Solid Oxide Electrolysis Cell“ Abk. SOEC).  Der produzierte Wasserstoff kann in das bestehende Erdgasnetz eingeleitet, in einem separatem Druckspeicher gespeichert und zu einem späteren Zeitpunkt in Energie umgewandelt oder durch den Methanisierungsprozess, durch Zugabe von Kohlenstoffdioxid, zu Methan (Erdgas) umgewandelt werden. Der Vorteil dieses weiteren Prozessschrittes ist es, dass das produzierte Methan unbegrenzt in das Erdgasnetz eingeleitet und von sämtlichen auf Erdgas basierenden Verbrauchern (GuD-Kraftwerke, Autos, Wohnhäuser, etc.) genutzt werden kann. Ein weiterer Vorteil liegt hier in der bereits bestehenden Infrastruktur des Erdgasnetzes.
Ziel dieser Arbeit ist es, anhand einer Modellsimulation das dynamische Betriebsverhalten von Elektrolyseuren zu vergleichen und  Aussagen zu Betriebsverhalten und Systemdienlichkeit abzuleiten. Neben dem technischen Aspekt des dynamischen Betriebsverhaltens spielt auch die wirtschaftliche Betrachtung eine wichtige Rolle dieser Arbeit.\\
% Zunächst wird ein Überblick über elektrische Energieversorgung und den Energiemarkt,Systemdienstleistungen, die Grundlagen der Wasserelektrolyse und die   gegeben. Anschließend werden die Rahmenbedingungen aufgezeigt und die Charakterisierung der Szenarien vorgestellt. Darauf aufbauend werden die verschiedenen Modelle und die Ergebnisse der Simulation vorgestellt, welche anschließend in die ökonomische Betrachtung einfließen. Abschließend werden die Ergebnisse diskutiert und ein Fazit über die Einsatzmöglichkeit einer Power-to-Gas Anlage als Systemdienstleistung für Stromnetze gegeben.

\subsection{Vorbemerkungen}
Farbige Grafiken können in der Druckversion ggf. Kontrastschwächen aufweisen. Diesbezüglich sei auf die Digitalversion dieser Arbeit verwiesen.\\
Im Folgenden werden die Wortkombinationen bzw. Abkürzungen EISMAN, RESIDUAL, WEA und COST sowie die Kurzformen EIS und RES für die Bezeichnung von betrachteten Betriebsarten verwendet.  


\subsection{Problemstellung}
%-IPCC Bericht -> Klimaveränderung, höhere Anstrengungen notwendig
%-sog. Energiewende noch am Anfang -> Anteil EE an Energiebedarf
%- absehbar wesentlich höherer Anteil EE an E-Bedarf (insbes. Strom)
%->-> steigender Anteil fluktuierender Erzeuger
%->-> Speicher notwendig
%->-> EE-Kraftstoffe!
%Elektrolyse -> Sektorenkopplung
%-Projekt in Heide
%bereits aktuell hoher EE-Anteil
%-> -> techno-ökonomische Abschätzung der Wasserstoff-Erzeugungskosten notwendig
%über Abschreibungs/ Lebensdauer d. Anlagen
%Einbezug von Degradationseffekten



\subsection{Zielsetzung}
\subsubsection{Modell}
Als Resultat der vorliegenden Arbeit soll ein dynamisches Modell der aktuell gängigen Elektrolyse-Technologien hervorgehen, welches auch nach Abschluss der vorliegenden Arbeit weiter optimiert und adaptiert werden kann.\\

\subsubsection{Simulation}
Die Betriebs-Simulation erfolgt auf Grundlage des Elektrolyseur-Modells sowie auf charakteristischen Zeitreihen. Um das erstellte Modell mit verschiedenen aussagekräftigen Zeitreihen zu betreiben werden folgende Betriebsszenarien als Eingangssignale verwendet:
\begin{itemize}
	\item Residual-Last
	\item Einspeise-Management-Maßnahmen
	\item WEA-Erzeugung
	\item Strompreis-Signal
\end{itemize}  
Allen Eingangsgrößen liegt zu Grunde, dass Sie ein indirektes Abbild verschiedener Netzsituation liefern, wodurch ein Vergleich der betrachteten Elektrolyse-Technologien hinsichtlich Effizienz und Systemdienlichkeit erfolgen kann.  

%\subsubsection{transientes/dynamisches Verhalten}
%\begin{itemize}
%	\item Aufbau der Doppelschicht: Spannungs-/Strom-Verzögerung (in wenigen Sekunden -->>QUELLE???)
%	
%\end{itemize}

%\subsubsection{Gaskompress.}
%ggf. Redlich-Kwong Eq
\section{Theoretische Grundlagen}
\subsection{Grundlagen Elektrische Energienetze und -märkte}
\label{sub_GrundET}
Das Stromnetz bildet die zentrale Energieinfrastruktur der modernen europäischen Gesellschaft. Die Möglichkeit, elektrische Energie mit verhältnismäßig geringen Verlusten über weite Strecken zu transportieren und durch den europäischen Verbund einen hohen Grad an Versorgungssicherheit zu gewährleisten, macht es zu einem der wichtigsten Bestandteile innerhalb der Grundversorgungsstruktur \cite{Crastan2018}.
 In Deutschland ist das Übertragungsnetz in vier Regelzonen mit den Übertragungsnetzbetreibern (ÜNB) Amprion, 50 Hertz, TenneT und TransnetBW unterteilt. Die ÜNB tragen die Verantwortung für die Sicherheit und Zuverlässigkeit des Elektrizitätsversorgungssystems in enger Zusammenarbeit mit den anderen europäischen Netzbetreibern. Das Verbundnetz besteht aus mehreren Netzebenen mit unterschiedlich hohen Betriebsspannungen. Es kann allgemein in Übertragungsnetze, welche im  Hoch- $60-110kV$ und Höchstspannungsbereich $220/380k$ betrieben werden, und in Verteilsnetze im Nieder- $(230/400V)$ und Mittelspannungsbereich $(6-30 kV)$ unterteilt werden. Als Bindeglied zwischen den Netzebenenen dienen Leistungstransformatoren in Umspannwerken (im Übertragungsnetzbereich) sowie Transformatoren in Ortsnetzstationen (im Verteilnetzbereich) \cite{Konstantin.2013}. Die Stromerzeugung wandelt sich im Zuge der Energiewende erheblich.  2015 deckten Erneuerbare Energien $32,5\%$ des Bruttostromverbrauches Deutschlands, wobei die Windenergieeinspeisung insbesondere zunahm.\cite{Konstantin.2013} Die Einspeisung der EEG-Anlagen erfolgt in Abhängigkeit ihrer installierten Nennleistungen in niedrigen bis hohen Spannungsebenen.

\begin{figure}[H]
	\centering
	\includegraphics[width=0.9\textwidth]{/home/dafu/Schreibtisch/Master-Projekt/Doku/Abb/vorlagen/Netzstruktur.JPG}
	\caption[Vereinfachte Darstellung der Netzstruktur]{Vereinfachte Darstellung der Netzstruktur sowie der angeschlossenen Erzeuger und Verbraucher aus \cite{Zapf.2017}}
	\label{fig:Netzstruktur} 
\end{figure}

Die zunehmende Einspeisungen von EEG-Anlagen auf Verteilnetzebene beeinflusst die Versorgungsaufgabe des Stromnetzes stark. Die zentrale Stromerzeugung konventioneller Kraftwerke erfolgt in hohen Spannungsebenen. Der Strom wird zu den Endkunden auf den niedrigeren Netzebenen verteilt. Abweichend zum zentralen Erzeugungsmodell muss jedoch zunehmend der Stromtransport in beide Richtungen auf der Verteilnetzebene ermöglicht werden, um die Erzeugung der dezentralen Anlagen nutzen zu können. 

Inhomogene Windgeschwindigkeits- und Sonneneinstrahlungs-Verteilung führen zu regional unterschiedlicher Energieerzeugung. Des Weiteren ist für den Erhalt einer hohen Versorgungssicherheit die Stabilität des Netzes und damit der Ausgleich von Erzeugung und Verbrauch zu jedem Zeitpunkt erforderlich. Aus diesen Gründen ist mit zunehmenden Anteil Erneuerbarer eine räumliche und zeitliche Entkopplung von Erzeugung und Verbrauch notwendig. Während konventionelle Kraftwerke ihre Erzeugungsleistung nach Bedarf regeln können (beispielsweise Gas-Kraftwerke), können dargebotsabhängige, volatile Erzeuger nur begrenzt zur Erbringung von Systemdienstleistungen herangezogen werden.
%Ein Index für die Netzsituation stellt der aktuelle Börsenstrompreis dar. [??]
%Das Fraunhofer ISE hat einen Ansatz entwickelt, um mittels Rundsteuertechnik entsprechende Signale an Erzeuger, oder Verbraucher zu senden. Damit kann eine NEtz- bzw. systemdienliche Anlagenreaktion erzielt werden. \cite{Schneider2018}

\subsubsection{Systemdienstleistungen}
\label{subsub_SysDien}
ÜNB und Verteilnetzbetreiber sind verpflichtet, mithilfe von Regelungsmaßnahmen dauerhaft einen stabilen Betrieb des Stromnetzes zu gewährleisten. Die Stabilität des Netzes ergibt sich aus den Vorkehrungen, die für einen planbaren und planmäßigen sicheren Betrieb der Netze und gegen Spannungs- und Frequenzschwankungen ergriffen werden. Den Netzbetreibern stehen für die Stabilisierung der Netze eine Reihe von Systemdienstleistungen zur Verfügung, welche nachfolgend anhand ihrer Funktion eingeordnet werden können: \cite{Zapf.2017}

\begin{itemize}
	\item \textbf{Frequenzhaltung} erfolgt mittels konventioneller Kraftwerke, da regelbare Lasten notwendig sind. EEG-Anlagen können als Teil eines regelbaren Anlagenpools ebenfalls zur Frequenzhaltung genutzt werden. Die Frequenzhaltung erfolgt in diesem Falle über Wirkleistungsregelung.
	\item \textbf{Statische und dynamische Spannungshaltung} erfolgt durch konventionelle Anlagen, Kompensationsanlagen sowie EEG-Anlagen über Bereitstellung von Blindleistung, Kurzschlussleistung und durch Spannungsregelung. 
%	\item Weiterhin können ÜNB und Verteilnetzbetreiber auf Maßnahmen nach \§ 13 Absatz 1 und 2 des Energiewirtschaftsgesetztes zurückgreifen. So erfolgt im Rahmen des Engpassmanagements Regelung durch Redispatch, Countertrading oder Einspeisemanagement-Maßnahmen.
\end{itemize}

\subsubsection{Systemdienlichkeit}
Systemdienliches Verhalten trägt dazu bei, die frequenz- sowie spannungsbezogene Netzstabilität aufrechtzuerhalten. Dies bedeutet im Allgemeinen, dass nach Möglichkeit in Zeiten niedriger Erzeugung und hoher Last der Leistungsbezug so niedrig wie möglich bzw. im umgekehrten Fall eine schnelle Reaktion auf Einspeise-Spitzen erfolgt. 
Eine Power-to-Gas Anlage, wie sie in diesem Modell ohne Rückverstromung und Wiedereinspeisung der Energie simuliert wird, kann lediglich als regelbare Last eingesetzt werden. Einen perspektivischen Ansatz zur Ansteuerung regelbarer Lasten hat das Fraunhofer ISE mittels Rundsteuerempfänger entwickelt. Damit ist das gezielte Zuschalten bestimmter Verbraucher für bestimmte Lastsituationen mit geringem Technologieaufwand möglich \cite{Schneider2018}.

%\subsubsection{Speicher-Einrichtungen}
%Energiespeichereinrichtungen werden in Zukunft eine zunehmend wichtige Rolle im Netzbetrieb einnehmen. \cite{Schmiegel2019}
%Da in zukünftigen Stromnetzen der Anteil Erneuerbare Energien deutlich höher als aktuell ausfallen wird, sind entsprechend veränderte Eigenschaften des Energiesystems absehbar. Volatilen Eigenschaften von Photovoltaik- und Windenergieanlagen stehen im Gegensatz zu möglichst frei regelbarer bzw. sogenannter einlastbarer Leistung.\cite{Matthes2018}\\
%Durch die Anforderungen an einen ständigen Last- und Erzeugungsausgleich können Power-to-Gas Anlagen einerseits als regelbare Lasten betrieben werden und gleichzeitig die notwendige Grundlage alternativer Brenngase für flexible Gaskraftwerke liefer. [????????]
%
%- X kW notwendige install Leistung im Jahr 20?? [???]
%- PtG langfristig wichtige Technologie für relativ große Energiemengen günstiger \cite{Schroeer2015}
%ZITAT: ''PtG
%ist dabei ein zentrales Element in der kostenoptimalen Konfiguration für eine 100% EE-
%Stromversorgung''
%- Wasserstoff für Verkehr langfristig günstiger \cite{Robinius2018}

\subsubsection{Strombörse}
Der liberalisierte  Strommarkt  existiert in Deutschland seit 1998. Neben dem sogenannten Unbundling, der wirtschaftlichen Trennung von Erzeugung, Transport und Vertrieb, wurden seither die Strombörse als Großhandelsplattform mit den Spot- und Terminmärkten sowie die Regelenergiemärkte eingeführt Mit einem Handelsvolumen von mehr als $800 TWh$ im Jahr 2014 ist die Strombörse EEX die größte im deutschen Marktgebiet.\cite{Zapf.2017} Die Strompreisbildung erfolgt nach dem Merit-Order-Prinzip: ''An den Märkten mit Einheitspreissystem, wie z.B. den Day-Ahead-Märkten, wird den Kraftwerken der Reihe nach ein Zuschlag erteilt, beginnend mit dem aus den Grenzpreisen abgeleiteten niedrigsten Gebot, bis der prognostizierte Bedarf abgedeckt ist. Dies wird auch als Merit Order oder Einsatzreihenfolge der Kraftwerke bezeichnet.''\cite{Zapf.2017} Der sogenannte Market-Clearing-Price wird anhand des Gebotes des letzten Kraftwerkes, ebenfalls als Grenzkraftwerk bezeichnet, ermittelt. 
Da im Rahmen des Unbundling-Prozesses Erzeugung und Verbrauch entkoppelt wurden, kann in Abhängigkeit des Strompreisverlaufes nicht direkt auf einen Netzzustand geschlossen werden.\cite{Zapf.2017}

%\section{Theoretische Grundlagen}
%\subsection{Grundlagen Elektrische Energienetze}
%\subsubsection{Funktion und Herausforderungen}
%Das Stromnetz bildet die zentrale Energieinfrastruktur der modernen europäischen Gesellschaft. Durch die Möglichkeit, elektrische Energie mit verhältnismäßig geringen Verlusten über weite Strecken zu transportieren und durch den europäischen Verbund einen hohen Grad an Versorgungssicherheit zu gewährleisten, macht es zu einem der wichtigsten Bestandteile innerhalb der Grundversorgungsstruktur. [Crastan???] Die Versorgung mit elektrischer Energie wird zu jedem beliebigen Zeitpunk vorausgesetzt. \cite{Zahoransky2019} 
%Mit der Unterzeichnung des Pariser Klimaschutzabkommens \cite{Rogelj2016} hat sich auch die Bundesrepublik Deutschland zu drastischen CO2-Einsparmaßnahmen verpflichtet, wodurch eine Transformation des Energiesystems unausweichlich wird. Eine entsprechende Abkehr von konventionellen Kraftwerken, welche mittels fossiler Energieträger betrieben werden, erfordert einen deutlichen Ausbau erneuerbarer Erzeugungsleistung. [???] Da in Zukunft der Großteil erneuerbarer Erzeugungsleistung von volatilen Erzeugern bereitgestellt wird, ergeben sich Veränderungen der Kraftwerksstruktur sowie der Erzeugungscharakteristik. 
%Inhomogene Windgeschwindigkeits- und Sonneneinstrahlungs-Verteilung führen zu regional unterschiedlicher Erzeugung(-sstrukturen). Des Weiteren ist für den Erhalt einer hohen Versorgungssicherheit die Stabilität des netzes und damit der Ausgleich von Erzeugung und Verbrauch zu jedem Zeitpunkt erforderlich. Aus diesen Gründen ist mit zunehmenden Anteil Erneuerbarer eine räumliche und zeitliche Entkopplung von Erzeugung und Verbrauch notwendig. bezogen auf den -tandortabhängigen 
%Während konventionelle Kraftwerke einlastbare (im Rahmen technischer Möglichkeiten frei wählbare Leistung) Kraftwerke gelten und durch flexible Steuerung einen großen Teil zur Netzregelung beitragen, können dargebotsabhängige, volatile Erzeuger nur begrenzt zur Erbringung von Systemdienstleistungen herangezogen werden.
%
%
%--> historisch: Verteilung von oben nach unten
%--> aktuell: veränderung durch EE im MS/NS-Netz
%
%\subsubsection{Systemdienlichkeit}
%Systemdienliches Verhalten ist demnach eine Betriebsweise, welche dazu beiträgt, die Netzstabilität aufrecht zu erhalten. Dies bedeutet, dass nach Möglichkeit in Zeiten niedriger Erzeugung und hoher Last der Leistungsbezug so niedrig wie möglich bzw. im umgekehrten Fall eine schnelle Reaktion auf Einspeise-Spitzen erfolgt.\\
%Ein Index für die Netzsituation stellt der aktuelle Börsenstrompreis dar. [??]
%Das Fraunhofer ISE hat einen Ansatz entwickelt, um mittels Rundsteuertechnik entsprechende Signale an Erzeuger, oder Verbraucher zu senden. Damit kann eine NEtz- bzw. systemdienliche Anlagenreaktion erzielt werden. \cite{Schneider2018}
%
%\subsubsection{Systemdienstleistungen}
%-->>Frequenzbereiche Schulz Abb 87.58 (
%-->>Vgl. Erzeugungsanlagen: Preis, CO2, Volllaststd. (Schulz tab 87.10)
%
%\subsubsection{Speicher-Einrichtungen}
%Energiespeichereinrichtungen werden in Zukunft eine zunehmend wichtige Rolle im Netzbetrieb einnehmen. \cite{Schmiegel2019}
%Da in zukünftigen Stromnetzen der Anteil Erneuerbare Energien deutlich höher als aktuell ausfallen wird, sind entsprechend veränderte Eigenschaften des Energiesystems absehbar. Volatilen Eigenschaften von Photovoltaik- und Windenergieanlagen stehen im Gegensatz zu möglichst frei regelbarer bzw. sogenannter einlastbarer Leistung.\cite{Matthes2018}\\
%Durch die Anforderungen an einen ständigen Last- und Erzeugungsausgleich können Power-to-Gas Anlagen einerseits als regelbare Lasten betrieben werden und gleichzeitig die notwendige Grundlage alternativer Brenngase für flexible Gaskraftwerke liefer. [????????]
%
%- X kW notwendige install Leistung im Jahr 20?? [???]
%- PtG langfristig wichtige Technologie für relativ große Energiemengen günstiger \cite{Schroeer2015}
%ZITAT: ''PtG
%ist dabei ein zentrales Element in der kostenoptimalen Konfiguration für eine 100% EE-
%Stromversorgung''
%- Wasserstoff für Verkehr langfristig günstiger \cite{Robinius2018}

\subsection{Grundlagen Wasser-Elektrolyse}
\label{subs_Grundl_EL}

Die Produktion von Wasserstoff lässt sich durch verschiedenste Verfahren realisieren, basierend auf unterschiedlichen Energiequellen. Häufig eingesetzte Verfahren und ihre Energiequelle sind zum Beispiel die Dampfreformierung (chemische Energie), die elektrochemische Wasserspaltung (elektrische Energie), Wasserdampfelektrolyse (Strom und Wärme), oder die photokatalytische Wasserspaltung (Lichtenergie/ Photonen \cite{Schnurnberge.2004}. Die in dieser Arbeit betrachteten Prozesse gehören zu den Kategorien der elektrochemischen Wasserspaltung und Wasserdampfelektrolyse. Die Verfahren um Wasserstoff herzustellen sind alkalische Elektrolyse, mit flüssigem, basischen Elektrolyten (\gls{ael}), saure Elektrolyse mit einem polymeren Festelektrolyten (\gls{pem}-\gls{el}) und Hochtemperaturelektrolyse mit einem Festoxid als Elektrolyt (\gls{soel}).\cite{Smolinka.05.07.2011} Diese basieren auf dem Prinzip, dass sich das Wassermolekül durch Anlegen eines Gleichstroms in seine Elemente Wasserstoff und Sauerstoff spaltet. Eine solche Reaktion wird als erzwungene Redox-Reaktion bezeichnet und folgt folgender Gesamtgleichung \cite{Ghaib.2017}:
\\
\begin{equation} 
2H_2O \xrightarrow{Elektrolyse} 2H_2+O_2
\label{eq:EL_REakt}
\end{equation}
\\
Je nach Technologie unterscheiden sich Temperaturbereich, Kathoden- bzw. Anodenreaktion sowie der Ladungsträger, wie aus Tabelle(\ref{tab:Tempber+Reakt}) ersichtlich wird.

\begin{table}[H]

	\label{tab:Tempber+Reakt}
	\caption{Temperaturbereich, Kathoden- und Anodenreaktion und Ladungsträger (LT) der drei Wasserelektrolysetechnologien nach \cite{Lindermeir.7.11.2017}}
	
	\begin{tabular*}{\textwidth}{lllll}
		
		\textbf{Technologie} & \textbf{Temperaturbereich} & \textbf{Kathodenreaktion} & \textbf{LT} & \textbf{Anodenreaktion} \\ \hline \hline
		&&&&\\
		AEL&$40-90 ^\circ{C} $&$ 2H_2O + 2e^- \rightarrow H_2 + 2OH^- $&$OH^-$&$2OH^- \rightarrow 0,5O_2 + H_2O + 2e^- $\\
		&&&&\\
		PEM-EL&$50-80 ^\circ{C} $&$ 2H^+ + 2e^- \rightarrow H_2$&$ H^+ $&$ H_2O \rightarrow 0,5O_2 + 2H^+ +2e^- $\\
		&&&&\\
		SOEL&$700-1000 ^\circ{C} $&$ H_2O + 2e^- \rightarrow H_2 + 2O^{2-}$&$ O^{2-} $&$ O^{2-} \rightarrow 0,5O_2 + 2H^+ + 2e^-$\\
		
	\end{tabular*}
	
\end{table}

\subsubsection{Thermodynamische Grundlagen}
\label{subsub_thermodynGrundL}
Damit Wasser in seine Bestandteile gespalten werden kann muss dem System ein Energiebetrag $\Delta H$ zugeführt werden.\cite{Jung} In einem isothermen und adiabaten Prozess entspricht diese Energie dem oberen Heizwert des Wassers $(HHV=39,41kWh/kg)$, da ein Phasenwechsel des Wasserstoffs von der flüssigen in die gasförmige Phase stattfinden muss. Die benötigte Energiemenge, welche zugeführt werden muss, lässt sich nach der thermodynamischen Grundgleichung für die freie Enthalpie wie folgt berechnen:

\begin{equation}
\Delta H_0 = \Delta G_0 + T\Delta S_0
\end{equation}

Die Gesamtenergie ist die Summe aus der Standardenthalpie (auch Gibbs-Energie) $\Delta G_0$ und dem Produkt der Temperatur und der Standardentropiedifferenz, $T * \Delta S_0$. Wird die benötigte Energie in Form von thermischer Energie (Wärme) zugeführt, so findet die Zerlegung des Wassers bei $4700K$ statt. Alternativ kann die Spaltung von Wasser durch die Kombination von elektrischer und thermischer Energiezufuhr ablaufen. Dabei entspricht die freie Reaktionsenthalpie der elektrischen Energie und der Entropieterm der Wärme, die dem System zugeführt werden muss.\\

Der Betrag der elektrischen Energie, welche der Elektrolyse zugeführt werden muss, wird durch die reversible Zellspannung $U_0$ (auch Zersetzungsspannung) beschrieben.\cite{Hey.26.Oktober2012} Sie entspricht der Differenz aus den elektrochemischen Potenzialen $E_{Kat}$,$E_{An}$ der verwendeten Halbzellen. Eine Halbzelle beschreibt jeweils eine Teilreaktion der Redox-Reaktion, wobei an der Anode die Oxidation und an der Kathode die Reduktion stattfindet.

\begin{equation}
	U_0 = E_{Kat} - E_{An}
	\label{eq:Teilreakt}
\end{equation}

\begin{table}[H]
	\begin{tabular*}{\textwidth}{lll}\\
		$E_{Kat}$&Kathodenpotentials&$V$\\
		$E_{An}$&Anodenpotentials&$V$\\
		
	\end{tabular*}
\end{table}

Die reversible Zellspannung stellt eine charakteristische Kennzahl der Elektrolyse dar und beschreibt den Punkt, an dem die Zersetzung eintritt. Sie entspricht der Differenz des Kathodenpotentials $E_{Kat}$ und des Anodenpotentials $E_{An}$ (Gleichung(\ref{eq:Teilreakt})). In der Wasserelektrolyse beträgt die Zersetzungsspannung unter Standardbedingungen ($1bar$ und $25 ^ \circ C$) $1,23V$ und wird durch Druck und Temperatur beeinflusst. Nach \cite{Kurzweil.2015} kann, wie in Gleichung(\ref{eq:U_0}) dargestellt, die reversible Zellspannung berechnet werden:
	
	\begin{equation}
	U_0 =  \frac{\Delta G_0}{zF}	
	\label{eq:U_0}
	\end{equation}
	
	\begin{table}[H]
		\begin{tabular*}{\textwidth}{lll}\\
			$U_0$&reversible Zellspannung&$V$\\
			$\Delta G_0$&freie Standardreaktionsenthalpie&$J/mol$\\
			$z$&Ladungszahl&\\
			$F$&Faraday-Konstante&$C/mol$\\
		\end{tabular*}
	\end{table}
	
	Stellt man nach der Gibbs-Energie um, so erhält man die Gibbs-
	Helmholtz-Gleichung(\ref{eq:Gibbs-Helmh}):\cite{Bayer.2000}
	
	\begin{equation}
	\Delta G_R^0 =\Delta H_R^0 - T\Delta S^0
	\label{eq:Gibbs-Helmh}
	\end{equation}
	
	Mit Hilfe der Standardbildungsenthalpien $\Delta H_B^0$ und -entropien $\Delta S_B^0$ bei Standardbedingungen, lassen sich die Standardreaktionsenthalpie $\Delta H_R^0$ (Gleichung(\ref{eq:Standenthalp})) und die Standardreaktionsentropie $\Delta S_R^0$ (Gleichung(\ref{eq:standentrop}
	)) aus der Summe der Produkte und Edukte berechnen.
	
	\begin{align}
	\Delta H_R^0 =\sum{\Delta H_{B~Prod}^0}-\sum{\Delta H_{B~Eduk}^0} =286,17 kJ/mol \label{eq:Standenthalp}\\
	\Delta S_R^0 =\sum{\Delta S_{B~Prod}^0}-\sum{\Delta S_{B~Eduk}^0} = 162,63 kJ/mol \label{eq:standentrop}
	\end{align}
	

	Die Werte für die Standardbildungsenthalpien und -entropien von Wasser, Wasserstoff und Sauerstoff können einschlägiger Fachliteratur entnommen werden.\\
	Eine weitere wichtige Größe ist die Thermoneutralspannung $U_{th}$, sie bezieht sich allein auf die Reaktionsenthalpie $\Delta H$ der Wasserelektrolyse und bezeichnet die Spannung oberhalb derer die Elektrolyse unter zusätzlicher Wärmeentwicklung abläuft \cite{Kurzweil.2015}. Nach Gleichung(\ref{eq:U_H}) kann diese wie folgt berechnet werden:
	
	\begin{equation}
	U_{th} = \frac{\Delta H^0}{zF} = 1,48 V
	\label{eq:U_H}
	\end{equation}
	
	Die Differenz zwischen der Thermoneutralspannung und der reversiblen Zellspannung entspricht dem Entropieanteil $T\Delta S$ der nicht in Nutzarbeit umgewandelt werden kann und dem System kontinuierlich entzogen wird. Wird dieser Anteil der Energie nicht in Form von Wärme von außen zugeführt, so ist eine zusätzliche Zufuhr elektrischer Energie nötig.\\
	Die reversible Zellspannung sowie die Thermoneutralspannung sind temperatur- und druckabhängige Größen, wobei höhere Temperaturen eine Verringerung Druckanstiege dagegen eine Erhöhung der reversiblen Zellspannung verursachen. In Abbildung(\ref{fig:Temperatur_U0}) ist die Temperatur-, in Abbildung(\ref{fig:Druck_U0}) die Druckabhängigkeit der Zell- oder Zersetzungsspannung dargestellt.\cite{Kurzweil.2015}
	
	\begin{figure}[H]
		\centering
		\includegraphics[width=0.6\textwidth]{/home/dafu/Schreibtisch/Master-Projekt/Doku/Abb/vorlagen/Temp_Abh_Zersetzungsspannung.PNG}
		\caption[Temperaturabhängigkeit der Zersetzungspannung]{Temperaturabhängigkeit der Zersetzungspannung aus \cite{Kurzweil.2015}}
		\label{fig:Temperatur_U0} 
	\end{figure}
	
	\begin{figure}[H]
		\centering
		\includegraphics[width=0.6\textwidth]{/home/dafu/Schreibtisch/Master-Projekt/Doku/Abb/vorlagen/Druckabh_der_Zersetzungsspannung.PNG}
		\caption[Druckabhängigkeit der Zersetzungspannung]{Druckabhängigkeit der Zersetzungspannung aus \cite{Kurzweil.2015}}
		\label{fig:Druck_U0} 
	\end{figure}
	
	\subsubsection{Zellspannung}
	\label{subsub_Zellspann}
	
	Die zum Ablauf der Elektrolyse notwendige Spannung liegt erfahrungsgemäß \cite{Zhang.2010,Buttler.2018} deutlich oberhalb der reversiblen Zellspannung. Diese Spannung wird als Zellspannung $U_{Zell}$ bezeichnet und setzt sich aus der reversiblen Zellspannung $U_0$ sowie der Überspannung $\eta$, bestehend aus verschiedenen Überspannungsanteilen, zusammen. Gleichung(\ref{eq:U_Zell}) beschreibt die reale Zusammensetzung der Zellspannung:
	
	\begin{equation}
	U_{zell}= -\Delta E = U_0 + \eta_{con,An} + \eta_{con,Ka} + \eta_{act,An} + \eta_{act,Ka} + \eta_{ohm} + \eta_{deg}
	\label{eq:U_Zell}
	\end{equation}
	
	\begin{table}[H]
		\begin{tabular*}{\textwidth}{lll}\\
			$U_0$&reversible Zellspannung&$V$\\
			$\Delta E$&Zellpotential&$V$\\
			$\eta_{con,Ka}$&Konzentrationsüberspannung an der Kathode&$V$\\
			$\eta_{con,An}$&Konzentrationsüberspannung an der Anode&$V$\\
			$\eta_{act,Ka}$&Aktivierungsüberspannung an der Kathode&$V$\\
			$\eta_{act,An}$&Aktivierungsüberspannung an der Anode&$V$\\
			$\eta_{ohm}$&Ohmsche Verluste&$V$\\
			$\eta_{deg}$&Degradationsüberspannung&$V$\\
		\end{tabular*}
	\end{table}
	
	%Nach (Ghaib 2017) wirkt sich ab einem Druck von $10 bar$ der Entropieeinfluss negativ auf die Zellspannung aus. Sie wird mit Hilfe von Gleichung (2.9-2) beschrieben.
	
	\subsubsection*{Reversible Zellspannung $U_0$ }
	
	Die reversible Zellspannung lässt sich als druck- und temperaturabhängige Größe mittels Nernst-Gleichung (\ref{eq:Nernst}) bestimmen. Allgemein gilt nach \cite{NI.2007}:
	
	\begin{equation}
	U(p,T) = U_0 + \frac{RT}{zF}\ln{\left({p_i}^{\nu_{i}}\right)}
	\label{eq:Nernst}
	\end{equation}
	
	\begin{table}[H]
		\begin{tabular*}{\textwidth}{lll}\\
			$U_0$&reversible Zellspannung&$V$\\
			$T$&Temperatur&$K$\\
			$p_i$&Partialdruck der Redox-Edukte und Produkte&$Pa$\\
			$\nu_i$&Stöchiometrischer Koeffizient&\\
		\end{tabular*}
	\end{table}

	Für die Wasserelektrolyse ergibt sich somit unter Berücksichtigung von Gleichung(\ref{eq:EL_REakt}) der folgende Zusammenhang:
	
	\begin{equation}
	U(p,T) = U_0 + \frac{RT}{zF}\ln{\left(\frac{p_{H_2}\cdot p_{O_2}^{\frac{1}{2}}}{p_{H_2O}}\right)}
	\label{key}
	\end{equation}
	
	\subsubsection*{Konzentrationsüberspannung $\eta_{con}$}
	Die Konzentrationsüberspannung beschreibt den durch den Stofftransport verursachten Spannungsabfall an den Elektrodenoberflächen der Kathode sowie der Anode.\cite{UniversitatUlm.2016} Aufgrund der elektrochemischen Umwandlung sinkt die Konzentration der Edukte, welche durch Diffusion nachgeliefert werden müssen. Gleichzeitig steigt die Konzentration der Produkte, welche entsprechend abtransportiert werden müssen, an. Die Konzentrationsüberspannung an der Kathode kann über Gleichung(\ref{eq:konz_Cath}) und die an der Anode mit Hilfe der Gleichung(\ref{eq:konz_AN}) berechnet werden.\cite{NI.2007}
	  Der Partialdruck $p_i^0$ des jeweiligen Stoffes (Wasserstoff, Sauerstoff, Wasser) gilt  im Elektrolyten und beschreibt die Ausgangskonzentration. Der Partialdruck $p_i^1$  gilt an der Elektroden-Elektrolyt-Grenzfläche und beschreibt den Zustand der gestiegenen bzw. verminderten Konzentrationen.
	
	\begin{equation}
		U_{conc,Ca} = \frac{RT}{2F}ln{\left(\frac{p_{H_2}^1 \cdot p_{H_2O}^0}{p_{H_2}^0\cdot p_{H_2O}^1}\right)}
		\label{eq:konz_Cath}
	\end{equation}
	
	\begin{equation}
		U_{conc,An} = \frac{RT}{2F}ln{\left(\frac{p_{O_2}^1}{p_{O_2}^0}\right)}^{\frac{1}{2}}
		\label{eq:konz_AN}
	\end{equation}
	
	
	Sinkt die Konzentration an der Elektrodenoberfläche auf null ab, ist die maximale Stromdichte der Elektrolysezelle erreicht. An diesem Punkt ist die Reaktion an der Elektrodenoberfläche sehr viel schneller, als der Stofftransport zur Elektrode.\cite{UniversitatUlm.2016} Die Diffusion der Edukte zur Elektrode können zusätzlich durch Gasblasenbildung behindert werden.
	
	\subsubsection*{Aktivierungsüberspannung $\eta_{act}$}
	Die Aktivierungsüberspannung beschreibt die Ladungsmenge, welche für chemische Reaktionen an den Elektroden benötigt wird, damit diese einen energetisch höherwertigen Zustand erreichen.\cite{Klein.2013} Die Buttler-Vollmer-Gleichung(\ref{eq:ButtlerVoll}) \cite{NI.2007} beschreibt die Abhängigkeit der Aktivierungsüberspannungen der Anode und Kathode von der Stromdichte.
	
	\begin{equation}
		i=i_{0,j}\left(\exp{\frac{\alpha -1)zF\eta_{act,j}}{RT}-\exp{\frac{(\alpha -1)zF\eta_{act,j}}{RT}}}\right)
		\label{eq:ButtlerVoll}
	\end{equation}
			
			
			\begin{table}[H]
				\begin{tabular*}{\textwidth}{lll}
					$j$&Anode / Kathode&\\
					$i_0$&Austauschstromdichte&$A/m^2$\\
					$\alpha$&Durchtrittsfaktor&\\
					$\eta_{act}$&Aktivierungsüberspannung&$V$\\
				\end{tabular*}
			\end{table}
			
			Logarithmieren und Umformen von Gleichung(\ref{eq:ButtlerVoll}) liefert die in Gleichung(\ref{eq:buttVoll_Umst}) Aktivierungsüberspannung nach :
			
			\begin{equation}
				\eta_{act}=\frac{RT}{\alpha_{An}F}\ln{\frac{i}{i_{0,an}}}+\frac{RT}{\alpha_{Ka}F}\ln{\frac{i}{i_{0,Ka}}}
				\label{eq:buttVoll_Umst}
			\end{equation}
			
			
			Die Austauschstromdichte $i_0$ ist ein Maß für die elektrokatalytische Aktivität des Elektrodenmaterials. Sie ist konzentrations- und druckabhängig und kann nach Gleichung(\ref{eq_i0}) berechnet werden.\cite{NI.2007} 
			
			\begin{equation}
				i_{0,j}= \gamma_j\cdot \exp{\frac{-E_{act,j}}{RT}}
				\label{eq_i0}
			\end{equation}
			
			\begin{table}[H]
				\begin{tabular*}{\textwidth}{lll}
					$\gamma_j$&Vorfaktor zur Bestimmung der Austauschstromdichte der Anode / Kathode&\\
					$\Delta G_{c,act,j}$&Aktivierungsenergie der Anode / Kathode&$J/mol$\\
				\end{tabular*}
			\end{table}
			
			Die Geschwindigkeit, mit der die Reaktion abläuft, ist abhängig von Temperatur, Druck und Elektrodenmaterial. So führen hohe Temperaturen, wie sie bei der \gls{soel} auftreten, zu hohen Reaktionsgeschwindigkeiten und somit zu geringen Aktivierungsüberspannungen.
			
			\subsubsection*{Ohmsche Überspannungen $\eta_{ohm}$}
			Die Überspannung, die durch den Spannungsabfall über dem Elektrolyten aufgebracht werden muss, wird als ohmscher Spannungsverlust beschrieben.\cite{Klein.2013} Der Anteil der ohmschen Überspannung hängt von der Leitfähigkeit des Elektrolyten ab. Grundsätzlich wird zwischen der elektronischen und der ionischen Leitfähigkeit unterschieden. Nach dem ohmschen Gesetz setzt sich die ohmsche Überspannung aus der Stromdichte und einem flächenspezifischen Widerstand nach Gleichung(\ref{eq:U_ohm}) zusammen.\cite{Goetz.2000}
			
			\begin{equation}
			U_{ohm}=i\cdot R_{spez}
			\label{eq:U_ohm}
			\end{equation}
			
			\begin{table}[H]
				\begin{tabular*}{\textwidth}{lll}
					$U_{ohm}$&Ohmsche Spannung&$V$\\
					$RE_{spez}$&spezifischer Widerstand&$\Omega$\\
				\end{tabular*}
			\end{table}

			Der ohmsche Widerstand einer Elektrolyse-Zelle setzt sich aus drei Anteilen zusammen: dem Kontaktwiderstand, dem Leitungswiderstand und dem Widerstand des Elektrolyten. In der Regel besitzen die Kontaktplatten und die Elektroden eine sehr viel höhere elektrische Leitfähigkeit als der Elektrolyt. Die Spannungsverluste, welche durch die ersten beiden Verlustarten entstehen, können somit vernachlässigt werden. Im Folgenden wird ausschließlich der Elektrolytwiderstand berücksichtigt.\cite{NI.2007}
			
			\subsubsection*{Degradationsüberspannug $\eta_{deg}$}
			Unter Degradation wird der Verlust von Leistung durch Alterung der Elektrolyse-Zellen und der damit verbundene Anstieg der Überspannung bezeichnet. Der Überspannungsanteil, der durch Alterung der Membran entsteht, wird als Degradationsüberspannung bezeichnet und ist abhängig vom betrachteten Elektrolysesystem. So weisen die drei Elektrolyseanlagen jeweils unterschiedliche Degradationsraten auf. Im Fall der \gls{soel} liegt die Lebensdauer einer Zelle im Bereich von $8.000 ~– ~20.000~h$ \cite{DeNiangThe.12.2015}, während die \gls{pem}-\gls{el} eine Lebendsauer von $60.000~ - ~100.000~h$ aufweist. Werte bezüglich der \gls{ael} bewegen sich zwischen $55.000$ und $120.000~h$.\cite{Buttler.2018} Die Degradation einer Elektrolyse-Zelle kann durch folgende Effekte hervorgerufen werden:
			\begin{itemize}
				\item chemische/ elektrochemische Degradation
				\item mikrostrukturelle Materialveränderung
				\item Materialzerstörung
			\end{itemize}
			Bei der chemischen / elektrochemischen Degradation findet eine unerwünschte Reaktion an der Grenzfläche zwischen Sauerstoffelektrode und Elektrolyt statt. Dadurch bilden sich sauerstoffhaltige Verbindungen. Die entstehenden Verbindungen können zu einer Delamination des Elektrolyten und somit zu einer mikrostrukturellen Veränderung des Materials, bis hin zu einer Materialzerstörung führen. Eine solche Veränderung hat zur Folge, dass es zu einer Erhöhung der Wasserstoff-Permeation durch die Membran kommt. Erreicht die Permeation ein kritisches Level und der Anteil von Wasserstoff in Sauerstoff übersteigt Werte von $4 ~Mol\%$, ist die Explosionsgrenze der Mischung und damit ein hochgefährlicher Zustand erreicht. In der Regel werden Elektrolyseure mit einem Sicherheitsfaktor von $2$ betrieben, weswegen ab eine molaren Konzentration von $H_2$ in $O_2$ $>~2~Mol\%$ die Membran ersetzt werden muss.\cite{Trinke2018}
			Zur Abbildung der Degradation wird innerhalb dieser Arbeit eine lineare Degradationsrate $r_{deg}$ (in $\mu V/h$) bezogen auf die absolute Betriebszeit $t_{op}$ angenommen und nach Gleichung(\ref{eq:degr_calc}) berechnet.
			
			\begin{equation}
				\eta_{deg} = r_{deg} * t_{op}
				\label{eq:degr_calc}
			\end{equation}	
			
			\subsubsection{Polarisationskurve und Betriebsbereich}
			Üblicherweise wird der Stromfluss einer Elektrolyse-Zelle auf ihre Elektrodenfläche bezogen und nach Gleichung(\ref{eq:i}) als Stromdichte formuliert.\cite{Hey.26.Oktober2012} Die Stromdichte wird in der Literatur häufig in $A/cm^2$ (oder alternativ in $A/m^2$) ausgedrückt.
			
			\begin{equation}
				i=\frac{I}{A}
				\label{eq:i}
			\end{equation}
			
			\begin{table}[H]
				\begin{tabular*}{\textwidth}{lll}\\
					$i$&Stromdichte&$A/m^2$\\
					$I$&elektrischer Strom&$A$\\
					$A$&Elektrodenoberfläche&$m^2$\\
				\end{tabular*}
			\end{table}
			
			Zur Auslegung eines Elektrolyseurs und Berechnung des Zellwirkungsgrades, wird die reale Zellspannung $U_{Zell}$ in Abhängigkeit der Stromdichte $i$ angegeben.\cite{Tjarks.2017} Der sich ergebene Zusammenhang ist für jeden Zellaufbau charakteristisch und wird als ''Polarisationskurve'' bezeichnet. Die Polarisationskurve berechnet sich aus der Summe reversibler Zellspannung $U_{rev}$ zuzüglich sämtlicher Überspannungen $\eta$ nach Gleichung(\ref{eq:U_Zell}). Die Polarisationskurve ist neben den gewählten Betriebsbedingungen, Temperatur und dem Druck, stark abhängig von den verwendeten Komponenten der Elektrolyse-Zelle. In Abbildung(\ref{fig:polar_Grundlagen}) werden die Polarisationskurve und ihre einzelnen Bestandteile am Beispiel eines PEM-Elektrolyseurs gezeigt.
			
			\begin{figure}[H]
				\centering
				
				\includegraphics[width=0.5\textwidth]{/home/dafu/Schreibtisch/Master-Projekt/Doku/Abb/vorlagen/Polaisationskurve3.png}
				\caption[Beispiel einer PEM-Polarisationskurve]{Beispiel einer Polarisationskurve der \gls{pem}-Elektrolyse aus \cite{Abdin.2017}}
				\label{fig:polar_Grundlagen} 
			\end{figure}
			
			Um die Zelleffizienz für einen bestimmten Betriebspunkt zu erhöhen muss die Zellspannung bei gleichbleibender Stromdichte reduziert werden, zum Beispiel durch erhöhen der Temperatur oder absenken des Drucks.
			
			\subsubsection{Kenndaten der Elektrolyse}
			\label{subsubb_Kenndaten}
			Um eine Vergleichbarkeit von den verschiedenen Elektrolyseuren herstellen zu können, werden folgende Kenndaten genutzt:
			
			\begin{table}[H]
				\begin{tabular*}{\textwidth}{lll}
					$\eta_{sys}$&Systemwirkungsgrad&\\
					$\eta_{util}$&Nutzungsgrad der verfügbaren Energiemenge&\\
					$E_{spez}$&Spezifischer Energiebedarf&$m^2$\\
					$k_{spez}$&Wasserstoffgestehungskosten&$EUR/m^3_N$\\
					$m_{H_2}$&Produzierte Wasserstoffmenge&$kg$\\
				\end{tabular*}
			\end{table}
			
			\subsubsection*{Wasserstoffproduktion}
			Eine der essentiellen Kennzahlen zum Vergleich von Elektrolyseuren ist die produzierte Wasserstoffmasse bzw. unter Berücksichtigung der Dichte im Normzustand das produzierte Wasserstoffvolumen. Nach dem Faraday’schen Gesetz wird die produzierte Masse Wasserstoff proportional zur Stromdichte nach Gleichung(\ref{eq:m_H2_calc}) berechnet.\cite{Petipas.2013} 
			
			\begin{equation}
				m_{H_2}=\frac{i\cdot A}{2F}\cdot M_{H_2}
				\label{eq:m_H2_calc}
			\end{equation}
			
			\begin{table}[H]
				\begin{tabular*}{\textwidth}{lll}\\
					$i$&Stromdichte&$A/m^2$\\
					$A$&Elektrodenoberfläche&$m^2$\\
					$M_{H_2}$&Molare Masse von Wasserstoff&$g/mol$\\
				\end{tabular*}
			\end{table}
			
			\subsubsection*{Ausnutzung der verfügbaren Energie}
			Die Skalierung einer Elektrolyseanlage bestimmt die technisch möglich umzusetzende Arbeit aufgrund ihrer Betriebsparameter und Nennleistung. Ist die Eingangsleistung in einem betrachteten Zeitraum höher als die Nennleistung der Elektrolyseeinheit, kann die zur Verfügung stehende Arbeit nicht vollständig in die Wasserstoffproduktion umgesetzt werden. Die Anlage wird in diesem Zeitraum unter Volllast betrieben und die Wasserstoffproduktion befindet sich am Maximum. Tritt der Fall ein, dass das Eingangssignal für einen betrachteten Zeitraum unterhalb der maximalen Leistung der Elektrolyseanlage liegt, so wird die zur Verfügung stehende Arbeit vollständig umgesetzt, die Anlage jedoch lediglich im Teillastbereich betrieben. Gleichung(\ref{eq:eta_util}) beschreibt das Verhältnis von genutzter Arbeit zu maximal verfügbarer Arbeit und somit die Ausnutzung der verfügbaren Energie.
			
			\begin{equation}
				\eta_{util}=\frac{W_{act}}{W_{in}}
				\label{eq:eta_util}
			\end{equation}
			
			\begin{table}[H]
				\begin{tabular*}{\textwidth}{lll}\\
					$\eta_{util}$&Nutzungsgrad der verfügbaren Energiemenge&$kW/h$\\
					$W_{act}$&Verrichtetete Arbeit der Elektrolyseanlage&$kW/h$\\
					$W_{in}$&Eingangs&$kW/h$\\
					
				\end{tabular*}
			\end{table}

			\begin{table}[H]
				\begin{tabular*}{\textwidth}{lll}\\
					$\eta_{util}$&Nutzungsgrad der verfügbaren Energiemenge&\\
					$W_{act}$&Gesamte Arbeit der Elektrolyse-Anlage&$kW/h$\\
					$W_{in}$&Zur Verfügung stehende elektrische Arbeit&$kW/h$\\
				\end{tabular*}
			\end{table}
		
		\subsubsection*{Systemwirkungsgrad}
		Eine weitere Möglichkeit eine Elektrolyse-Anlage zu bewerten ergibt sich durch Einbeziehen des tatsächlichen Stromverbrauchs sämtlicher Verbraucher, wie zum Beispiel Pumpen, Verdichter, Speisewasseraufbereitung oder Heizung. Je nachdem wie das Wasser zugeführt und wie der Wasserstoff weiter verwertet wird, muss entweder Heizwert oder Brennwert des Wasserstoffs verwendet werden.\cite{Smolinka.05.07.2011} Will man den Gesamtwirkungsgrad der Wasserstoffumwandlung berechnen, muss der Wirkungsgrad des Elektrolyseurs auf den Heizwert ($LHV~=~3,00 ~kWh/m^3_N$) bezogen werden. Wird Wasser vom flüssigen in den gasförmigen Zustand überführt, muss der Brennwert ($HHV~=~3,54~ kWh/m^3_N$) zur Wirkungsgrad Berechnung von Wasserstoff verwendet werden. Der Energiegehalt des produzierten Wasserstoffs wird ins Verhältnis zu der Energiemenge aller elektrischen Verbraucher gesetzt, wodurch sich nach Gleichung(\ref{eq:eta_sys}) der Systemwirkungsgrad ergibt.\cite{Bayer.2000} 
		
		\begin{equation}
		\label{eq:eta_sys}
		\eta_{sys}=\frac{\dot{m}_{H_2}\cdot LHV}{W_{el}}
		\end{equation}
		
		\begin{table}[H]
			\begin{tabular*}{\textwidth}{lll}
				$\dot{m}_{H_2}$&Massenstrom des produzierten Wasserstoffs&$kg/s$\\
				$LHV$&Heizwert des Wasserstoffs&$J/mol^3$\\
				$W_{el}$&Zugeführte elektrische Arbeit&$Ws$\\
			\end{tabular*}
		\end{table}
%#############################################################	
		\subsubsection*{Spezifischer Energiebedarf}
		Die Energiemenge, die benötigt wird, um einen Normkubikmeter Wasserstoff herzustellen, kann mit Hilfe des spezifischen Energiebedarfes berechnet werden.\cite{Hey.26.Oktober2012} Werden die Annahmen getroffen, dass keine Nebenreaktionen auftreten und der eingesetzte Strom zu nahezu 100\% für die Elektrolyse verwendet wird, so kommt es zu einer hundertprozentigen Stoffumsetzung. Mit dieser Annahme und dem 2. Faraday’schen Gesetz (Gleichung(\ref{eq:Q})) lässt sich der spezifische Energiebedarf berechnen.
		
		\begin{equation}
		\label{eq:Q}
		Q=n\cdot z \cdot F
		\end{equation}
		
		\begin{table}[H]
			\begin{tabular*}{\textwidth}{lll}
				$Q$&Ladung&$C$\\
				$n$&Stoffmenge&$mol$\\
				$z$&Ladungszahl für die Wasserstoffreaktion & 2\\
				$F$&Faraday-Konstante&$C/mol$
			\end{tabular*}
		\end{table}
		
		Bezieht man die Ladung auf die Stoffmenge von 1 mol Wasserstoff und multipliziert sie mit der realen Zellspannung, erhält man den spezifischen Energieverbrauch $e_{spez}$ (Gleichung(\ref{eq:e_spez1})).\cite{Bayer.2000}
		
		\begin{equation}
		\label{eq:e_spez1}
		e_{spez} = \frac{Q}{n}\cdot U_{Zell}=z\cdot F\cdot U_{Zell}=2\cdot 96485 \frac{C}{mol}\cdot U_{Zell} = 53,603 \dfrac{Ah}{mol}\cdot U_{Zell}
		\end{equation}
			
			Da der spezifische Energieverbrauch üblicherweise auf $1 m^3_N$ (entspricht $44,6 mol$ Wasserstoff mit dem Normvolumen eines idealen Gases von $22,4 l/mol$) bezogen wird, ergibt sich die Formel zu Gleichung(\ref{eq:e_spez2}):
			
			\begin{equation}
			\label{eq:e_spez2}
			e_{spez} = 2392,98\frac{Ah}{m^3_N}\cdot U_{Zell}
			\end{equation}
			
			Wird für die Zellspannung die Thermoneutralspannung eingesetzt, erhält man als spezifischen Energiebedarf:
			
			\begin{equation}
			\label{eq:e_spez3}
			e_{spez} = 2392,98\frac{Ah}{m^3_N}\cdot 1,48V=3,54\frac{kWh}{m^3_{N,H_2}}
			\end{equation}
			
			Dieser Wert entspricht dem Brennwert von Wasserstoff.
			
			\subsubsection*{Wasserstoffgestehungskosten}
			Um einen Vergleichswert der jeweiligen Produktionskosten zur produzierten Wasserstoffmenge zu bekommen, werden mit Hilfe von Gleichung(\ref{eq:k_spez}) die spezifischen Kosten der Wasserstoffproduktion bestimmt. Sie können auch als Wasserstoffgestehungskosten bezeichnet werden.
			
			\begin{equation}
			\label{eq:k_spez}
			k_{spez} = \frac{\sum{k_i}}{V_{H_2}}
			\end{equation}
			
			\begin{table}[H]
				\begin{tabular*}{\textwidth}{lll}
					$\sum{k_i}$&Summe aller Kosten der Wasserstoffproduktion&$EUR$\\
					$V_{H_2}$&Gesamtvolumen des produzierten Wasserstoffs&$m^3_N$\\
				\end{tabular*}
			\end{table}
			
			Aus welchen Bestandteilen sich die Kosten für die Wasserstoffproduktion zusammensetzen wird in Abschnitt(\ref{subsub_Zellspann}) beschrieben.
			
			Als weitere Kennzahl dient die Auslastung $\alpha_{OP}$ der Anlage. Diese gibt an wie das Verhältnis von Betriebszeit $t_{OP}$ zum betrachteten Zeitraum ist. In dieser Arbeit bezieht sich der betrachtete Zeitraum $t_{ges}$ auf 8760 Stunden, was einem Jahr entspricht. Nach Gleichung(\ref{eq:alph_op}) berechnet sich die Auslastung wie folgt:
			
			\begin{equation}
			\label{eq:alph_op}
			\alpha_{OP} = \frac{t_{OP}}{t_{ges}}
			\end{equation}
			
			\subsubsection*{Die Leistung einer Elektrolyse-Zelle}
			Die Leistung einer Elektrolyse-Zelle berechnet sich nach der Gleichung(\ref{eq:P_zell})
			
			\begin{equation}
			\label{eq:P_zell}
			P_{Zell} = U_{Zell}\cdot I
			\end{equation}
			
			\begin{table}[H]
				\begin{tabular*}{\textwidth}{lll}
					$P_{Zell}$&elektrische Zellleistung&$W$\\
					$I$&Strom&$A$\\
				\end{tabular*}
			\end{table}
			
			
			Um verschiedene Elektrolyse-Stacks besser vergleichen zu können, werden häufig flächenspezifische Leistungswerte, wie die Leistungsdichte (Gleichung(\ref{eq:p_zell})), angegeben.\cite{Klein.2013} 
			
			\begin{equation}
			\label{eq:p_zell}
			p_{Zell} = U_{Zell}\cdot i
			\end{equation}
			
			\begin{table}[H]
				\begin{tabular*}{\textwidth}{lll}
					$p_{Zell}$&Leistungsdichte der Zelle&$W/m^2$\\
				\end{tabular*}
			\end{table}
			
			Multipliziert man die Summe der Überspannungen mit dem Strom, so ergibt sich nach Gleichung(\ref{eq:P_v}) die Verlustleistung der Zelle, welche als Wärme in der Zelle anfällt. Aus Abschnitt(\ref{subsub_thermodynGrundL}) ist bekannt, dass es bei der Wasserspaltung zu einer Entropieerhöhung kommt, diese Erhöhung wird zum Teil durch die entstehende Wärme aus der Zelle ausgeglichen.
			
			\begin{equation}
			\label{eq:P_v}
			P_{verl} = \sum{\eta}\cdot I
			\end{equation}
			
			\begin{table}[H]
				\begin{tabular*}{\textwidth}{lll}
					$P_{verl}$&Verlustleistung&$W$\\
					$\eta$&Überspannung&$V$\\
				\end{tabular*}
			\end{table}
			
			
			\subsubsection*{Ökonomische Kennwerte}
			Anhand einer späteren ökonomischen Betrachtung der einzelnen Elektrolyseverfahren, bezogen auf die verschiedenen Eingangssignale, soll eine Aussage darüber getroffen werden, ob die dynamische Betriebsweise eine sinnvolle Alternative darstellt. Gleichzeitig soll sie aufzeigen, welcher Elektrolyseurtyp sich am ehesten für ein solches Betriebsmodell eignet und in welchem Kostenbereich sich die Anlage befindet. Für die Signale EIS, RES und WEA wird ein fixer Strompreis von $k_{el}= 0,07~EUR/kWh$ angenommen.\cite{DeutscheEnergieAgenturGmbH.06.2018} Im Fall des Strompreissignals wird eine Preisgrenze von $40~EUR/MWh$ angenommen. Bei unterschreiten des Strompreissignals wird Wasserstoff produziert. Anhand der produzierten Wasserstoffmenge und der sich ergebenen Betriebszeit und Betriebskosten, kann anschließend ein Verkaufspreis für Wasserstoff berechnet und die Konkurrenzfähigkeit am Markt beurteilt werden. Die ökonomische Betrachtung wird anhand der folgenden Daten durchgeführt:
			\begin{itemize}
				\item Kapitalkosten
				\item Wartungs- und Betriebskosten
				\item Sonstige Kosten
				\item Wasserstoffproduktionskosten
			\end{itemize}
	
			\subsubsection*{Kapitalkosten}
			Als Kapitalkosten werden die Kosten betrachtet, die durch die Anschaffung der Elektrolyse-Anlage als jährlich wiederkehrende Kosten über einen gewählten Abschreibungszeitraum anfallen. Die Kapitalkosten errechnen sich dann aus den Investitionskosten für die Anlage und einem Annuitätenfaktor. Auf der Basis des gewählten Simulationszeitraums von einem Jahr, wird für die ökonomische Betrachtung der Investitionskosten die Annuitätenmethode gewählt. Diese Methode ist ein Verfahren mit welchem festgestellt werden kann was eine Investition, in einem betrachteten Zeitraum, kostet und erwirtschaftet. Dadurch sind die Investitionskosten gleichmäßig über den Abschreibungszeitraum verteilt. Typische Abschreibungszeiträume oder auch Nutzungsdauer genannt, sind bei Elektrolyseuren $\tau = 20-30 Jahre$.\cite{Missal.12.03.2014} Die Investitionskosten berechnen sich aus den spezifischen Investitionskosten und der Nennleistung der geplanten Anlage nach Gleichung(\ref{eq:P_verl}). 
			
			\begin{equation}
			\label{eq:P_verl}
				K_{inv} = e_{spec} \cdot P_N
			\end{equation}
			
			\begin{table}[H]
				\begin{tabular*}{\textwidth}{lll}
					$K_{inv}$&Investitionskosten&$EUR$\\
					$e_{spez}$&spezifische Investitionskosten&$EUR/kW$\\
					$P_N$&Nennleistung&$kW$\\
				\end{tabular*}
			\end{table}
			
			In Tabelle(\ref{tbl:spez_Kost}) sind die aus \cite{Buttler.2018} spezifischen Investitions- und Wartungskosten für die Elektrolyseurtypen AEL, PEM und SOEL aufgeführt und die Nennleistung ergibt sich aus dem jeweiligen Eingangssignal und der damit verbundenen Größe.
				
			\begin{table}[H]
				
				\caption{Spez. Investitions-, Wartungs-, und Betriebskosten für die \gls{ael},\gls{pem}-\gls{el}, \gls{soel}}
					\begin{tabular*}{\textwidth}{lll|l|l}
						
						&& {\textbf{\gls{ael}}}& {\textbf{\gls{pem}-\gls{el}}} & {\textbf{\gls{soel}}} \\ 
						\hline\hline
						&&&&\\
						spez. Investitionskosten& $EUR/MWh$ &$800-1500$&$1400-2100$&$>2000$\\
						&&&&\\
						Wartungs u. Betriebskosten& $\%~von~k_{inv}$ &$2-3$&$3-5$&$n.a.$\\
					\end{tabular*}
					\label{tab:spez_Kost}
				\end{table}
					
				Der für die Investitionskosten benötigte Annuitätsfaktor lässt sich mit Gleichung(\ref{eq:Ann}) berechnen.\cite{Gluck.}
				
\begin{equation}
\label{eq:Ann}
	A_F=\frac{\frac{z}{100\%}\cdot\left(1+\frac{z}{100\%}\right)^{\tau}}{left(+\frac{z}{100\%}right)^{\tau}}1
\end{equation}
		
				\begin{table}[H]
					\begin{tabular*}{\textwidth}{lll}
						$A_F$&Annuitaetenfaktor&$1/a$\\
						$z$&Zinssatz&$\%$\\
						$\tau$&Abschreibungszeitraum&$a$\\
						
					\end{tabular*}
				\end{table} 
	
				Werden die spezifischen Investitionskosten mit dem Annuitätsfaktor multipliziert, lassen sich die spezifischen jährlichen Kapitalkosten $k_{inv}$ berechnen (Gleichung(\ref{eq:k_inv})).\cite{Hey.26.Oktober2012}
				
				\begin{equation}
				\label{eq:k_inv}
				k_{inv} = e_{spez} \cdot A_F
				\end{equation}
				
				\begin{table}[H]
					\begin{tabular*}{\textwidth}{lll}
						$k_{inv}$&jaehrliche Kapitalkosten&$EUR$\\
						$e_{spez}$&spezifische Investitionskosten&$EUR/kW$\\
						$A_F$&Annuitaetenfaktor&$1/a$\\
					\end{tabular*}
				\end{table}
					
				\subsubsection*{Wartungs- und Betriebskosten}
				Eine übliche Art die Kosten für Wartung und Betrieb anzugeben ist es, sie als jährlichen Anteil der Investitionskosten anzunehmen. Aus Tabelle(\ref{tab:spez_Kost}) ist ersichtlich, dass die Kosten für Wartung und Betrieb im Bereich von 2 – 5\% in Abhängigkeit des Anlagentyps liegen. Die sich daraus ergebenen jährlichen Kosten berechnen sich nach Gleichung(\ref{eq:k_WartBetr}).\cite{Hey.26.Oktober2012}
				
				\begin{equation}
				\label{eq:k_WartBetr}
				k_{WB} = \frac{4\%/a}{100\%}\cdot k_{inv}
				\end{equation}
				
				\begin{table}[H]
					\begin{tabular*}{\textwidth}{lll}
						$k_{WB}$&Jährliche Kosten für Wartung und Betrieb&	$EUR/a$\\
						$f_{WB}$&	Anteil der jährlichen Investitionskosten&$\%$\\
					\end{tabular*}
				\end{table}
							
				\subsubsection*{Sonstige Kosten}
				Unter sonstigen Kosten werden Kosten für die elektrische Energie und das verbrauchte Wasser aufgeführt. Die elektrischen Kosten ergeben sich aus der Menge der bezogenen elektrischen Arbeit und dem Preis (Gleichung(\ref{eq:k_el})). Als Preis wurde der von der \cite{DeutscheEnergieAgenturGmbH.06.2018} ermittelte Strompreis von $p_{el} = 14,5 ct/kWh$gewählt. Analog dazu ergeben sich die Wasserkosten aus der Menge des für die Elektrolyse verwendeten Wassers und dem dafür gewählten Preis (Gleichung(\ref{eq:k_h2o})). Der Preis pro verbrauchtem Kubikmeter Wasser wurde von \cite{Hey.26.Oktober2012} gewählt und liegt bei $p_{H_2O} = 1,5 EUR/m^3$. 
				
				\begin{equation}
				\label{eq:k_el}
				k_{el} = p_{el}\cdot W_{el}
				\end{equation}
				
				\begin{table}[H]
					\begin{tabular*}{\textwidth}{lll}
						$k_{el}$&Kosten für elektrische Energie&$EUR$\\
						$p_{el}$&Strompreis&$EUR/kWh$\\
						$W_{el}$&Elektrische Arbeit&$kWh$\\
					\end{tabular*}
				\end{table}
				
				\begin{equation}
				\label{eq:k_h2o}
				k_{H_2O} = p_{H_2O} \cdot V_{H_2O}
				\end{equation}
				
				\begin{table}[H]
					\begin{tabular*}{\textwidth}{lll}
					$k_{H_2O}$&	Wasserkosten	&$EUR$\\
					$p_{H_2O}$&Wasserpreis		&$EUR/m^3$\\
					$V_{H_2O}$&Volumen des verbrauchten Wassers&$m^3$\\
				\end{tabular*}
			\end{table}
			
			\subsubsection*{Wasserstoffproduktionskosten}
			Aus den oben aufgeführten Kosten lassen sich die Kosten für die Wasserstoffproduktion $k_{H_2}$ berechnen. Diese setzen sich aus der Summe der Kapital, Wartungs-, Betriebs- und den sonstigen Kosten nach Gleichung (2.37) zusammen.
			
			\begin{equation}
			%\label{eq:k_h2}
			k_{H_2} = k_{inv}+k_{WB}+k_{el}+k_{H_2O}
			\end{equation}
			
			%$k_(H_2 )= k_inv+k_WB+k_el+k_(H_2O)$
			
			\subsubsection*{Alkalische Elektrolyse (\gls{ael})}	
			Der Prinzipielle Aufbau einer alkalischen Elektrolysezelle ist in Abbildung(\ref{fig:schem_EL}) dargestellt.
			
			\begin{figure}[H]	
				\centering
				\includegraphics[height=4cm]{/home/dafu/Schreibtisch/Master-Projekt/Doku/Abb/sc_Schmidt2017_EL-Rech.png}
				\caption[Schematische Darstellung Elektrolyse-Technologien]{Schematische Darstellung der drei betrachteten Elektrolyse-Technologien}
				\label{fig:schem_EL}
			\end{figure}
			
			Eine AEL besteht aus zwei Elektroden, Anode und Kathode, gleicher oder unterschiedlicher Metalle z.B. Nickel für die Anode und Kathode oder Nickel/Nickelsulfid.\cite{Kurzweil.2015} Diese werden in eine alkalische Lösung aus Kalium- oder Natriumhydroxid getaucht und durch einen Separator aus z.B. Zirfon getrennt. Typischer weise werden Massengehalte im Bereich von 30 – 40 \% gewählt \cite{Ghaib.2017} um eine erhöhte elektrische Leitfähigkeit zu erhalten. Wie Tabelle 1 zeigt, werden an der Anode Hydroxidionen zu Sauerstoff und Wasser oxidiert. Durch eine externe Stromquelle fließen die Elektronen zur Kathode, wo sie mit Wasser zu Wasserstoff und Hydroxidionen reagieren. Durch die umgesetzten Anionen an der Anode entsteht ein Konzentrationsgradient, aufgrund dessen die Hydroxidionen durch den Separator permeieren.
			Alkalische Elektrolyseure arbeiten in einem Temperaturbereich von $50–80 ^\circ C$, bei Drücken von $1–150 bar$ , Stromdichten von $0,2 - 0,45 A/cm^2$, Zellspannungen von $1,8~–~2,4~ V$ \cite{Kurzweil.2015} und mit Zellflächen $< 3,6 m^2$\cite{Buttler.2018} . 
			
			\subsubsection*{Kennzahlen der \gls{ael}}	
			Um große Mengen an Wasserstoff produzieren zu können werden alkalische Elektrolyseure aus mehreren parallel geschalteten Modulen zusammengebaut. Besteht der Elektrolyseur aus bipolaren Zellen, des Filterpressentyp, so sind mehrere hintereinander angeordnete Elektroden, deren Vorderseite als Anode und Rückseite als Kathode arbeiten, mit einander verschaltet.\cite{Kurzweil.2015} Dadurch wird jede Zelle mit eigenem Elektrolyten durchflossen und eigener Zellspannung betrieben. Jedes Modul / jeder Stack kann für sich eine Leistungsaufnahme im Bereich von wenigen Kilowatt bis hin zu einigen MW aufnehmen.\cite{Smolinka.05.07.2011}  Die Wasserstoffproduktionsrate einer Zelle liegt typischer Weise in einem Bereich von $1.000–4.000 m^3/h$und die einer Anlage bei ca. $200.000 m^3/h$.\cite{Kurzweil.2015} Der spezifische Energieverbrauch einer Elektrolyse-Zelle wird in \cite{Kurzweil.2015} mit $4,3-4,6 kWh/m^3$ und in \cite{Buttler.2018} mit $5,0–5,9 kWh/m^3$ angegeben, mit einer Systemeffektivität von $51–60\%$. Die Lebensdauer von ca. $55.000–120.000h$ (ca. 6 – 13 Jahre) entspricht einer Degradationsrate von etwa $2–4 \mu V/h$ entspricht.\cite{Buttler.2018} Für die ökonomische Betrachtung des Elektrolyseprozesses werden aus \cite{Buttler.2018} die Investmentkosten von $800 – 1500 EUR/kW$ und Wartungs- und Betriebskosten in Höhe von $2 – 3 \%$ der Investmentkosten pro Jahr übernommen. 
			\subsubsection*{Dynamisches Verhalten der \gls{ael}}	
			Ein dynamisches Leistungseingangssignal, durch welches der Elektrolyseur betrieben wird, fordert ein gutes dynamisches Lastverhalten der gesamten Anlage, da periodisch ein Wechsel zwischen Volllast und Teillast oder unterschiedlichen Teillaststufen stattfindet. Eine \gls{ael}-Elektrolyse-Einheit kann laut \cite{Smolinka.05.07.2011} einen Teillastbereich von $20–40 \%$ fahren. Dieser Bereich wirkt sich jedoch negativ auf die Produktgasqualität des produzierten Wasserstoffs aus. Die Hauptursache einer schlechteren Qualität liegt darin, dass sich aufgrund der verringerten Produktgasmenge vermehrt Fremdgase im Elektrolyten anreichern. Dieser Effekt wird bei höheren Drücken noch verstärkt, da die Löslichkeit von Gasen in Flüssigkeiten mit steigendem Druck zunimmt. Weiterhin beeinflussen die Komponenten des Gesamtsystems die dynamische Betriebsweise kritisch. Während die Elektrochemie quasi verzögerungsfrei auf Lastwechsel reagieren kann, reagieren die für die Elektrolyse zusätzlich benötigten Komponenten, wie Druckregler, Laugenpumpe oder Produktgas-Separator eher träge. Hier ist auch die Wärmekapazität des Systems von Bedeutung, da bei höheren Stromdichten größere Überspannungen auftreten, durch die mehr Wärme ins System eingebracht wird. Um die Elektrolyse bei gleichbleibenden Temperatur zu betreiben muss der zusätzliche Wärmestrom schnell aus dem System abgeführt werden. 
			
			\subsubsection*{PEM Elekrolyse \gls{pem}-\gls{el}}	
			Eine weitere Möglichkeit Wasserstoff mittels Elektrolyse herzustellen bietet die PEM-Elektrolyse mit welcher Wasserstoff aus reinem Wasser hergestellt werden kann. Der Prinzipielle Aufbau einer PEM-Elektrolysezelle ist in Abbildung(\ref{fig:schem_EL}) dargestellt.
			ABBILDUNG
			Die PEM-Elektrolysezelle besteht aus einer Anode, an der die Sauerstoffproduktion stattfindet und einer Kathode, an der die Wasserstoffproduktion stattfindet.\cite{Smolinka.05.07.2011} Beide Elektroden sind durch eine saure Protonenaustauschmembran, wie z.B. Nafion, voneinander getrennt (PEM, engl.: proton exchange membrane). Das Wasser wird an der Anodenseite zugeführt, wo es in Protonen und Sauerstoff gespaltet wird.\cite{Kurzweil.2015} Die Protonen durchqueren die Membran und wandern zur Kathode, wo sie mit H+ zu Wasserstoff reagieren.
			Typische Betriebsparameter von PEM-Elektrolyseuren sind Temperaturbereiche von $50–80 ^\circ C$, Stromdichten zwischen $0,5–3 A/cm^2$ \cite{EspinosaLopez.2018}, Spannungen im Bereich von $1,7–2,1 V$ \cite{Kurzweil.2015}, Drücken bis ca. $50 bar$ und mit einer Zellfläche $<130 m^2$ \cite{Buttler.2018}. 
			
			\subsubsection*{Kennzahlen der \gls{pem}-\gls{el}}	
			Ähnlich wie bei der \gls{ael} ist die \gls{pem} aus bipolaren Stacks vom Filterpresstyp aufgebaut. Sie bestehen aus bis zu $200$ hintereinander geschalteten Elektroden-Elektrolyt Einheiten mit einer Fläche von bis zu $2500 cm^2$.\cite{Kurzweil.2015} PEM Elektrolyse sind eher im Bereich der Kleinanlagen zu finden mit einer elektrischen Leistungsaufnahme bis ca. $150 kW$ pro Modul \cite{Smolinka.05.07.2011} und einer Wasserstoffproduktionsrate von $20 m^3/h$.\cite{Kurzweil.2015} Nach \cite{Buttler.2018} können Anlagen jedoch eine Größe von bis zu $2 MW$ pro Stack erreichen mit einer Wasserstoffproduktionsrate von bis zu $400 m^3_N/h$ pro Stack. 
			Der spezifische Energieverbrauch einer Elektrolyse-Zelle wird in \cite{Kurzweil.2015} mit$ 1,9–4,0 kWh/m^3$ und in \cite{Buttler.2018} mit $5,0–6,5 kWh/m^3$ angegeben mit einer Systemeffizienz im Bereich von $46–60\%.$ Die Lebensdauer der PEM – Elektrolyse reicht von $60.000–100.000h$ (ca. 7 – 11 Jahre), was einer Degradationsrate von $4-8 \mu V/h$ entspricht. Für die ökonomische Betrachtung des Elektrolyse-Prozesses werden aus \cite{Buttler.2018} die Investmentkosten von $1400-2100 EUR/kW$ und Wartungs- und Betriebskosten in Höhe von $3–5 \%$ der Investmentkosten pro Jahr angenommen.
			
			\subsubsection*{Dynamisches Lastverhalten der \gls{pem}-\gls{el}}	
			Der dynamische Betrieb einer \gls{pem} Elektrolyse ist im Vergleich zu der \gls{ael} etwas besser und zeichnet sich zum einen durch einen größeren Teillastbereich aus. Als untere Grenze wird in \cite{Smolinka.05.07.2011} ein Wert von 0\% für die Zell- bzw. Stackebene angegeben. Das bedeutet, dass auch bei kleinsten Teillast kein kritisches Niveau der Fremdgaskonzentration erreicht wird. Als technisch sinnvolle Teillastgrenze wird ein Wert von 5\% der Nennleistung angegeben. Ähnlich wie bei der AEL kommt es in der PEM zu Diffusionsströmen von Wasserstoff durch die Membran, welche durch die Partialdruckdifferenz, von Kathode zu Anode und die Temperatur angetrieben ist. Ein Problem des Teillastbetriebs ist, dass dieser keinen Einfluss auf die Wasserstoffdiffusionsrate hat. Obwohl die Produzierte Menge an Wasserstoff sinkt, bleibt die Permeationsrate konstant, wodurch der Wasserstoffanteil in Sauerstoff ansteigt. Mit Erhöhung des Drucks steigt die Permeation linear an. 
			Zum anderen weist die Systemperipherie wie z.B. Zirkulationspumpe oder Flüssiggasseparator jedoch eine geringere thermische Kapazität auf, wodurch dem dynamischen Leistungssignal gut gefolgt werden kann. 
			
			\subsubsection*{Hochtemperaturelektrolyse (\gls{soel})}	
			Verglichen mit \gls{ael} und\gls{pem}-\gls{el} Technologien wird die SOEL bei sehr viel höheren Temperaturen im Bereich von $700–1.000 ^\circ C$ \cite{Zhang.2010} betrieben. Der Vorteil, der sich daraus ergibt, ist, dass mit steigender Temperatur die reversible Zellspannung abnimmt. Der Prinzipielle Aufbau einer \gls{soel}, auch Dampfelektrolyse mit Festelektrolyten (SOEL, engl.: solid oxide electrolysis) genannt, ist in Abbildung(\ref{fig:schem_EL}) beschrieben.
			
			Ebenso wie bei der \gls{pem} sind Anode und Kathode durch eine Membran getrennt. Im Fall der SOEL besteht die Membran aus einem Festelektrolyten wie z.B. Yttriumstabilisiertes Zirconiumdioxid.\cite{Ghaib.2017} Wasser wird an der Kathode zugeführt und durch Aufnahme von zwei Elektronen zu Wasserstoff und $O^{2-}$ zersetzt. Das $O^{2-}$ diffundiert durch den Elektrolyten zur Anode, wo es zu Sauerstoff oxidiert wird.
			Aus der Literatur bekannte Betriebsparameter der \gls{soel} sind neben den oben erwähnten Temperaturen Stromdichten zwischen $0,1–1,0 A/cm^2$ \cite{Buttler.2018}, Zellspannungen im Bereich von $0,95–1,3 V$ \cite{Ghaib.2017} , Drücken bis hin zu $15 bar$ \cite{Buttler.2018} und einer Zellfläche von $<0,06 m^2$.
			
			
			\subsubsection*{Kennzahlen der (\gls{soel})}	
			Neben den bipolar-plattenförmigen Zellen existieren für die SOEL noch röhrenförmige Elektrolysezellen. Die Plattenförmigen Elektrolysezellen lassen sich leicht herstellen und weisen eine vorteilhafte Leistungsdichte auf, haben jedoch das Problem, dass sie schwer abzudichten sind, wodurch eine unerwünschte Gasdurchlässigkeit entsteht. Dieses Problem besteht bei den röhrenförmigen nur im geringen Maß, da der Festelektrolyte direkt mit dem keramischen Elektrodenmaterial beschichtet wird. Dafür besitzen die Röhren eine geringere Leistungsdichte als die planaren Zellen. 
			Zurzeit befindet sich die SOEL eher im Bereich aktueller Forschungen, wodurch es kaum kommerziell betriebene Anlagen gibt. Lediglich die Sunfire GmbH hat eine Anlage mit $150kW$ und einer Produktionsrate von ca. $40 m^3/h$ entwickelt.\cite{Sunfire.2017} Nach \cite{Buttler.2018} sind Anlagen in der Größe $<0,01 MW$ pro Stack mit einer Wasserstoffproduktionsrate von bis zu $<10 m^3_N/h$ pro Stack. Der spezifische Energieverbrauch einer Elektrolysezelle wird in \cite{Buttler.2018} mit $3,7–3,9 kWh/m^3$ angegeben mit einer Systemeffizienz im Bereich von $76–81\%$. Die Lebensdauer der SOEL – Elektrolyse reicht von $8.000–20.000 h$ (ca. 1 – 2,5 Jahre) \cite{DeNiangThe.12.2015}, was einer Degradationsrate von  $21 \mu V/h$ entspricht. Für die ökonomische Betrachtung des Elektrolyseprozesses werden aus \cite{Buttler.2018}  Investmentkosten von $>2500 EUR/kW$ angenommen. Da für die Wartungs- und Betriebskosten verhältnismäßig wenige Referenzen aufzuweisen sind, wurden $5 \%$ der Investmentkosten pro Jahr angenommen.
			
			\subsubsection*{Dynamisches Lastverhalten der (\gls{soel})}	
			Durch die hohe Betriebstemperatur der \gls{soel} führen Laständerungen automatisch zu einer Temperaturänderung im Stack und dadurch bedingt kann es zu feinen Rissen in der keramischen Membran kommen, was einen direkten Einfluss auf die Lebensdauer des Stacks hat. Dies hat zur Folge, dass die \gls{soel}, obwohl sie theoretisch den elektrischen Laständerungen schnell folgen kann, durch die mechanischen, Lebensdauer reduzierenden Probleme eingeschränkt ist. Eine weite Schwierigkeit bei einem dynamischen Leistungssignal ist die lange Zeitspanne bis zum Erreichen der Betriebstemperatur.


%Innerhalb dieser Arbeit werden 3 Technologie-Typen der Wasser-Elektrolyse unterschieden bzw. betrachtet:
%\begin{itemize}
%	\item Alkalische Elektrolyse (AEL)
%	\item[]ebenfalls abgekürzt als \glqq AEC \grqq $\rightarrow$ alkaline electrolyzer cell
%	\item Polymer-Elektrolyt-Elektrolyse ??? (PEM-EL)
%	\item[]ebenfalls abgekürzt als \glqq PEMEC \grqq $\rightarrow$ polymere electrolyte membrane electrolyzer cell \textit{oder} protone exchange membrane electrolyzer cell
%	\item Hochtemperatur Elektrolyse (SOEL)
%	\item[]ebenfalls abgekürzt als \glqq SOEC \grqq $\rightarrow$ solide oxide electrolyzer cell
%	
%\end{itemize}





%%\subsection{PEM - EL}
%%MYRTE???
%%\subsection{Grundsätzliches}
%- hohe Temperaturen -> niedrigere Spannung an Zelle anzulegen
%aber: hohe Temp. -> stärkere Degradation
%- deshalb lieber hohe Drücke aber: -> Sicherheitsaspekte! H2-permeation -> Konz in O2
%[Schalenbach 2013]
%
%
%\subsection{Teilreaktionen}
%-->> Tabelle nach DLR (QUELLE???)
%\begin{figure}
%	\label{EL-Teilreakt}
%	\includegraphics[width=\textwidth]{Abb/sc_Schmidt2017_EL-Rech.png}
%	\caption[Schematische Darstellung Elektrolyse-TEchnologiearten]{Schematische Darstellung der 3 betrachteten Elektrolysetypen; aus \cite{Schmidt.2017}}
%\end{figure}
%
%\subsubsection{PEM}
%Anoden-Reaktion:
%\begin{equation}\label{Teilreakt_Anode_PEM}
%H_2O \Rightarrow \frac{1}{2} O_2 + 2H^+ +2e^-
%\end{equation}
%Kathoden-Reaktion:
%\begin{equation}\label{Teilreakt-Kathode_PEM}
%2H^+ 2e^- \Rightarrow H_2
%\end{equation}
%
%\subsubsection{SOEL}
%Anoden-Reaktion:
%\begin{equation}\label{Teilreakt_Anode_SOEL}
%H_2O \Rightarrow \frac{1}{2} O_2 + 2H^+ +2e^-
%\end{equation}
%Kathoden-Reaktion:
%\begin{equation}\label{Teilreakt-Kathode_SOEL}
%2H^+ 2e^- \Rightarrow H_2
%\end{equation}
%
%\subsubsection{weitere (hier nicht betrachtete) EL-Typen}
%\begin{itemize}
%	\item AEM (?) % % Anion exchange
%	\item membraneless EL \cite{Esposito2017} %(Esposito 2017, Joule)
%	\item co-electrolysis
%	\item high temp. PEM
%\end{itemize}
%
%
%
%\subsubsection{Aufbau und Funktion}
%
%
%-> Temperaturkontrolle:
%>> Aufbau Lettenmeier (Diss):
%-Temp.-Sensor im Anodenseitigen Wasseraustritts-Strom
%%-$60°C$ max (Modell bedingt)
%-Aufheizen nur über Verluste (Überspannung)
%
%---Ionentauscher...?
%
%\subsubsection{Stand der Technik}
%\label{SdT}
%PEMEL , PEMWE, PEMEC, ...
%
%\subsubsection{AEL}
%zu AEL: Hammoudi + HAug !!!!!
%
%\subsubsection{Überspannung}
%
%\subsubsection*{Diffusionsüberspannung}
%Hamann: Grundlagen der Kinetik
%-> 
%-> 
%-> S. 246
%-->> siehe Bemerkung S. 248!!
%
%
%\subsubsection{Performance - Kriterien}
%\label{peformance-krit}
%https://www.sciencedirect.com/science/article/pii/S0360319917336868
%
%https://ieeexplore.ieee.org/abstract/document/8494523
% 
%siehe Abschnitt(\ref{Bewertungs-Krit})!!!
%\begin{itemize}
%	\item Reaktionszeit
%	\item Produktgasqualität
%	\item Effizienz / spez. Energiebedarf
%\end{itemize}
%
%
%\subsubsection{Marktsituation}
%
%
%\subsection{Degradation}
%Wie jedes real betriebene System unterliegen auch Elektrolyseure unterschiedlichen Verschleißmechanismen, welche Gegenstand der aktuellen Forschung sind.\\
%Die Einwirkung auf sämtliche Komponenten der drei betrachteten EL(am Anfang Abkürzen!!!???)-Typen wird von unterschiedlichen Vorgängen beeinflusst.\\
%Unterschiedliche Auswirkungen: Materialabbau, Widerstand-Zunahme, Überspannungs-Zunahme...
%
%Beeinflussung von Performance-Kriterien von Interesse. % Verweis auf
%Abschnitt(\ref{peformance-krit})
%
%\subsubsection{Elektroden}
%
%\subsubsection{Membran-Dregadation}
%% % paper Lettenmeier2016
%
%PEM:
%% % bei Chandesris nur teilw. Temperatrurabhängigkeit!
%
%
%\cite{wu2008review}
%-> WU - PEM fuel cell degradation: Tab2: degr. rates in  lit.
%
%\cite{Rakousky2016}
%!!! -> verringerung von $i_0$ auf (!) 37 \% des Ursprungswertes
%%-> Rakousky: vA durch MEA-assembly beeinflusst (schwierig modellierbar
%% NAfion 117 Katalysatoren: Anode: IrO2, TiO2 (Ir-Loading of $2,25 mg/cm^2$)
%%Kathode: Pt/C (Pt-Loading of 0.8 g/cm2)
%%cath.- / an.-PTLs (?)
%%A=17.64
%
%%Test über 1009 h
%%high purity feed water $18.2 M Ohm/cm (?), 25 ml/min, 75 C$
%%cell-temp.: const. @ 80 grad C
%
%%niedrige Stromdichte $(1 A/cm^2 // 1,70 V)$: quasi keine Degr.
%%ab $1.84 V // 2 A/cm^2$ --Spannung UND Stromdichte ausschlaggebend
%%Verweis -> Ayers_2015 -->> hohe Zellspannung vermutlich nicht einziger Treiber f. Degr.
%
%%bei größerem i:
%%-zunehmend Gasblasenbildung -> mechaniche beanspruchung
%%- höhere Sauerstoff-Konz in Anoden-Kompartement (?) -> Korrosion/ Passivierung
%% größerer Wasserbedarf -> beschleunigte Ansammlung potentieller "contaminants" in CCM
%% potentielle Bildung von Hotspots bei mangelhaftem Wasser-Management
%
%%Stromdichte UND Schalt-Zeiten beeinflussen Degr
%%- Intervall-Länge beeinflusst maßgeblich Degr. (größere Intervalle -> geringere Degr)
%
%%2 unterschiedl. Degradations-Werte:
%%- durchschnittl. Degr.: ges. Spannungs-Steigerung über ges. Test-Zeit
%%- molar volt. degr. rate: durchschn. degr.-rate bzgl. 1 mol H2 prod.
%
%%degradation ist unabängig von prod. H2-menge
%
%%findings:
%%- konstanter Betrieb:
%%-> 2 A/cm2 -->>hohe Degr.
%%-> 1 A/cm2 -->> keine Degr.
%%--->>> lange Intervalle mit großen Stromdichten vermeiden!
%
%%- dynamischer Betrieb:
%%-degr.-raten (q 2 A/cm2) verringert, wenn zwischenzeitlich geringere Stromstärken vorliegen
%%-> besonders starke verringerung der Degr. bei vollst. Strom-Unterbrechung (effekt noch nicht vollständig verstanden)
%
%- Strom-Unterbrechungen sind vorsichtig einzusetzen:
%-> größere Perioden sinnvoller
%-> häufige Abschaltungen vermeiden
%
%-gemessen Halbzellen-Pot. ??? -> 3.2?!
%- Anoden-Degr. höher als KAth.
%
%- reversible degr. phenomenom stems from cathode side (für versch. zellen unterschiedlich...also unklar, wo)
%
%EIS:
%%Zellen mit großem Spannungs-Anstieg (hohe Stromdichte) weisen ebenfalls größere Zunahme von R_ohm auf
%%-Widerstandsverringerung durch CCM-topology changes
%%--- was ist R_PTL?
%
%-> hohe Stromdichte: Zunahme von R-ohm
%-> niedrige Stromdichte: R-PTL = const. -->> R-ohm nimmt ab
%-> im Fall hoher Stromdichten sorgen Unterbrechungen für geringere Zunahme von R-ohm (kann aktuell noch nicht erklärt werden)
%-------
%2 Paramter für Degr.-Effekt in Polarisationskurven-modell:
%-Austausch-Stromdichte (Abnahme: Gesamt-Verlust d. Elektroden-Performance) //keine An./Kath.-Unterscheidung
%- R-total -> ohmscher Gesamtwiderstand d. zelle
%
%in ersten 400h etwa 50\% Verringerung d. Austausch-Stromdichte (unabhängig von betriebsform) -> effektive betriebszeit intermittierend kürzer!
%
%R-total korreliert mit Spannungszunahme -> wichtiger Degr.-Parameter
%-> reduktion d Zellspannung auf 1.7 V (über 6 h) -> gleicher effekt, wie 10-minütige Abschaltung 
%
%- degradationswerte -> Berechnung??
%
%j-0 verringerung: vermutlich durch Ti-Ionen in Anode
%
%
%
%\begin{itemize}
%\item mechanisch
%
%\begin{itemize}
%\item Perforation
%\item Risse
%\item Poren
%\item Ursachen\\
%Durch Herstellungsprozess oder fehlerhafte Montage\\
%Interfaces: ungleichmäßige mechan. Beanspruchung\\
%low-humdification, low-humidification, relative humidity (?)\\
%\end{itemize}
%
%\end{itemize}
%
%-------------------Chandesris (2015):\\
%Regarding PEMWE, one of the most complete studies that
%gives evidence of membrane degradation was conducted at
%PSI in the 1990's [25] where substantial thinning of the mem-
%branes has been detected. Regarding the dissolution process,
%the ion exchange capacity measurements on thinned mem-
%branes reveal that the composition of the remaining polymer
%is not changed with respect to ionic groups: the degradation
%mechanism does not involve preferential attack on the ion
%exchange groups. Furthermore, complementary experiments
%indicated that the membrane degrading reaction can be
%localized on the cathode side of the cell. The interfaces with the anode as well as the bulk of the membrane were quite less
%affected [17,25]. In the present study, as reported in section
%!!!
%Single cell experimental set-up, we have observed a non
%negligible fluoride release only on the cathode side, which also
%support the hypothesis that membrane degrading reactions
%occur mainly at the cathode side for PEMWE.
%!!!
%
%( While performance can be tested quite rapidly,
%lifetime estimation is much more difficult to evaluate.
%Furthermore, in PEM water electrolyzer, degradation mecha-
%nisms occur very slowly with a typical characteristic time of
%thousands of hours, compared to hundreds of hours in PEMFC.)
%
%the PEMFC is not the exact opposite of
%an electrolysis cell. In PEMWEs, rutile oxydes like IrO 2 and
%RuO 2 are used as anode catalysts and the membrane is far
%thicker.
%The main sources of performance losses are
%related to catalysts and catalyst layers degradation, mem-
%brane degradation and bipolar plates and current collectors
%corrosion, however it is quite difficult to isolate the different
%degradation phenomena. Nevertheless, in PEMWEs, perfor-
%mance decreases and durability restrictions are mostly
%attributed to membrane pollution or degradation, reason why
%we will focus on this mechanism.
%
%Nevertheless, these data should not be taken into
%account as the explosivity limit has been reached. Indeed, an
%important criterion for the electrolyzer is the molar percent-
%age of H 2 in O 2 , as this mixture becomes explosive above 4% of
%H 2 in O 2 Ref. [39]. Fig. 15 plots the evolution of this quantity at
%333 and 353 K. An exponential increase is observed due to the
%coupling between the thinning of the membrane leading to
%the gas cross-over increase and the chemical degradation of
%the membrane. This trend is completely similar to the one
%observed experimentally by Inaba et al. [40] when performing
%accelerated stress test in PEMFC.
%
%Diffusion:
%First, as oxygen (resp. hydrogen) is present under gas form
%only at the anode (resp. cathode), a concentration gradient
%appears, once the gases are dissolved in the ionomer of the
%membrane. This concentration gradient induces an oxygen
%flux from the anode to the cathode and a hydrogen flux from
%the cathode to the anode via a diffusive process.
%
%Furthermore,
%water is crossing the membrane from the anode to the cath-
%ode due to both electro-osmotic phenomenon and water
%diffusion. Since part of the produced gas gets soluble in the
%water, it can be assumed that it is convectively carried
%through the membrane by this water flow.
%
%---params:\\
%The values for the solubility and diffusion coefficients are
%chosen among a large base of experimentally given laws
%[20,26,27].
%--->>>> Table 4 ( medium permeation behaviour)
%-----\\
%
%tion. The cathode side is at very low potential (< 0 V vs. SHE),
%as hydrogen evolution reaction occurs at the platinum elec-
%troactive sites. At potential lower than 0.4 V, ORR is considered
%to predominantly occurs via H 2 O 2 formation pathway [32] and
%one can neglect water recombination: (?????????)

%\subsection{Elektrische Energieversorgung und Energiemarkt}




%\section{Rahmenbedingungen}
%angewandter Betriebsbereich:
%-typische Stromdichten
%-Reaktionszeit-Zeit (Signal) -> Rampen
%- maximale Leistungsgradienten


%\subsection{Standort -> in betriebsarten aufführen} 


%\subsection{Eingangssignale}
\section{Betriebsarten}
\label{sect_Betriebsarten}
Im Folgenden wird die Generierung verschiedener Leistungszeitreihen für unterschiedliche Betriebsszenarien der betrachteten \gls{el}-Technologien beschrieben.

\subsection{Einspeisemanagement-Maßnahmen (EIS)}
Netzengpässe treten auf, wenn durch Stromflüsse, die sich aus den am Strommarkt vereinbarten Einspeisungen ergeben, die Strom- und Spannungsgrenzen der jeweiligen Netzebene  nicht eingehalten werden können und die Stabilität und Qualität (n–1-Sicherheit) der Stromnetze nicht gewährleistet werden kann. Einspeisemanagementmaßnahmen zählen neben Redispatch oder Countertrading zu Engpassmanagementmassnahmen, welche den Netzbetreibern zur Verfügung stehen um die Versorgungssicherheit trotz eines Engpasses sicherzustellen. Ferner schreibt \cite{Zapf.2017}: ''Eine speziell geregelte Netzsicherheitsmaßnahme besteht gegenüber Erneuerbare-Energien-, Grubengas- und KWK-Anlagen nach § 14 EEG 2014. Wenn die Netzkapazitäten nicht ausreichen und vorrangige Abregelungsmaßnahmen gegenüber konventionellen Erzeugern ausgeschöpft wurden, kann die bevorrechtigte Einspeisung vorübergehend abgeregelt werden.'' Im Falle einer Abregelung können mögliche Erzeugungsleistungen nicht genutzt werden, womit die Stromgestehungskosten der betroffenen EEG-Anlagen in die Höhe getrieben werden. Können Entschädigungsansprüche seitens der Anlagenbetreiber geltend gemacht werden , werden diese von den Netzbetreibern über das Netzentgelt auf alle Stromkonsumenten umgelegt.
\subsubsection{EIS-Datengrundlage}
Grundlage der Einspeisemanagement-Daten sind die Veröffentlichungen der jeweiligen ÜNB. EEG-Anlagenbetreiber bekommen so die Möglichkeit, eventuelle Entschädigungsansprüche geltend zu machen. Die Schleswig-Holstein-Netz AG veröffentlicht zu diesem Zwecke seit Januar 2015 alle bisher getätigten Eisman-Maßnahmen ihrer Netzregion. Anhand der folgenden Kriterien können diese gefiltert und als Datensatz bezogen werden: \cite{SchleswigHolstenNetzAG.31.Januar2019}

\begin{itemize} 
	\item Einsatz-ID 
	\item Start-Zeitpunkt (1-minütige Auflösung)
	\item End-Zeitpunkt (1-minütige Auflösung)
	\item Ort des Engpasses
	\item Stufe und
	\item EEG-Anlagenschlüssel
\end{itemize}

Anhand der Kriterien wurden die Einspeisemanagement-Maßnahmen für den Engpass-Ort ''Heide 110/20 Trafo 122'' für den Zeitraum Januar bis Dezember 2017 exportiert.\newline
Die Online-Informationsplattform der deutschen ÜNB \cite{TennetTSOGmbH50hertzTransmissionGmbHAmprionGmbHTransnetBWGmbH.31.Januar2019} dient der Informationsveröffentlichung allgemeiner netzrelevanter Daten. So werden auch jährlich EEG-Anlagenstammdaten aktualisiert. Anhand der folgenden Kriterien (exemplarisch) können diese gefiltert und zum Download zur Verfügung gestellt werden: %CITE: \cite{•}
\begin{itemize} 
	\item EEG-Anlagenschlüssel
	\item Installierte Anlagenleistung
	\item Energieträger
\end{itemize}

Die Datengrundlage der Einspeisemanagement-Maßnahmen wird anhand der Anlagenschlüssel um die Informationen der jeweiligen EEG-Anlage erweitert. Es ergibt sich für den entsprechenden Netzknoten ein Profil von 69 EEG-Anlagen, welche eine durchschnittlich installierte Leistung von 497 kW aufweisen.
\subsubsection{EIS-Datenaufbereitung}
Um die EISMAN-Maßnahmen als Leistungssignal für die EL-Betriebsmodelle nutzen zu können, ist die Bildung einer kontinuierlichen Zeitreihe notwendig. In ein-minütiger Auflösung werden die betroffenen Anlagenleistungen aufaddiert, wenn diese sich in jener Minute in einer EISMAN-Regelung befinden. Diese Iteration erfolgt für jede Minute der Zeitreihe, demzufolge vom $01. ~Januar ~2017 ~00:00 ~Uhr$ bis zum $31.~ Dezember~ 2017 ~23:59~ Uhr$. Weiterhin dient die Angabe der Regelungsstufe der Bestimmung der tatsächlich abgeregelten Anlagenleistung. \\
Oben beschriebenes Vorgehen ist in Abbildung(\ref{fig:EISMAN-Generierung}) schematisch dargestellt.
%GRAFIK und GRAFIK aus Zwischenpräsentation.
\begin{figure}[H]
	\centering
	\includegraphics[width=\textwidth]{/home/dafu/Schreibtisch/Master-Projekt/Doku/Abb/Modell/EISMAN_SCHEM_v01.png}
	\caption[Schema: EISMAN-ZEitreihen GEnerierung]{Schematisches Darstellung des Vorgehens zur Generierung der EISMAN-Leistungszeitreihe}
	\label{fig:EISMAN-Generierung}
\end{figure}

\subsubsection{EIS-Resultierende Zeitreihen}
%Abbildung: Rohdaten und JDL+Popt

Es ergibt sich eine EISMAN-Leistungszeitreihe für den Netzknoten \textsc{Heide 110/20 Trafo 122} in einer ein-minütigen  Auflösung für den Zeitraum Januar bis Dezember 2017. in Abbildung(\ref{fig:EIS_roh_betrART}) ist zu erkennen, dass gleichzeitig eine hohe Zahl an Erzeugungsanlagen gleichzeitig geregelt werden. Folglich entstehen hohe Leistungspeaks bis zu $25.000 ~kW$. Ebenso verdeutlicht die Jahresdauerlinie, dass nur an ca. $1000$ Stunden im Jahr überhaupt Regelungsmaßnahmen stattfinden.\\

\begin{figure}[H]
	\centering
	\begin{minipage}[t]{0.49\textwidth}
		\includegraphics[width=\textwidth]{/home/dafu/Schreibtisch/Master-Projekt/Doku/Abb/Graph/JDL/2019-03-03--10-37_EIS.pdf}
		
		\caption[Leistungszeitreihe der EISMAN-Betriebsart]{Leistungszeitreihe der EISMAN-Betriebsart; Summe der Einspeisemanagement-Maßnahmen für den Engpassort \textsc{Heide 110/20 Trafo 122} im Jahr 2017}
		\label{fig:EIS_roh_betrART} 
	\end{minipage}
	\hfill
	\begin{minipage}[t]{0.49\textwidth}
		\includegraphics[width=\textwidth]{/home/dafu/Schreibtisch/Master-Projekt/Doku/Abb/Graph/JDL/2019-03-03--10-37_EIS_JDL.pdf}
		\caption[Geordnete Leistungszeitreihe der EISMAN-Betriebsart]{Geordnete Leistungszeitreihe der EISMAN-Betriebsart; Absteigend geordnete Summe der Einspeisemanagement-Maßnahmen für den Engpassort \textsc{Heide 110/20 Trafo 122} im Jahr 2017}
		\label{fig:EIS_JDL_betrART} 
	\end{minipage}
\end{figure}

%\begin{figure}[H]
%	
%	\centering
%	\includegraphics[width=0.49\textwidth]{/home/dafu/Schreibtisch/Master-Projekt/Doku/Abb/test.png}
%	\includegraphics[width=0.49\textwidth]{/home/dafu/Schreibtisch/Master-Projekt/Doku/Abb/test.png}
%	\caption{ }
%	\label{fig:JDL_Eis} 
%\end{figure}

\subsection{Negative Residuallast (RES)}
Nach \cite{Schiffer.2019} ergibt sich die Residuallast aus der Differenz zwischen Stromverbrauch und nicht oder nur eingeschränkt regelbarer Einspeisung (Stromerzeugung) aus erneuerbaren Energien (z. B. Wind und Sonne). \\
Abweichend wird der Residuallast-Verlauf in dieser Modellierung anhand der Differenz zwischen Stromverbrauch und gesamter Stromerzeugung ermittelt. Positive Residuallast wird vor allem durch konventionelle Kraftwerke, z. B. durch Speicher- und Reservekraftwerke, gedeckt.\cite{Schiffer.2019} P2G-Anlagen hingegen können eingesetzt werden, um negative Residuallast zu kompensieren.
\subsubsection{RES-Datengrundlage}
Auf der von der Bundesnetzagentur (BNetzA) betriebenen Plattform \textsc{smard}
%Die Veröffentlichungen der Plattform ''smard.de'' der Bundesnetzagentur CITE ermöglichen 
die Ausgabe der Verläufe des Stromverbrauches sowie der Stromerzeugung in 15-minütiger Auflösung für die Regelzone des ÜNBs TenneT, jeweils in der Einheit $MW/h$. Der Residuallast-Verlauf wird für den Zeitraum zwischen dem 01. Januar 2019 und dem 31. Dezember 2017 erstellt.
\subsubsection{RES-Datenaufbereitung}
Es wird angenommen, dass verrichtete Erzeugung und Verbrauch (in $MW/h$) konstant für den jeweiligen 15-minütigen Zeitraum gelten - somit ist eine direkte Angabe der Residualleistung für Zeitpunkte in 15-minütiger Auflösung möglich. Weiterhin wird lediglich die negative Residualleistung betrachtet, in denen die Erzeugung den Verbrauch überwiegt.
Ohne weitere Skalierung sind die Leistungsbeträge der Residualleistungs-Zeitreihe im dreistelligen MW-Bereich, da zunächst die gesamte Regelzone berücksichtigt wird. Daher erfolgt eine Skalierung mit dem Faktor $1/4000$.
\subsubsection{RES-Resultierende Zeitreihe}
%Abbildung: Rohdaten und JDL+Popt

In Abbildung(\ref{fig:RES_roh_betrART}) ist die RESIDUAL-Leistungszeitreihe in 15-minütiger Auflösung für die Regelzone des ÜNB TenneT über den Zeitraum Januar bis Dezember 2017 dargestellt. Die entsprechende, geordnete Jahresdauerlinie (Abbildung(\ref{fig:RES_JDL_betrART})) weist eine maximale Residualleistung von  ca. 2.200 kW auf. Negative Residualleistungen treten lediglich an ca. 3.500 Stunden im Jahr 2017 auf.\\
\begin{figure}[H]
	\centering
	\begin{minipage}[t]{0.49\textwidth}
		\includegraphics[width=\textwidth]{/home/dafu/Schreibtisch/Master-Projekt/Doku/Abb/Graph/JDL/2019-03-03--10-38_RES.pdf}
		
		\caption[Leistungszeitreihe der RESIDUAL-Betriebsart]{Leistungszeitreihe der RESIDUAL-Betriebsart; Skalierter Residuallastverlauf der Tennet-Regelzone für das Jahr 2017}
		\label{fig:RES_roh_betrART} 
	\end{minipage}
	\hfill
	\begin{minipage}[t]{0.49\textwidth}
		\includegraphics[width=\textwidth]{/home/dafu/Schreibtisch/Master-Projekt/Doku/Abb/Graph/JDL/2019-03-03--10-40_RES_JDL.pdf}
		\caption[Geortnete Leistungszeitreihe der RESIDUAL-Betriebsart]{Geordnete Leistungszeitreihe der RESIDUAL-Betriebsart; Absteigend geordneter, skalierter Residuallastverlauf der Tennet-Regelzone für das Jahr 2017}
		\label{fig:RES_JDL_betrART} 
	\end{minipage}
\end{figure}


\subsection{Reale Wind-Einspeisung (WEA)}
Abweichend von den Betriebsarten nach EIS und RES dient der Betrieb einer Power-to-Gas-Anlage nach einer Windpark-Leistungszeitreihe nicht direkt der Netzstabilisierung. Es wird angenommen, dass der Windpark direkt mit der P2G-Anlage gekoppelt ist und somit die Erzeugungsleistung als Leistungszeitreihe für die P2G genutzt werden kann. Fortan wird das Akronym ''WEA'' als Bezeichnung für die Leistungszeitreihe des Windparks verwendet.\\
\subsubsection{WEA-Datengrundlage}
Die Windpark-Leistungszeitreihe wurde institutsintern zur Verfügung gestellt und repräsentiert einen Windpark in der norddeutschen Region mit einer Nennleistung von $10 ~MW$, bestehend aus vier Anlagen des Typs AN Bonus 2000/76 sowie einer Anlage des Typs Vestas V90. Sie beschreibt den Erzeugungsleistungsverlauf des Zeitraumes vom 01. Januar 2017 bis zum 31. Dezember 2017. Die Auflösung der Zeitreihe beträgt 10 Minuten.
\subsubsection{WEA-Datenaufbereitung}
Die Zeitreihe weist zunächst einige fehlende Zeiträume von wenigen Stunden bis zu mehreren Tagen auf, welche anhand linearer Interpolation zwischen dem vorigen und nächsten gültigen Leistungswert ersetzt werden. Es ist davon auszugehen, dass aufgrund dieser Korrektur die Leistungswerte  mit den tatsächlich realisierten Erzeugungsleistungen nicht für jeden Zeitpunkt der Zeitreihe übereinstimmen.
\subsubsection{WEA-Resultierende Zeitreihe}


Nachfolgend wird eine WEA-Leistungszeitreihe in 10-minütiger Auflösung über den Zeitraum Januar bis Dezember 2017 betrachtet. Die Jahresdauerlinie weist eine maximale Erzeugungsleistung von  ca. $10.000 ~kW$ auf. Nahezu das gesamte Jahr über weist die Zeitreihe Leistungen ungleich null auf, wobei nur an etwa der Hälfte des Jahres Erzeugungsleistungen von mehr als $1000 ~kW$ auftreten. \\

\begin{figure}[H]
	\centering
	\begin{minipage}[t]{0.49\textwidth}
		\includegraphics[width=\textwidth]{/home/dafu/Schreibtisch/Master-Projekt/Doku/Abb/Graph/JDL/2019-03-03--15-09_WEA.pdf}
		
		\caption[Leistungszeitreihe der WEA-Betriebsart]{Leistungszeitreihe der WEA-Betriebsart; Einspeiseleistung eines realen Windparks für das Jahr 2017}
		\label{fig:WEA_roh_betrART} 
	\end{minipage}
	\hfill
	\begin{minipage}[t]{0.49\textwidth}
		\includegraphics[width=\textwidth]{/home/dafu/Schreibtisch/Master-Projekt/Doku/Abb/Graph/JDL/2019-03-03--15-10_WEA_JDL.pdf}
		\caption[Geortnete Leistungszeitreihe der WEA-Betriebsart]{Geordnete Leistungszeitreihe der WEA-Betriebsart; Einspeiseleistung eines realen Windparks für das Jahr 2017???}
		\label{fig:WEA_JDL_betrART} 
	\end{minipage}
\end{figure}

\subsection{Börsenstrompreis (COST)}
Weiterhin wird der Betrieb der P2G-Anlagen in Abhängigkeit des Strommarktes simuliert. Der \textsc{EEX}-Spotmarkt wird von \textsc{EPEX} Spot SE mit Sitz in Paris betrieben. Handelsebene ist das Höchstspannungsnetz $220/380~kV$. Lieferort sind die die deutschen, österreichischen, französischen und schweizerischen Übertragungsnetze. KONSTANTIN.  Wissentlich, dass ein direkter Bezug des Stromes vom Höchstspannungsnetz technisch kaum umsetzbar wäre, wird dennoch die Strompreis-Zeitreihe des Spotmarktes genutzt, um in Abhängigkeit des Strompreises Signale für den Anlagenbetrieb zu generieren. Hierzu wird ein Grenzwert in der Einheit $EUR/MWh$ gesetzt, unterhalb dessen Betrieb in voller Nennleistung und oberhalb Stillstand der Anlage geschehen soll.
\subsubsection{COST-Datengrundlage}
Die von der Bundesnetzagentur betriebene Online-Plattform \textsc{smard.de} \cite{Bundesnetzagentur.31.Januar2019} stellt historische Spotmarkt-Preise in der Einheit $EUR/MWh$ mit einer zeitlichen Auflösung 60 Minuten zur Verfügung.
%\subsubsection{COST-Datenaufbereitung}
\subsubsection{COST-Resultierende Zeitreihe}
Abbildung (\ref{fig:COST_roh_betrART}) bildet die Strompreiszeitreihe in 60-minütiger Auflösung über den Zeitraum Januar bis Dezember 2017 ab. Die Jahresdauerlinie weist eine maximalen Strompreis von  ca. $160~ EUR/MWh$, sowie einen minimalen Strompreis von ca. $-80 ~EUR/MWh$ auf. An etwa 5.450 Stunden im Jahr 2017 weist die Zeitreihe Strompreise kleiner $40 ~EUR/MWh$ auf. \\

\begin{figure}[H]
	\centering
	\begin{minipage}[t]{0.49\textwidth}
		\includegraphics[width=\textwidth]{/home/dafu/Schreibtisch/Master-Projekt/Doku/Abb/Graph/JDL/2019-03-03--16-28_STROMPREIS.pdf}
		
		\caption[Strompreis-Zeitreihe der COST-Betriebsart]{Strompreis-Zeitreihe der COST-Betriebsart; Verlauf des Börsenstrompreises im Jahr 2017}
		\label{fig:COST_roh_betrART} 
	\end{minipage}
	\hfill
	\begin{minipage}[t]{0.49\textwidth}
		\includegraphics[width=\textwidth]{/home/dafu/Schreibtisch/Master-Projekt/Doku/Abb/Graph/JDL/2019-03-03--16-28_STROMPREIS_JDL.pdf}
		\caption[Geortnete Strompreis-Zeitreihe der COST-Betriebsart]{Geortnete Strompreis-Zeitreihe der COST-Betriebsart; Werte des JAhres 2017, absteigend geordnet}
		\label{fig:COST_JDL_betrART} 
	\end{minipage}
\end{figure}
%\begin{figure}[H]
%	
%	\centering
%	\includegraphics[width=0.49\textwidth]{/home/dafu/Schreibtisch/Master-Projekt/Doku/Abb/test.png}
%	\includegraphics[width=0.49\textwidth]{/home/dafu/Schreibtisch/Master-Projekt/Doku/Abb/test.png}
%	\caption{ }
%	\label{fig:JDL_COST} 
%\end{figure}
%\subsubsection{Residuallast}
%an welchem punkt? Knoten-Daten verfügbar? sonst skalierung?


%\subsubsection{Datenlage}





\section{Bewertungskriterien}
\label{Bewertungs-Krit}

Die Simulationen der vier Betriebsarten \gls{eis}, \gls{res}, \gls{wea} und \gls{cost}  liefern Kennwerte, anhand derer die Elektrolyseur-Anlagentypen \gls{ael}, \gls{pem}-\gls{el}, und \gls{soel} 
hinsichtlich ihrer Eignung für den jeweiligen Betrieb bewertet werden können.

\subsection{ Systemdienliches Verhalten}%Systemdienstleistung}

Wie aus \ref{sub_GrundET} hervorgeht, sind Elektrolyseur-Anlagen, wie in diesem Modell dargestellt, lediglich in der Lage  Systemdienleistung zu erbringen, indem sie das Netz durch Energieverbrauch entlasten. Die  Ausnutzungsrate $\eta_{util}$ beschreibt das Verhältnis der verrichteteten Arbeit der Elektrolyse-Anlage $W_{act}$ zur gesamten zur Verfügung stehenden elektrische Arbeit $W_{in}$.
Bei einem Verhältnis von 1 wäre die Anlage in der Lage,
\begin{itemize}
\item die gesamte aufgrund von Einspeisemanagement-Maßnahmen (\gls{eis}) geregelte Last vom Netz zu nehmen, Netzengpässe ideal zu verhindern und zudem die ungenutzte Erzeugung zu nutzen,
\item die gesamte negative Residuallast (\gls{res}) zu nutzen und im Sinne der Systemdienlichkeit das Netz ideal zu stabilisieren.
\end{itemize}
Je höher $\eta_{util}$ ausfällt, desto mehr Arbeit konnte die \gls{el}-Anlage verrichten und umso netzdienlicher ist der Anlagenbetrieb.

\subsection{Effizienz}
Die Systemeffizienz $\eta_sys$ setzt die Energie des produzierten Wasserstoffs in Relation zur der Elektrolyse-Anlage zugeführten elektrischen Arbeit. Somit beschreibt der Kennwert die Effizienz der Elektrolyse-Anlage. %, aus elektrischer Energie Wasserstoff zu erzeugen.
Ebenso gibt der spezifische Energiebedarf $E_spez$ ein wichtiger Effizient-Kennwert der

%\subsection{}
%
%-proportional zur Stromdichte -> erhöhung der Stromdichte -->> Erhöhung der Ausbeute, ohne Vergrößerung des Stack |Lettenmeier, Diss
%aber: Degradation
%
%Systemdienstleistungen sind für Anschlussnehmer des elektrischen Verbundnetzes (??? stimmt das?) nach -->>Verordnung XY ???<<< definiert.
%Im hier betrachteten Zusammenhang wird die Entlastung von Teilbereichen der verwendeten Netzebene im Fall hoher Einspeiseleistung durch Erneuerbare Energien als Systemdienstleistung verstanden 





%\section{Bewertungskriterien}
%\label{Bewertungs-Krit}
%
%Vergleichskriterien der unterschiedlichen Technologien.
%
%Produktgas reinheit
%-> \url{https://www.riessner.de/documents/gase/wasserstoff.pdf}
%
%\subsection{Produktgasqualität}
%Tjarks S. 22: Durch die hohen
%Reinheitsanforderungen an den produzierten Wasserstoff wird als Regenerationsgas
%ebenfalls Wasserstoff eingesetzt, um eine Fremdgas-kontamination zu vermeiden.
%
%>>>>was ist mit O2 in H2 ??????? >>>>>>>>>Permeation für AEL und SOEL nicht ermittelt\\
%
%
%\subsection{Wasserstoffproduktionsrate}
%-proportional zur Stromdichte -> erhöhung der Stromdichte -->> Erhöhung der Ausbeute, ohne Vergrößerung des Stack |Lettenmeier, Diss
%aber: Degradation
%
%\subsection{ Systemdienliches Verhalten}%Systemdienstleistung}
%Systemdienstleistungen sind für Anschlussnehmer des elektrischen Verbundnetzes (??? stimmt das?) nach -->>Verordnung XY ???<<< definiert.
%Im hier betrachteten Zusammenhang wird die Entlastung von Teilbereichen der verwendeten Netzebene im Fall hoher Einspeiseleistung durch Erneuerbare Energien als Systemdienstleistung verstanden.
%
%\subsection{Effizienz}
%
%\subsubsection{Ökonomie}
%
%\subsection{Utilisationrate Enutz/Epot}
%
%\subsection{Effizienz}
%%\section{Charakterisierung der Szenarien}


\newpage
\section{Modellbildung}

Das Modell des Elektrolyse-Vergleichsystems wird mit Hilfe der Programmiersprache \textsc{python} und auf Grundlage aktueller Veröffentlichungen zur Modellbildung der Elektrolyse erstellt (vgl. Abschnitt(\ref{subs_Grundl_EL})).\\
In vorliegender Literatur bestehen verschieden komplexe Ansätze für nahezu sämtliche technologisch relevanten System-Komponenten der betrachteten Elektrolyse Systeme. Die Zahl verfügbarer Veröffentlichungen repräsentiert den derzeitigen Forschungsstand bzw. -schwerpunkt. Entsprechend besteht für Hochtemperatursysteme eine geringere Dichte und Vielfalt an Veröffentlichungen.\\
Die zu Grunde liegenden elektrochemischen Prinzipien werden für sämtliche Systeme strukturell in gleicher Art und Weise modelliert. Simulation und Auswertung erfolgen gegenüberstellend mit entsprechender Parametrisierung (Abschnitt(\ref{subs_Par_je_TEch})) sowie Skalierung (Abschnitt(\ref{subs_Skal_TEch})) je Technologie .\\
Ziel ist es, innerhalb der Modellierung die entsprechende Technologie ausreichend detailliert abzubilden, um wesentliche Einflüsse auf die Performance über signifikante Zeitbereiche %Langzeit-Betrieb (??? angestrebte Betriebsdauer)
herauszustellen. %Der Vergleich der 3 betrachteten Elektrolyse-Typen soll als Entscheidungsgrundlage für die tatsächlich umgesetzte Installation einer Demonstrationsanlage dienen(???).\\
Vorgenommene Vereinfachungen werden in Abschnitt(\ref{subsubs_Mod_Vereinfach}) dokumentiert.
\begin{figure}[H]
	
	\centering
	%\includegraphics[width=0.49\textwidth]{/home/dafu/Schreibtisch/Master-Projekt/Doku/Abb/test.png}
	\includegraphics[width=\textwidth]{/home/dafu/Schreibtisch/Master-Projekt/Doku/Abb/Modell/modell_ges_v37.png}
	\caption[Inhalt und Systemgrenzen der Modellierung]{Schematische Darstellung von Inhalt und Systemgrenzen der Modellierung}
	\label{fig:Mod_Zielsetz} 
\end{figure}
Abbildung(\ref{fig:Mod_Zielsetz}) stellt die wichtigsten Elemente der Modellierung dar:
Auf Grundlage verschiedener dynamischer Eingangsdaten bezüglich der elektrischen Versorgung der Anlage werden die drei ausgewählten \gls{el}-Typen inklusive wichtiger Nebenaggregate abgebildet. Festgelegte Systemgrenzen schließen Anlagenperipherie, wie Gleichrichter/ Leistungselektronik, Förderung des Edukt- und Kühlwasserstromes sowie eine Gastrocknungsanlage ein. Des Weiteren sind für das \gls{ael}-Modell Elektrolyt-Aufbereitung und für das \gls{soel}-Modell ein Überhitzer vorgesehen. Über die Systemgrenzen tritt der Edukt-und Kühlwasserstrom sowie elektrische Energie in das System ein. Produktgase (bislang lediglich Wasserstoff) verlassen das System über die Grenze. Vorausgesetzt bzw. nicht innerhalb des Modells berücksichtigt ist die Bereitstellung ausreichend reinen Wassers sowie die Nachverdichtung des Produktgases auf entsprechende Druckniveaus.
Im linken Drittel von Abbildung(\ref{fig:Mod_Zielsetz}) sind außerdem verwendete Eingangssignale dargestellt. Diese entsprechen den unter (\ref{sect_Betriebsarten}) erläuterten Betriebsarten. 
Im Folgenden wird häufig eine Unterscheidung zwischen Hoch- und Niedertemperatur-\gls{el} vorgenommen. Ersteres entspricht der \gls{soel}- während letztere Bezeichnung auf \gls{ael}- und \gls{pem}-\gls{el} bezogen ist.
%\subsection{Vorüberlegungen / Modellausrichtung}
%ABZUSTIMMEN
%
%- Zeit-Abhängigkeit durch deltaT des Eingangsignals
%-->> Dynamik eher makroskopisch?
%elektrochemische Dynamik nicht berücksichtigt
%
%-> dynamisches Verhalten über Rampen angenähert
%-->> Leistungsgradient als Begrenzung?
%
%-> Druckbetrieb: Tjarks S. 93 ....bis zu 29 \% steigerung d. spez. Arbeit 
%
%- Anpassung der Regelungs-Komponenten (?) -->> Wassermanagement // Temperatur
%- optimierter Betriebsbereich (p, T)
%
%- notwendige Ausgangsgrößen für technoökonomische Betrachtung/ Bewertung:
%-> Wasserstoff-Volumen / Volumenstrom
%(-> Sauerstoff)
%-> Wasserbedarf
%-> Leistung/ Energie-Input
%
%Betriebskosten
%-Fix:
%
%--> Energiekosten




\subsection{Peripherie-Anlagenschemata}
\label{subs_peri_schem}

\begin{figure}[!tbp]
	\centering
	\begin{minipage}[t]{0.32\textwidth}
		\includegraphics[height=4cm]{/home/dafu/Schreibtisch/Master-Projekt/Doku/Abb/Modell/st_ael.png}
		
		\caption[Schema: Wasserführung \gls{ael}-Stack]{Schematische Darstellung der Wasserführung des \gls{ael} Stack-Modells}
		\label{fig:stack_schem_ael} 
	\end{minipage}
	\hfill
	\begin{minipage}[t]{0.32\textwidth}
		\includegraphics[height=4cm]{/home/dafu/Schreibtisch/Master-Projekt/Doku/Abb/Modell/st_pem.png}
		
		\caption[Schema: Wasserführung \gls{pem}-Stack]{Schematische Darstellung der Wasserführung des \gls{pem} Stack-Modells}
		\label{fig:stack_schem_pem} 
	\end{minipage}
	\hfill
	\begin{minipage}[t]{0.32\textwidth}
		\includegraphics[height=4cm]{/home/dafu/Schreibtisch/Master-Projekt/Doku/Abb/Modell/st_soel.png}
		
		\caption[Schema: Wasserführung \gls{soel}-Stack]{Schematische Darstellung der Wasserführung des \gls{soel} Stack-Modells}
		\label{fig:stack_schem_soel} 
	\end{minipage}
\end{figure}


%\begin{figure}[H]
%	
%	\centering
%	\includegraphics[height ={4cm}]{/home/dafu/Schreibtisch/Master-Projekt/Doku/Abb/Modell/st_ael.png}
%	\includegraphics[height ={4cm}]{/home/dafu/Schreibtisch/Master-Projekt/Doku/Abb/Modell/st_pem.png}
%	\includegraphics[height ={4cm}]{/home/dafu/Schreibtisch/Master-Projekt/Doku/Abb/Modell/st_soel.png}
%	\caption{Schematische Darstellung verwendeter Stack-Modelle; links: \gls{ael}; mitte: \gls{pem}; rechts: \gls{soel} }
%	\label{fig:peri_schem_Modell} 
%\end{figure}
Für die Modellierung der drei betrachteten \gls{el}-Technologien wird jeweils ein vereinfachter Stack-Aufbau zu Grunde gelegt. Annahmen zu Komponenten und Aufbau sind aktueller Fachliteratur (vgl. Abschnitt(\ref{subs_Grundl_EL})) entnommen. In den Abbildungen(\ref{fig:stack_schem_ael},\ref{fig:stack_schem_pem},\ref{fig:stack_schem_soel}) sind die in dieser Arbeit verwendeten Schemata abgebildet.\\
\subsubsection{NT-Elektrolyse}
\label{subsubs_peri_schem_NT}
Der Wasserkreislauf beider \gls{nt}-\gls{el}-Anlagen (Abbildungen(\ref{fig:stack_schem_ael},\ref{fig:stack_schem_pem})) kombiniert die Zufuhr des Wassers als Edukt sowie den Kühlwassermassenstrom. Der notwendige Gesamtwasserstrom tritt mit Kühlwassertemperatur  in den Wasserkreislauf ein ($T_0$) und wird zum zirkulierenden Kühlwasserstrom beigemischt, wodurch sich eine Mischtemperatur $T_1$ einstellt. Eine Wärmeabgabe der Pumpe an den Wassermassenstrom ist theoretisch zu berücksichtigen, wird jedoch in der vorliegenden Arbeit vernachlässigt. Aus diesem Grund entspricht die Stack-Eintritts-Temperatur $T_2$ der Mischtemperatur $T_1$. Der aus dem Stack austretende Massenstrom stellt eine Mischung aus Wasser und dem Produktgasstrom dar und weist Stack-Temperatur $T_3$ auf. Für die Rückführung des Kühlwassers ist ein Wasserabscheider erforderlich, welcher jedoch im Modell nicht weiter berücksichtigt wird. Der zurückgeführte Wasserstrom wird mittels Wärmeübertrager abgekühlt ($T_4$) und dem Eduktstrom beigemischt. Der Produktgasstrom verlässt das System an Stelle (6) und wird anschließend der Gastrocknung zugeführt. Das Modell der \gls{ael} weist zudem ein Elektrolyt-Aufbereitungssystem auf.

\subsubsection{HT-Elektrolyse}  
\label{subsubs_peri_schem_HT}
Die Wasser-Zirkulation ($0$),($1$),($2$),($3$),($4$) folgt der für die \gls{nt}-\gls{el} beschriebenen Systematik. Diese unterscheidet sich jedoch insofern, dass sämtliche Stoffströme in gasförmigem Zustand vorliegen. Innerhalb dieser Arbeit wird die Annahme getroffen, dass am Anlagenstandort ein Dampf-Strom mit Abdampftemperatur von $280~^\circ C$ verfügbar ist. Da der Dampfstrom von einer externen Wärmequelle zugeführt wird, entfällt die Notwendigkeit einer weiteren Förder-Einheit. Das Aufheizen auf Soll-Temperatur des Stack erfolgt mittels eines elektrischen Überhitzers ($1-2$). Innerhalb der Modellierung entsprechender Stoffströme ist zu beachten, dass ein molarer Mindestanteil innerhalb des Stack-Zustroms zu gewährleisten ist (siehe Abschnitt(\ref{subs_Grundl_EL})). Entsprechend ist die Implementierung einer Rezirkulation mittels Stoffstrom-Aufteilung ($3-4$ und $3-5$), unter Berücksichtigung sich verändernder Stoffstromanteile von Dampf und Wasserstoff, vorzunehmen. Der rezirkulierte Stoffstrom muss ausreichend groß gewählt werden, um den Mindestanteil von Wasserstoff im Stack-Zustrom zu erreichen. Eine Wärmerückgewinnung aus dem Produktgasstrom ist im vorliegenden Modell nicht berücksichtigt.

\subsubsection{Gastrocknung}
\label{subsubs_peri_schem_Gastrockn}
Um einen Produkgasstrom ausreichender Reinheit bereitstellen zu können, ist der Betrieb einer Gastrocknungsanlage erforderlich. Anlagenschema und Funktionsprinzip sind \cite{Tjarks2017} entnommen und in Abbildung(\ref {fig:schema_TB}) dargestellt.

\begin{figure}[H]
	
	\centering

	\includegraphics[height ={4cm}]{/home/dafu/Schreibtisch/Master-Projekt/Doku/Abb/Modell/TB_02.png}
	\caption[Schema: Gastrocknung]{Schema des zur Gastrocknung verwendeten Modells einer Temperatur-Wechsel-Absorptionsanlage nach \cite{Tjarks2017}}
	\label{fig:schema_TB} 
\end{figure}
Grundsätzlich beinhaltet die Gastrocknung zwei Stufen. Nach einer reinen Kondensation ($K$) gegen Umgebungstemperatur erfolgt durch Adsorption eine nahezu vollständige Abscheidung des im Produktgasstrom enthaltenen Restwassers.\\ 
Eine Temperatur-Wechsel-Adsorption wird mittels zwei (oder mehr) abwechselnd durchströmten Trockenbetten ausgeführt. Eine wechselnde Durchströmung ermöglicht gleichtzeitige Gastrocknung und Regeneration. Sofern beispielsweise das erste Trockenbett ($TB1$) zur Wasseraufnahme betrieben wird, erfolgt gleichzeitig die Regeneration des zweiten ($TB2$) Trockenbettes. Die Regeneration (Desorption des aufgenommenen Wassers) erfolgt mittels Rückleitung und Aufheizen eines Teils des trockenen Gasstromes. Dieser nimmt bei hoher Temperatur Wasser des zu regenerierenden Trockenbetts auf und wird anschließend erneut dem Kondensator zugeführt.
In Abbildung(\ref{fig:schema_TB}) ist eine Trocknungssituation mit $TB1$ im Absorptions- und $TB2$ im Desorptions-Betrieb dargestellt. Der Produktgasstrom folgt den Volllinien ($0 -2-3-4$), während ein Teil zur Regeneration  umgeleitet ($5-7-8-9$) wird. Sobald das regulär durchströmte Trockenbett ($TB1$) eine gesetzte Kapazitätsgrenze erreicht hat, werden entsprechende Stoffströme so umgeleitet, dass eine entgegengesetzte Durchströmung eintritt; Trocknung:($0-8-7-6$), Desorption: ($5-3-2-1$). Unterbrochene Linien stellen die geänderte Leitungsführung im entgegengesetzten Betrieb dar. Dadurch wird eine Regeneration des ersten Trockenbettes und Wasseraufnahme im zweiten Trockenbettes bewirkt. 
In Anlehnung an das Vorgehen in oben genannter Quelle wird vereinfachend ein kontinuierlicher Betrieb angenommen.\\
Dieses Modell wird für sämtliche Technologien gleichermaßen angewandt.

\subsection{Grundlagen}
%\subsubsection{Numerische Simulation}
%Möglichst detailgetreue Beschreibung bzw. Abbildung einer realen Komponente oder eines Systems durch Zahlenwerte, welche durch Grundlagengleichungen das System-Verhalten beschreiben. [??? Quelle]


\subsubsection{Programmiersprache: Python}
Die Modellierung sowie sämtliche zusätzlichen Berechnungen werden innerhalb einer \textsc{python}-IDE vorgenommen.\\
Die ''open source high-level'' Programmiersprache \textsc{python} wird durch die \textsc{Python Software Foundation\footnote{\url{https://www.python.org/psf/}}} bereitgestellt und beinhaltet eine große Zahl verschiedener Funktionen und Erweiterungen für eine Vielzahl programmiertechnischer Anwendungen.\\
Die gewählte Entwicklungsumgebung ermöglicht eine objektorientierte (OOP) Programmierung auf Grundlage von Klassen und Methoden (Funktionen).
Da seitens der Autoren keine einschlägigen Kenntnisse bezüglich OOP vorliegen, weißt die erstellte Programmstruktur lediglich teilweise entsprechende Eigenschaften auf: Zwar sind verschiedene Funktionen auf unterschiedlichen Skript-Ebenen (an-)geordnet und miteinander verknüpft, jedoch wird keine Klassen-Struktur verwendet. 


%\subsection{Modell}

\subsubsection{Vereinfachungen /Modellrestriktionen}
\label{subsubs_Mod_Vereinfach}

Dynamisches Verhalten entsteht innerhalb der Modellierung makroskopisch durch das zeitabhängige Eingangssignal sowie hinsichtlich der elektrochemischen Eigenschaften durch eine dynamische Temperaturänderung. Das entsprechend verwendete Temperaturmodell bedient sich der Annahme einer zusammengefassten Wärmekapazität der Stacks sowie einer vereinfachten Betrachtung entsprechender Wassermassenströme (vgl. Abschnitt(\ref{subs_peri_schem})). Weitere dynamische Größen bzw. Vorgänge, wie:
\begin{itemize}
	\item Druckdynamik 
	\item Doppelschicht-Vorgänge
	\item Diffusions-Vorgänge 
\end{itemize}
werden auf aktuellem Stand des Modells nicht berücksichtigt. Bezüglich der Druckdynamik sind bereits entsprechende Variablen vorgesehen bzw. angelegt, wodurch eine nachträgliche Implementierung sehr einfach möglich ist. Sofern bezüglich beider letztgenannten elektrochemischen Vorgänge hinreichend genaue Modelle erstellt werden können, ist auch deren Implementierung innerhalb des Elektrochemie-Teils (Abschnitt(\ref{subsubs_mod_elchem})) verhältnismäßig unproblematisch.\\
Trägheiten hinsichtlich der Reaktion auf Eingangsleistungsänderungen werden mittels Rampen auf Grundlage von Literaturwerten \cite{Buttler.2018} angenähert. 
Des Weiteren werden Eigenschaften von Elektrode und Membranen bzw. Elektrolyt vereinfachend homogen angenommen. Eine räumlich aufgelöste Betrachtung sämtlicher Einflussgrößen und Eigenschaften ist nicht vorgesehen.\\ 
Einflüsse von Konzentrationsänderungen auf entsprechende Zellspannungen werden auf Grund unsicherer Berechnungsgrundlagen und lediglich geringer Größenordnung nicht in allen Teilmodellen berücksichtigt.\\
Eine Wärmerückgewinnung aus Produktgasströmen der \gls{ht}-\gls{el} ist ebenfalls nicht implementiert.\\

%%- Membran / Elektroden eindimensional abgebildet
%- keine Massenbilanzen im Elektrolyt
%%- keine Druck-Dynamik
%- Zeitvesetzte berechnug der Gasaufbereitung
%
%%-> wasserkreisläufe: keine Mischung bei NT
%%-> kein produktgas-Wärmerückgewinn
%
%%-keine Konzentrationsüberspannungen
%%- keine diffusionsdynamik
%%- keine doppelschicht-dynamik
%
%
%-Polarisationskurve auf Grundlage d nernst-Gleichung
%->Aktivität der beteiligten Reaktionspartner erforderlich
%->
%Modellierung nach [??? Schalenbach]



\subsection{Skript-Struktur und funktionaler Aufbau}
\label{subs_Skript-Strukt}
Aus der in Abbildung(\ref{fig:Mod_Zielsetz}) verdeutlichten Zielsetzung ist ein Berechnungs-Tool hervorgegangen, welches insgesamt $12$ Skripte beinhaltet.
Die Berechnung wird von einem zentralen Skript aus gesteuert, welches sämtliche weiteren Skripte aufruft bzw. verwaltet. Das Rückrad des Programms bilden das erwähnte Steuerungs-Skript sowie zwei weitere Haupt-Skripte zum Koordinieren unterschiedlicher Schleifen Ebenen. 

\subsubsection{Funktionale Struktur}
\label{subs_Funkt-Strukt}
Das Modell basiert auf zwei verschachtelten Schleifen. Während die äußere Schleife (\textsc{main\_loop}) hauptsächlich dazu dient, Variablen, Parameter sowie Eingangsdaten aufzubereiten bzw. zu verwalten, findet innerhalb der inneren Schleife (\textsc{sub\_loop}) die Koordination nahezu sämtlicher notwendiger Berechnungen statt. Lediglich Berechnungen und Funktionen, welche je Zeitschritt des Datensatzes zu konstanten Werten führen, werden innerhalb der äußeren Schleife ausgeführt. Die Anzahl der äußeren Schleifen-Durchläufe entspricht der Länge des Eingangs-Datensatzes. Dagegen kann die Anzahl der inneren Schleifendurchläufe je Leistungs- bzw. Zeitwert festgelegt werden.\\

Die Funktionale Struktur bzw. Interaktion der Haupt-Skripte ist in Abbildung(\ref{fig:Strukt_Skript}) dargestellt. 
\newpage
\subsubsection{Skript-Struktur}
\label{subsubs_skrpt-strukt}
Die Struktur erstellter Skripte- bzw. Ordner sowie wichtige Dateiinhalte sind wie folgt aufgebaut:

\begin{itemize}
	\item master\_ctrl\_ (\textit{startet bzw. steuert Berechnung})
	\item log-Datei (\textit{listet Berechnungszeit})
	\item log-Datei f. ökonomische Berechnung (\textit{direkte Ausgabe ökonomischer Kennwerte})
	\item Haupt-Funktions-Ordner (mainf)
	\begin{itemize} 
		\item \_main\_loop\_
		\item \_sub\_loop\_
		\item \_plot\_
		\item Funktions-Ordner (Deff)
		\begin{itemize}
			\item \_def\_glob\_(\textit{globale Funktionen})
			\item \_def\_AEL\_ (\textit{AEL-spezifische Funktionen})
			\item \_def\_PEM\_ (\textit{PEM-spezifische Funktionen})
			\item \_def\_SOEL\_ (\textit{SOEL-spezifische Funktionen})
		\end{itemize}
		\item Parameter-Ordner (Par)
		\begin{itemize}
			\item \_par\_glob\_(\textit{globale Parameter})
			\item \_par\_AEL\_ (\textit{AEL-spezifische Parameter})
			\item \_par\_PEM\_ (\textit{PEM-spezifische Parameter})
			\item \_par\_SOEL\_ (\textit{SOEL-spezifische Parameter})
		\end{itemize}
	\end{itemize}
	\item Input-Ordner (In)
	\item[] Leistungs-Datensatz von
	\begin{itemize}
		\item WEA
		\item RES
		\item EIS
		\item COST
	\end{itemize}
	\item Output-Ordner (Out)
	\begin{itemize}
		\item sämtl. Ausgangs-Datensätze
	\end{itemize}
	
\end{itemize}

In Abbildung(\ref{fig:Strukt_Skript}) ist die zu Grunde liegende Interaktion der Modell-Skripte schematisch dargestellt.\\
Die Steuerung der Berechnung erfolgt innerhalb des \textsc{ctrl}-Skripts (dreifach umrandet). Hier werden Basis-Parameter zu Simulierter Technologie (Typ, Gesamt-Zellenzahl, Regel-Parameter) und Eingangsdatensatz (Länge, Skalierung, Berechnungs-Auflösung) festgelegt.
 Das gezielte Setzen bzw. Verändern von weiteren technologie- bzw. anlagenspezifischen Parametern erfolgt innerhalb der Parameter-Skripte \textsc{par\_glob} und \textsc{par\_tech} (unterbrochene Umrandung), wobei ''tech'' für die entsprechend berechnete Technologie (\gls{ael}, \gls{pem}, \gls{soel}) steht. Für weitere Skripte ist innerhalb der Anwendung keine Werte-Veränderung vorgesehen.\\
\begin{figure}[H]
	
	\centering
	%\includegraphics[width=0.49\textwidth]{/home/dafu/Schreibtisch/Master-Projekt/Doku/Abb/test.png}
	\includegraphics[width=0.6\textwidth]{/home/dafu/Schreibtisch/Master-Projekt/Doku/Abb/Skript/skript_strukt_v33.png}
	\caption[Skript-Interaktions-Schema]{Vereinfachtes Interaktions-Schema erstellter Skripte sowie deren Interaktion; große Pfeile: Aufrufe, Zugriffe sowie Variablen- und Parameterübergabe; kleine Pfeile (Volllinie): Berechnungsrückgabe; kleine Pfeile (unterbrochene Linie): Parameterrück- bzw. Übergabe}
	\label{fig:Strukt_Skript} 
\end{figure}

Sämtliche Datensätze werden im ''.csv''-Dateiformat eingelesen und müssen eine Datums-Spalte sowie eine weitere Spalte mit entsprechenden Leistungswerten enthalten. Auf dieser Grundlage erfolgen sämtliche Berechnungen. Nach erfolgreichem Programmdurchlauf wird eine ''out.csv''-Datei ausgegeben, welche die relevante Größen sämtlicher berechneter Zeitschritte enthält. Eine Auflistung berechneter und ausgegebener Größen ist in Tabelle(\ref{tab:mod_calc_Ausgabewerte}) dargestellt. Des Weiteren ist eine ökonomische Auswertung implementiert, welche auf Basis von spezifischen Investitionskosten (bezogen auf installierte Leistung), diesbezüglich relativen, jährlichen Wartungskosten sowie Strombezugskosten ermittelt werden. Eine entsprechende Ausgabe erfolgt in einer gesonderten ''cost.csv''-Datei. Außerdem wird eine ''log''-Datei mit Informationen zum Zeitpunkt der Berechnung sowie benötigter Berechnungszeit angelegt. 

\begin{table}[H]
	\caption{Ausgabewerte der Simulation}
	\begin{tabular}{llc}
		Größe&Zeichen&Einheit\\
		\hline
		\hline
		berechnete Zeit&$t_{abs}$ 	&$s$\\
		Stack-Temperatur&$T$ 		& $^\circ C$\\
		(Kühl-)Wassermassenstrom&$m_c$		&$kg/s$\\
		Zellspannung&$u_cell$	&$V$\\
		Stromdichte&$i_cell$	&$A/cm^2$\\
		Eingangsleistung&$P_{in}$	&$W$\\
		Anlagen-Leistungsaufnahme&$P_{act}$	&$W$\\
		Stack-Leistungsaufnahme&$P_{st}$	&$W$\\
		Peripherie-Leistungsaufnahme&$P_{aux}$	&$W$\\
		Kathodendruck&$p_K$		&$bar$\\
		Anodendruck&$p_A$		&$bar$\\
		Produktgas-Stoffstrom, Wasserstoff&$n_{H_2}$	&$mol/s$\\
		Produktgas-Stoffstrom, Sauerstoff & $n_{O_2}$		&$mol/s$\\
		Reations-Wasser-Verbrauch&$n_{H_{2}O}$&$mol/s$\\
		Stoffmengenanteil $H_2$ in $O_2$&$\theta_{H_2inO_2}$&$mol/mol$\\
		
	\end{tabular}
	\label{tab:mod_calc_Ausgabewerte}
\end{table}

  
 



%\begin{figure}[H]
%	
%	\centering
%	\includegraphics[width=0.49\textwidth]{/home/dafu/Schreibtisch/Master-Projekt/Doku/Abb/test.png}
%	\includegraphics[width=0.49\textwidth]{/home/dafu/Schreibtisch/Master-Projekt/Doku/Abb/test.png}
%	\caption{ }
%	\label{fig:Strukt_Funkt_oa} 
%\end{figure}
%\begin{itemize}
%	\item Elektrochemie der jeweiligen Technologie
%	\begin{itemize}
%		\item AEL
%		\begin{itemize}
%			\item Gibbs
%			\item Austauschstromdichte
%			\item Partialdrücke
%			\item Überspannungen
%			\item Polarisationskurve
%		\end{itemize}
%	\end{itemize}
%	\begin{itemize}
%		\item PEM
%		\begin{itemize}
%			\item 
%		\end{itemize}
%	\end{itemize}
%	\begin{itemize}
%		\item SOEL
%		\begin{itemize}
%			\item 
%		\end{itemize}
%	\end{itemize}
%\end{itemize}
\subsection{Modell-Bestandteile}
Im Folgenden werden die wichtigsten Modell-Komponenten aufgeführt und erläutert.\\
% Teilmodelle und Funktionen, welche für sämtliche \gls{el}-Typen gültig sind werden mittels $\blacktriangleright$ gekennzeichnet, während ausschließlich technologiespezifisch anzuwendende Teilmodelle mit $\triangleright$ markiert werden.\\
Der Aufbau des grundlegenden elektrochemischen Modells erfolgt maßgeblich auf Grundlage von \cite{Chandesris2015,Schalenbach2013,Tjarks.2017,EspinosaLopez.2018,Gabrielli2016}
\subsubsection{Grundlagen}
Ziel des vorgestellten Modells ist es, das Verfahren der Wasser-Elektrolyse mit hinreichendem Detaillierungsgrad abzubilden, um Anhaltspunkte über das systemrelevante Verhalten über einen aussagekräftigen Zeitbereich zu erlangen.\\ Wichtige Kenngrößen sind Leistungsaufnahme und Produktgasmenge sowie ableitbare Effizienz-Kennwerte.   

Die Betriebstemperatur des Stacks besitzt maßgeblichen Einfluss auf viele wichtige Größen der \gls{el}-Berechnung. 
In Abbildungen(\ref{fig:Strukt_Funkt_innerloopNT},\ref{fig:Strukt_Funkt_innerloopHT}) wird die Struktur der dynamischen Berechnung wichtiger Kenngrößen für \gls{nt}- und \gls{ht}-\gls{el} dargestellt.\\

%\begin{figure}[H]
%	
%	\centering
%	\includegraphics[width=0.49\textwidth]{/home/dafu/Schreibtisch/Master-Projekt/Doku/Abb/Modell/inner_NT_13.png}
%	\includegraphics[width=0.49\textwidth]{Abb/Modell/inner_HT_13.png}
%	\caption{Struktur des dynamischen Kern-Modells; links: \gls{nt}-, rechts: \gls{ht}-\gls{el} }
%	\label{fig:Strukt_Funkt_innerloop} 
%\end{figure}

\begin{figure}[!tbp]
	\centering
	\begin{minipage}[t]{0.49\textwidth}
		\includegraphics[width=\textwidth]{/home/dafu/Schreibtisch/Master-Projekt/Doku/Abb/Modell/inner_NT_13.png}
		
		\caption[Kern-Modell \gls{ht}-\gls{el}]{Struktur des dynamischen Kern-Modells der \gls{nt}-\gls{el}; Berechnungsablauf und Interdependenzen wichtiger Größen}
		\label{fig:Strukt_Funkt_innerloopNT} 
	\end{minipage}
	\hfill
	\begin{minipage}[t]{0.49\textwidth}
		\includegraphics[width=\textwidth]{Abb/Modell/inner_HT_13.png}
		
		\caption[Kern-Modell \gls{ht}-\gls{el}]{Struktur des dynamischen Kern-Modells der \gls{ht}-\gls{el}; Berechnungsablauf und Interdependenzen wichtiger Größen}
		\label{fig:Strukt_Funkt_innerloopHT}  
	\end{minipage}
\end{figure}

Die Stacktemperatur hängt unter anderem von Stackspannung und -strom ab.
Auf Grundlage der Stacktemperaturänderung $T(U,I)$ sowie korrelierenden Stoffströmen ($n_{i}$,$m_c$) ist die Leistungsaufnahme der Peripherie ($P_{aux} = P_{gt}+P_{p}(+P_{ec}(+P_{h}))$) bestimmbar. Unter Berücksichtigung der Leistungselektronik sowie zulässiger Leistungsgradienten ($P_{grad}$) und Betriebsbereichen wird anhand der verfügbaren Eingangsleistung $P_{in}$die verfügbare Leistung $P_{diff}$  des betrachteten Zeitschrittes bestimmt. Anhand technologiespezifischer, elektrochemischer Eigenschaften, welche in die Polarisationskurve (bzw. Strom-Spannungs-Verhältnis der Zelle) einfließen, ist anschließend die optimale Stromdichte bestimmbar. Auf deren Grundlage ist anschließend die Ermittlung der Produkgasströme $n_{H_2}, ~n_{O_2}$, der tatsächlichen Stackleistung $P_{st}$ sowie weiterer Eingangswerte des nachfolgenden Berechnungsschrittes zu berechnen.  

In den folgenden Abschnitten werden weitere Funktionen sowie Details zu Temperaturmodell, Peripherie und grundlegender elektrochemischer Berechnungen erläutert.  

% % Temperatur-Modelle nach Tjarks, Lettenmeier, Ruuskanen, Gabrielli

%\begin{figure}[H]
%	
%	\centering
%	\includegraphics[width=0.49\textwidth]{/home/dafu/Schreibtisch/Master-Projekt/Doku/Abb/Graph/test_JDL_WEA_scale-marker_Markerpoints_01.pdf}
%	
%	\includegraphics[width=0.49\textwidth]{/home/dafu/Schreibtisch/Master-Projekt/Doku/Abb/2019-03-02--12-06_barplot_EISMAN_RESIDUAL_WEA_E_spec.pdf}
%	\caption{TEST }
%	\label{fig:Test} 
%\end{figure} 


 \subsubsection{Temperaturmodell}
 \label{subsubs_mod_TEmp}
Das Temperaturmodell wird auf vereinfachender Grundlage einer zusammengefassten Wärmekapazität sämtlicher Stack-Komponenten aufgebaut. Dieses Modell ist in verschiedenen Ansätzen zu finden (\cite{Espinosa-Lopez2018,Gabrielli2016}), wobei bezüglich des zu Grunde liegenden Ansatzes \cite{Ulleberg2003} zitiert wird. Für die \gls{nt}-\gls{el} kann dieses Modell direkt übernommen werden, während im Fall der \gls{ht}-\gls{el} eine abweichende Bilanzierung unter gleicher Voraussetzung einer zusammengefassten Wärmekapazität erfolgt.\\


%$\blacktriangleright$ 
\begin{wrapfigure}{l}{1cm}
	\includegraphics[width=1.0cm]{/home/dafu/Schreibtisch/Master-Projekt/Doku/Abb/Modell/it_dTdt.png}
	%\caption{Bildunterschrift}
\end{wrapfigure}
Den zentralen Baustein des Temperaturmodells bildet die Bestimmung der Temperaturänderung des Stacks mittels build-in DGL-solver. 

Die Differentialgleichung der Wärmebilanz berücksichtigt für die \gls{nt}-\gls{el} auf Grundlage der genannten Modelle Wärmegewinne durch Überspannungen und  Wärmeabgabe an die Umgebung sowie an den Kühlwasserstrom. Für die Berechnung der \gls{ht}-\gls{el} werden Wärmeeinträge durch Überspannungen sowie den Eingangs-Dampfstrom und  Wärmeabgabe an die Umgebung berücksichtigt.\\


%Für den ersten Berechnungsschritt müssen erwartete Eingangswerte festgelegt werden, während im Laufe der weiteren Berechnung (Anzahl der Berechnungsschritte $n~>~0$) jeweils auf entsprechende Werte des vorherigen Berechnungsschrittes ($n-1$) zurückgegriffen werden kann.



Wichtige Parameter sind Wärmekapazität sowie Wärmedurchgangskoeffizient $H_{AX}$ in Abhängigkeit der Stack- bzw. Anlagengröße. 
\newline


%[Gabrielli???]
%- wichtig: PID und HAx
%-> lumped heat cap. // vereinfachung für WÜ -> T\_st  const
%
%-da\\
%-kommt noch text hin\\
%\newline


%$\blacktriangleright$ 
\begin{wrapfigure}{l}{1cm}
	%\vspace{-0.8cm}
	\includegraphics[width=1.0cm]{/home/dafu/Schreibtisch/Master-Projekt/Doku/Abb/Modell/it_PID.png}
	%\caption{Bildunterschrift}
\end{wrapfigure}
Zur Regelung der Stacktemperatur wird ein PID-Regler implementiert, wobei einzelne Regelglieder mittels finiter Differenzen angenähert werden.
Dieser berechnet im Falle der Niedertemperatur-Elektrolyse den Kühlwassermassenstrom bzw. für die \gls{ht}-\gls{el} die elektrische Zuheizleistung des Zirkulationsstromes. Grundsätzlich ist eine ausreichend hohe Zahl an Berechnungsdurchläufen je Eingangswert erforderlich, um die Stabilität des Temperaturmodells zu gewährleisten.\\ Im Fall der \gls{ht}-\gls{el} wird je nach Eingangs-Datensatz eine konstante oder variable Soll-Temperatur verwendet: Sofern Eingangs-Zeitreihen lediglich kurze Zeiträume positiver Leistungswerte aufweisen, ist es erforderlich, eine konstante Soll-Temperatur auf Niveau der Arbeitstemperatur zu wählen. Bei ausreichend großen Intervallen positiver Leistungswerte kann die Soll-Temperatur dynamisch gewählt werden. Für Zeitbereiche ohne Leistungs-Input ist dadurch die notwendige Überhitzer-Leistung des Rezirkulationsstromes (vgl. Abschnitt(\ref{subsubs_peri_schem_HT})) reduzierbar.
%Da die Temperaturänderung des Stacks von der Höhe des Kühlwassermassenstromes abhängt, dieser jedoch auf Grundlage der Soll-Ist-Differenz innerhalb der PID-Funktion ermittelt wird, 
\newline
%?????
%Zu Elektrochemie bzw. Polarisationskurve im Detail, siehe folgender Abschnitt(\ref{subsubs_mod_elchem})????

\subsubsection{Steuerungstechnische Implementierungen}
\label{subsubs_mod_Anlagen-strg}

%$\triangleright$ 
%\newline



%$\blacktriangleright$
\begin{wrapfigure}{l}{1cm}
	%\vspace{-0.8cm}
	\includegraphics[width=1.0cm]{/home/dafu/Schreibtisch/Master-Projekt/Doku/Abb/Modell/it_Popt.png}
	%\caption{Bildunterschrift}
\end{wrapfigure} Die Berechnung des optimalen Anlagenstromes erfolgt an Hand einer build-in Optimierungsfunktion.
 Auf Grundlage der Polarisationskurve wird die optimale Stromdichte des Stacks in Abhängigkeit verfügbarer Leistung und unter Berücksichtigung des Peripherie-Leistungsbezugs ermittelt. Die zu Grunde liegende Optimierungsfunktion wird hier mittels \textsc{Methode kleinster Quadrate} durchgeführt.\\




%\subsubsection{Inner Loop (HT)}
%- Temp. Modell nach???
%- zusätzliche Stoffstrom-Mischung Betrachtet



\subsubsection{Elektrochemie}
\label{subsubs_mod_elchem}
%\begin{wrapfigure}{l}{1cm}
%	%\vspace{-0.8cm}
%	\includegraphics[width=1.0cm]{/home/dafu/Schreibtisch/Master-Projekt/Doku/Abb/Modell/it_polar.png}
%	%\caption{Bildunterschrift}
%\end{wrapfigure}


Die Anlagencharakteristik wird hauptsächlich durch die zu Grunde liegenden Eigenschaften des jeweiligen \gls{el}-Typs bestimmt. 
Das Haupt-Funktionsskript der betrachteten Technologien   (\textsc{def\_ael}, \textsc{def\_pem}, \textsc{def\_soel}; vgl. Abschnitt(\ref{subsubs_skrpt-strukt})) enthält verschiedene Funktionen, welche für die Berechnung der Spannungs- und Strom-Abhängigkeit erforderlich sind.\\
\begin{figure}[H]
	
	\centering
	%\includegraphics[width=0.49\textwidth]{/home/dafu/Schreibtisch/Master-Projekt/Doku/Abb/test.png}
	\includegraphics[height = 8cm]{/home/dafu/Schreibtisch/Master-Projekt/Doku/Abb/Modell/polar_schem_v02.png}
	\caption[Schematische Darstellung zur Berechnung Strom-Spannungs-Abhängigkeit]{Schematische Darstellung einzelner Funktionen zur Berechnung der charakteristischen Strom-Spannungs-Abhängigkeit; }
	\label{fig:schem_polar} 
\end{figure}
%\newline
 %Polarisationskurven-Berechn.:
%\begin{figure}[H]
%	
%	\centering
%	\includegraphics[width=0.49\textwidth]{/home/dafu/Schreibtisch/Master-Projekt/Doku/Abb/test.png}
%	\includegraphics[width=0.49\textwidth]{/home/dafu/Schreibtisch/Master-Projekt/Doku/Abb/test.png}
%	\caption{ }
%	\label{fig:Strukt_Funkt_elchem} 
%\end{figure}
Je Technologietyp sind Funktionen zur Berechnung folgender Größen auf Grundlage der in  Abschnitt(\ref{subs_Grundl_EL}) erläuterten Berechnungen bzw. Näherungen implementiert:\\

%\newline
\begin{wrapfigure}{l}{1cm}
	%\vspace{-0.8cm}
	\includegraphics[width=1.0cm]{/home/dafu/Schreibtisch/Master-Projekt/Doku/Abb/Modell/it_Gibbs.png}
	%\caption{Bildunterschrift}
\end{wrapfigure}
%Freie Gibbsche Enthalpie: 
Für sämtliche Technologietypen wird eine identische Berechnungsgrundlage zur Ermittlung der freien Gibbschen Enthalpie %$\Delta G$
\gls{Gibbs} angewendet, wobei bezüglich verwendeter Enthalpie- und Entropiewerte die entsprechende Stacktemperatur  zu berücksichtigen ist.\\
%($>/< 100^\circ C$)
\begin{wrapfigure}{l}{1cm}
	%\vspace{-0.8cm}
	\includegraphics[width=1.0cm]{/home/dafu/Schreibtisch/Master-Projekt/Doku/Abb/Modell/it_i0.png}
	%\caption{Bildunterschrift}
\end{wrapfigure}
Die Ermittlung der Austauschstromdichte je Elektrode ($i_{0c}$,$i_{0a}$) erfolgt  nach den in Abschnitt(\ref{subs_Grundl_EL}) erläuterten Grundprinzipien und wird ebenfalls gleichermaßen für sämtliche \gls{el}-typen angewendet.
\newline

\begin{wrapfigure}{l}{1cm}
	%\vspace{-0.8cm}
	\includegraphics[width=1.0cm]{/home/dafu/Schreibtisch/Master-Projekt/Doku/Abb/Modell/it_press.png}
	%\caption{Bildunterschrift}
\end{wrapfigure}
 In die Berechnung von \gls{ael} und \gls{pem}-\gls{el} fließen Partialdruckmodelle für $H_2$ und $O_2$ ein, während dies für die \gls{soel} nicht berücksichtigt werden kann. Die innerhalb der Literatur beschriebenen Modelle weichen je Technologie deutlich voneinander ab und stellen lediglich Näherungslösungen dar.\\
\newline

\begin{wrapfigure}{l}{1cm}
	%\vspace{-0.8cm}
	\includegraphics[width=1.0cm]{/home/dafu/Schreibtisch/Master-Projekt/Doku/Abb/Modell/it_dU.png}
	%\caption{Bildunterschrift}
\end{wrapfigure} 
Hinsichtlich auftretender Überpotentiale ($\eta_{tech}$, mit $tech ~=~AEL,~PEM,~SOEL$) werden Anteile des Elektrolyten ($\eta_{ohm}$) sowie der Kinetik ($\eta_{act}$) in sämtlichen technologiespezifischen Modellen berücksichtigt. Innerhalb des \gls{soel}-Modells werden zusätzlich Konzentrationsüberspannungen ($\eta_{con}$)abgebildet.\\
\newline

%$\blacktriangleright$ Zell-Spannung (Polarisation)
%$\blacktriangleright$ Degradation
\begin{wrapfigure}{l}{1cm}
	%\vspace{-0.8cm}
	\includegraphics[width=1.0cm]{/home/dafu/Schreibtisch/Master-Projekt/Doku/Abb/Modell/it_degr.png}
	%\caption{Bildunterschrift}
\end{wrapfigure} Da die innerhalb der nachfolgenden Betriebsanalyse betrachteten Zeiträume im Bereich von bis zu einem Jahr liegen, ist es sinnvoll, Degradationseffekte zu berücksichtigen. Auf Grundlage der in Abschnitt(\ref{subsub_Zellspann}) beschriebenen Effekte und Modelle wird eine zusätzlich auftretende Degradations-Überspannung, welche in linearer Abhängigkeit der Betriebszeit ansteigt, innerhalb des Überspannungs-Modells berücksichtigt.\\
Aus \cite{Chandesris2015} kann ein zusätzliches Modell der Membrandegradation hinsichtlich deren absoluter Dicke $\delta_mem $ abgeleitet bzw. daraus resultierende Werte mittels Curve-Fit-Methode angewendet werden. 
%<<<<<<Beispiel-Abbildung für 1 Jahr zu Degrad?????



\subsubsection{Stoffströme}
\label{subsubs_mod_stoffstroeme}

%$\blacktriangleright$ 
Die Berechnung auftretender Stoffströme orientiert sich auf Anlagen-Ebene an den unter (\ref{subs_peri_schem}) aufgeführten Modellen.
\begin{wrapfigure}{l}{1cm}
	%\vspace{-0.8cm}
	\includegraphics[width=1.0cm]{/home/dafu/Schreibtisch/Master-Projekt/Doku/Abb/Modell/it_flow.png}
	%\caption{Bildunterschrift}
\end{wrapfigure} Produktgasströme sämtlicher Technologien werden nach den erläuterten Vorschriften in Abschnitt(\ref{subsubb_Kenndaten}) unter Berücksichtigung des Auftritts-Ortes (Kathode, Anode) berechnet.\\
Für die \gls{pem}-\gls{el} ist außerdem ein Membran-Permeationsmodell nach \cite{Tjarks2017} implementiert, welches entsprechende Gas-Durchgänge berücksichtigt, implementiert. Damit können Fremdgas-Kontamination und Verluste abgebildet werden. Innerhalb der \gls{ael}- und \gls{soel}-Modelle existiert zum aktuellen Zeitpunkt keine diesbezügliche Abbildung, wobei unzulässige Produktgaskontaminationen mittels unterere Teillastbegrenzung erzielt werden (Abschnitt(\ref{subsubs_mod_Anlagen-strg})). 

%\subsubsection{Gibbs}
%Enthalpie des Wassers:
%\begin{equation}\label{Enthalpie}
%\Delta H = \Delta G + T\Delta S
%\end{equation}
%Die reversible Zellspannung je Halbreaktion ist definiert nach
%\begin{equation}\label{E_rev}
%E_{rev}=\frac{\Delta G}{n \cdot F}
%\end{equation}
%Mit der Anzahl beteiliger Elektronen $n$ und der Faraday-Konstanten $F=96485,3~C/mol$
%[Schalenbach] setzt folgende Werte ein: $\Delta S=-159,6~J/kg~mol$ und $\Delta H =2,847 \cdot 10^{5}~J/mol$
%-> GibbsFreeEnergy berechnung
%
%\subsubsection{reversible Zellspannung; Nernst-Gleichung}
%Nernst-Spannung für Teilreaktion der Anode mit $E_0 = 0V$ gegen NHE:
%\begin{equation}
%E^{An} = E_0 + \frac{RT}{nF} ln \bigg( \frac{a(H^+)^2 \cdot \sqrt[]{a(O_2)}}{a(H_2O)} \bigg)
%\end{equation}
%
%Nernst-Spannung für Teilreaktion der Anode mit $E_0 = E_{rev}$ gegen NHE:
%\begin{equation}
%E^{Kat} = E^{Kat}_0 + \frac{RT}{nF} ln \bigg( \frac{a(H^+)^2}{a(H_2)} \bigg)
%\end{equation}
%
%\subsubsection{Überspannungen}
%\subsubsection*{Ohmsche}
%\subsubsection*{Diffusive}
%\subsubsection*{Konzentrations-}
%-> HAmann S. 188ff. (! S.191) \cite{Hamann2005}
%
%
%\subsubsection{Degradation}
%
%
%\subsection{Stoffströme}
%
%\subsubsection{AEL - Stoffströme}
%??? Skizze !
%\subsubsection{PEM - Stoffströme}
%Diffusions- und Permeations-Vorgänge vereinfachend (-> linear prop. zu Stromdichte) abgebildet nach [Tjarks, Diss]\\
%eigentlich: 3D-DGL 
%
%\subsubsection*{Schalenbach}
%S.14925!
%S. 14926 -> oben:In this model, the partial pressures of hydrogen in the
%anodic catalyst layer and oxygen in the cathodic catalyst layer
%are assumed to be independent on the gas crossover.
%Furthermore, the influence of pressure on the solubility and
%permeability is neglected, which is below 1% for the assumed
%conditions [27].
%
%bzgl. Partialdruck: (!)
%gas outlets. Since the production rate densities of the gases are
%proportional to the current density (eq (13)), a partial pressure
%enhancement of hydrogen in the cathodic catalyst layer with
%the same dependence in the current density is assumed. The
%
%While gas crossover dominates the Faraday efficiency, the
%voltage efficiency is affected by activation losses, mass
%transport losses and ohmic losses. Furthermore, as shown
%before, pressurized electrolysis enhances the required ther-
%modynamical voltage according to the Nernst equation (eq
%(10)).
%
%membrane influences faraday AND voltage efficiency!!
%
%
%
%\subsubsection{SOEL - Stoffströme}


%\subsection{???Temperatur-Regelung}
%Saisonale Simulation...Außentemperaturanpassung?
%
%Temperatur-Modelle nach Tjarks, Lettenmeier, Ruuskanen
%
%Temp.-Vränderung des internen Wasserkreislaufs durch Pumpenverluste (siehe Abschnitt (\ref{overall-water-managem}))
%
%
%\subsection{???Wasser-Management}
%\label{overall-water-managem}
%Wasserbedarf aus Stoffströmen
%+ Kühlwasser-Zustrom
%
%-> Pumpenwirkungsgrad (Espinoza-Lopez2018)


\subsubsection{Peripherie}
\label{subsubs_mod_Periph}
Zusätzlich zur notwendigen Stackleistung werden außerdem auftretende Leistungsbezüge von Nebenaggregaten ermittelt. Innerhalb der vorliegenden Arbeit werden Fördereinheiten, Überhitzer sowie eine Gastrockung zur Produktgasaufbereitung berücksichtigt. 
Leistungsbezug der Nebenaggregate (Wasserpumpe und Elektrolytpumpe(\gls{ael})) werden mit vereinfacht angenommen, konstanten Druckverlusten, proportional zu Massenströmen nach \cite{KSB2005} berechnet.
\newline


\begin{wrapfigure}{l}{1cm}
	%\vspace{-0.8cm}
	\includegraphics[width=1.0cm]{/home/dafu/Schreibtisch/Master-Projekt/Doku/Abb/Modell/it_pump.png}
	%\caption{Bildunterschrift}
\end{wrapfigure} 
Innerhalb der \gls{nt}-\gls{el} ist das vereinfachte Modell einer Pumpe zur Förderung des Edukt- bzw. Kühlwasserstromes implementiert. Die Berechnung erfolgt nach \cite{KSB2005} %auf Grundlage von Gleichung(???)
, mit einem vereinfachend konstant angenommenen Druckverlust sowie einem konservativ gewählten Gesamtwirkungsgrad.
\newline


\begin{wrapfigure}{l}{1cm}
	%\vspace{-0.8cm}
	\includegraphics[width=1.0cm]{/home/dafu/Schreibtisch/Master-Projekt/Doku/Abb/Modell/it_Pgt.png}
	%\caption{Bildunterschrift}
\end{wrapfigure} Der Leistungsbezug der Produktgasaufbereitung ist abhängig von Massenstrom des Produktgases und Höhe der Reinheitsanforderungen und basiert auf der Arbeit von \cite{Tjarks2017}. Erstere besitzen eine direkte Abhängigkeit vom Anlagenstrom, wobei Permeationsverluste ggf. zu berücksichtigen sind. Letztere werden innerhalb der vorliegenden Arbeit konstant betrachtet. %(Verweis??? ggf. Sensitivität?).\\
\newline
%\subsubsection{Leistungselektronik}

%\subsubsection{Produktgasbehandlung}

%\subsubsection{Elektrolyt-Aufbereitung (AEL)}

%\subsubsection{Gastrocknung}

%\subsubsection*{Peripherie}
\begin{wrapfigure}{l}{1cm}
	%\vspace{-0.8cm}
	\includegraphics[width=1.0cm]{/home/dafu/Schreibtisch/Master-Projekt/Doku/Abb/Modell/it_KOH.png}
	%\caption{Bildunterschrift}
\end{wrapfigure}
Die Peripherie der \gls{ael} beinhaltet zusätzlich eine gesonderte Elektrolyt-Aufbereitung, um eine optimalen $KOH$-Konzentration sicherzustellen. Das Teilmodell ist nach \cite{Abdin.2015} erstellt und beinhaltet zum aktuellen Zeitpunkt lediglich eine weitere Pumpe zur Bewegung des Elektrolyten. Die zusätzliche Leistungsaufnahme wird im Modell in ($P_ec$) berücksichtigt.
\newline

\begin{wrapfigure}{l}{1cm}
	%\vspace{-0.8cm}
	\includegraphics[width=1.0cm]{/home/dafu/Schreibtisch/Master-Projekt/Doku/Abb/Modell/it_Ph.png}
	%\caption{Bildunterschrift}
\end{wrapfigure}
Innerhalb der \gls{ht}-\gls{el} beinhaltet die Peripherie eine elektrische Heizeinrichtung, um den eintretenden Dampfstrom auf Stack-Soll-Temperatur zu erhitzen. 
\vspace{2cm}
\\Weiterhin wird innerhalb der Peripherie der Wirkungsgrad der Leistungselektronik mittels Curve-Fit-Funktion anhand von Literaturwerten für entsprechende Betriebsbereiche berechnet und auf den Stack-Leistungsbezug $P_{st}$ angewandt.
%\subsubsection{Kompression}
%-isentrope Kompression nach |Thermodynamics of pressurized gas storage
%-->> Angaben für isentropen exponent kappa und compressibility factor Z // aktuell gemittelt
%ggf. zu interpolieren
%
%\subsection{???Effizienz}
%Ein Teilziele dieser Arbeit besteht in der Bestimmung eines optimalen Arbeitspunktes für jede auftretende Eingangsleistung.\\
%In eine entsprechende Optimierungsfunktion fließen die Folgenden Größen ein:
%\begin{itemize}
%	\item verfügbare Leistung
%	\item ggf. Zeitraum der verfügbaren Leistung
%	\item produzierte H2-Menge -> Effizienz
%	\item ggf. Arbeitspreis der bezogenen Energie 
%\end{itemize}
%Es wird angestrebt, 

%$\blacktriangleright$ 

\newpage
\subsubsection{Dynamik}
\label{subsubs_mod_Dyn}

%Für die Sicherstellung einer validen Betriebssimulation werden Leistungswerte  von $P~<~0$???????, Leistungsreduktion bei Übertemperatur und Berücksichtigung des 
Die Abbildung eines realistischen Betriebsverhaltens erfolgt mittels Berücksichtigung von maximalen Leistungsgradienten und Mindestleistungswerten.
\begin{wrapfigure}{l}{1cm}
	%\vspace{-0.8cm}
	\includegraphics[width=1.0cm]{/home/dafu/Schreibtisch/Master-Projekt/Doku/Abb/Modell/it_powgrad.png}
	%\caption{Bildunterschrift}
\end{wrapfigure} Ersteres ermöglicht die Abbildung von Einschalt- bzw. Lastwechselverhalten mittels Leistungsrampen. Die Implementierung von Mindestleistungswerten bewirkt das Abschalten der Anlage bei Unterschreiten des entsprechenden Wertes. Sämtliche dafür notwendigen Parameter sind Fachliteratur \cite{Buttler.2018} entnommen und in Tabelle(\ref{tab-param2}) dargestellt.  Im Fall der \gls{ael} liegt der Wert minimaler Leistung verhältnismäßig hoch, was der Verhinderung unzulässig hoher Konzentrationen von $H_2$ in $O_2$ (Produkgaskontamination) \cite{Trinke2018} geschuldet ist (siehe dazu auch Abschnitt(\ref{subsubs_mod_elchem})). Gleiches gilt für die \gls{soel}, wobei zulässige Minimalwerte deutlich niedriger ausfallen.
\newline

%\subsection{???Reaktionszeit/ Dynamik}
%Ein- und Ausschaltvorgänge elektrochemischer Systeme werden maßgeblich durch Doppelschicht- und Diffusionsvorgänge geprägt.
%
%Doppelschichtvorgänge können deshalb durch Totzeiten (???) angenähert werden, während Diffusionsvorgänge konkret modelliert werden müssen.

%\subsubsection{Doppelschichtvorgänge}
%in Form von Rampen / Totzeiten bzgl. d. Wasserstoff-Produktion berücksichtigt

%\subsubsection{Diffusionsvorgänge}
%% % Elektrochemie, Hamann // S. 188 ff.! insbes.: 196...?
%% dazu auch Abdin2015
%% % -> Han2015 !! eq.:12 / 13
%Diffusionsvorgänge unterscheiden sich maßgeblich durch den Zustand des Elektrolytes: bewegt (erzwungene Konvektion) oder ruhend. Während erstere innerhalb sehr kleiner Zeitbereiche Ablaufen ($10^{-1} \dots 1~s$), weisen letztere Zeiten (???stimmt so nicht!!!->nachlesen?????) im Bereich von $30 \dots 60~s$ auf. Die in dieser Arbeit angestrebte zeitliche Auflösung beträgt maximal $>1$ Minute, wodurch Diffusionsvorgänge nicht stationär betrachtet werden können.\\
%
%Diffusionsgrenzstromdichte (für zeitlich konstante Diffusionsschicht-Dicke)
%\begin{equation}
%i_0 = n \cdot F \cdot D \cdot \frac{c^0}{\delta_N}
%\end{equation}
%
%Mit Anzahl ausgetauschter Elektronen $n$, Faraday-Konstante $F$, Diffusionskoeffizient $D$, Ausgangskonzentration Elektrolyt $c_0$ sowie der Dicke der Nernstschen Diffusionsschicht $\delta_N$.


%\subsection{Vereinfachungen / Abweichungen innerhalb von Funktionen}
%->>>>>>>>SImu- Pola
%- PEM -> nur i0 an  berücksichtigt (cath vernachlässigt), da Berechnungen der cath zu Inplausibilitäten führen.
%- Konz-Überspannungen innerhalb AEL und PEM vernachlässigt
%- Membran-Degr. (Dicke) aus Chandesries nur für 2 (konstante) Temp.// zusätzliche  Unsicherheit durch ablesen



\section{Simulation}

Festlegen des inneren Berechnungszeitraumes auf 10 Sekunden.???

Im folgenden Abschnitt werden Validierung der elektrochemischen Modelle, Auswahl der Anlagengrößen für unterschiedliche Betriebsbereiche, entsprechende Parametrisierung sowie Simulationsablauf und Sensitivitätsanalyse beschrieben.

\subsection{Parametrisierung Technologie-spezifisch}
\label{subs_Par_je_TEch}
Keine keine zeitlichen und infrastrukturellen Ressourcen für Experimente vorhanden.
In Ermangelung von Versuchsaufbauten zur Bestimmung von bzw. dem fehlenden Zugriff auf Laborwerten ist lediglich eine Parametrisierung mittels Literaturangaben möglich. Durch eine entsprechende Notwendigkeit, Werte verschiedener Quellen und damit unterschiedlicher Skalierung bzw. Mess-Bedingungen zu verwenden, wird die Aussagekraft vorliegender Ergebnisse eingeschränkt. Einzelne Teilmodelle können häufig nur qualitativ überprüft bzw. verifiziert werden.\\
Dennoch ist dadurch die Grundsätzliche Funktion des Modells nachweisbar. % % verschieben!!

% % Espinoza-Lopez beschreibt PSO (S.167 unten links); dennoch sind experimente notwendig....
% % Goeßling -> Tab 1

\begin{table}[]
	\label{tab-param}
	\caption{Tabelle verwendeter Parameter}
		\begin{tabular*}{\textwidth}{lllllccc}
			
		\multirow{ 2}{*}{ \textbf{Kategorie}} & \multirow{ 2}{*}{\textbf{Größe}} &\multirow{ 2}{*}{\textbf{Symbol}}&\multirow{ 2}{*}{\textbf{Einheit}}& \multicolumn{3}{c}{\textbf{Wert}}&\multirow{ 2}{*}{\textbf{Quelle}}\\
		&&&& AEL & PEM & SOEL&\\
		\hline \hline
		&&&&&&&\\
		&Membrandicke &$\delta_m$& &&&&\cite{Espinosa-Lopez2018}\\
		&Austauschstromdichte&$i_0$&& &&&\cite{Goesling2018}\\
		&&&&&&&\\
		Temp.&Wärmekapazität, Stack&$C_th$&$J/K$&&$162116$&&\cite{Espinosa-Lopez2018}\\
		&Wärmeübergangswiderstand&$k/W$&$0.0668$&$$&$$&$$&\cite{Espinosa-Lopez2018}\\
		&Pumpen-Wirkungsgrad&$\eta_{Pu}$&$$&$$&$$&$$&\\
		&&$$&$$&$$&$$&$$&\\
		&&$$&$$&$$&$$&$$&\\
		&&$$&$$&$$&$$&$$&\\
	\end{tabular*}
\end{table}

\begin{table}[]
	\label{tab-param2}
	\caption{Tabelle verwendeter Parameter}
	\begin{tabular*}{\textwidth}{llllccccc}
		% \multirow{ 2}{*}{\textbf{Quelle}}
		%\multirow{ 2}{*}{ \textbf{Kategorie}} 
		 \multirow{ 2}{*}{\textbf{Größe}} &\multirow{ 2}{*}{\textbf{Symbol}}&\multirow{ 2}{*}{\textbf{Einheit}}& \multicolumn{3}{c}{\textbf{Wert}}&&&\\
		&&& AEL & Quelle&PEM &Quelle& SOEL&Quelle\\
		\hline \hline
		&&&&&&&&\\
		Membrandicke &$\delta_m$& &&\cite{Espinosa-Lopez2018}&&&&\\
		Austauschstromdichte&$i_0$&&&\cite{Goesling2018}&&&&\\
		&&&&&&&&\\
		Wärmekapazität, Stack&$C_th$&$J/K$&&$162116$&&\cite{Espinosa-Lopez2018}&&\\
		Wärmeübergangswiderstand&$k/W$&$0.0668$&$$&$$&$$&\cite{Espinosa-Lopez2018}&&\\
		Pumpen-Wirkungsgrad&$\eta_{Pu}$&$$&$$&$$&$$&&&\\
		
		&$$&$$&$$&$$&$$&&&\\
		&$$&$$&$$&$$&$$&&&\\
		&$$&$$&$$&$$&$$&&&\\
	\end{tabular*}
\end{table}

\subsection{Anlagenskalierung}
\label{subs_Skal_TEch}

\subsubsection{Anlagen Parameter}

\begin{table}[]
	\begin{tabular}{cllll}
		&&AEL&PEM&SOEL\\
		Betriebsart &Größe& &&\\
		\hline \hline
		\multirow{ 4}{*}{ \textbf{EIS}}&&&&\\
		&hallo&&&\\
		&hallo&&&\\
		\multirow{ 4}{*}{ \textbf{RES}}&&&&\\
		\multirow{ 4}{*}{ \textbf{WEA}}&&&&\\
		\multirow{ 12}{*}{ \textbf{COST}}&&&&\\
		&\multirow{ 4}{*}{ \textbf{s}}&&&\\
		&&&&\\
		&\multirow{ 4}{*}{ \textbf{m}}&&&\\
		&\multirow{ 4}{*}{ \textbf{m}}&&&\\
	\end{tabular}
\end{table}

\subsection{Funktionstest und Abgleich der Einzelkomponenten}
\subsubsection{Elektrochemie}
%Zur Validierung der zu Grunde liegenden Modelle erfolgt in diesem Abschnitt die Berechnung der Polarisationskurven (Zellspannung $u_{cell}$über Stromdichte $i$) der betrachteten Technologien.
%Anhand von Variation der Stack-Temperatur $T_{st}$??? sowie der Austauschstromdichte $i_0$??? und Abgleich mit Literaturwerten sind erste Aussagen über die Validität der Modelle ???ableitbar.\\

Zur Validierung der zu Grunde liegenden Modelle erfolgt in diesem Abschnitt die Berechnung der Polarisationskurven (Zellspannung $U_{Zell}$ über Stromdichte $i$) der betrachteten Technologien.
Durch Variation von Stack-Temperatur \gls{T_stack}, Membrandegradation \gls{rDEG} sowie Austauschstromdichte \gls{i0} werden verschiedene Polarisations-Kurvenverläufe aufgenommen und miteinander verglichen. Außerdem erfolgt ein Abgleich mit Literaturwerten. So können Aussagen über die Validität der Einzel-Modelle abgeleitet werden.\\
Auf Grundlage von Literaturangaben werden maximale Stromdichten von  $i=0,3~A/cm^2$ (\gls{ael}),  $i=3,0~A/cm^2$ (\gls{pem}) sowie $i=0,8~A/cm^2$ (\gls{soel}) für sämtliche weiteren Betrachtungen gewählt.
Die Variation der Stack-Temperatur $T_{st}$ erfolgt für die \gls{ael} zwischen $298~K$ und $358~K$, für die \gls{pem}-\gls{el} im Temperaturbereich zwischen $303~K$ und $353~K$ sowie für die \gls{soel} zwischen $973~K$ und $1273~K$.\\
Zusätzlich wird für die \gls{pem}-\gls{el} im gleichen Temperaturbereich der im Modell nach \cite{Tjarks.2017} als konstant angenommene Wert des Elektrolytwiderstand durch einen nach \cite{Chandesris2015} berechneten Wert ersetzt.

Des Weiteren wird für sämtliche Systeme eine Variation der Austauschstromdichten mittels verschiedener Literaturwerte bei konstanter Stacktemperatur durchgeführt.. %Dazu wird der Wert der Austauschstromdichte von Anode und Kathode variiert.
Für die Variation der \gls{ael} stehen ausreichend Literaturwerte zur Verfügung.
Innerhalb der \gls{pem}-\gls{el}-Berechnung findet auf Grund vernachlässigbarer Werte für die Kathoden-Seite ausschließlich die Austauschstromdichte der Anode Berücksichtigung.\\
 Bezüglich der \gls{soel} wird eine Kombination von jeweils zwei Wertepaaren für die Austauschstromdichte der Anode sowie Kathode vorgenommen, woraus vier Werte zur Variation zur Verfügung stehen.
Des Weiteren erfolgt eine Bewertung der Elektrolytdegradation für sämtliche Systeme, indem der Verlauf der Polarisationskurve zu Anfang, Mitte sowie am Ende der jeweils angegebenen Betriebszeiträume berechnet wird.\\


\subsubsection{Auswertung und Analyse der Polarisationskurven}
\label{subsub_polar}

Sämtliche Polarisationskurven weisen unabhängig von veränderten Parametern einen logarithmischen Verlauf der Zellspannung in Abhängigkeit der Stromdichte auf. Weiterhin ist zu erkennen, dass eine Temperaturerhöhung bei \gls{ael} und \gls{pem} eine Parallelverschiebung der Polarisationskurve hin zu niedrigeren Spannungswerten bewirkt. Im Fall der \gls{soel} nimmt die Steigung der Polarisationskurve mit steigender Temperatur ab. Rein qualitativ liegen die Polarisationskurven der \gls{pem} oberhalb des \gls{soel}-Niveaus sowie unterhalb der \gls{ael}-Spannungswerte. Ein ähnliches Verhalten zeigt sich bei der Variation der Austauschstromdichte für die jeweilige Technologie. Mit steigender Austauschstromdichte verschiebt sich die Polarisationskurve von \gls{pem} und \gls{ael} parallel nach unten, wohingegen die Steigung der Polarisationskurve der \gls{soel} mit steigender Austauschstromdichte abnimmt. Die Polarisationskurve für \gls{ael} und \gls{pem} befinden sich im Bereich von etwa $U_{Zell}(0)=1,25~V$ bis etwa $U_{Zell}=2,50~V$. Die \gls{soel} ist deutlich unterhalb der anderen beiden Elektrolysetechnologien.
Die Änderung der Betriebszeit und die damit vorangeschrittene Degradation des Elektrolyten, zeigt für alle Technologien eine Parallelverschiebung der Polarisationskurve nach oben.

\subsubsection*{AEL}
%Abbildung: Polarisationskurve
%Der Betriebsbereich der \gls{ael} wird auf Grundlage von Literaturangaben auf maximale Stromdichten von $0,3~A/cm^2$ begrenzt.
%Für die \gls{ael} erfolgt eine Variation der Stack-Temperatur $T_{st}$??? zwischen $298$ und $358~K$, während die Soll-Betriebstemperatur für weitere Betrachtungen auf $333~K$ festgelegt wird (vgl.Abbildung(\ref{fig:polk_ael_Tvar})). Erwartungsgemäß steigen Zellspannungen mit Zunahme der Stromdichte. Außerdem fällt das Niveau der Zellspannungen mit steigender Temperatur. Auffallend ist, dass die kurven insgesamt relativ flach verlaufen bzw. im unteren Teillastbereich stark ansteigen. In Abgleich mit \cite{Buttler2018} weist dieses Modell tendenziell zu hohe Zellspannungen auf.\\
% Abbildung(\ref{fig:polk_ael_i0var})  sind Polarisationskurven für unterschiedliche Austauschstromdichten [?????]QUELLEN bei Betriebstemperatur dargestellt. Durch die Wahl anderer Faktoren innerhalb der Berechnung 

Die Temperaturabhängigkeit der Polarisationskurve ist in (vgl.Abbildung(\ref{fig:polk_ael_Tvar})) dargestellt. Ab einer Stromdichte von ca. $i=0,05~A/cm^2$ ist die Steigung quasi Konstant. Bei einer Temperatur von $298~K$ steigt die Zellspannung mit zunehmender Stromdichte von ca. $U_{Zell}(0)=~1,5~V$ auf $U_{Zell}(0,3)=2,3~V$ an. Bei einer Temperatur von $358~K$ fällt die Zellspannung in einen Bereich von ca. $U_{Zell}(0)=1,25~V$ auf $U_{Zell}(0,3)=2,1~V$ mit steigender Stromdichte.
Für weitere Betrachtungen ist die Soll-Betriebstemperatur auf $333~K$ festgelegt, was dem mittleren Temperaturniveau der Polarisationskurve aus Abbildung(\ref{fig:polk_ael_Tvar})) entspricht.
\begin{figure}[H]
	\centering
	\begin{minipage}[t]{0.49\textwidth}
		\includegraphics[width=\textwidth]{/home/dafu/Schreibtisch/Master-Projekt/Doku/Abb/Graph/polar/ael/AEL_Polar_Tvar_5.pdf}
		
		\caption[Polarisationskurven der \gls{ael} für variierte Zelltemperaturen]{Polarisationskurven der \gls{ael} für variierte Zelltemperaturen von $298$ bis $358~K$}
		\label{fig:polk_ael_Tvar} 
	\end{minipage}
	\hfill
	\begin{minipage}[t]{0.49\textwidth}
		\includegraphics[width=\textwidth]{/home/dafu/Schreibtisch/Master-Projekt/Doku/Abb/Graph/polar/ael/AEL_Polar__AEL_polarsens_i0_at609.pdf}
		
		\caption[Polarisationskurven der \gls{ael} für variierte Austauschstromdichten]{Polarisationskurven der \gls{ael} für variierte Austauschstromdichten bei Betriebstemperatur und Umgebungsdruck}
		\label{fig:polk_ael_i0var}  
	\end{minipage}
\end{figure}
Der Einfluss der Austauschstromdichte auf die gesamte Polarisationskurve wird mit Hilfe von Abbildung(\ref{fig:polk_ael_i0var}) beschrieben. Im Fall der \gls{ael} wurden dafür vier verschiedene Austauschstromdichten bestehend aus jeweils einem Wert für die Austauschstromdichte der Anode und Kathode gewählt.\\ Die anodischen Austauschstromdichten liegen im Bereich von $10^{(-6)}$ bis $10^{-(12)}~A/cm^2$ und die der Kathode von $10^{(-3)}$ bis $10^{-(5)}~A/cm^2$. Im Fall niedriger Austauschstromdichten der Anode von $i_{0,An}=1,1 \cdot10^{-(12)}~A/cm^2$ und Kathode $i_{0,Ka}=9,4 \cdot10^{-(5)}~A/cm^2$ ergibt sich ein Zellspannungsbereich von ca. $U_{Zell}(0)=1,5$ bis $2,5~V$ bei $i=0,3~A$, für Austauschstromdichten der Anode von $i_{0,An}=7,0 \cdot10^{-(6)}~A/cm^2$ und Kathode $i_{0,Ka}=4,0 \cdot10^{-(3)}~A/cm^2$ liegt ein Zellspannungsbereich von ca. $U_{Zell}(0)=1,0~V$ bis etwa $U_{Zell}(0,3)=1,8~V$ vor.

Zu beginn der Elektrolyse, wenn noch keine Degradationserscheinungen (vgl. Abbildung(\ref{fig:polk_ael_tmemvar})aufgetreten sind, liegt die Zellspannungskurve im Bereich von  $U_{Zell}(0)=1,75~V$ bis $U_{Zell}(0,3)=2,25~V$. Die höchsten Werte erreicht die Polarisationskurve nach einer Betriebszeit von $t_{OP}=120.000~h$ mit einem Spannungsbereich von ca. $U_{Zell}(0)=1,90~V$ bis $U_{Zell}(0,3)=2,40~V$
\\

\begin{figure}[H]
	
	\centering
	%\includegraphics[width=0.49\textwidth]{/home/dafu/Schreibtisch/Master-Projekt/Doku/Abb/test.png}
	\includegraphics[height =6cm]{/home/dafu/Schreibtisch/Master-Projekt/Doku/Abb/Graph/polar/ael/AEL_Polar_tmemvar_6.pdf}
	\caption[Polarisationskurven der \gls{ael} für variierte absolute Betriebszeiten]{Polarisationskurven der \gls{ael} für variierte absolute Betriebszeiten von $0$ bis $120000~h$ bei Betriebstemperatur und Umgebungsdruck}
	\label{fig:polk_ael_tmemvar} 
\end{figure}


\subsubsection*{PEM}
%Abbildung: Polarisationskurve
%Zur Validierung des \gls{pem}-Verhaltens erfolgt eine Auftragung von Werten nach oben genanntem Schema. Für die folgende Analayse wird der Betriebsbereich auf Stromdichten von $3~A/cm^2$ und Temperaturen von maximal $358~K$ (Soll-Temperatur: $353~K$) begrenzt.\\
%Die erwartbare basis-Charakteristig ( steigende Zellspannung mit Stromdichte sowie fallendes Spannungsniveau mit steigender Temperatur) ist vorhanden, wobei der Temperatureinfluss weniger deutlich als bei der \gls{ael} ausfällt. Auftretende Zellspannungen liegen (auch in Abgleich mit \cite{Buttler2018}) auf sehr niedrigem Niveau. 
%- starke Abweichung von Espinoza Lopez!!!
%Erklärungsansätze sind diesbezüglich das unzureichende Elektrolytmodell bzw. der entsprechend verwendete feststehende Faktor, wodurch eine Temperaturabhängigkeit entfällt ???sowie der Betrieb auf Atmosphärischem Druckniveau.
%Unter Verwendung des vorliegenden elektrochemischen Modells ist absehbar mit hoher Anlagenreferenz zu rechnen.\\
%In Abbildung(\ref{fig:polk_PEM_i0var}) ist ebenfalls der Einfluss variierter Austauschstromdichten dargestellt. Im Unterschied zum verwendeten Modell, sind hier jedoch Festwerte für eine Betriebstemperatur von $353~K$ aus \cite{Espinosa-Lopez2018} entnommen. Es wird deutlich, dass zumindest für Soll-Temperatur, der berechnete und verwendete Wert für $i_0$ zu deutlich niedrigeren Zellspannungen führt, als weiterer Literaturwerte. 
%--- variables i0 aber gewünscht!
%------>>>>>> Abbildung zu Rmem-calc!!!!!!

%Abbildung: Polarisationskurve


Zur Validierung des \gls{pem}-Verhaltens wird zunächst, wie in Abbildung(\ref{fig:polk_PEM_Tvar}) dargestellt, das Temperaturverhalten der Polarisationskurve dargestellt.
%
Sämtliche folgende Simulationsdurchläufe werden bei Soll-Betriebstemperatur von $T_op~=~353~K$ bzw. $T_{max} ~=~ 358~K$ durchgeführt.\\
Ab einer Stromdichte von etwa $i=0,05~A/cm^2$ gehen die Polarisationskurven-Verläufe der \gls{pem}-\gls{el} in eine Konstante Steigung über. Der Gesamtbereich über den sich die Polarisationskurven verteilen beginnend bei einer Temperatur von $303~K$ mit Zellspannung von ca. $U_{Zell}(0)=1,25~V$ und steigt über die zunehmende Stromdichte bis auf einen Wert von etwa $U_{Zell}(3)=1,8~V$ an. Bei einer Temperatur von $353~K$ fällt die Zellspannung mit steigender Stromdichte auf einen Bereich von ca. $U_{Zell}(0)=1,20~V$ bis $U_{Zell}(3)=1,75~V$.\\
 Abbildung(\ref{fig:polk_PEM_Tvar_ReleCalc}) zeigt die Polarisationskurven, für einen nach \cite{Chandesris2015} berechneten Wert des Elektrolyt
widerstandes. Wie im ersten Fall liegen Zellspannungen bei niedrigen Stromdichten auf dem gleichen Niveau, allerdings liegt eine deutlich niedrigere Steigung vor, wodurch Spannungen bei einer Stromdichte von $i=3~A/cm^2$ im Bereich von $U_{Zell}=1,4$ bis $1,6~V$ liegen.\\
In Abbildung(\ref{fig:polk_PEM_i0var}) wird der Einfluss variierter Austauschstromdichten dargestellt. Im Unterschied zum verwendeten Modell, sind hier jedoch Festwerte für eine Betriebstemperatur von $353~K$ aus \cite{Espinosa-Lopez2018} entnommen.
\begin{figure}[!tbp]
	\centering
	\begin{minipage}[t]{0.49\textwidth}
		\includegraphics[width=\textwidth]{/home/dafu/Schreibtisch/Master-Projekt/Doku/Abb/Graph/polar/pem/PEM_Polar__polar_plaus_base13.pdf}
		
		\caption[Polarisationskurven der \gls{pem}-\gls{el} für variierte Betriebstemperaturen]{Polarisationskurven der \gls{pem}-\gls{el} für variierte Zelltemperaturen von $303$ bis $353~K$}
		\label{fig:polk_PEM_Tvar} 
	\end{minipage}
	\hfill
	\begin{minipage}[t]{0.49\textwidth}
		\includegraphics[width=\textwidth]{/home/dafu/Schreibtisch/Master-Projekt/Doku/Abb/Graph/polar/pem/PEM_Polar_i0var_polar_plaus_R_ele_calc16.pdf}
		\caption[Polarisationskurven der \gls{pem}-\gls{el} für variierte Austauschstromdichten]{Polarisationskurven der \gls{pem}-\gls{el} für variierte Austauschstromdichten bei Betriebstemperatur und Umgebungsdruck}
		\label{fig:polk_PEM_i0var} 
	\end{minipage}
\end{figure}
Der Verlauf der Polarisationskurve beginnt mit einer Austauschstromdichte der Anode von $i_{0,An}=2,50\cdot10^{(-12)}~A/cm^2 $ bei ca. $U_{zell}(0)=~1,75~V$ und steigt auf einen Wert von bis zu $2,20~V$ bei $i=3~A/cm^2$. Für niedrige Austauschstromdichten sinkt die Zellspannung auf Werte zwischen  $U_{Zell}(0)=1,30$ und $U_{Zell}(3)=1,75~V$.\\
Auch die \gls{pem}-\gls{el} weist für unterschiedlich angenommene Betriebszeiten und entsprechende Zunahme der Degradationsüberspannungen deutliche Unterschiede auf, welche in Abbildung(\ref{fig:polk_PEM_tmemvar}) dargestellt werden. Zu beginn der Elektrolyse, wenn noch keine Degradationserscheinungen aufgetreten sind, liegt die Zellspannungskurve im Bereich von  $U_{Zell}(0)=1,50~V$ bis $U_{Zell}(3)=1,95~V$. Die höchsten Werte werden nach einer Betriebszeit von $t_{OP}=100.000~h$ mit einem Spannungsbereich von ca. $U_{Zell}(0)=2,0~V$ bis $U_{Zell}(3)=2,40~V$ erreicht.

\begin{figure}[!tbp]
	\centering
	\begin{minipage}[t]{0.49\textwidth}
		\includegraphics[width=\textwidth]{/home/dafu/Schreibtisch/Master-Projekt/Doku/Abb/Graph/polar/pem/PEM_Polar__polar_plaus_R_ele_calc15.pdf}
		
		\caption[Beispielhafte Darstellung von Polarisationskurven der \gls{pem}-\gls{el}bzgl.Elektrolyt-Widerstand]{Beispielhafte Darstellung von Polarisationskurven der \gls{pem}-\gls{el} mit nach \cite{Chandesris2015} berechnetem Elektrolyt-Widerstand und variierter Zelltemperaturen von $303$ bis $353~K$}
		\label{fig:polk_PEM_Tvar_ReleCalc} 
	\end{minipage}
	\hfill
	\begin{minipage}[t]{0.49\textwidth}
		\includegraphics[width=\textwidth]{/home/dafu/Schreibtisch/Master-Projekt/Doku/Abb/Graph/polar/pem/PEM_Polar__polar_plaus_t_mem_minmittelmax4_new22_2.pdf}
		
		\caption[Polarisationskurven der \gls{pem}-\gls{el} für variierte absolute Betriebszeiten]{Polarisationskurven der \gls{pem}-\gls{el} für variierte absolute Betriebszeiten von $0$ bis $100000~h$ bei Betriebstemperatur und Umgebungsdruck}
		\label{fig:polk_PEM_tmemvar} 
	\end{minipage}
\end{figure}



\subsubsection*{SOEL}

Die Auswirkung der Temperaturänderung auf die Polarisationskurve der \gls{soel} ist in Abbildung (\ref{fig:polk_SOEL_Tvar}) dargestellt.\\
Im Bereich kleiner Stromdichten liegt die Zellspannung, für sämtliche Temperaturen, im Bereich von $U_{zell}=1,15~V$ und steigt bei $i=0,8~A/cm^2$ auf Spannungswerte im Bereich von $U_{zell}=1,25~V$ bis $1,50~V$ an, wobei die höchste Temperatur den niedrigsten Zellspannungswert aufweist.\\
\begin{figure}[H]
	\centering
	\begin{minipage}[t]{0.49\textwidth}
		\includegraphics[width=\textwidth]{/home/dafu/Schreibtisch/Master-Projekt/Doku/Abb/Graph/polar/soel/SOEL_Polar__SOEL_polarsensitivity_700-1000_tmem=0h_next18.pdf}
		
		\caption[Polarisationskurven der \gls{pem}-\gls{el} für variierte Zelltemperaturen]{Polarisationskurven der \gls{pem}-\gls{el} für variierte Zelltemperaturen von $973$ bis $1273~K$}
		\label{fig:polk_SOEL_Tvar} 
	\end{minipage}
	\hfill
	\begin{minipage}[t]{0.49\textwidth}
		\includegraphics[width=\textwidth]{/home/dafu/Schreibtisch/Master-Projekt/Doku/Abb/Graph/polar/soel/SOEL_Polar__SOEL_polarsensitivity_T850_tmem0h_4500h_9000h_20000h_next20.pdf}
		
		\caption[Polarisationskurven der \gls{soel} für variierte Austauschstromdichten]{Polarisationskurven der \gls{soel} für variierte Austauschstromdichten bei Betriebstemperatur und Umgebungsdruck}
		\label{fig:polk_SOEL_i0var} 
	\end{minipage}
\end{figure}
%Wie bei den beiden anderen Technologien wurde anhand der Temperaturverläufe für die \gls{soel} die Betriebstemperatur auf einen Wert von $T_{OP}=1123K$ fetgelegt. 
Für die Betrachtung unterschiedlicher Austauschstromdichten (Abbildung(\ref{fig:polk_SOEL_i0var})) wird eine Betriebstemperatur von $T_{OP}=1123K$ festgelegt. Für sämtliche Austauschstromdichten und  im Fall niedriger Werte von $i$ weist die Zellspannung Werte im Bereich von $U_{zell}=1,15~V$ auf. Mit dem Anstieg der Stromdichte auf $i=0,8~A/cm^2$ steigt die Polarisationspannung auf einen Bereich von $U_{zell}=1,25$ bis $1,35~V$ an. Dabei weist die Kombination des Wertepaares aus anodischer Austauschstromdichte $(A_1)$ und kathodischer Austauschstromdichte $(C_2)$ einen höchsten Wert der gewählten Kombinationen auf. 
\begin{figure}[H]
	
	\centering
	%\includegraphics[width=0.49\textwidth]{/home/dafu/Schreibtisch/Master-Projekt/Doku/Abb/test.png}
	\includegraphics[height =6cm]{/home/dafu/Schreibtisch/Master-Projekt/Doku/Abb/Graph/polar/soel/SOEL_Polar__SOEL_polarsensitivity_T850tmem0h_4500h_9000h_20000h_next19.pdf}
	\caption[Polarisationskurven der \gls{soel} für variierte absolute Betriebszeiten]{Polarisationskurven der \gls{soel} für variierte absolute Betriebszeiten von $0$ bis $20000~h$ bei Betriebstemperatur und Umgebungsdruck}
	\label{fig:polk_SOEL_tmemvar} 
\end{figure}
% während die Kombination aus anodischer Austauschstromdichte $(A_2)$ und kathodischer Austauschstromdichte $(C_2)$ den niedrigen Wert annimmt.
%Das Verhalten der Polarisationskurve entspricht dem der vorangegangenen Elektrolysetechnologien.
Das Degradationsverhalten der \gls{soel} ist in Abbildung(\ref{fig:polk_SOEL_tmemvar}) dargestellt. Zu Beginn der absoluten Betriebszeit und somit ohne Einfluss der Degradationserscheinungen liegt die Zellspannung im Bereich kleiner Stromdichten bei $U_{Zell}(0)=1,15~V$. Bei einer Stromdichte von $i=0,8~A/cm^2$ steigt die Zellspannung auf einen Bereich von $U_{Zell}(0)=1,30~V$. Mit zunehmender Betriebszeit ist ein deutlicher Spannungsanstieg für sämtliche Stromdichten festzustellen. Die höchsten Überspannungen erreicht die \gls{soel} nach einer Betriebszeit von $t_{OP}=20.000~h$ mit einer Zellspannung $U_{Zell}(0,8)= 1,70~V$ bei einer Stromdichte von $i=0,8~A/cm^2$ auf.\\

%Abbildung: Polarisationskurve
%Auch für die \gls{soel} werden Variationen von Temperatur und Austauschstromdichte durchgeführt und betrachtet, wobei hier der Temperaturbereich entsprechend höher ($973$ bis $1273~K$) ausfällt. Der Betriebsbereich wird im Folgenden auf Stromdichten bis $0,8~A/cm^2$ begrenzt.
%Die sich einstellende Kurvenform für Ergebnisse der Temperaturvariation weicht von den charakteristischen Formen beider anderen Technologien ab. Literaturwerte aus \cite{Buttler2018} bestätigen Form und Niveau, wobei sich Werte des verwendeten Modells im oberen Drittel und damit tendentiell im niedrigen Effizienzbereich der aufgeführten ??? bewegt.\\
%Für die Variation von der Austauschstromdichte werden folgende Kombinationen von k0????? und Aktivierungsenergie verwendet:
%---?????????\\
%Variationen weisen nur wenig Auswirkungen auf entsprechende Zellspannungen auf. Zellspannungen variieren lediglich für hohe Stromdichten um maximal $0,05~V$.?????


%#######################################




%\subsubsection{Thermo-Verhalten}
%?
%\subsubsection{chrakteristisches Verhalten (Doublet-Test)}
%\subsubsection*{AEL}
%Abbildung: Doublet- oder charakteristischer Test
%\begin{figure}[H]
%	
%	\centering
%	%\includegraphics[width=0.49\textwidth]{/home/dafu/Schreibtisch/Master-Projekt/Doku/Abb/ .png}
%	\includegraphics[width=0.8\textwidth]{/home/dafu/Schreibtisch/Master-Projekt/Doku/Abb/test.png}
%	\caption{ }
%	\label{fig:doublet_AEL} 
%\end{figure}
%\subsubsection*{PEM}
%Abbildung: Doublet- oder charakteristischer Test
%\begin{figure}[H]
%	
%	\centering
%	%\includegraphics[width=0.49\textwidth]{/home/dafu/Schreibtisch/Master-Projekt/Doku/Abb/ .png}
%	\includegraphics[width=0.8\textwidth]{/home/dafu/Schreibtisch/Master-Projekt/Doku/Abb/test.png}
%	\caption{ }
%	\label{fig:doublet_PEM} 
%\end{figure}
%\subsubsection*{SOEL}
%Abbildung: Doublet- oder charakteristischer Test
%
%\begin{figure}[h]
%	
%	\centering
%	%\includegraphics[width=0.49\textwidth]{/home/dafu/Schreibtisch/Master-Projekt/Doku/Abb/ .png}
%	\includegraphics[width=0.8\textwidth]{/home/dafu/Schreibtisch/Master-Projekt/Doku/Abb/test.png}
%	\caption{ }
%	\label{fig:doublet_SOEL} 
%\end{figure}
\subsubsection*{Analyse}
Innerhalb der Temperaturvariation ist für sämtliche Technologien eine deutliche Temperaturabhängigkeit zu erkennen. Für die \gls{ael} ist auffallend, dass ermittelte Polarisationskurven insgesamt relativ flach verlaufen, wobei diese im unteren Teillastbereich stark ansteigen. In Abgleich mit \cite{Buttler.2018}
Hinsichtlich der \gls{pem}-\gls{el} weisen entsprechende Polarisationskurven bei vergleichbarer Temperatur einen insgesamt sehr viel flacheren Verlauf auf.
Die erwartbare Basis-Charakteristik (steigende Zellspannung mit Stromdichte sowie fallendes Spannungsniveau mit steigender Temperatur) ist vorhanden, wobei die Spannungsveränderung weniger deutlich als bei der \gls{ael} ausfällt. Auftretende Zellspannungen liegen (auch in Abgleich mit \cite{Buttler.2018}) auf sehr niedrigem Niveau. Für den Fall, dass der Elektrolytwiderstand nicht konstant gewählt, sondern rechnerisch ermittelt wird, sinken Verläufe der Polarisationskurven zusätzlich, was als unrealistisches Verhalten interpretiert und damit von weiteren Betrachtungen ausgeschlossen wird.\\ 
Erklärungsansätze bezüglich der dennoch sehr niedrigen Polarisationsspannungen sind ggf. trotzdem das unzureichende Elektrolytmodell, da hierdurch eine diesbezügliche Temperaturabhängigkeit entfällt sowie der Betrieb auf Atmosphärischem Druckniveau.\\
Unter Verwendung des vorliegenden elektrochemischen Modells ist absehbar mit einer hohen Anlageneffizienz zu rechnen.\\
Es wird deutlich, dass zumindest für die Soll-Temperatur der berechnete und verwendete Wert für $i_0$ zu deutlich niedrigeren Zellspannungen führt, als dies für alternative, jedoch konstante Literaturwerte der Fall ist.\\ 


Die sich einstellende Kurvenform der \gls{soel}-Polarisationsspannungen weichen für Ergebnisse der Temperaturvariation von den charakteristischen Formen beider anderer Technologien ab. Literaturwerte aus \cite{Buttler.2018} bestätigen Form und Niveau der ermittelten Kurven, wobei sich Werte des verwendeten Modells im oberen Drittel und damit tendenziell im niedrigen Effizienzbereich der aufgeführten Systeme bewegt.\\
Variationen der Austauschstromdichte weisen lediglich im Berecih hoher Stromdichten leicht steigende Spannungswerte gegenüber dem verwendeten Basiswert auf. Zellspannungen variieren lediglich für hohe Stromdichten um maximal $0,05~V$.\\
Dagegen ist der Einfluss degradationsbedingter Überspannungen (in Abbildung(\ref{fig:polk_SOEL_tmemvar})) deutlich erkennbar und auf verhältnismäßig hohe Degradationsraten zurückzuführen.\\
Insgesamt treten für sämtliche Modelle Abweichungen zu mittleren Literaturwerten auf. Dies ist unter anderem auf die ausschließlich aus verschiedenen Literaturquellen zusammengetragenen Parameter und Berechnungsansätze zurückzuführen. Viele Modelle der Literatur werden mittels experimenteller Parameterermittlung optimiert bzw. validiert. Dennoch ist eine grundsätzliche Validität der verwendeten Modelle gegeben. In weiterführenden Anlaysen sind jedoch entsprechend beschriebene Abweichungen zu berücksichtigen. 
\newpage
\subsection{Anlagenskalierung}
\label{subs_ANL_SKAL}
Auf Grundlage der in Abschnitt(\ref{sect_Betriebsarten}) dargestellten Leistungsdatensätze werden im Folgenden verschiedene Zeitreihen bzw. Betriebsszenarien abgebildet. Dazu ist eine sinnvolle Anlagenskalierung je Betriebsart und Technologietyp erforderlich.\\
Zu diesem Zweck werden aus den o.g. Datensätzen je Betriebsart charakteristische Kurven errechnet, welche eine optische, tendenziell qualitative Auslegung ermöglicht. Aus entsprechenden Leistungs- bzw. Zeitreihen können für verschiedene (potentielle) Nennleistungswerte mögliche Energieumsätze sowie erreichbare Volllaststunden ermittelt werden. Auf dieser Grundlage ist ein können (indirekt) perspektivische Produktionsmenge und jährliche Volllaststunden für ausgewählte Nennleistungswerte abgelesen werden. Die zu Grunde gelegte Prämisse ist innerhalb dieser Arbeit eine möglichst hohe Nutzung verfügbarer Energiemengen, ohne dabei unnötig hohe Anlagenleistungen bzw. Installationkosten zu erfordern.   
\subsubsection{EIS-Skalierung}
In Abbildung(\ref{fig:Skal_EIS}) sind für das Jahr 2017 nutzbare Energiemengen sowie erzielbare Volllaststunden in Abhängigkeit verschiedener Anlagenleistungen auf Grundlage des EISMAN-Datensatzes dargestellt.
\begin{figure}[H]
	
	\centering
	%\includegraphics[width=0.49\textwidth]{/home/dafu/Schreibtisch/Master-Projekt/Doku/Abb/ .png}
	\includegraphics[width=0.7\textwidth]{/home/dafu/Schreibtisch/Master-Projekt/Doku/Abb/Graph/jdl/JDL_EIS_scale-marker_new-range_scal___12.pdf}
	\caption[Anlagenskalierung für EISMAN-Betrieb]{Anlagenskalierung für EISMAN-Betrieb; Umsetzbare Energiemenge \gls{umgesEnergie} (linke Ordinate) sowie erreichbare Volllaststunden \gls{volllaststd} (rechte Ordinate) über installierte Anlagenleistung ; Markierungslinien für $23475~kW$}
	\label{fig:Skal_EIS} 
\end{figure}
Die dargestellten Werte in o.g. Abbildung vermitteln auf den ersten Blick den Eindruck eines groben Berechnungsfehlers. In Abgleich mit den Abbildungen(\ref{fig:EIS_roh_betrART},\ref{fig:EIS_JDL_betrART}) wird jedoch deutlich, wie diese Werte zustande kommen: Da die EISMAN-Zeitreihe nahezu ausschließlich Werte in sehr hohen Leistungsbereichen, oder aber Leistungswerte von $P_{EIS}~=~0$ aufweist und darüber hinaus Zeitintervalle auftretender Leistungen $>0$ im Schnitt sehr kurz ausfallen, entsteht ein scharfer $P/E$-Zusammenhang. Analog entstehen für entsprechende Betriebszeiten bzw. Volllaststunden kaum Veränderungen bei Variation der installierten Leistung. Ab einer (potentiell) installierten Leistung von $P_{inst,EIS} ~=~23300~ kW$ ist auch mit zusätzlicher Anlagenleistung kaum noch eine Steigerung umgesetzter Energiemengen möglich. Für gewählte Leistungswerte, welche unter diesem Wert liegen, ergeben sich lediglich leichte Einbußen ($< ~5~\%$). Für nachfolgende Simulationen wird ein Leistungs-Richtwert von $23500~kW$ festgelegt. 
%????? warum nicht 23350????

\subsubsection{RES-Skalierung}
Im Verhältnis zur zuvor betrachteten EISMAN-Zeitreihe weisen entsprechende Leistungswerte des RESIDUAL-Datensatzes eine erwartbarere Charakteristik auf. In Abbildung(\ref{fig:Skal_RES}) ist das nach oben erläuterten Prinzipien erstellte Skalierungs-Diagramm für den RESIDUAL-Simulatonsfall dargestellt. Im unteren Drittel des potentiellen Leistungsbereiches weist die Kurve bezüglich umsetzbarer Energiemengen eine höhere Steigung auf, als die einhergehende Reduktion der Volllaststunden. Mit zunehmender, installierter Anlagenleistung nimmt die (relative) Steigerung umsetzbarer Energiemengen kontinuierlich ab. Der für die im Folgenden durchzuführenden Simulationen gewählte Leistungswert wird im oberen Drittel des Wertebereichs angesetzt, wodurch etwa $90~\%$ der verfügbaren Energiemenge genutzt werden kann, ohne in einen Bereich unnötig hoher Leistungswerte zu geraten. Entsprechend absehbare Volllaststunden fallen allerdings niedrig aus.
Die installierte Anlagenleistung des RESIDUAL-Datensatzes wird auf $P_{inst,RES} ~=~1500~kW$ festgelegt.
\begin{figure}[H]
	
	\centering
	%\includegraphics[width=0.49\textwidth]{/home/dafu/Schreibtisch/Master-Projekt/Doku/Abb/ .png}
	\includegraphics[width=0.7\textwidth]{/home/dafu/Schreibtisch/Master-Projekt/Doku/Abb/Graph/jdl/JDL_RES_scale-marker__02.pdf}
	\caption[Anlagenskalierung für RESIDUAL-Betrieb]{Anlagenskalierung für RESIDUAL-Betrieb; Umsetzbare Energiemenge \gls{umgesEnergie} (linke Ordinate) sowie erreichbare Volllaststunden \gls{volllaststd} (rechte Ordinate) über installierte Anlagenleistung; Markierungslinien für $1500~kW$}
	\label{fig:Skal_RES} 
\end{figure}
\newpage
\subsubsection{WEA-Skalierung}
Die resultierende Leistungskurve des WEA-Datensatzes weist eine bezüglich des bereits erläuterte RESIDUAL-Signals ähnliche $E/P$-Charakteristik auf.
zu beachten ist, dass der Schnittpunkt beider Kurven keine Aussage beinhaltet, sondern lediglich auf Grund der gewählten Achsenskalierung erzeugt wird.
 Wohingegen die Kurve entsprechender Volllaststunden insgesamt eine deutlich stärkere Krümmung innerhalb eines größeren Leistungsspektrums aufweist. Da in diesem Fall schon ab Leistungswerten im mittleren Bereich des Gesamtspektrums nur geringe Zugewinne umsetzbarer Energiemengen absehbar sind, gleichzeitig jedoch die Zahl erreichbarer Volllaststunden deutlich sinkt, wird eine Anlagenleistung von $P_{inst,WEA} ~=~6000~kW$ festgelegt.
\begin{figure}[H]
	
	\centering
	%\includegraphics[width=0.49\textwidth]{/home/dafu/Schreibtisch/Master-Projekt/Doku/Abb/ .png}
	\includegraphics[width=0.7\textwidth]{/home/dafu/Schreibtisch/Master-Projekt/Doku/Abb/Graph/jdl/JDL_WEA_scale-marker__02.pdf}

	\caption[Anlagenskalierung für WEA-Betrieb]{Anlagenskalierung für WEA-Betrieb; Umsetzbare Energiemenge \gls{umgesEnergie} (linke Ordinate) sowie erreichbare Volllaststunden \gls{volllaststd} (rechte Ordinate) über installierte Anlagenleistung; Markierungslinien für $6000~kW$}
	\label{fig:Skal_WEA} 
\end{figure}

%\subsection{Skalierungsergebnisse und Eingangsparameter für Simulation}
%\begin{table}
%	\begin{tabular}{ll|l|l|l|l|l|l}%
%	
%	
%	&&{\bfseries EIS}&{\bfseries RES}&{\bfseries COST S}&{\bfseries COST M	}&{\bfseries COST L} \hline\hline
%	&&&&&&\\
%	& 	geplante Anlagenleistung in W		& 23.500.000		&	 1500.000	& 6.000.000		& 10.000		&180.000	&     1.000.000\\
%	&&&&&&\\\hline
%	&&&&&&\\
%	&Leistung in W	&	23.461.755		&1.502.039		&6.003.356		&	9.998		&18.2356	&1.002.959	\\
%	&&&&&&\\
%	{\textbf{AEL} 	&Zellen	&117.336	&7.512	&30.024	&50&	912	&5.016\\
%	&&&&&&\\
%	&Anzahl Stacks	&4.889	&313	&1.251	&&	38	&209\\
%	&&&&&&\\\hline
%	&&&&&&\\
%	&Leistung in kW	&23.571.433	&1.619.411	&6.117.776	&9.896	&179.934	&1.079.667\\
%	&&&&&&\\
%	{\textbf{PEM} &Zellen&	26.200&	1.800&	6.800&	11&	200	&1.200\\
%	&&&&&&\\
%	&Anzahl Stacks&	131	&9	&34	&&	1	&6\\
%	&&&&&&\\ \hline
%	&&&&&&\\
%	&Leistung in kW	&23.537.523	&1.502.398	&6.009.593	&10016&	208.666	&1.001.599\\
%	&&&&&&\\.
%	{\textbf{SOEL}	&Zellen	&225.600	&14.400	&57.603	&96&	2.000	&9.600\\
%	&&&&&&\\
%	&Anzahl Stacks	&564	&36	&144.0075	&&	5	&24\\
%	&&&&&&\\
%	
%	\end{tabular}
%\end{table}

\newpage
\subsection{Simulationsdurchläufe und Plausibilitätsbetrachtung}
Innerhalb dieses Abschnitts erfolgt eine Darstellung der Simulationsergebnisse für wichtige Berechnungsgrößen der drei betrachteten \gls{el}-Technologien. Diesbezügliche Abbildungen zeigen die über den in der vorliegenden Arbeit maximal gewählten Zeitraum von einem Jahr je Eingangsdatensatz ermittelten Werte. %Während in diesem Abschnitt lediglich eine reine Abbildung und BEschreibung 
Sämtliche Simulationsverläufe werden in identisch gegliederten Diagrammen dargestellt. Eine Abbildung enthält jeweils vier untereinander angeordnete Diagramme mit je zwei Ordinaten mit (i.d.R.) unterschiedlicher Skalierung. Sämtliche Diagramme besitzen eine identische Abszissen-Skalierung (1 Jahr in Sekunden), welche lediglich für das unterste Diagramm dargestellt wird. Auf der linken Ordinate werden (von oben nach unten) Stacktemperatur \gls{T_stack}, Zellspannung \gls{u_cell}, verfügbare \gls{P_in} und tatsächliche Anlagenleistung \gls{P_act} sowie die Stoffströme der Produktgase \gls{n_H2} und \gls{n_O2} dargestellt.
Mit Bezug zur rechten Ordinate werden Kühlwassermassenstrom (bzw. Wasserbezug im Fall der \gls{soel}), Stromdichte \gls{i}, Stackleistung \gls{P_st} sowie die Produktgaskontamination von $H_2$ in $O_2$ dargestellt.
Grundsätzlich fällt auf, dass eine Zeitreihe dieser Größenordnung ohne die Möglichkeit einer höheren Auflösung von Teilbereichen lediglich mäßige Aussagekraft besitzt und in erster Linie zur Darstellung der Programm-Funktionsweise dienen kann. Wichtige Ergebnisse und Ableitungen werden im Anschluss auf Grundlage übergeordneter Kennwerte getroffen.\\ %VERWEIS!!!
Aus diesem Grund werden an dieser Stelle lediglich exemplarische Simulationsergebnisse dargestellt, während verbleibende Abbildungen in den Anhang verschoben werden.
  
\subsubsection{Simulationsdurchlauf: EIS-AEL}
In Abbildung(\ref{fig:plt_se_EIS_AEL}) sind exemplarisch für die EISMAN-Zeitreihe Simulationsergebnisse der \gls{ael} mit dargestellt. 
\begin{figure}[H]
	\centering
	\includegraphics[width=\textwidth]{/home/dafu/Schreibtisch/Master-Projekt/Doku/Abb/Graph/eis_se/EIS_AEL__EIS_se__.pdf}
	
	\caption[Simulation EISMAN im Jahresbetrieb \gls{ael}]{Simulation EISMAN im Jahresbetrieb \gls{ael}; Darstellung von Stack-Temperatur \gls{T_stack} und Kühlwassermassenstrom \gls{m_c} (oben), Zellspannung \gls{u_cell} und Zellstrom \gls{i}(2. v.o.), Eingangsleistung \gls{P_in},  tatsächlich abgerufene Leistung \gls{P_act} und Stackleistung \gls{P_st} (2. v. u.) sowie molare Produktgasströme \gls{n_H2},\gls{n_O2} und Produktgaskontamination \gls{H2inO2}}
	\label{fig:plt_se_EIS_AEL} 
\end{figure}
Grundsätzlich besteht plausibles Verhalten. Die Anlage heizt jeweils bis zur vorgesehenen Betriebstemperatur (von $333~K$) auf und regelt den Kühlwasserstrom entsprechend nach. Stromdichte und Zellspannungen bewegen sich in technologietypischen Bereichen.
Verfügbare Leistung wird bis zur maximal festgelegten Anlagenleistung genutzt; es werden keine unzulässig hohen Werte erreicht. Auch die Produktgasströme weisen plausibles Verhalten auf, da die Stoffströme der Sauerstoffproduktion in etwa die Hälfte der Wasserstoff-Produktion ausmachen. Eine Produktgaskontamination ist deshalb nicht, da für diese Funktion für die \gls{ael} bisher nicht vorgesehen ist.


\subsubsection{Simulationsdurchlauf: RES-PEM}
Exemplarische Simulationsergebnisse der RESIDUAL-Zeitreihe sind in Abbildung(\ref{fig:plt_se_RES_PEM}) für die \gls{pem}-\gls{el} dargestellt.\\

\begin{figure}[H]
	\centering
	\includegraphics[width=\textwidth]{/home/dafu/Schreibtisch/Master-Projekt/Doku/Abb/Graph/res_se/RES_PEM__RES_se__.pdf}
	
	\caption[Simulation RESIDUAL im Jahresbetrieb \gls{pem}-\gls{el}]{Simulation RESIDUAL im Jahresbetrieb \gls{pem}-\gls{el}; Darstellung von Stack-Temperatur \gls{T_stack} und Kühlwassermassenstrom \gls{m_c} (oben), Zellspannung \gls{u_cell} und Zellstrom \gls{i}(2. v.o.), Eingangsleistung \gls{P_in},  tatsächlich abgerufene Leistung \gls{P_act} und Stackleistung \gls{P_st} (2. v. u.) sowie molare Produktgasströme \gls{n_H2},\gls{n_O2} und Produktgaskontamination \gls{H2inO2}}
	\label{fig:plt_se_RES_PEM} 
\end{figure}
Auch für diesen Simulationsdurchlauf weisen abgebildete Ergebnisse plausibles Verhalten auf. Die Stacktemperatur \gls{T_stack} bewegt sich im vorgesehenen Bereich ($T_{max}~=~358~K$). Ebenfalls die Kühlwasserregelung verhält sich plausibel: Zu Zeitpunkten niedriger Stacktemperaturen wird der Massenstrom deutlich gedrosselt, während eine Anregelung im Fall hoher Stacktemperaturen erfolgt. Auch für diesen Durchlauf bewegen sich Stromdichte und Zellspannungen in sinnvollen Größenordnungen. Der Leistungswert des Stacks folgt der verfügbaren Leistung, ohne maximal zulässige Werte zu überschreiten.
Höhe und Verlauf der Produktgasströme ist in o.g. Abbildung nicht scharf genug abgebildet, um Wasserstoff- und Sauerstoff-Anteile deutlich unterscheiden zu können. Erkennbar ist, dass der Wert der Produktgaskontamination zwar variiert, sich jedoch stets innerhalb des unkritischen Bereichs bewegt.



\subsubsection{Simulationsdurchlauf: WEA-SOEL}
Ebenfalls für die WEA-Zeitreihe erfolgt eine exemplarische Darstellung der Ergebnisse. In Abbildung(\ref{fig:plt_se_WEA_SOEL}) sind Simulationsergebnisse der \gls{soel} für o.g. Betriebsart aufgetragen. 
\begin{figure}[H]
	
	\centering
	\includegraphics[width=\textwidth]{/home/dafu/Schreibtisch/Master-Projekt/Doku/Abb/Graph/wea_se/WEA_SOEL__WEA_se___.pdf}
	
	\caption[Simulation WEA im Jahresbetrieb \gls{soel}]{Simulation WEA im Jahresbetrieb \gls{soel}; Darstellung von Stack-Temperatur \gls{T_stack} und Kühlwassermassenstrom \gls{m_c} (oben), Zellspannung \gls{u_cell} und Zellstrom \gls{i}(2. v.o.), Eingangsleistung \gls{P_in},  tatsächlich abgerufene Leistung \gls{P_act} und Stackleistung \gls{P_st} (2. v. u.) sowie molare Produktgasströme \gls{n_H2},\gls{n_O2} und Produktgaskontamination \gls{H2inO2}}
		\label{fig:plt_se_WEA_SOEL} 
\end{figure}
Grundsätzlich fällt auf, dass die Anlage sich auf einem wesentlich höheren Temperaturniveau bewegt, während deutlich niedriger Zellspannungen als bei \gls{ael} und \gls{pem}-\gls{el} vorliegen. Der Massenstrom $m_c$ entspricht hier lediglich dem zusätzlich zuzuführenden Dampfstrom. Innerhalb der vorliegenden Darstellung sind verfügbare und tatsächlich genutzte Leistung gut unterscheidbar. Auch hier wird der zulässige bzw. festgelegte Bereich eingehalten. Für die \gls{soel} wird zum derzeitigen Stand lediglich der erzeugte Wasserstoff-Strom ausgegeben bzw. angezeigt. Eine Berechnung der entsprechenden Sauerstoffmenge ist dennoch implementiert. Für aktuelle Berechnungen ist allerdings die verminderte Ausgabe ausreichend. 

\subsubsection{Simulationsdurchlauf: COST-PEM}
In Abbildung(\ref{fig:plt_se_COST_PEM}) ist exemplarisch die Simulation eines Betriebsjahres einer \gls{pem}-\gls{el} mit COST-Betriebsart und mittlerer (m) Skalierung dargestellt.
\begin{figure}[H]
	
	\centering
	\includegraphics[width=\textwidth]{/home/dafu/Schreibtisch/Master-Projekt/Doku/Abb/Graph/cost_m_se/COST_M_PEM__COST_M_se___.pdf}
	
	\caption[Simulation COST im Jahresbetrieb \gls{pem}]{Simulation COST im Jahresbetrieb \gls{pem}; Darstellung von Stack-Temperatur \gls{T_stack} und Kühlwassermassenstrom \gls{m_c} (oben), Zellspannung \gls{u_cell} und Zellstrom \gls{i}(2. v.o.), Eingangsleistung \gls{P_in},  tatsächlich abgerufene Leistung \gls{P_act} und Stackleistung \gls{P_st} (2. v. u.) sowie molare Produktgasströme \gls{n_H2},\gls{n_O2} und Produktgaskontamination \gls{H2inO2}}
	\label{fig:plt_se_COST_PEM} 
\end{figure}
Innerhalb der dargestellten Berechnung ist das Strompreis-Limit auf $40~ EUR/MWh$ festgelegt. Zu erkennen ist, dass insbesondere in den Sommermonaten liegen große Zeiträume vor, in denen der festgelegte Wert unterschritten wird.\\
Auch für diesen Betriebsfall besteht grundsätzlich Plausibilität der Ergebnisse. Die Temperaturregelung scheint relativ genau zu agieren: Die Stacktemperatur liegt durchgehend innerhalb sinnvoller bzw. zulässiger Bereiche.\\
Auf Grund der Signalcharakteristik wird die Anlage lediglich zwischen Stillstand und Nennleistung bewegt, wodurch für sämtliche weiteren dargestellten Größen wenig Aussagekraft innerhalb dieser Darstellung existiert. Gut erkennbar ist dagegen das Niveau der Wasserstoff- und Sauerstoff-Produktion sowie das Niveau der Produktgaskontamination. Auch diese Werte liegen in erwarteten Bereichen.\\
\subsubsection{Bewertung der Simulation}
%???HIER?
Auch wenn das vorgestellte Modell zahlreiche Lücken und VEreinfachungen aufweist, konnte anhand der Betrachtung von Simulationsergebnissen (dieser Abschnitt) und zu Grunde gelegten elektrochemischen Modellen (Polarisationskurven, Abschnitt(\ref{subsub_polar})) die grundsätzliche Funktionstüchtigkeit des Programms bestätigt werden. Anhand wichtiger Kenngrößen ist in den Folgenden Abschnitten eine weitere Verifizierung implementierter Berechnungsfunktionen und vorliegender Ergebnisse möglich.

\newpage
\subsection{Abgleich der Anlagenskalierung}
Aus den im vorangestellten Abschnitt (sowie im Anhang) dargestellten Simulationsdurchläufen bzw. auf Grundlage des erzeugten Daten-Outputs können weiterführende Berechnungen und Analysen durchgeführt werden. Im folgenden Abschnitt wird eine Überprüfung der Anlagenskalierung (vgl. Abschnitt(\ref{subs_ANL_SKAL}))anhand von tatsächlich erreichten Volllaststunden bzw. umgesetzten Energiemengen durchgeführt.

\subsubsection{EIS}
Abbildung(\ref{fig:ABGL_SKAL_EIS}) zeigt eine bezüglich der EISMAN-Betriebsart um Markierungspunkte für genutzter Energiemenge (farbige Kreuze) und erreichte Volllaststunden (farbige Rauten) ergänzte Abbildung(\ref{fig:Skal_EIS}). 
\begin{figure}[H]
	
	\centering
	%\includegraphics[width=0.49\textwidth]{/home/dafu/Schreibtisch/Master-Projekt/Doku/Abb/ .png}
	\includegraphics[width=0.8\textwidth]{/home/dafu/Schreibtisch/Master-Projekt/Doku/Abb/Graph/jdl/JDL_EIS_scale-marker_Markerpoints_new-range_scal+MP_1__02.pdf}
	\caption[Abgleich Anlagenskalierung -EIS]{Abgleich der theoretischen mit der tatsächlichen EIS-Anlagenskalierung auf Grundlage erstellter Jahresdauerlinien}
	\label{fig:ABGL_SKAL_EIS} 
\end{figure}
Die tatsächlichen Anlagen-Leistungen entsprechen mit geringfügigen Abweichungen den Auslegungspunkten. Sämtliche Anlagen bewegen sich im Bereich prognostizierter Volllaststunden. Während \gls{ael} und \gls{pem}-\gls{el} die theoretisch nutzbare Energiemenge nur leicht unterschreiten, weicht die umgesetzte Energiemenge der \gls{soel} deutlich ab. 
\newpage
\subsubsection{RES}
Bezüglich der RESIDUAL-Betriebsart zeigt Abbildung(\ref{fig:ABGL_SKAL_RES}) die um Markierungspunkte für genutzte Energiemenge und erreichte Volllaststunde ergänzte Abbildung(\ref{fig:Skal_RES}).\\


\begin{figure}[H]
	
	\centering
	%\includegraphics[width=0.49\textwidth]{/home/dafu/Schreibtisch/Master-Projekt/Doku/Abb/ .png}
	\includegraphics[width=0.8\textwidth]{/home/dafu/Schreibtisch/Master-Projekt/Doku/Abb/Graph/jdl/JDL_RES_scale-marker_Markerpoints__02.pdf}
	\caption[Abgleich Anlagenskalierung -RES]{Abgleich der theoretischen mit der tatsächlichen RES-Anlagenskalierung auf Grundlage erstellter Jahresdauerlinien}
	\label{fig:ABGL_SKAL_RES} 
\end{figure}
Während für die \gls{pem}-\gls{el} eine etwas zu große Anlagenleistung hinsichtlich des Auslegungspunktes besteht, bewegen sich sämtliche Werte der Volllaststunden im Rahmen des erwarteten Bereichs. Hinsichtlich umsetzbarer Energiemengen erreicht die \gls{pem}-\gls{el} vergleichsweise hohe Werte, wobei diese für eine angepasste Anlagenleistung entsprechend zu reduzieren sind.Dadruch ist ein Angleich von \gls{pem} und \gls{soel} zu erwarten. Die \gls{ael} weist den geringsten Umsatz verfügbarer Energiemengen auf.  
\newpage
\subsubsection{WEA}
Auch für die WEA-Betriebsart wird Abbildung(\ref{fig:Skal_WEA}) um entsprechende Markierungspunkte für umgesetzte Energiemenge und erreichte Volllaststunden ergänzt.\\

\begin{figure}[H]
	
	\centering
	%\includegraphics[width=0.49\textwidth]{/home/dafu/Schreibtisch/Master-Projekt/Doku/Abb/ .png}
	\includegraphics[width=0.8\textwidth]{/home/dafu/Schreibtisch/Master-Projekt/Doku/Abb/Graph/jdl/JDL_WEA_scale-marker_Markerpoints_02.pdf}
	\caption[Abgleich Anlagenskalierung -WEA]{Abgleich der theoretischen mit der tatsächlichen WEA-Anlagenskalierung auf Grundlage erstellter Jahresdauerlinien}
	\label{fig:ABGL_SKAL_WEA} 
\end{figure}
Während innerhalb der RES-Betriebsart die \gls{pem}-\gls{el} den höchsten Energieumsatz aufweist, dominiert in diesem Fall deutlich die \gls{soel}, während die \gls{ael} ebenfalls für diese Betriebsart die niedrigsten Energieumsätze aufweist. Hinsichtlich erreichter Vollaststunden dominiert ebenfalls die \gls{soel} vor \gls{pem}-\gls{el} und \gls{ael}. 

%\subsection{Sensitivität}
%\begin{itemize}
%	\item Anlagen verhalten
%	\begin{itemize}
%		\item ...
%		\item elchem -> welche Größe? -> anhand polarc rausfinden!
%		\item dPdt (?) -> ökon?
%		\item PID (?) -> PI statt PID? 
%		\item dU\_fact (?)
%	\end{itemize}
%	\item ökonom. Ergebn
%	\begin{itemize}
%		\item ...
%		%\item elchem -> welche Größe? -> anhand polarc rausfinden!
%		\item Inv.-Kosten
%		\item Amort
%		%\item PID (?) -> PI statt PID?
%		%\item dU\_fact (?)
%	\end{itemize}
%\end{itemize}



%\subsection{Grenzen des Modells}
%\subsubsection{Technologie-Übergreifend}
%\begin{itemize}
%	\item keine Druck-Dynamik
%	\item Permeation unzureichend abgebildet
%	\item Degradation unzur. abgebildet
%	\item eingeschränkte Vergleichbarkeit der Technologien
%\end{itemize}
%
%\subsubsection{PEM}
%\begin{itemize}
%	\item Elektrolytwiderstand
%	\item Dynamik
%	\item Thermo (unvollständige Bilanz)
%	\item Peripherie
%	\item Degradation:
%	\begin{itemize}
%		\item nur über curve-fit
%		\item 
%	\end{itemize}
%\end{itemize}


\newpage
\section{Analyse und Auswertung}
%\subsection{Methodik}
%%\subsection{Analyse grundsätzlichen Verhaltens}
%%Dynamik... Verhalten bei P-Sprüngen....
%%aus großen Plots
\subsection{Kennwert-Analyse}
Um die betrachteten Technologien bewerten zu können, ist ein Abgleich verschiedener Kennwerte 
%(vgl. Abschnitt(???->Grundlagen Kenndaten???)) 
notwendig. Im Folgenden werden entsprechende Kennwerte je Technologietyp und Eingangssignal vergleichend aufgetragen. In sämtlichen Diagrammen besteht eine Gruppierung nach Betriebsart über welchen die \gls{el}-Typen \gls{ael} (blau),\gls{pem} (grün) und \gls{soel} (rot) aufgetragen sind. Zum Abgleich von Systemwirkungsgrad und spezifischem Energiebedarf mit Literaturwerten sind entsprechende Referenzwerte durch graue Bereiche dargestellt.

\subsubsection{Systemwirkungsgrad}
Wie in Abschnitt(\ref{subsubb_Kenndaten}) erläutert, gibt der Systemwirkungsgrad Auskunft über die Effizienz der Umsetzung elektrischer Energie während des Produktionsprozesses. Berechnete Werte für die Betriebsarten \gls{eis}, \gls{res} und \gls{wea} sind in Abbildung(\ref{fig:analy_nC_eta}) aufgetragen.
\begin{figure}[H]
	
	\centering
	%\includegraphics[width=0.49\textwidth]{/home/dafu/Schreibtisch/Master-Projekt/Doku/Abb/ .png}
	\includegraphics[width=0.8\textwidth]{/home/dafu/Schreibtisch/Master-Projekt/Doku/Abb/Graph/eco/2019-03-02--11-16_barplot_EISMAN_RESIDUAL_WEA_eta.pdf}
	\caption[Systemwirkungsgrad EIS, RES, WEA]{Systemwirkungsgrad auf Grundlage der Jahressimulation mit Zeitreihen:EISMAN, RESIDUAL, WEA }
	\label{fig:analy_nC_eta} 
\end{figure}
Für die \gls{eis}-Betriebsart weisen die drei \gls{el}-Technologien Systemwirkungsgrade zwischen $55 $ und $69$ bzw. $70~\%$ auf. Während\gls{pem} und \gls{soel} ein ähnliches Niveau aufweisen, liegt die Effizienz der \gls{ael} deutlich darunter, befindet sich jedoch damit im Bereich aktueller Literaturangaben.\cite{Buttler.2018}
Dagegen übersteigt die \gls{pem}-\gls{el} Literaturwerte deutlich (um etwa mindestens $10\%$). Die \gls{soel} hingegen liegt mit $69~\%$ um etwa $6-8~\%$ unterhalb erwartbarer Werte.\\
Ein ähnliches Bild ergibt sich für die \gls{res}-Betriebsweise: \gls{ael} und \gls{soel} liegen etwa $1~\%$ oberhalb der für \gls{eis} beschriebenen Werte. Die \gls{pem}-\gls{el} übersteigt den entsprechenden \gls{eis}-Wert um weitere $5~\%$. Abweichungen zur Literatur liegen hier in gleichem Maße vor, wie zuvor beschrieben.\\
Auch für den Betriebsfall \gls{wea} bleiben diese Verhältnisse erhalten, wobei die Effizinez der \gls{pem}-\gls{el} nochmals um $1~ \%$ steigt.\\

Mögliche Gründe für die hohe Systemeffizienz der \gls{pem}-\gls{el} lassen sich auf die in Abschnitt (\ref{subsub_polar}) erklärte Polarisationskurve zurückführen. Ein geringer Anteil an Überspannungen führt bei gleichem Leistungseingang zu einem höheren Strom, wodurch die Menge an produziertem Wasserstoff steigt. Des Weiteren weist die \gls{pem}-\gls{el} im Gegensatz zu der \gls{ael} eine bessere Teillastausnutzung auf, da diese erst ab einem Leistungseingang von $30\%$ der Nennleistung den Betrieb aufnimmt (und Wasserstoff produziert). Im Gegensatz dazu arbeitet die \gls{pem}-\gls{el} bereits ab einer Eingangsleistung von $5\%$ der Nennleistung.\\ 
Ein deutlicher Nachteil entsteht für die \gls{soel} durch das hohe Temperaturniveau in Kobination mit Leistungssignalen, welche große Zeiträume mit $\gls{P_in}=~0$ aufweisen. Dies hat für die \gls{soel} häufiges Aufheizen und Abkühlen zur Folge. Die Dauer der Anfahrzeit von mehreren Stunden hat zudem eine stark verzögerte Wasserstoffproduktion zur Folge, wohingegen \gls{ael} und \gls{pem}-\gls{el} nahezu sofort bei Signaleingang den Betrieb aufnahemn können.
Ist wie im Fall von \gls{eis} die absolute Betriebszeit der Elektrolyseanlage gering, jedoch mit einer hohen zur Verfügung stehenden Eingangsleistung verbunden, ergibt sich der Vorteil einer hohen Wasserstoffproduktionsrate. Diesbezüglich ist jedoch ebenfalls zu beachten, dass Elektrolyseure im Teilllastbereich höhere Wirkungsgrade aufweisen.
%EIS/RES/WEA:
%
%- eta
%• Eis: AEL = 0,56; PEM=0,7; SOEL= 0,69
% aufsteigend von AEL zu PEM zu SOEL
%◦ PEM und SOEL ungefähr gleich
%◦ AEL im Lit Bereich; PEM deutlich über Lit Bereich; SOEL deutlich unter Lit Bereich
%• Res: AEL = 0,57; PEM=0,75; SOEL= 0,70
%◦ aufsteigend von AEL zu SOEL zu PEM
%◦ AEL im Lit; PEM weit über Lit; SOEL deutlich unter Lit
%• Wea: AEL = 0,57; PEM=0,77; SOEL= 0,69
%◦ aufsteigend von AEL zu SOEL zu PEM
%◦ AEL im Lit; PEM deutlich über Lit; SOEL deutlich unter Lit.
%• AEL schlechter, weil Pol-Kurve zu hoch
%◦ Gleichbleibende eta bei allen Input-Sig
%◦ pminfrac keine Auswirkung, weil eta\_sys bezogen auf Pact ist
%◦ keinen Bezug zum aktuellen Leistungsbezug
%• PEM zu gut, weil gute Pol-Kurve  niedrige Überspannungen
%◦ Aufsteigende eta von Eis zu Res zu Wea
%◦ Bessere Teillastausnutzung
%• SOEL 
%◦ Gleichbleibende eta
%◦ Keine Unterschiede Teil- (RES/WEA) zu Volllastbetrieb (EIS)

\begin{figure}[H]
	
	\centering
	%\includegraphics[width=0.49\textwidth]{/home/dafu/Schreibtisch/Master-Projekt/Doku/Abb/ .png}
	\includegraphics[width=0.8\textwidth]{/home/dafu/Schreibtisch/Master-Projekt/Doku/Abb/Graph/eco/2019-03-02--10-25_barplot_Cost_S_Cost_M_Cost_L_eta.pdf}
	\caption[Systemwirkungsgrad-COST]{Systemwirkungsgrad auf Grundlage der Jahressimulation mit COST (S, M , L)-Zeitreihen  }
	\label{fig:analy_COST_eta} 
\end{figure}
Im Kontrast zu den zuvor beschriebenen Betriebsarten entsteht für die \gls{cost}-Betriebsweise ein abweichendes, jedoch innerhalb der Skalierungen sehr ähnliches Bild.\\
Hier liegt die  \gls{soel} mit gleichbleibend $80~-~82~ \%$ leicht oberhalb von Literaturangaben und damit deutlich über Effizienzwerten von \gls{ael} ($55~\%$) und \gls{pem} ($66~-~67~\%$). Für diese beiden Technologien ergeben sich für jede Skalierungsstufe (S, M , L) ähnliche bis gleiche Werte. Auch hier befindet sich die \gls{pem}-\gls{el} deutlich oberhalb von Literaturwerten, während Werte der \gls{ael} durchweg in der Mitte des Literaturwertebereichs liegen.
Die hohe Effizienz der \gls{soel} ist damit zu erklären, dass anstatt eines dynamischen Temperaturverlaufs die Temperatur nach einmaligem Anfahren auf Betriebstemperatur von $T=850~^\circ C$ gehalten wird. 
%Auf Grund der speziellen Charakteristik des COST-Signals, ist aus den Effizienzwerten jeweiliger Technologien ebenfalls ein Rückschluss auf die Produzierten Wasserstoffmengen möglich: Unter der Annnahme, dass sämtliche \gls{el}-Typen nahezu den 
%Cost S/M/L:
%- eta
%• Cost S: aufsteigend von AEL zu PEM zu SOEL
%◦ AEL = 0,55; PEM=0,66; SOEL= 0,82
%• Cost M: aufsteigend von AEL zu PEM zu SOEL
%◦ AEL = 0,55; PEM=0,67; SOEL= 0,80
%• Cost M: aufsteigend von AEL zu PEM zu SOEL
%◦ AEL = 0,55; PEM=0,67; SOEL= 0,82
%• Cost S: AEL trifft Lit mittig; PEM über dem Lit, SOEL leicht über Lit
%• Cost M: AEL trifft mittig den Lit.; PEM über dem Lit.; SOEL am oberen Rand der Lit.
%• Cost L: AEL trifft mittig, PEM über dem Lit; SOEL knapp über dem Lit

\subsubsection{Nutzungsgrad verfügbarer Energiemenge}
\begin{figure}[H]
	\centering
	%\includegraphics[width=0.49\textwidth]{/home/dafu/Schreibtisch/Master-Projekt/Doku/Abb/ .png}
	\includegraphics[width=0.8\textwidth]{/home/dafu/Schreibtisch/Master-Projekt/Doku/Abb/Graph/eco/2019-03-02--11-44_barplot_EISMAN_RESIDUAL_WEA_eta_N.pdf}
	\caption[Energie-Ausnutzung EIS, RES, WEA]{Grad der Ausnutzung verfügbarer Energie auf Grundlage der Jahressimulation mit Zeitreihen:EISMAN, RESIDUAL, WEA }
	\label{fig:analy_nC_etaN} 
\end{figure}
Bezüglich der Ausnutzung verfügbarer Energie entsteht für \gls{eis}, \gls{res} und \gls{wea} ein deutlich inhomogenes Bild (Abbildung(\ref{fig:analy_nC_etaN})).
Für die \gls{eis}-Betriebsart weisen \gls{ael} und \gls{pem}-\gls{el} ähnlich hohe Werte im Bereich von $96$ bzw. $97~\%$ auf, während die \gls{soel} mit $72~\%$ deutlich darunter liegt.\\
Dagegen weist die \gls{ael} für beide weiteren Betriebsarten deutlich niedrigere Werte mit fallender Tendenz ($77$ bzw. $69~\%$) auf. Während für den Fall des \gls{res}-Betriebs die \gls{pem}-\gls{el} mit $94~\%$ ein hohes Niveau beibehält, steigt die \gls{soel} in beiden Fällen um etwa $15~\%$ auf $87~\%$.\\
 
Die \gls{ael} weist für das \gls{eis}-Signal einen hohen Grad an Ausnutzung auf. Dies wird durch die hohe, zur Verfügung stehende Eingangslast zu verursacht, während der Betriebszeit fast ausschließlich Vollast vorliegt. Dies wird durch den direkten Vergleich mit den Signalen \gls{res} und \gls{wea} deutlich. Für beide letztgenannten Signale ist dies diese Charakteristik deutlich geringer ausgeprägt. Höhere Anteile niedriger Leistung führen zu einem kontinuierlicheren Betrieb. Weiterhin ist erkennbar, dass sich das dynamische Eingangssignal negativ auf die \gls{soel} auswirkt. Aufgrund häufig auftretender Null-Signale kommt es zu einem ständigen Wechsel zwischen Aufheizen und Abkühlen der \gls{soel}.\\
Niedrigere Effizienzwerte der \gls{pem} gegenüber der \gls{soel} sind ggf auf einen Steuerungs- bzw- Regelungstechnischen Fehler innerhalb der Simulation zurückzuführen: %Wie in Abbildung() erkennbar und in Abschnitt() näher erläutert VERWEISE?????
Wie zuvor erläutert, werden ab einer bestimmten Leistung (und damit zusammenhängender Temperatur des Stacks)  auch technisch erzielbare hohe Werte abgeregelt bzw. die Anlage temporär gedrosselt betrieben.  

%Eta\_util
%• Eis: AEL = 0,96; PEM=0,97; SOEL= 0,72
%◦ aufsteigend von SOEL zu AEL zu PEM
%◦ PEM und AEL ungefähr gleich
%◦ AEL hoher Wirkungsgrad da fast immer P_nenn, kaum pminfrac
%◦ SOEL schlecht, weil häufig 0 Sig  häfig hochfahren  viel Ph
%
%• Res: AEL = 0,77; PEM=0,94; SOEL= 0,87
%◦ aufsteigend von AEL zu SOEL zu PEM
%◦ AEL schlecht, weil hoher Anteil Pin unterhalb pminfrac
%
%• Wea: AEL = 0,69; PEM=0,78; SOEL= 0,87
%◦ aufsteigend von AEL zu PEM zu SOEL
%◦ PEM regelt viel weg, wegen Temp. Schutz (Einst. vom PID)
%▪ 4mio sec. – 5,5mio sec.
%▪ Für ne bessere Aussage neuer Lauf
%• SOEL bei RES/WEA quasi gleiche eta_util, da durchgängiges Sig
%◦ Verluste durch Abschneiden des Sig bei überschreiten der Pnenn



\subsubsection{Spezifischer Energiebedarf}
Der spezifische Energiebedarf beschreibt die für die durchschnittliche Erzeugung eines Normkubikmeters Wasserstoffs notwendige Energiemenge. Auch hier ist ein Abgleich mit Literaturwerten möglich, wobei Abbildungsskalierungen so gewählt sind, dass deren Maximalwerte nicht mit abgebildet wird.\\
\begin{figure}[H]
	
	\centering
	%\includegraphics[width=0.49\textwidth]{/home/dafu/Schreibtisch/Master-Projekt/Doku/Abb/ .png}
	\includegraphics[width=0.8\textwidth]{/home/dafu/Schreibtisch/Master-Projekt/Doku/Abb/Graph/eco/2019-03-02--12-06_barplot_EISMAN_RESIDUAL_WEA_E_spec.pdf}
	\caption[Spezifischer Energiebedarf EIS, RES, WEA]{Spezifischer Energiebedarf auf Grundlage der Jahressimulation mit Zeitreihen:EISMAN, RESIDUAL, WEA }
	\label{fig:analy_nC_Espec} 
\end{figure}
Für die Betriebsarten \gls{eis}, \gls{res} und \gls{wea} entsteht insgesamt ein homogenes Bild. Die \gls{ael} liegt mit $5.2$ bis $5.4~kWh/m^3_N$ für sämtliche Fälle im Literaturbereich und deutlich oberhalb der anderen Technologien. Auch die \gls{soel} befindet sich in allen Fällen mit $4.3~kWh/m^3_N$ im mittleren Bereich von Literaturangaben. Dagegen unterschreitet die \gls{pem}-\gls{el} Werte der Literatur, wie auch der anderen Technologien, von \gls{eis} ($4.3~kWh/m^3_N$) zu \gls{wea} ($3.9~kWh/m^3_N$) zunehmend. Insgesamt spiegeln die in Abbildung(\ref{fig:analy_nC_Espec}) dargestellten Werte die Effizienzwerte aus Abbildung(\ref{fig:analy_nC_eta}) wider. Es ist anzumerken, dass sich die \gls{soel} trotz einer relativ deutlichen Unterschreitung von Literatur-Effizienzwerten (um $ca.~7\%$) bezüglich des spezifischen Energiebedarfs im Bereich von diesbezüglichen Literaturangaben bewegt.

%e_spez
%• (Ref. Werte AEL und PEM Obergrenze nicht angezeigt)
%• Eis: AEL = 5,40; PEM= 4,3; SOEL= 4,3
%◦ aufsteigend von PEM zu SOEL zu AEL
%◦ AEL im Lit. A und B Bereich; PEM deutlich unterhalb des Lit. A und B Bereichs; SOEL deutlich über Lit A und innerhalb von Lit. B
%◦ PEM/ SOEL vom Wert ähnlich
%◦ 
%
%• Res: AEL = 5,2; PEM= 4,0; SOEL= 4,3
%◦ aufsteigend von PEM zu SOEL zu AEL
%◦ 
%
%• Wea: AEL = 5,20; PEM= 3,90; SOEL= 4,30
%◦ aufsteigend von PEM zu SOEL zu AEL
%◦ 
%
%• AEL deutlich schlechter, weil 
%◦ weniger H2 produziert wird
%◦ von Pol-Kurv T\_betrieb < T\_betrieb PEM
%• PEM 
%◦ Eis  wenig Betriebsstunden aber große Anlagenleistung  e\_spez = hoch
%◦ Res weniger Volllaststd. als Wea  bessere Auslastung der Anlage bei Wea  im vgl. kleinere Anlage Res kaum Einfluss auf die e\_spez
%• SOEL 
%◦ gleichbleibende e-spez für EIS/RES/WEA
%▪ EIS: Zwar wenig Betriebsstd. (viel Volllast) dafür in der Zeit hohes P\_in (hohe H2 Prod)
%▪ WEA/ RES: mehr Betriebsstd. (wenig Vollast) dafür niedrigeres P\_in niveau

\begin{figure}[H]
	
	\centering
	%\includegraphics[width=0.49\textwidth]{/home/dafu/Schreibtisch/Master-Projekt/Doku/Abb/ .png}
	\includegraphics[width=0.8\textwidth]{/home/dafu/Schreibtisch/Master-Projekt/Doku/Abb/Graph/eco/2019-03-02--10-25_barplot_Cost_S_Cost_M_Cost_L_E_spec.pdf}
	\caption[Spezifischer Energiebedarf -COST]{Spezifischer Energiebedarf auf Grundlage der Jahressimulation mit Zeitreihen:COST(S, M, L) }
	\label{fig:analy_COST_Espec} 
\end{figure}

Hinsichtlich der spezifischen Energiemengen im Fall der COST-Betriebsweise kehrt sich das Verhältnis von \gls{pem}-\gls{el} und \gls{soel} um: Die \gls{soel} weist hier einen deutlich geringeren spezifischen Energiebedarf auf. Während die \gls{ael} im Bereich großer Analagenskalierungen leichte Effizienzverluste verzeichnet und sich ebenfalls die \gls{pem}-\gls{el} durch leichte diesbezügliche Einbußen Literaturwerten annähert, ist für die \gls{soel} eine Effizienzsteigerung bzw. Reduktion des spezifischen Energiebedarfs um etwa $0,5~kWh/m^3_N$ zu verzeichnet. Ein Vergleich der \gls{soel}-Effizienz mittels COST- und weiterer Signale ist jedoch nicht direkt möglich, da im Fall des COST-Signals eine veränderte Parametrierung vorliegt. %(dynT, vgl. Abschnitt(????????))
%#########################################################
%COST:
%- e\_spez
%• Cost S: aufsteigend von SOEL zu PEM zu AEL
%◦ SOEL = 3,7; PEM=4,5; AEL = 5,4
%◦ AEL über Lit A innerhalb lit B
%◦ PEM knapp innerhalb A, unterhalb B
%◦ SOEL oberhalb Lit A, knapp unterhalb Lit. B
%• Cost M: aufsteigend von SOEL zu PEM zu AEL
%◦ SOEL = 3,7; PEM=4,5; AEL = 5,4
%◦ AEL über Lit A innerhalb lit B
%◦ PEM knapp innerhalb A, unterhalb B
%◦ SOEL oberhalb Lit A, knapp unterhalb Lit. B
%• Cost M: aufsteigend von SOEL zu PEM zu AEL
%◦ SOEL = 3,7; PEM=4,5; AEL = 5,4
%◦ AEL über Lit A innerhalb lit B
%◦ PEM knapp innerhalb A, unterhalb B
%◦ SOEL oberhalb Lit A, knapp unterhalb Lit. B


\subsection{Sensitivität der Zeitreihen-Ausschnitte} %9800er
Um den Einfluss bzw. dessen Signifikanz verschiedener Parameter auf wichtige Kenngrößen der betrachteten Technologien herauszustellen, ist eine Sensitivitätsanalyse erforderlich.\\
\subsubsection{Zeitreihen-Ausschnitt}
Für diesen Zweck fällt die Wahl auf das \gls{wea}-Signal. Lastverlauf und sowie Anteile von Voll- und Teillast sowie deren Verteilung zeichnen dieses Signal als sinnvolles Testsignal aus.\\ 
Auf Grund der relativ hohen Berechnungszeit des Programms, ist die Berechnung einer vollständigen (jährlichen) Zeitreihe für sämtliche als notwendig erachteten Parameter Variationen überaus zeitintensiv. Aus diesem Grund wird die zuvor erläuterte Datenreihe (\gls{wea}) nicht vollständig, sondern lediglich als Ausschnitt von $700~h$ herangezogen. Um dennoch eine möglichst deckungsgleiche Charakteristik zu erhalten, werden verschiedene Statistische Kennwerte des vollständigen Datensatzes mit denen des Ausschnitts verglichen.\\
In Tabelle(\ref{tab:WEA_slice_stat_vgl}) sind statistische Werte des Originaldatensatzes sowie des $700~h$-Ausschnittes gegenübergestellt. 

\begin{table}[H]
	\caption{Abgleich des original WEA-Datensatzes mit dem gewählten Ausschnitt anhand von Statistischen Werten sowie ermittelter Betriebszeiten je Technologie}
	\begin{tabular}{l|ccc}
		
		stats& $df_{orig}$ & $df_{9800}$ &rel. Abweichung\\ 
		\hline 
		\hline
		max()			& $10189$ & $10097$& $-0.009$\\ 
		\hline 
		mean()			&$1786$  & $1781$ &$-0.003$\\ 
		\hline 
		25\%			&$295$  & $326$ & $0.11$\\ 
		\hline 
		50\%			& $1055$ & $1056$& $0.001$\\ 
		\hline 
		75\%			& $2573$ &  $2600$&$0.01$\\ 
		\hline 
		
		$t_{op,ael}$	& 2847,6 & 227& $0.0797$\\ 
		\hline 
		$t_{op,pem}$	& 6385 & 520& $0.081$\\
		\hline
		$t_{op,soel}$	& 5301 & 435& $0.082$\\
		\hline
		& 8760 & 700& $0.079$\\
	\end{tabular} 
	\label{tab:WEA_slice_stat_vgl}
\end{table}
Des Weiteren werden die je Technologie erzielten Betriebszeiten abgeglichen.
Der gewählte Ausschnitt weist eine sehr gute Übereinstimmung mit den Statistischen Werten des ganzjährigen Datensatzes auf. Lediglich das untere Quartil ($25\%$) weicht mit $11\%$ relativ deutlich von Werten des Originaldatensatzes ab. Sämtliche sonstige verglichene Statistik-Werte liegen weichen um weniger als $1\%$ ab. Damit ist dieser Ausschnitt unter Beachtung des Einflusses auf niedrige Teillastbereiche verwendbar.

\subsubsection{Sensitivitäts-Parameter}
Zur Überprüfung der Sensitivität wichtiger Kenngrößen werden folgende Parameter herangezogen bzw. variert:
\begin{table}[H]
	\caption{Parameterliste für Sensitivitätsanalyse}
	\begin{tabular}{ll}
		Betriebszeit,start	&$t_{mem0}$ \\
		Betriebszeit, mittel	&$t_{mem,mid}$\\
		Betriebszeit, ende 		&$t_{mem,end}$\\
		Druckverluste, Pumpe 	&$dp_{380k}$\\
		Minimal-Leistung, hoch 	&$pminfrac_{up}$\\
		Minimal-Leistung, niedrig &$pminfrac_{low}$\\
		Reinheit Produktgas &$X_{tar}$\\
		Maximaler Leistungsgradient &$dPdt_{max}$\\
	\end{tabular}
	\label{tab:sens_params}
\end{table}

Im Folgenden werden Ergebnisse der Parameter Variation auf wichtige Kenngrößen der \gls{el} anhand von Balkendiagrammen dargestellt. Für eine bessere Vergleichbarkeit sind Linien auf höhe der Mittelwerte der jeweiligen Technologie eingetragen.
Zum Abgleich sind außerdem die Werte des vollständigen Datensatzes($WEA_{ref}$) linksstehend abgebildet.

\newpage
\subsubsection{Sensitivität- Systemwirkungsgrad}
In Abbildung(\ref{fig:barplot_eco_9800_eta}) ist der Systemwirkungsgrad jedes Technologietyps in Abhängigkeit der in Tabelle(\ref{tab:sens_params}) aufgeführten Parameter bzw. deren Variation von Standard-Einstellungen dargestellt.\\


\begin{figure}[H]
	\centering
	%\includegraphics[width=0.49\textwidth]{/home/dafu/Schreibtisch/Master-Projekt/Doku/Abb/ .png}
	\includegraphics[width=\textwidth]{/home/dafu/Schreibtisch/Master-Projekt/Doku/Abb/Graph/eco/2019-03-02--17-17_barplot_9800_eta.pdf}
	\caption[Systemwirkungsgrad Sensitivitätsanalyse]{Systemwirkungsgrad der Sensitivitätsanalyse für Ausschnitt der WEA-Zeitreihe}
	\label{fig:barplot_eco_9800_eta} 
\end{figure}
Auffallend ist, dass Werte der \gls{pem} sowie der \gls{soel} von Ergebnissen des vollständigen Datensatzes um mehr als $5\%$ nach unten (\gls{pem}) bzw. nach oben (\gls{soel}) abweichen.\\ Des Weiteren ist ein deutlicher Einfluss zunehmender Degradationsüberspannungen auf Grund höherer, absoluter Betriebszeit für jede Technologie erkennbar.\\
Am deutlichsten werden die Einflüsse der Degradation bei der \gls{pem} sowie der \gls{soel}, wobei die Werte in Relation gesehen werden müssen, da die \gls{pem}-\gls{el} mit einer Laufzeit von von $30.000~h$ bzw. $60.000h$ eine um den Faktor 6 höhere Lebensdauer als die \gls{soel} aufweist. Mit einer maximalen Abweichung von etwa $5\%$ vom Mittelwert hat die Degradation auf die \gls{ael} einen deutlich geringeren Einfluss. Resultierend aus dem Anstieg der Degradation folgt die Verringerung der Wasserstoffproduktion\\
Den einzigen beiden Effekte die sich positiv auf alle drei Technologien auswirken ist eine Verringerung der Produktgasqualität und die Verringerung der Minimal-Leistung, woduch die Systemeffizienz um ca. $2\%$ ansteigt.

auch: Wasserstoffgestehungskosten

Ein deutlich anderes Verhalten der drei Technologien zeigt der Ausnutzungsgrad  bezogen auf die in Tabelle(\ref{tab:sens_params}) aufgeführten Parameter, welcher in Abbildung(\ref{fig:barplot_eco_9800_etaN}) dargestellt ist. \\
Während sich \gls{ael} sowie \gls{soel} Grundsätzlich auf dem gleichen Niveau wie der Jahresdatensatz befinden, zeigt die \gls{pem}-\gls{el} eine um $10~$-$13~\%$ positive Abweichung für sämtliche Kennwerte.\\
Den größten negativen Einfluss auf die Ausnutzung der zur Verfügung stehenden Energie, für \gls{ael} und \gls{soel}, zeigt sich durch die Erhöhung der Minimal-Leistung. Weshalb ein großer Teil des \gls{wea}-Signals für die \gls{ael} und \gls{soel} wegfällt. Der gegenteilige Effekt, also eine Verringerung der Minimallast zeigt sich besonders deutlich für die \gls{ael}, wodurch ein größerer Anteil im unteren Teillastbereich abgedeckt wird und somit eine höhere Ausnutzung zustande kommt.\\
Aus Abbildung(\ref{fig:barplot_eco_9800_eta}) wird zudem deutlich, dass die Degradationseffekte keinen Einfluss auf die Ausnutzung der zur Verfügung stehenden Energie aufweisen. 
\begin{figure}[H]
	\centering
	%\includegraphics[width=0.49\textwidth]{/home/dafu/Schreibtisch/Master-Projekt/Doku/Abb/ .png}
	\includegraphics[width=\textwidth]{/home/dafu/Schreibtisch/Master-Projekt/Doku/Abb/Graph/eco/2019-03-02--17-13_barplot_9800_eta_N.pdf}
	\caption[Ausnutzungsgrad Sensitivitätsanalyse]{Grad der Ausnutzung verfügbarer Energiemengen der Sensitivitätsanalyse für Ausschnitt der WEA-Zeitreihe}
	\label{fig:barplot_eco_9800_etaN} 
\end{figure}
Mit Hilfe von Abbildung(\ref{fig:barplot_eco_9800_Espec}) werden die Auswirkungen der einzelnen Kennwerte auf den spezifischen Energiebedarf beurteilt. Für alle drei 
\gls{el}-Technologien ergeben sich die gleichen Auswirkungen bezogen auf den jeweiligen Kennwert. Den größen Einfluss auf den spezifischen Energiebedarf zeigt sich für alle Elektrolyseurtypen mit Zunahme der Degradation. Diese negativen Effekte zeigen sich ebenfalls in der Abnahme des Systemwirkungsgrades, welcher bereits mit Hilfe von Abbildung (\ref{fig:barplot_eco_9800_eta}) beschrieben wurde.\\
Die restlichen Parmeter zeigen in den gewählten Bereichen für die \gls{pem}-\gls{el} nur minimale Auswirkungen, für die \gls{soel} Veränderungen im Bereich von ca. $0,01~kWh$ und für die \gls{ael} einen Einfluss von $0,06~kWh$ bei der Verringerung der Minimal-Last. 

\begin{figure}[H]
	\centering
	%\includegraphics[width=0.49\textwidth]{/home/dafu/Schreibtisch/Master-Projekt/Doku/Abb/ .png}
	\includegraphics[width=\textwidth]{/home/dafu/Schreibtisch/Master-Projekt/Doku/Abb/Graph/eco/2019-03-02--17-13_barplot_9800_E_spec.pdf}
	\caption[Spezifischer Energiebedarf; Sensitivitätsanalyse]{Spezifischer Energiebedarf der Sensitivitätsanalyse für Ausschnitt der WEA-Zeitreihe}
	\label{fig:barplot_eco_9800_Espec}
\end{figure}
Als letzte Sensitivitätsanalyse wurden die spezifischen Produktionskosten in Abhängigkeit der aus Tabelle (\ref{tab:sens_params}) bekannten Kennwerte in Abbildung (\ref{fig:barplot_eco_9800_Cost} ) dargestellt. Diese zeigt für alle \gls{el}-Technologien eine deutliche Erhöhung der Kosten bezogen auf die produzierte Menge an Wasserstoff. Die Ursache dafür liegt in den spezifischen Investmentkosten, die auf ein Jahr bezogen sind nicht Anteilig auf den Verkürzten Zeitraum bezogen werden.\\
Die mit Abstand größten Abweichungen weist die \gls{soel} mit einer maximalen Abweichung von $3,6~EUR/m^3_N$ auf. Dieser Wert ergibt sich durch die Erhöhung der Minimal-Leistung der Anlage. Für alle drei Technologien hat ansonsten die Membrandegradation den größten Einfluss.

\begin{figure}[H]
	\centering
	%\includegraphics[width=0.49\textwidth]{/home/dafu/Schreibtisch/Master-Projekt/Doku/Abb/ .png}
	\includegraphics[width=\textwidth]{/home/dafu/Schreibtisch/Master-Projekt/Doku/Abb/Graph/eco/2019-03-02--17-10_barplot_9800_Cost.pdf}
	\caption[Spezifische Produktionskosten; Sensitivitätsanalyse]{Spezifische Produktionskosten von Wasserstoff der Sensitivitätsanalyse für Ausschnitt der WEA-Zeitreihe}
	\label{fig:barplot_eco_9800_Cost} 
\end{figure}
\subsubsection{Zusammenfassung-Sensitivitätsanalyse}
Entweder wenig Einfluss, oder Werte Bereich nicht sinnvoll gewählt
Die Häufigkeit niedriger Leistungswerte besitzt offenbar relativ deutlichen Einfluss auf \gls{pem}-\gls{el} und \gls{soel}, jedoch mit entgegengesetzte Wirkung.

\subsection{Ökonomische Analyse}
\subsubsection{Spezifische Wasserstoff(H2) Produktionskosten} %?????
Für einen Kostendeckenden Betrieb sowie wirtschaftliche Wettbewerbsfähigkeit eines Elektrolyseurs sind die spezifischen Kosten des Produktgases von zentraler Bedeutung.

Für die \gls{eis}- und \gls{res}-Signale weist die \gls{soel} die höchsten Kosten im Bereich von $1,65~\euro{}/m^3_N$ für \gls{eis} und $1,1~\euro{}/m^3$ für \gls{res} auf.
Im Fall des \gls{eis}-Betriebes liegen die Kosten der \gls{ael} etwas unterhalb der Vergleichbaren Kosten die bei einem \gls{pem}-Betrieb resultierenden. Im \gls{res}- und \gls{wea}-Fall kann die \gls{pem}-\gls{el} am günstigsten Produzieren. %(Abschnitt())???????

Für sämtliche \gls{cost}-Berechnungen liegen nahezu identische Werte vor, wobei die \gls{ael} mit Werten von ca. $0,8~\euro{}/m^3_N$ am höchsten liegt, die  \gls{pem}-\gls{el} mit $0,7~\euro{}/m^3_N$ etwas darunter liegt und  die \gls{soel} Werte von $0,6~\euro{}/m^3_N$ aufweist.\\
Aus dem \gls{cost}-Sigal wird deutlich das die Skalierung bei einem Signal mit hohem Volllastanteil keinen wahrnehmbaren Effekt bezüglich der Anlagenkennwerte aufweist.
%WEA EIS RES:
%cost_spez 
%• Eis: AEL = 1,04; PEM= 1,08; SOEL= 1,61
%◦ aufsteigend von AEL zu PEM zu SOEL
%◦ PEM/ AEL ähnlich groß
%
%• Res: AEL = 0,94; PEM= 0,86; SOEL= 1,09
%◦ aufsteigend von PEM zu AEL zu SOEL
%◦ SOEL schlecht  viel Dynamik durch das Input-Sig.
%• Wea: AEL = 0,86; PEM= 0,72; SOEL= 0,87
%◦ aufsteigend von PEM zu SOEL zu AEL
%◦ viel Dynamik, aber hohe Anzahl an Volllaststd.
-----------------------------------------------------------
\begin{figure}[!tbp]
	\centering
	\begin{minipage}[b]{0.49\textwidth}
	\includegraphics[width=\textwidth]{/home/dafu/Schreibtisch/Master-Projekt/Doku/Abb/Graph/eco/2019-03-02--14-37_barplot_EISMAN_RESIDUAL_WEA_Cost.pdf}
\caption[Vergleich spezifischer Produktionskosten EIS,RES, WEA]{Vergleich spezifischer Produktionskosten EISMAN, RESIDUAL, WEA}
\label{fig:analy_nC_speCo} 
	\end{minipage}
	\hfill
	\begin{minipage}[b]{0.49\textwidth}
	\includegraphics[width=\textwidth]{/home/dafu/Schreibtisch/Master-Projekt/Doku/Abb/Graph/eco/2019-03-02--10-29_barplot_Cost_S_Cost_M_Cost_L_Cost.pdf}
\caption[Vergleich spezifischer Produktionskosten COST]{Vergleich spezifischer Produktionskosten COST (S, M, L)}
\label{fig:analy_COST_speCo} 
	\end{minipage}
\end{figure}
%COST:
%cost\_spez
%• Cost S: AEL = 0,81; PEM=0,71; SOEL= 0,59
%• Cost M: AEL = 0,82; PEM=0,70; SOEL= 0,61
%• Cost L: AEL = 0,82; PEM=0,70; SOEL= 0,60
%• Cost S: aufsteigend SOEL zu PEM zu AEL
%• Cost M: aufsteigend SOEL zu PEM zu AEL
%• Cost L: aufsteigend SOEL zu PEM zu AEL
%• Skalierung im Volllastbereich kaum Effekt, weil viele lineare Abh. zB ct…
%• AEL am teuersten  allg. Betriebszustand schlecht
%• AEL und PEM unterschiedlich, weil PEM Auslegung sehr gut, AEL Auslegung eher schlecht

\subsubsection{Spezifische (H2) Produktionskosten und spezifischer Invest}
\begin{figure}[H]
	
	\centering
	%\includegraphics[width=0.49\textwidth]{/home/dafu/Schreibtisch/Master-Projekt/Doku/Abb/ .png}
	\includegraphics[width=\textwidth]{/home/dafu/Schreibtisch/Master-Projekt/Doku/Abb/Graph/eco/2019-03-02--15-41_plot_EIS_RES_WEA.pdf}
	\caption[{Spezifische Produktionskosten; variierte Investitionskosten EIS, RES, WEA}]{Spezifische Produktionskosten über variierten spezifischen Investitionskosten }
	\label{fig:analy_ecoVAls_all} 
\end{figure}

\begin{figure}[H]
	
	\centering
	%\includegraphics[width=0.49\textwidth]{/home/dafu/Schreibtisch/Master-Projekt/Doku/Abb/ .png}
	\includegraphics[width=\textwidth]{/home/dafu/Schreibtisch/Master-Projekt/Doku/Abb/Graph/eco/2019-03-02--10-33_plot_Costs.pdf}
	\caption[Spezifische Produktionskosten; variierte Investitionskosten COST]{Spezifische Produktionskosten über variierten spezifischen Investitionskosten COST (S, M, L)}
	\label{fig:analy_ecoVAls_COST} 
\end{figure}




%\subsection{Einzel - Ergebnisse}
%\subsubsection{Strom-Spannungszusammenhänge}
%
%\subsection{Verifizierung}
%-> Abgleich mit Literatur
%\subsection{Gesamt-Simulation}
%
%\section{Ökonomische Betrachtung}
%\subsection{Komponenten-Wahl}
%\subsubsection{PEM}
%-> S. 13 [Lettenmeier-Diss]-> Katalysator, Anode!!!
%
%\subsection{Skalierung}
%\subsection{Wasserstoff-Vertriebsmöglichkeiten}
%- demibras (Hiwi/H2-Distr/sonstige) -> Tab 4, ff
%
%\subsection{Sauerstoff-Vertriebsmöglichkeiten}

\section{Bewertung}
Das angestrebte Modell zur Abbildung relevanter Betriebscharakteristiken verschiedener Elektrolyse-Typen konnte innerhalb des Bearbeitungszeitraums erstellt und erfolgreich getestet werden. Im Folgenden soll nun eine Bewertung der innerhalb der vorliegenden Arbeit erzielten Ergebnisse erfolgen. 
\subsection{Simulationsergebnisse}
Unter Berücksichtigungen von Abweichungen und Vereinfachungen innerhalb der Modellierung der Elektrochemie kann das vorliegende Modell als valide erachtet werden. Zwar ist keine Abbildung sämtlicher elektrochemischer Einflussgrößen möglich, dennoch kann eine charakteristische Darstellung der unterschiedlichen \gls{el}-Typen erreicht werden. Damit ist ebenfalls die Untersuchung bzw. der Abgleich der entsprechenden Technologietypen bezüglich einer bestimmten Anwendungssituation möglich. Für die betrachteten Betriebsarten sind zum teil deutliche Unterschiede bzw. Vor- und Nachteile einzelner \gls{el}-Typen möglich.

\subsection{Ökonomie}
\subsection{Systemdienlichkeit}


\section{Diskussion/ Fazit}

\subsection{Überprüfung der Methodik}
%Für eine aussagekräftige analyse wäre es ggf. zielführender, einen gezielteren Abgleich elektrochemischer Parameter vorzunehmen, um Abweichungen zu Liteatur- und Testanlagen zu minimieren und eine gleichmäßigere basis für vergleichende Aussagen zu schaffen.\\
%
%PEM-> durch Werte aus verschiedenen Modellen Validität schwierig zu erreichen
%
%Durch intensiveren Abgleich von Modell-Parametern im Vorfeld bzw. insgesamt die Verwendung von realistischeren Modellen wird die Aussagekraft deutlich erhöht.
 

\subsection{Ausblick}
\subsubsection{Alternative Anlagen-Konfig}
%\begin{itemize}
%	\item zur Steigerung der Anlagendynamik bzw. sinnvoller Eingrenzung ''verträglicher'' Stromdichten/-bereiche -> -> Kombination mit Akkumulator
%	\item -->>dadurch ggf. durch Kombination mit EE Inselnetz-Aufbau möglich? 
%\end{itemize}
\subsubsection{Perspektivisch zu implementieren}
%\begin{itemize}
%	\item elektrochemische Dynamik
%	\item[] möglich, da Zahl innerer Schleifendurchläufe beliebig hoch wählbar, ABER: Berechnungszeit
%	\item Druck-Dynamik
%	\item Ausgabe von Wirkungsgraden ( Faraday, Spannungs-)???
%	\item saisonale Umgebungs-Temperatur (Einhausung d. EL) bzw. Heizenergie
%	\item Modellierung über mehrere Jahre:
%	\item isentrope Kompression nach |Thermodynamics of pressurized gas storage
%	\begin{itemize}
%		\item Korrelation mit Wetterdaten
%		\item[] $\rightarrow$ jährlich verschiedene Datensätze
%	\end{itemize}	
%	\item optimale Betriebsweise über x Jahre:
%	\begin{itemize}
%		\item Druck <-> Membrandicke
%		\item Temp./ Degr.
%		
%	\end{itemize}
%	\ zu Stand-By / Off
%	\begin{itemize}
%		\item ggf. zusätzliche Spalte in df zu An/Aus -> über Zeitabstände von Null-Leistung
%		\item für Echtzeit-Argumentation: ggf. auch über Wetterdaten % %https://openfredproject.wordpress.com/
%	\end{itemize} 
%	\item Thermo-Management
%	\begin{itemize}
%		\item Saisonale Simulation...Außentemperaturanpassung?
%	\end{itemize}
%\end{itemize}
%\subsubsection{Ökonomie}
%-> Day-Ahead-Market implementieren 

%\subsubsection{Reflektion Projektmanagement}
%\begin{itemize}
%	\item Zeitplaneinhaltung schwierig
%	\item Protokollierung ok
%	\item Ergebnissicherung mangelhaft
%	\item -> Striktere Überprüfung des individuellen Fortschritts -> Ehrlichkeit
%	\item Projektpartner <-> Kommilitonen....schwierig
%\end{itemize}

%En general no se numera las subsecciones. Esto se logra agregando un asterisco: %\verb|\subsection*{titulo}|
% \subsection*{Literaturverzeichnis}
% Citar es facil:
% \citet*[pag. 4]{silverstein1964giving}
% y \cite{serway2010fisica}, \citep{Erdos65}. Agreguen sus referencias a \verb|labibliografia.bib|.

% \subsubsection*{Minisección}
% Se puede ir hasta un nivel bajo.
% \section{Método}
% Otros deberían poder repetir el trabajo con la información disponible aquí.
% \section{Resultados}
% Hagan click sobre el indice o la referencias para navegar el documento.
% \section{Discusión o conclusión}
% La discusión debería ser corta y no tener subsecciones.

%\clearpage 
%\newpage

\bibliographystyle{alpha}
%\bibliographystyle{natdin}
\bibliography{proj-refs.bib}

\appendix
%\addcontentsline{toc}{chapter}{Anhang}
%\addtocontents{toc}{}%\protect\addtokomafont{chapterentry}{Anhang\ }
\section{Parameter-Liste}

\begin{table}[]
	\tiny
	
	\caption{Tabelle verwendeter Parameter}
	\label{tab-param2}
	\begin{tabular*}{\textwidth}{lllllllll}
		
		
		
		\multirow{2}{*}{\textbf{Größe}} & \multirow{2}{*}{\textbf{Symbol}} & \multirow{2}{*}{\textbf{Einheit}} & \multicolumn{6}{c}{\multirow{2}{*}{\textbf{Wert}}} \\
		
		&&&&&&&& \\ \hline \hline
		
		&&&&&&&& \\
		
		\textbf{Globale Kennwerte}&&&&&&&& \\
		
		Umgebungstemperatur 			& $T_{amb}$			& $\degree{C}$		& \multicolumn{6}{l}{25} \\
		Umgebungsdruck		 			& $p_{amb}$			& $bar		$		& \multicolumn{6}{l}{1,013} \\
		
		&&&&&&&& \\
		
		
		&&&  \textbf{PEM} & Ref. & \textbf{AEL} & Ref. & \textbf{SOEL} & Ref. \\
		
		\textbf{Anlagenspezifische Kennwerte}&&&&&&&& \\
		Betriebsdruck 							& $p$ 					& $bar$ 			& 1,013		& 	 		& 1,013 				& 	 		& 1,013					& 	 		\\
		Betriebstemperatur						& $T_{tar}$				& $\degree{C}$ 		& 77 		& Tjakrs	& 60 					& Hammoudi	& 850 					& Ni 		\\[2ex]
		maximale Temperatur						& $T_{max}$				& $\degree{C}$ 		& 85 		& Tjakrs	& 80 					& Hammoudi	& 1050 					& Ni 		\\
		minimale Temperatur						& $T_{min}$				& $\degree{C}$ 		& 20 		& Tjakrs	& 20 					& Hammoudi	& 700 					& Ni 		\\
		maximale Stromdichte 					& $i_{max}$ 			& $A/{cm^2}$ 		& 3.0 		& Tjakrs	& 0.3 					& Hammoudi	& 0.8 					& Petipas 	\\
		Zellen pro Stack						& $N_{ce_{st}}$			& -				 	& 200		& Gabr		& 24					& Hammoudi 	& 400 					& Petipas 	\\
		spez. Wärmekapazität					& $C_{t_{spec}}$		& $kJ/KN_{ce}$ 		& 1.2500	& Gabr		& 4.8906 				& Ulle		& 0,0969				& Petipas 	\\
		spez. Wärmedurchgangskoeffizient		& $H_{AX_{spec}}$		& $W/Kcm^{2}N_{ce}$				& 0.028		& Gabr					& 0,011 	& Gabr		& ??? 		& - 		\\
		spez. Wärmedurchgangswiderstand			& $R_{t_{spec}}$		& $K/W$				& 0,167 	& Ulle		& 0,167					& Ulle		& 0,840					& Petipas 	\\
		Kühlwassereingangstemperatur			& $T_{cw_{in}}$			& $\degree{C}$ 		& 15 		& Ulle		& 15 					& Ulle		& 	 					& 	 		\\	
		Degradationsgradient 					& $deg$ 				& ${\mu}V/h$		& 4-8 		& Buttler	& 1-2 					& Buttler	& 21 					& De-Niang 	\\
		
		&&&&&&&& \\
		\textbf{Anoden-Kennwerte}&&&&&&&& \\
		Elektrodenmaterial 						& - 					& -					& - 		& -	 		& - 					& Hammoudi	& - 					& -			\\
		Elektrodenoberfläche 					& $A_{an}$				& $cm^{2}$			& 170 		& Gabriel	& 300					& Hammoudi 	& 100 					& Petipas 	\\
		Elektrodendurchmesser 					& $\delta_{an}$			& $cm$ 				& 0,14 		& Chande 	& 0,2 					& Hammoudi	& 0,05 					& Zhang 	\\
		Seperatordistanz 						& $l_{an}$ 				& $cm$ 				& 	 		& 	 		& 0,125 				& Hammoudi	& 	 					& 			\\
		
		Tortuosität 							& $\tau_{an}$			& - 				& 	 		& 	 		& 3,65 					& Abdin		& 6 					& Zhang 	\\
		Porosität 								& $\epsilon_{an}$		& - 				& 	 		& 	 		& 0,30 					& Abdin		& 0,3 					& Zhang 	\\
		Porenradius (Knudsen-Diff.) 			& $r_{P_{an}}$ 			& $cm$ 				& 	 		& 	 		& $1\times{10^{-4}}$	& Abdin		& $0,5\times{10^{-4}}$ 	& Zhang 	\\
		Rugosität 								& $rug_{an}$			& $m^{2}/m^{2}$		& 150 		& Chande	& 	 					& 			& 	 					& 	 		\\
		Elektrodenrauheit 						& $\gamma_{an}$ 		& -	 				& 	 		& 	 		& 1,25 					& Abdin		&  						& 	 		\\
		
		Ladungstransferkoeffizient 				& $\alpha_{an}$ 		& - 				& 0,65 		& Chande	& 1,65 					& Abdin		& 	 					& 	 		\\
		Freie Aktivierungsenergie 				& ${\Delta}G_{c_{an}}$	& $J/mol$ 			& 62836		& Chande	& 80510 				& Abdin		& 120000 				& Zhang 	\\
		Referenzaustauschstromdichte			& $i_{0ref_{an}}$		& $A/cm^{2}$ 		& 	 		& 	 		& $1\times{10^{-11}}$ 	& Abdin		& 	 					& 	 		\\
		Referenztemperatur $i_{0ref_{an}}$ 		& $T_{ref_{an}}$ 		& $\degree{C}$	 	& 	 		& 	 		& 90 					& Abdin		& 	 					& 	 		\\	
		Anodic rate parameter					& $k_{0_{an}}$			& $mol/Ksm^{2}$		& $4,63\times{10^{-3}}$ & Chande 				&& 	 		& 	 					& 	 		\\
		Elektrodenleitfähigkeit					& $\sigma_{an}$ 		& $S/m$				& 13700	 	& Chande 	& 						& 	 		& 	 					& 	 		\\
		
		&&&&&&&& \\
		\textbf{Kathoden-Kennwerte}&&&&&&&& \\
		Elektrodenmaterial 						& - 					& - 				& - 		& - 		& -		 				& Hammoudi 	& - 					& - 		\\
		Elektrodenoberfläche 					& $A_{ca}$ 				& $cm^{2}$ 			& 170 		& Gabriel	& 300 					& Hammoudi 	& 100 					& Petipas	\\
		Elektrodendurchmesser 					& $\delta_{ca}$ 		& $cm$ 				& 0,235		& Chande	& 0,2 					& Hammoudi 	& 0,5 					& Zhang 	\\
		Seperatordistanz 						& $l_{ca}$ 				& $cm$ 				& 	 		& 	 		& 0,125 				& Hammoudi 	& 	 					& 	 		\\
		
		Tortuosität 							& $\tau_{ca}$	 		& -	 				& 	 		& 			& 3,65 					& Abdin 	& 6 					& Zhang 	\\
		Porosität 								& $\epsilon_{ca}$		& -	 				& 	 		& 			& 0,30 					& Abdin 	& 0,3 					& Zhang 	\\
		Porenradius (Knudsen-Diff.) 			& $r_{P_{ca}}$			& $cm$ 				& 	 		& 	 		& $1\times{10^{-4}}$	& Abdin 	& $0,5\times{10^{-4}}$ 	& Zhang 	\\
		Rugosität 								& $rug_{ca}$			& $m^{2}/m^{2}$		& 150 		& Chande	& 	 					& 		 	& 	 					&	 		\\
		Elektrodenrauheit 						& $\gamma_{ca}$ 		& 	 				& 	 		& 	 		& 1,05 					& Abdin 	& 	 					& 	 		\\
		
		Ladungstransferkoeffizient 				& $\alpha_{ca}$ 		& - 				& 0,51 		& Chande	& 0,73 					& Abdin 	& 	 					& 	 		\\
		Freie Aktivierungsenergie 				& ${\Delta}G_{c_{ca}}$ 	& $J/mol$		    & 24359		& Chande	& 80510 				& Abdin 	& 100000 				& Zhang 	\\
		Referenzaustauschstromdichte			& $i_{0ref_{ca}}$ 		& $A/cm^{2}$		& 	 		& 	 		& $1\times{10^{-3}}$ 	& Abdin 	& 	 					& 	 		\\
		Referenztemperatur $i_{0ref_{ca}}$ 		& $T_{ref_{ca}}$ 		& $\degree{C}$	 	& 	 		& 	 		& 90 					& Abdin 	& 	 					& 	 		\\
		Cathodic rate parameter					& $k_{0_{ca}}$			& $mol/Ksm^{2}$		& $1,10\times{10^{-3}}$ & Chande 				&&			& 						& 	 		\\
		Elektrodenleitfähigkeit					& $\sigma_{ca}$ 		& $S/m$				& 46	 	& Chande	& 	 					&  			& 	 					& 	 		\\
		
		&&&&&&&& \\
		\textbf{Diaphragma-Kennwerte}&&&&&&&& \\
		Oberfläche 								& $A_{dia}$ 			& $cm^{2}$ 			&&& 300 	& Abdin		&&\\
		Durchmesser 							& $\delta_{dia}$ 		& $\mu{m}$ 			&&& 320 	& Abdin		&&\\
		Tortuosität 							& $\tau{dia}$			& - 				&&& 3,65 	& Abdin		&&\\
		Porosität 								& $\epsilon_{dia}$ 		& - 				&&& 0,30 	& Abdin		&&\\
		Wettability 							& $\omega_{dia}$ 		& - 				&&& 0,85 	& Abdin		&&\\
		Spezifische Resistivität 				& $\rho_{dia}$ 			& ${\Omega}/cm$ 	&&& 1,5 	& Perl Zirf &&\\
		
		&&&&&&&& \\
		\textbf{Elektrolyt-Kennwerte}&&&&&&&& \\
		Durchmesser								& $\delta_{mem}$		& $\mu{m}$			& 200		& Gabriel	&			&			& 500   & Zhang \\			
		Membranfeuchtigkeit						& $\lambda_{mem}$		& -					& 16  		& Chande	& 			& 			& 		& 		\\
		Membranwiderstand						& $R_{mem}$ 			& $\Omega$			& 0.096		& Tjarks	& 	 		& 			& 	 	& 		\\
		Diff. Permeabilität $O_{2}$				& $\epsilon_{O_{2}}$	& $mol/cmsbar$		& $2,00\times{10^{-11}}$	& Schalenbach			&			&						&	 \\
		Diff. Permeabilität $H_{2}$				& $\epsilon_{H_{2}}$	& $mol/cmsbar$		& $4,65\times{10^{-11}}$	& Schalenbach			&			&						&	 \\
		
		&&&&&&&& \\
		KOH-Konzentration 						& - 					& $wt\%$ 			&			&			& 30 		& Henao		&		& 		\\
		KOH-Volumenstrom pro Stack 				& $V_{KOH}$ 			& $l/min$ 			&			&			& 0.5 		& Henao		&		& 		\\
		
		
		
		
	\end{tabular*}
\end{table}
\section{Simulationergebnisse}
\subsection{EISMAN}
\subsubsection{PEM}
\begin{figure}[H]
	\centering
	\includegraphics[width=\textwidth]{/home/dafu/Schreibtisch/Master-Projekt/Doku/Abb/Graph/eis_se/EIS_PEM__EIS_se__.pdf}
	
	\caption[Simulation EISMAN im Jahresbetrieb \gls{pem}-\gls{el}]{Simulation EISMAN im Jahresbetrieb \gls{pem}-\gls{el}; Darstellung von Stack-Temperatur \gls{T_stack} und Kühlwassermassenstrom \gls{m_c} (oben), Zellspannung \gls{u_cell} und Zellstrom \gls{i}(2. v.o.), Eingangsleistung \gls{P_in},  tatsächlich abgerufene Leistung \gls{P_act} und Stackleistung \gls{P_st} (2. v. u.) sowie molare Produktgasströme \gls{n_H2},\gls{n_O2} und Produktgaskontamination \gls{H2inO2}}
	\label{fig:plt_se_EIS_PEM} 
\end{figure}
\subsubsection{SOEL}
\begin{figure}[H]
	
	\centering
	\includegraphics[width=\textwidth]{/home/dafu/Schreibtisch/Master-Projekt/Doku/Abb/Graph/eis_se/EIS_SOEL__EIS_se__.pdf}
	\includegraphics[width=0.8\textwidth]{/home/dafu/Schreibtisch/Master-Projekt/Doku/Abb/Graph/eis_se/EIS_SOEL__EIS_se__.pdf}
	
	\caption[Simulation EISMAN im Jahresbetrieb \gls{soel}]{Simulation EISMAN im Jahresbetrieb \gls{soel}; Darstellung von Stack-Temperatur \gls{T_stack} und Kühlwassermassenstrom \gls{m_c} (oben), Zellspannung \gls{u_cell} und Zellstrom \gls{i}(2. v.o.), Eingangsleistung \gls{P_in},  tatsächlich abgerufene Leistung \gls{P_act} und Stackleistung \gls{P_st} (2. v. u.) sowie molare Produktgasströme \gls{n_H2},\gls{n_O2} und Produktgaskontamination \gls{H2inO2}}
	\label{fig:plt_se_EIS_SOEL} 
\end{figure}
\subsection{RES}
\subsubsection{AEL}
\begin{figure}[H]
	\centering
	\includegraphics[width=\textwidth]{/home/dafu/Schreibtisch/Master-Projekt/Doku/Abb/Graph/res_se/RES_AEL__RES_se__.pdf}
	
	\caption[Simulation RESIDUAL im Jahresbetrieb \gls{ael}]{Simulation RESIDUAL im Jahresbetrieb \gls{ael}; Darstellung von Stack-Temperatur \gls{T_stack} und Kühlwassermassenstrom \gls{m_c} (oben), Zellspannung \gls{u_cell} und Zellstrom \gls{i}(2. v.o.), Eingangsleistung \gls{P_in},  tatsächlich abgerufene Leistung \gls{P_act} und Stackleistung \gls{P_st} (2. v. u.) sowie molare Produktgasströme \gls{n_H2},\gls{n_O2} und Produktgaskontamination \gls{H2inO2}}
	\label{fig:plt_se_RES_AEL} 
\end{figure}

\subsubsection{SOEL}
\begin{figure}[H]
	
	\centering
	\includegraphics[width=\textwidth]{/home/dafu/Schreibtisch/Master-Projekt/Doku/Abb/Graph/res_se/RES_SOEL__RES_se__.pdf}
	\caption[Simulation RESIDUAL im Jahresbetrieb \gls{soel}]{Simulation RESIDUAL im Jahresbetrieb \gls{soel}; Darstellung von Stack-Temperatur \gls{T_stack} und Kühlwassermassenstrom \gls{m_c} (oben), Zellspannung \gls{u_cell} und Zellstrom \gls{i}(2. v.o.), Eingangsleistung \gls{P_in},  tatsächlich abgerufene Leistung \gls{P_act} und Stackleistung \gls{P_st} (2. v. u.) sowie molare Produktgasströme \gls{n_H2},\gls{n_O2} und Produktgaskontamination \gls{H2inO2}}
	\label{fig:plt_se_RES_SOEL} 
\end{figure}

\subsection{WEA}
\subsubsection{AEL}
\begin{figure}[H]
	\centering
	\includegraphics[width=\textwidth]{/home/dafu/Schreibtisch/Master-Projekt/Doku/Abb/Graph/wea_se/WEA_AEL__WEA_se__.pdf}
	
	\caption[Simulation WEA im Jahresbetrieb \gls{ael}]{Simulation WEA im Jahresbetrieb \gls{ael}; Darstellung von Stack-Temperatur \gls{T_stack} und Kühlwassermassenstrom \gls{m_c} (oben), Zellspannung \gls{u_cell} und Zellstrom \gls{i}(2. v.o.), Eingangsleistung \gls{P_in},  tatsächlich abgerufene Leistung \gls{P_act} und Stackleistung \gls{P_st} (2. v. u.) sowie molare Produktgasströme \gls{n_H2},\gls{n_O2} und Produktgaskontamination \gls{H2inO2}}
	\label{fig:plt_se_WEA_AEL} 
\end{figure}
\subsubsection{PEM}
\begin{figure}
	\centering
	\includegraphics[width=\textwidth]{/home/dafu/Schreibtisch/Master-Projekt/Doku/Abb/Graph/wea_se/WEA_PEM__WEA_se___.pdf}
	
	\caption[Simulation WEA im Jahresbetrieb \gls{pem}-\gls{el}]{Simulation WEA im Jahresbetrieb \gls{pem}-\gls{el}; Darstellung von Stack-Temperatur \gls{T_stack} und Kühlwassermassenstrom \gls{m_c} (oben), Zellspannung \gls{u_cell} und Zellstrom \gls{i}(2. v.o.), Eingangsleistung \gls{P_in},  tatsächlich abgerufene Leistung \gls{P_act} und Stackleistung \gls{P_st} (2. v. u.) sowie molare Produktgasströme \gls{n_H2},\gls{n_O2} und Produktgaskontamination \gls{H2inO2}}
	\label{fig:plt_se_WEA_PEM} 
\end{figure}
\subsection{COST-M}
\subsubsection{AEL}
\begin{figure}
	\centering
	\includegraphics[width=\textwidth]{/home/dafu/Schreibtisch/Master-Projekt/Doku/Abb/Graph/cost_m_se/COST_M_AEL__COST_M_se___.pdf}
	
	\caption[Simulation COST\_M im Jahresbetrieb \gls{ael}]{Simulation COST\_M im Jahresbetrieb \gls{ael}; Darstellung von Stack-Temperatur \gls{T_stack} und Kühlwassermassenstrom \gls{m_c} (oben), Zellspannung \gls{u_cell} und Zellstrom \gls{i}(2. v.o.), Eingangsleistung \gls{P_in},  tatsächlich abgerufene Leistung \gls{P_act} und Stackleistung \gls{P_st} (2. v. u.) sowie molare Produktgasströme \gls{n_H2},\gls{n_O2} und Produktgaskontamination \gls{H2inO2}}
	\label{fig:plt_se_COST_ALE} 
\end{figure}
\subsubsection{SOEL}
\begin{figure}
	\centering
	\includegraphics[width=\textwidth]{/home/dafu/Schreibtisch/Master-Projekt/Doku/Abb/Graph/cost_m_se/COST_M_SOEL__COST_M_se___.pdf}
	
	\caption[Simulation COST\_M im Jahresbetrieb \gls{soel}]{Simulation COST\_M im Jahresbetrieb \gls{soel}; Darstellung von Stack-Temperatur \gls{T_stack} und Kühlwassermassenstrom \gls{m_c} (oben), Zellspannung \gls{u_cell} und Zellstrom \gls{i}(2. v.o.), Eingangsleistung \gls{P_in},  tatsächlich abgerufene Leistung \gls{P_act} und Stackleistung \gls{P_st} (2. v. u.) sowie molare Produktgasströme \gls{n_H2},\gls{n_O2} und Produktgaskontamination \gls{H2inO2}}
	\label{fig:plt_se_COST_SOEL} 
\end{figure}



\end{document}
